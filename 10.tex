\section{Lecture -- Week 10}


The idea used in Example \ref{exa:for_HK} can be generalized. 

\begin{optional}
\begin{example}
\index{Double coset}
Let $G$ be a group and $H$ and $K$ be subgroups of $G$. Let the group $L=H\times K$ act on $G$ by
\[
(h,k)\cdot g=hgk^{-1}.
\]
The orbits are sets of the form 
\[
HgK=\{hgk:h\in H,\,k\in K\}.
\]
These sets are called \emph{double $(H,K)$-coset}. 
In particular, two double cosets are either disjoint 
or equal. Moreover, $G$ admits a decomposition 
as a disjoint union of double cosets, that is 
\[
G=\bigcup_{i\in I}Hg_iK,
\]
for some set $I$. Now we compute 
\[
L_g=\{(h,k)\in H\times K:hgk^{-1}=g\}=\{(h,g^{-1}hg)\in H\times K\}.
\]
Then $|L_g|=|H\cap gKg^{-1}|$, because there is a bijection between $L_g$ and
$H\cap gKg^{-1}$. By the fundamental counting principle (Theorem~\ref{thm:fundamental}), 
\[
|HgK|=(L:L_g)=\frac{|H\times K|}{|H\cap gKg^{-1}|}=\frac{|H||K|}{|H\cap gKg^{-1}|}.
\]
\end{example}
\end{optional}

As another application, we compute the
order of the group $\GL_n(p)$ for $n\geq1$ and 
a prime number $p$. 
The argument also works for
the group $\GL_n(q)$ in the case where
$q$ is a power of the prime number $p$.

\begin{example}
Let $p$ be a prime number and $K=\Z/p$.
We claim that 
\[
|\GL_n(p)|=(p^n-1)p^{n-1}|\GL_{n-1}(p)|,
\]
and hence 
\[
|\GL_n(p)|=(p^n-1)(p^n-p)\cdots (p^n-p^{n-1}).
\]

The formula is valid if $n\in\{1,2\}$. 
Assume that it holds for $n-1\geq1$.
The group $G=\GL_{n}(p)$ acts on
$K^{n\times 1}$ by left multiplication. 
How are the orbits? 
Since for every 
non-zero $v,w\in K^{n\times 1}$, then there exists 
$g\in G$ such that $gv=w$. Thus there are only two orbits.
One orbit is the one-element orbit 
of the zero column vector of $K^{n\times1}$, and 
the other orbit is the set $\mathcal{O}$ of non-zero vectors of $K^{n\times1}$. 
By the fundamental counting principle,  
\[
p^{n+1}-1=|\mathcal{O}|=(G:G_{v}),
\]
for every $v\in\mathcal{O}$, that is every $v\in K^{n\times 1}$. 

To compute the stabilizer $G_v$ easily, take 
\[
v=\colvec{4}{1}{0}{\vdots}{0}\in\mathcal{O}. 
\]
If $g=(g_{ij})\in G$ is such that
$gv=v$, then 
\[
g=
\begin{pmatrix}
1 & g_{12} & \cdots & g_{1n}\\
0 & g_{22} & \cdots & g_{2n}\\
\vdots & \vdots & \ddots &\vdots\\
0 & g_{n1} & \cdots & g_{nn}
\end{pmatrix}.
\]
Therefore $|G_{v}|=p^{n-1}|\GL_{n-1}(p)|$, as the submatrix 
$(g_{ij})_{2\leq i,j\leq n}$ is invertible and the 
$g_{1j}$'s can be chosen 
arbitrarily for all $j\in\{2,\dots,n\}$.
Hence 
\[
p^{n}-1=\frac{|G|}{|G_{v}|}=\frac{|\GL_n(p)|}{p^{n-1}|\GL_{n-1}(p)|},
\]
which implies the formula we wanted to prove.
\end{example}

\subsection{$p$-groups}

\index{Action!fixed points}
\index{Class equation}
Let $G$ be a finite group acting on a finite 
set $X$. Let 
\[
\Fix(X)=\{x\in X:g\cdot x=x\text{ for all $g\in G$}\}
\]
be the set of \emph{fixed points} of $X$, that is the set of one-elements 
orbits. We know that $X$ decomposes as a disjoint 
union of orbits. In particular, 
\[
X=\Fix(X)\cup \mathcal{O}_1\cup\cdots\mathcal{O}_k,
\]
where $\mathcal{O}_1,\dots,\mathcal{O}_k$ are orbits such that 
$|\mathcal{O}_j|\geq2$ for all $j\in\{1,\dots,k\}$. 
If we apply cardinality and use the  
fundamental counting principle, 
\begin{equation}
\label{eq:clases}
|X|=|\Fix(X)|+\sum_{i=1}^k|\mathcal{O}_i|
=|\Fix(X)|+\sum_{i=1}^k(G:G_{x_i}),
\end{equation}
where $x_j\in\mathcal{O}_j$ and 
$(G:G_{x_i})\geq2$ for all $j\in\{1,\dots,k\}$. 
Equality~\eqref{eq:clases} is extremely 
useful and is called the 
\emph{class equation}.

\begin{example}
Let a finite group $G$ act on $G$ by conjugation. 
Then $\Fix(G)=Z(G)$ and 
\[
|G|=|Z(G)|+\sum_{i=1}^k(G:C_G(x_i)),
\]
for some $x_1,\dots,x_k\in G$ such that 
$(G:C_G(x_i))\geq2$ for all $i\in\{1,\dots,k\}$.
\end{example}

\begin{definition}
\index{$p$-group}
Let $p$ be a prime number. We say 
that $G$ is a \emph{$p$-group} if $|G|=p^m$ for some $m\geq0$.
\end{definition}

\begin{theorem}
Let $p$ be a prime number and 
$G$ be a $p$-group. If $N$ is a non-trivial normal subgroup of $G$, then
$N\cap Z(G)\ne\{1\}$.
\end{theorem}

\begin{proof}
Since $N$ is normal in $G$, $G$ acts on $N$ by conjugation. 
By the fundamental counting principle,
each orbit has prime-power size. Write 
\[
N=\underbrace{\mathcal{O}_1\cup\cdots\cup \mathcal{O}_k}_{\text{one-element orbits}}\cup\underbrace{\mathcal{O}_{k+1}\cup\cdots\cup\mathcal{O}_m}_{\text{orbits of size $>1$}},
\]
Since $N\cap Z(G)=\mathcal{O}_1\cup\cdots\cup\mathcal{O}_k$, 
the integers $k=|N\cap Z(G)|$ and $|N\setminus(N\cap Z(G))|$ are divisible by $p$. Thus 
\[
|N|\equiv|N\cap Z(G)|\bmod p.
\]
Since $1\in N\cap Z(G)$, then
$|N\cap Z(G)|>1$. In particular, $N\cap Z(G)\ne\{1\}$.
\end{proof}

The following corollary follows immediately: 

\begin{corollary}
Let $p$ be a prime number and 
$G$ a $p$-group. Then 
$Z(G)\ne\{1\}$.
\end{corollary}

In Exercises \ref{xca:size4} and \ref{xca:size9}
we proved that groups of order four and nine are always abelian. 

\begin{corollary}
    Let $p$ be a prime number. If $G$ is a group of order $p^2$, 
    then $G$ is abelian. 
\end{corollary}

\begin{proof}
By Lagrange's theorem, $|Z(G)|\in\{1,p,p^2\}$. Since $G$ 
is a $p$-group, $Z(G)\ne\{1\}$. If $|Z(G)|=p$, then $G/Z(G)$ 
is cyclic. By Exercise \ref{xca:G/Z(G)}, 
$G$ is abelian, a contradiction. 
Thus $|Z(G)|=p^2$ and hence $G=Z(G)$.
\end{proof}

\subsection{Cauchy's theorem}

\begin{theorem}[Cauchy]
\index{Cauchy's theorem}
\label{thm:Cauchy}
Let $G$ be a finite group, and $p$ be a prime number
that divides $|G|$. 
Then there exists $g\in G$ of order $p$.
\end{theorem}

\begin{proof}
Let $C=\Z/p$ and 
\[
X=\{(x_1,\dots,x_p)\in G\times\cdots\times G:x_1\cdots x_p=1\}.
\]
Then $C$ acts on $X$ by $k\cdot (x_1,\dots,x_p)=(x_{k+1},\dots,x_{k+p})$, where the indices are taken modulo $p$. To see that
this is an action, note that 
\[
x_{i_1}\cdots x_{i_p}=1
\implies (x_{i_1}^{-1}x_{i_1})x_{i_2}\cdots x_{i_p}=x_{i_1}^{-1}
\implies x_{i_2}\cdots x_{i_p}x_{i_1}=1.
\]
If $x_1,\dots,x_{p-1}$ are fixed, then 
$x_p=x_{p-1}^{-1}\cdots x_{1}^{-1}$. Thus $|X|=|G|^{p-1}$. Each $C$-orbit
has either one or $p$ elements, as $|C|=p$. Write
\[
X=\underbrace{\mathcal{O}_1\cup\cdots\cup \mathcal{O}_k}_{\text{one-element orbits}}\cup\underbrace{\mathcal{O}_{k+1}\cup\cdots\cup\mathcal{O}_m}_{\text{orbits of size $p$}}.
\]
Hence $0\equiv |G|^{p-1}=|X|\equiv k\bmod p$, that is $p$ divides $k$. Since $(1,1,\dots,1)\in X$, $k\geq 1$. Therefore $p\leq k$. In particular,
there exists $x\in G\setminus\{1\}$ such that $(x,x,\dots,x)\in X$. Hence $|x|=p$.
\end{proof}

\begin{exercise}
\label{xca:p_groups}
    Let $p$ be a prime number and $G$ be a finite group. 
    Then $G$ is a $p$-group if and only if 
    every element of $G$ has order a power of $p$. 
\end{exercise}

\begin{corollary}
    Let $p>2$ be a prime number 
    and $G$ be a group of order $2p$. 
    Then either $G\simeq\Z/2p$ or $G\simeq\D_p$.
\end{corollary}

\begin{proof}
    By Cauchy's theorem, there exist $r,s\in G$ such that
    $|r|=p$ and $|s|=2$. Let $H=\langle r\rangle$. Then
    $(G:H)=2$ and $H\unlhd G$. We can decompose $G$ as 
    $G=H\cup Hs$ (disjoint union), 
    as $s\not\in H$. In particular, 
\[
G=\{1,r,\dots,r^{p-1},s,rs,\dots,r^{p-1}s\}.
\]
Since $srs^{-1}\in H$, it follows that $srs^{-1}=r^k$ for some
$k\in\{0,1,\dots,p-1\}$. Since $s^2=1$,
\[
r=s^2rs^{-2}=s(srs^{-1})s^{-1}=sr^ks^{-1}=r^{k^2}.
\]
Thus $k^2\equiv 1\bmod p$ and either 
$k\equiv 1\bmod p$ or $k\equiv-1\bmod p$. 
If $k\equiv -1\bmod p$, then $srs^{-1}=r^{-1}$ and hence $G\simeq\D_p$.
If $k\equiv 1\bmod p$, then $rs=sr$ and hence, since $G$ is abelian, $G\simeq\Z/{2p}$.
\end{proof}

\begin{theorem}
    Let $p$ be a prime number. 
    A group of order $p^m$ has a normal subgroup of order
    $p^n$ for every $n\leq m$.
\end{theorem}

\begin{proof}
    We proceed by induction on $m$. The case where $m=1$ is trivial. So let $m\geq1$ and 
    assume the result holds for groups of order $p^m$. Let
    $G$ be a group of order $p^{m+1}$.
    We claim that if $n\leq m$, $G$ contains a normal subgroup of
    order $p^{n+1}$. Since $Z(G)\ne\{1\}$, there exists
    $g\in Z(G)\setminus\{1\}$ of order $p$. Let
$N=\langle g\rangle\unlhd G$. The quotient 
group $G/N$ has order $p^m$. By the inductive hypothesis, 
there exists a normal subgroup $Y$ of $G/N$ of order $p^n$. Let 
$\pi\colon G\to G/N$ be the canonical map. 
By the correspondence theorem, $G$ contains a normal subgroup $K$ of $G$ 
that contains $N$, that is $N\leq K\leq G$. In fact, 
$Y=\pi(K)$ and 
$(G:K)=(\pi(G):\pi(K))=p^{m-n}$. Hence $|K|=p^{n+1}$.
\end{proof}

% The following exercise uses group actions to 
% prove Fermat's theorem on the representation 
% of primes as sum of two squares. 
% Elsholtz, C. (2010). A Combinatorial Approach to Sums of Two Squares and Related Problems. Additive Number Theory, 115–140. doi:10.1007/978-0-387-68361-4_8 
% \begin{bonus}

%     \label{xca:twosquares}
%     Let 
%     \[
%     X_1=\begin{pmatrix}
%        0&1&0\\
%        1&0&0\\
%        0&0&-1
%     \end{pmatrix},\quad 
%     X_2=\begin{pmatrix}
%         0&1&0\\
%         1&0&0\\
%         0&0&1
%     \end{pmatrix},\quad 
%     X_3=\begin{pmatrix}
%         1&-1&1\\
%         0&1&0\\
%         0&2&-1
%     \end{pmatrix}.
%     \]
%     Let 
%     \begin{align*}
%         S &= \{(x,y,z)\in\Z^3:p=4xy+z^2,\quad x,y>0\},\\
%         T &= \{(x,y,z)\in S:z>0\},\\
%         U &= \{(x,y,z)\in S:x+z>y\}.
%     \end{align*}
%     Prove the following statements:
%     \begin{enumerate}
%         \item $X_1^2=X_2^2=X_3^2=I$, the identity matrix.
%         \item $X_1$ maps $S$ to $S$, $X_2$ maps $T$ to $T$ and $X_3$ maps $U$ to $U$.  
%         \item $|T|=|X_1(T)|$. 
%         \item %$S=T\cup X_1(T)$ (disjoint union). Thus 
%         $|S|=2|T|=2|U|$. 
%         \item The map $X_3$ acting on $U$ has exactly one orbit of length one and since all other orbits have length two. In particular, $|U|$ is odd. 
%         \item The action of $X_2$ on $T$ has an orbit of length one. Conclude that $p=4x^2+y^2$. 
%     \end{enumerate}
% \end{bonus}