\section{}

\subsection{Actions}

\begin{definition}
\index{Action}
    Let $G$ be a group and $X$ a set. 
    A (left) \textbf{action} of $G$ on $X$ is a map 
    $G\times X\to X$, $(g,x)\mapsto g\cdot x$, such that 
    \begin{enumerate}
        \item $1\cdot x=x$ for all $x\in X$, y
        \item $g\cdot (h\cdot x)=(gh)\cdot x$ for all $g,h\ n G$ and $x\in X$.
    \end{enumerate}
\end{definition}

If a group $G$ acts on a set $X$, we also say that
$X$ is a $G$-set.

\begin{example}
\index{Action!trivial}
Every group $G$ acts on $G$ trivially: $g\cdot h=h$ for all $g,h\in G$.
\end{example}

\begin{example}
\index{Action!left multiplication}
Every group $G$ acts on $G$ by left multiplication, that is
$g\cdot h=gh$ for all $g,h\in G$.
\end{example}

\begin{example}
\index{Action!conjugation}
Every group $G$ acts on $G$ by conjugation, that is
$g\cdot h=hg^{-1}$ for all $g,h\in G$. More generally, 
if $N$ is a normal subgroup of $G$, then $G$ acts on
$N$ by conjugation: $g\cdot x=gx
g^{-1}$ for all $g\in G$ y $x\in N$. 
\end{example}

\begin{example}
\index{Action!on left cosets}
Let $G$ be a group $H$ be a subgroup of $G$. Then $G$ 
acts on the set of left cosets $G/H$ by left multiplication, that 
is $g\cdot (xH)=(gx)H$ for all $g,x\in G$.
\end{example}

There is a bijective correspondence between 
left actions of a group $G$ on a set $X$ and
group homomorphisms 
$\rho\colon G\to\Sym_X$. The correspondence is given by
the formula 
\[
\rho(g)(x)=g\cdot x,\quad g\in G,x\in X.
\]
We will write $\rho_g=\rho(g)$.

As an example, if $G\times X\to X$, $(g,x)\mapsto x$, is 
an action of $G$ on $X$, then
each $\rho_g\colon X\to X$ is a bijective map with inverse 
$(\rho_g)^{-1}=\rho_{g^{-1}}$. Moreover, 
 $\rho$ is a group homomorphism, as 
\[
\rho(gh)(x)=(gh)\cdot x=g\cdot (h\cdot x)=\rho_g(h\cdot x)=\rho_g(\rho_h(x))
\]
for all $g,h\in G$ and $x\in X$.

\begin{example}
Let $G=\Sym_3$ and $H=\langle (123)\rangle=\{\id,(123),(132)\}$. Let $G$ act on the set $X=G/H=\{H,(12)H\}$ of left cosets of $H$ 
by left multiplication. Write 
$x_1=H$ and $x_2=(12)H$. Then
\begin{align*}
&(12)\cdot x_1=x_2,
&&(12)\cdot x_2=x_1,
&&(123)\cdot x_1=x_1,
&&(123)\cdot x_2=x_2.
\end{align*}
Since $G=\langle (12),(123)\rangle$, one has the group 
homomorphism 
$\rho\colon G\to\Sym_{X}\simeq\Sym_2$ given by
$(12)\mapsto (12)$, $(123)\mapsto\id$.
\end{example}

\begin{example}
As before, let $G=\Sym_3$ and $H=\langle (12)\rangle=\{\id,(12)\}$. Let $G$ act on the set $X=G/H=\{H,(123
)H,(132)H\}$ of left cosets of $H$ by left multiplication. Write 
$x_1=H$, $x_2=(123)H$ and $x_3=(132)H$. Then
\begin{align*}
(12)\cdot x_1=x_1,&& (12)\cdot x_2=x_3, && (12)\cdot x_3=x_2,\\
(123)\cdot x_1=x_2, && (123)\cdot x_2=x_3, &&(123)\cdot x_3=x_1.
\end{align*}
Since $G=\langle (12),(123)\rangle$, one has the group
homomorphism 
$\rho\colon G\to\Sym_{X}\simeq\Sym_3$ given by $(12)\mapsto (23)$, $(123)\mapsto (123)$.
\end{example}


\begin{example}
Let $G=Q_8=\{1,-1,i,-i,j,-j,k,-k\}$ and $N=\{1,-1,i,-i\}$. Since
$N$ is normal in $G$, $G$
acts by conjugation on $X=N$.
If $x_1=1$, $x_2=-1$, $x_3=i$ and $x_4=-i$, then
$i\cdot x=x$ for all $x\in N$. Moreover, 
\begin{align*}
j\cdot x_1=x_1, && j\cdot x_2=x_2, && j\cdot x_3=x_4, && j\cdot x_4=x_3.
\end{align*}
Since $G=\langle i,j\rangle$, a group homomorphism $\rho\colon G\to\Sym_X\simeq\Sym_4$ is determined by 
$\rho_i=\id$ and $\rho_j=(34)$.
\end{example}

The following example is important. 

\begin{example}
    The group $\Sym_n$ acts on $\R^n$ by
    \[
    \sigma\cdot (x_1,\dots,x_n)=(x_{\sigma^{-1}(1)},\dots,x_{\sigma^{-1}(n)}).
    \]
    It is very important to use $\sigma^{-1}$ and not 
    $\sigma$, as we need to permute the elements of the standard basis of $\R^3$.

    As a concrete example, let us see that the operation 
    \[
    \sigma\cdot (x_1,x_2,x_3)=(x_{\sigma(1)},x_{\sigma(2)},x_{\sigma(3)})
    \]
    is not an action of $\Sym_3$ on $\R^3$.
    If $\sigma=(12)$ and $\tau=(23)$, then $\sigma\tau=(123)$. Since 
    \begin{align*}
    &(123)\cdot (5,6,7)=(6,7,5),\\
    &(12)\cdot ((23)\cdot (5,6,7))=(1,2)\cdot (5,7,6)=(7,5,6),
    \end{align*}
    this does not define an action. If we compute 
    \begin{align*}
        \sigma\cdot (\tau\cdot (x_1,\dots,x_n))
        =\sigma\cdot (x_{\tau(1)},\dots,x_{\tau(n)})
    \end{align*}
    and for each $i\in\{1,\dots,n\}$ we set $y_i=x_{\tau(i)}$, then 
    \[
    \sigma\cdot (\tau\cdot (x_1,\dots,x_n))=\sigma\cdot (y_1,\dots,y_n)=(y_{\sigma(1)},\dots,y_{\sigma(n)})
    =(x_{\tau\sigma(1)},\dots,x_{\tau\sigma(n)}),
    \]
    even if $\sigma$ and $\tau$ do not commute.

    Now we show that by using inverses, we do have an action. 
    For $j\in\{1,\dots,n\}$, let $y_j=x_{\tau(j)}$,
    that is 
    \[
    (y_1,y_2,\dots,y_n)=\tau\cdot (x_1,x_2,\cdots,x_n)=(x_{\tau^{-1}(1)},x_{\tau^{-1}(2)},\dots,x_{\tau^{-1}(n)}).
    \]
    Then 
    \begin{align*}
        \sigma\cdot (\tau\cdot (x_1,x_2,\dots,x_n))&=\sigma\cdot (y_1,y_2,\dots,y_n)\\
        &=\left(y_{\sigma^{-1}(1)},y_{\sigma^{-1}(2)},\dots,y_{\sigma^{-1}(n)}\right)\\
        &=\left(x_{\tau^{-1}(\sigma^{-1}(1))},x_{\tau^{-1}(\sigma^{-1}(2))},\dots,x_{\tau^{-1}(\sigma
^{-1}(n))}\right)\\
        &=\left(x_{(\sigma\tau)^{-1}(1))},x_{(\sigma\tau)^{-1}(2))},\dots,x_{(\sigma\tau)^{-1}(n))}\right).
    \end{align*}
\end{example}

The following example is also important: 

\begin{example}
    The group $\Sym_n$ acts on the set of polynomials on
    $n$ variables $X_1,\dots,X_n$
    by permitting the variables. For example, for three variables, if 
    $\sigma=(123)$ y $f=X_2X_3-X_1+5X_2X_3^2X_1$, then 
    $\sigma\cdot f=X_2^2X_3-X_1+5X_2X_3^2X_1$.

    Restricting the action, we see that 
    $\Sym_n$ acts on the set 
    \[
    \{\lambda_1X_1+\cdots\lambda_nX_n:\lambda_1,\dots,\lambda_n\in\R\}.
    \]
    Then 
    \begin{align*}
    \sigma \cdot (\lambda_1X_1+\cdots+\lambda_nX_n) &= (\lambda_1X_{\sigma(1)}+\cdots+\lambda_nX_{\sigma(n)})
    =(\lambda_{\sigma(1)}X_1+\cdots+\lambda_{\sigma(n)}X_n).
    \end{align*}
\end{example}

It is relevant to compute the kernel of the action homomorphism. 

\begin{example}
Let $G$ be a group and $H$ be a subgroup of $G$. Then $G$ 
acts on $G/H$ by left multiplication, that is 
$g\cdot (xH)=(gx)H$ for all $g,x\in G$. Let $\rho\colon G\to\Sym_{G/H}$ be the group homomorphism induced by the action. 

We claim that $\ker\rho=\cap_{x\in G}xHx^{-1}$. 
We first prove $\supseteq$. If $g\in xHx^{-1}$ for all 
$x\in G$, then, for a fixed $x\in G$,
 \[
 \rho(g)(xH)=g\cdot (xH)=(gx)H=(xhx^{-1})xH=(xh)H=xH
 \]
because $g=xhx^{-1}$ for some $h\in H$. Thus $\rho(g)=\id$ 
and hence $g\in\ker\rho$. We now prove 
$\subseteq$. If $g\in\ker\rho$, then
 $\rho(g)=\id$. So for all $x\in G$,
 \begin{align*}
\rho(g)(xH)=xH
\Longleftrightarrow (gx)H=xH
\Longleftrightarrow x^{-1}gx\in H
\Longleftrightarrow g\in xHx^{-1}.
 \end{align*}
It is an exercise to show that
$\ker\rho$ is the largest normal subgroup of $G$ 
contained in~$H$.
\end{example}

With these results, we can provide a third 
solution to Exercise~\ref{xca:p_smallest} of 
page~\pageref{xca:p_smallest}.
We let  $G$ act on $G/H$ by left multiplication. 
The induced group homomorphism  $\rho\colon G\to\Sym_p$ has 
kernel 
\[
K=\ker\rho=\bigcap_{x\in G}xHx^{-1}\subseteq H.
\]
By the first isomorphism theorem, 
$G/K\simeq\rho(G)\lesssim\Sym_p$ (this means that 
$\rho(G)$ is isomorphic to a subgroup of $\Sym_p$). 
Thus $|G/K|$ divides $p!$.
Let $m=(H:K)$. By Lagrange's theorem,
\[
(G:K)=(G:H)(H:K)=pm
\]
and hence $pm$ divides $p!$. This implies that $m$ divides $(p-1)!$. If $q$ a prime number dividing 
 $m$, then $q\geq p$, by the minimality of $p$. Moreover, 
 every prime factor of $(p-1)!$ is 
 $<p$. Hence $m=1$ and therefore $H=K$.

\begin{exercise}
    Let a group $G$ act on a set $X$. 
    On $X$, we define the following relation: $x\sim y$ if and only if
    there exists $g\in G$ such that $g\cdot x=y$. Prove 
    that this is an equivalence relation on $X$. 
\end{exercise}

\begin{definition}
\index{Orbit}
Let $G$ be a group acting on a set $X$. If $x\in X$, the
orbit of $x$ is the set
\[
G\cdot x=\{g\cdot x:g\in G\}.
\]
\end{definition}

The orbits of the action of $G$ on $X$ are 
the equivalence classes of the equivalence 
relation induced by the action. In particular, 
every two orbits will be either disjoint or equal. Moreover, 
$X$ decomposes as a disjoint union of orbits. 

\begin{definition}
        \index{Stabilizer of a point}
        Let $G$ be a group that acts on $X$. If $x\in X$, the \textbf{stabilizer} of $x$ in $G$
        is the set   
        \[
        G_x=\{g\in G:g\cdot x=x\}.
        \]
\end{definition}

The reader must prove that the stabilizer is a subgroup. 

\begin{definition}
\index{Action!transitive}
We say that an action of a group $G$ on a set $X$
is \textbf{transitive} if for any $x,y\in X$ there exists $g\in G$ such that $g\cdot x=y$.
\end{definition}

\begin{example}
    Let $G$ be a group and $H$ a subgroup of $G$. Let $G$ act
    on $G/H$ by left multiplication. The action is transitive: if 
    $xH,yH\in G/H$, there exists $g\in G$ such that
    $(gx)H=yH$ (take for example $g=yx^{-1}$). 
\end{example}

\begin{example}
Por evaluación,
el grupo simétrico $\Sym_n$ actúa transitivamente en el conjunto $\{1,\dots,n\}$.
\end{example}

In the definition of a transitive action, there is no assumption
on the number of elements $g$ such that $g\cdot x=y$.

\begin{definition}
\index{Action!faithful}
We say that an action of a group $G$ on a set 
$X$ is \textbf{faithful} if 
$\{g\in G:g\cdot x=x\text{ for all $x\in X$}\}=\{1\}$.
\end{definition}

The definition is equivalent to the injectivity of 
the group homomorphism induced
by the action. 

\begin{theorem}[Fundamental counting principle]
\index{Theorem!fundamental counting principle}
\label{thm:fundamental}
Let $G$ be a finite group acting on a finite set $X$. If 
$x\in X$, then $|G\cdot x|=(G:G_x)$.
\end{theorem}

\begin{proof}
    Let $\varphi\colon G/G_x\to G\cdot x$, $gG_x\mapsto g\cdot x$. Then $\varphi$ is well-defined, as 
        \[
        gG_x=hG_x\implies h^{-1}g\in G_x
        \implies h^{-1}g\cdot x=x\implies g\cdot x=h\cdot x.
        \]
    Moreover, $\varphi$ is injective: 
        \[
        \varphi(gG_x)=\varphi(hG_x)\implies
        g\cdot x=h\cdot x\implies
        h^{-1}g\in G_x\implies gG_x=hG_x.
        \]
    Finally, $\varphi$ is surjective. Hence 
    $|G/G_x|=|G\cdot x|$.
\end{proof}

Theorem \ref{thm:fundamental} is also known 
as the orbit--stabilizer theorem. 

If $G$ is a group and $X$ and $Y$ are $G$-sets, 
we say that a map $\varphi\colon X\to Y$ is a 
\textbf{homomorphism} of $G$-sets if $\varphi(g\cdot x)=g\cdot \varphi(x)$ for all $g\in G$ y $x\in X$. The bijection 
$\varphi$ constructed in the proof of Theorem \ref{thm:fundamental}
is a homomorphism of $G$-sets, where
$G$ acts on $G/G_x$ by left multiplication: 
\[
\varphi(g\cdot hG_x)=\varphi((gh)G_x)=(gh)\cdot x=g\cdot (h\cdot x)=g\cdot\varphi(hG_x).
\]
Thus $G\cdot x\simeq G/G_x$ as $G$-sets.

\begin{example}
    If$G$ acts on $G$ by conjugation, that is $g\cdot x=gxg^{-1}$, the orbits of this action are called the \textbf{conjugacy classes} 
    of $G$. They are sets of the form
        \[
        G\cdot x=\{gxg^{-1}:g\in G\}.
        \]
    In particular, $G$ decomposes as a disjoint union of conjugacy classes. Moreover, 
    the stabilizers are the centralizers:
        \[
        G_x=\{g\in G:g\cdot x=x\}=\{g\in G:gxg^{-1}=x\}=C_G(x).
        \]
    In particular, $|G\cdot x|=(G:C_G(x))$.
\end{example}

\begin{example}
    Let $H$ be a subgroup of $G$ and $X$ the set of subsets of $G$. Let $G$ act on 
    $X$ by conjugation, that is $S\in X$. Then
        $g\cdot S=gSg^{-1}$. The orbit of $H$ 
        is 
        \[
        G\cdot H=\{g\cdot H:g\in G\}=\{gHg^{-1}:g\in G\},
        \]
        the set of conjugates of $H$. The stabilizer of $H$ in $G$ 
        is 
        \[
        G_H=\{g\in G:g\cdot H=H\}=\{g\in G:gHg^{-1}=H\}=N_G(H),
        \]
        the normalizer of $H$ in $G$. It follows that
        $H$ has exactly $(G:N_G(H))$ conjugates in $G$. In particular,
        if $G$ is finite, 
        the number of conjugates of $H$ divides $|G|$. 
\end{example}

As an application, we provide an alternative proof
of Theorem~\ref{thm:|HK|}. 

\begin{example}
\label{exa:for_HK}
Let $G$ be a group and $H$ and $K$ be subgroups of $G$. 
The group $L=H\times K$ acts on $X=HK$ by 
\[
(h,k)\cdot x=hxk^{-1},\quad x\in X,\,h\in H,\,k\in K.
\]
Note that $1\in HK$ and the action of $L$ on $X$ is transitive, as 
$(h,k^{-1})\cdot 1 = hk$. Since 
\[
L_1=\{(h,k)\in H\times K: (h,k)\cdot 1=1\}=\{(h,k)\in H\times K:h=k\},
\]
it follows that $|L_1|=|H\cap K|$ because there exists a bijection
between $L_1$ and 
$H\cap K$. By the fundamental counting principle, 
\[
|HK|=(L:L_1)=\frac{|H\times K|}{|H\cap K|}=\frac{|H||K|}{|H\cap K|}.
\]
\end{example}

As another application, we compute the
order of the group $\GL_n(p)$ para $n\geq1$ and 
a prime number $p$. 
The argument also works for
the group $\GL_n(q)$ in the case where
$q$ is a power of the prime number $p$.

\begin{example}
Let $K=\Z/p$.
We claim that 
\[
|\GL_n(p)|=(p^n-1)p^{n-1}|\GL_{n-1}(p)|,
\]
and hence 
\[
|\GL_n(p)|=(p^n-1)(p^n-p)\cdots (p^n-p^{n-1}).
\]
The formula is valid if $n\in\{1,2\}$. 
Assume that it holds for $n-1\geq1$.
The group $G=\GL_{n}(p)$ acts on
$K^{n}$ by left multiplication. There are two orbits, so 
\[
X=\{0\}\cup (K^{n}\setminus\{0\}),
\]
as if $v,w\in K^{n}\setminus\{0\}$, then there exists 
$g\in G$ such that $gv=w$.
By the fundamental counting principle,  
\[
p^{n+1}-1=|K^{n+1}\setminus\{0\}|=(G:G_{e_1}),
\]
where $e_1=(1,0,\dots,0)^T$. If $g=(g_{ij})\in G$ is such that
$ge_1=e_1$, then 
\[
g=
\begin{pmatrix}
1 & g_{12} & \cdots & g_{1n}\\
0 & g_{22} & \cdots & g_{2n}\\
\vdots & \vdots & \ddots &\vdots\\
0 & g_{n1} & \cdots & g_{nn}
\end{pmatrix}.
\]
Therefore $|G_{e_1}|=p^{n-1}|\GL_{n-1}(p)|$, as the submatrix 
$(g_{ij})_{2\leq i,j\leq n}$ is invertible and the 
$g_{1j}$'s can be chosen 
arbitrarily for all $j\in\{2,\dots,n\}$.
Hence 
\[
p^{n}-1=\frac{|G|}{|G_{e_1}|}=\frac{|\GL_n(p)|}{p^{n-1}|\GL_{n-1}(p)|},
\]
which implies the formula we wanted to prove.
\end{example}


