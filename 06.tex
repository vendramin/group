\section{Lecture -- Week 6}

\subsection{Permutable subgroups}

\index{Product of subgroups}
If $H$ and $K$ are subgroups of $G$, let 
\[
        HK=\{hk:h\in H,\,k\in K\}.
\]
Note that 
\[
H\cup K\subseteq HK\subseteq\langle H\cup K\rangle.
\]
When $HK$ is a subgroup of $G$? 
Note that $HK\leq G$ if and only if $\langle H\cup K\rangle=HK$.

\begin{proposition}
        Let $H$ and $K$ be subgroups of $G$. Then $HK$ is a subgroup of
        $G$ if and only if $HK=KH$.
\end{proposition}

\begin{proof}
    Assume that $HK=KH$. Since $1\in H\cap K$, $HK\ne\emptyset$. 
    If $h\in H$ and $k\in K$, then $(hk)^{-1}=k^{-1}h^{-1}\in KH=HK$. Moreover, 
    \[
    (HK)(HK)=H(KH)K=H(HK)K=(HH)(KK)=HK.
    \]
    Thus $HK$ is closed under multiplication. 

    Now assume that $HK$ is a subgroup of $G$. Since $H\subseteq HK$,
    $K\subseteq HK$ and $HK$ closed under multiplication,
        \[
        KH\subseteq (HK)(HK)\subseteq HK.
        \]
        Conversely, let $g\in HK$.
        Since $g^{-1}\in HK$, there exist $h\in H$ and $k\in K$ such that
        $g^{-1}=hk$.
        Thus $HK\subseteq KH$, as 
        $g=k^{-1}h^{-1}\in KH$.
\end{proof}

\begin{exercise}
\label{xca:HK_normal}
Let $H$ and $K$ be subgroups of $G$. Prove that 
if $H$ is normal in $G$, then $HK$ is a subgroup of $G$.
\end{exercise}

\begin{example}
Let $G=\Sym_4$. The subgroups $H=\langle (12)\rangle$ and $K=\langle (34)\rangle$ are such that 
\[
HK=KH=\{
\id,(12),(34),(12)(34)\}
\]
is a subgroup of 
$\Sym_4$. Note that not $H$ nor $K$
are normal in $G$.
\end{example}

\begin{exercise}
Let $G$ be a group and $S$ be a subgroup of $G$. 
If $T\leq N_G(S)$, then $TS$ is a group and $S\leq TS$.
\end{exercise}

\index{Permutable subgroups}
Two subgroups $H$ and $K$ of $G$ are said to be
\emph{permutable} if $HK=KH$.

\begin{theorem}
\label{thm:|HK|}
    Let $H$ and $K$ be finite subgroups of $G$. 
    Then 
    \[
        |HK|=\frac{|H||K|}{|H\cap K|}.
    \]
\end{theorem}

\begin{proof}
Let $L=H\cap K$.
We decompose $H$ as a disjoint union of left coclases of $L$, say 
$H=\cup_{i=1}^k x_iL$, where $k=(H:L)$. Note that $LK=K$, as $L\subseteq K$ and $K\subseteq 1K\subseteq LK$.
Then
\[
HK=\bigcup_{i=1}^k x_iLK=\bigcup_{i=1}^k x_iK,
\]
In particular, since the union is disjoint, 
\[
|HK|=\sum_{i=1}^k |x_iK|=k|K|=\frac{|H||K|}{|H\cap K|}.\qedhere
\]
\end{proof}

In the theorem, we do not assume that $HK$ is a subgroup of $G$. 

As an application, the theorem yields a different solution to Exercise~\ref{xca:p_smallest} of page~\pageref{xca:p_smallest}. 
If $\{gHg^{-1}:g\in G\}=\{H\}$, then $H$ is normal in $G$. Assume that
there exists $g\in G$ such that
$H\ne g^{-1}Hg=K$. Since $(H:H\cap K)$ divides $|H|$ 
and all prime divisors of $|G|$ 
are $\geq p$, it follows that $(H:H\cap K)\geq p$. Thus 
\[
|HK|=\frac{|H||K|}{|H\cap K|}\geq p|K|=|G|
\]
as $(G:H)=p$ and $|K|=|H|$. In particular, $HK=G$. Since $K=g^{-1}Hg$, 
$g=h(g^{-1}h_1g)$ for some $h,h_1\in H$. Thus 
\[
1=hg^{-1}h_1\implies h_1h=g\in H\implies H=K,
\]
a contradiction.

\begin{example}
Let $G=\Sym_3$, $H=\langle (12)\rangle$ and $K=\langle (23)\rangle$. Then 
\[
HK=\{\id,(12),(23),(123)\}
\]
is not a subgroup of $G$, as by Lagrange's theorem, 
$G$ cannot have subgroups of four elements. Another way 
to see that $HK$ is not a subgroup of $G$ follows from 
the fact that 
\[
KH=\{\id,(12),(23),(132)\}\ne HK.
\]
\end{example}

\begin{example}
Let $G=\Sym_3$, $H=\langle (12)\rangle$ and $K=\langle (123)\rangle$.
Since $K$ is normal in $G$, $HK$ is a subgroup of $G$. By Lagrange's theorem, $|HK|=6$ and hence $G=HK$.
Each $g\in G$ can be written uniquely as $g=hk$ for some $h\in H$ and $k\in K$ (one can prove this either considering all possible cases or 
using the fact that $H\cap K=\{\id\}$). It follows that the map 
\[
H\times K\to G,\quad
(h,k)\mapsto hk,
\]
is bijective. Note that this bijective map is not compatible 
with the operation of $G$, as 
\[
(h_1k_1)(h_2k_2)\ne (h_1h_2)(k_1k_2).
\]
\end{example}

\subsection{Homomorphisms}

\begin{definition}
        \index{Homomorphism!of groups}
        Let $G$ and $H$ be groups. 
        A map $f\colon G\to H$ is said to be a 
        \emph{group homomorphism} if 
        $f(xy)=f(x)f(y)$ for all $x,y\in G$.
\end{definition}

If $f\colon G\to H$ is a group homomorphism, 
then $f(1)=1$. Why? 

\index{Homomorphism!injective}
\index{Homomorphism!surjective}
\index{Homomorphism!bijective}
If a group homomorphism is injective, it will be called
a \emph{monomorphism}. If it is subjective, 
an \emph{epimorphism}. If it is bijective, 
an \emph{isomorphism}. Two groups $G$ and $H$ 
are said to be \emph{isomorphic} (notation: $G\simeq H$)
is there exists an isomorphism $G\to H$.

\begin{example}\
\begin{enumerate}
\item If $G$ is a group, the identity map $\id\colon G\to G$ is a group
homomorphism. 
\item If $G$ and $H$ are groups, the map $e\colon G\to H$, $e(g)=1_H$,  is a group homomorphism. 
\item For each $n\in\Z$, the map $\Z\to\Z$, $x\mapsto nx$, is a group homomorphism. 
\item If $G$ is an abelian group and $n\in\Z$, the map $G\to G$, $g\mapsto g^n$, is a group
homomorphism. 
\end{enumerate}
\end{example}

\begin{example}
\index{Conjugation}
Let $G$ be a group and $g\in G$. The map $\gamma_g\colon G\to G$, $\gamma_g(x)=gxg^{-1}$, is called 
\emph{conjugation} by $g$ and it is a group homomorphism. 
\end{example}

\begin{example}
The map $\exp\colon\R\to\R^\times$, $\exp(x)=e^x$, is a group homomorphism. 
\end{example}

\begin{example}
\index{Inclusion}
The inclusion map $\Z\hookrightarrow\Q$ is an injective group homomorphism. 
\end{example}

Generally, if $S$ is a subgroup of $G$, 
then the \emph{inclusion map}  $S\hookrightarrow G$ is a group
homomorphism. 

\begin{example}
The determinant $\det\colon\GL_2(\R)\to\R^\times$ is a group homomorphism. 
\end{example}

\begin{example}
\index{Restriction homomorphism}
    Let $f\colon G\to H$ be a group homomorphism and
    $S$ be a subgroup of $G$.
    The \emph{restriction} $f|_S\colon S\to H$ is a group homomorphism. 
\end{example}

\begin{example}
The map $f\colon\R\to\C^\times$, $f(x)=\cos x+i\sin x$, is a group
homomorphism, as 
$f(x+y)=f(x)f(y)$ for all $x,y\in\R$.
\end{example}

\begin{exercise}
Let $f\colon G\to H$ be a group homomorphism. Prove the following properties: 
\begin{enumerate}
    \item $f(1)=1$.
    \item $f(g^{-1})=f(g)^{-1}$ for all $g\in G$.
    \item $f(g^n)=f(g)^n$ for all $g\in G$ and $n\in\Z$.
\end{enumerate}
\end{exercise}

\begin{example}
Let $f\colon\R_{>0}\to\R$, $f(x)=\log(x)$. The formula 
\[\log(xy)=\log(x)+
\log(y)
\]
implies that $f$ is a group homomorphism. 
The previous exercise resembles 
the following properties of the logarithm function: 
\[
\log(1)=0,
\quad
\log\left(\frac{1}{x}\right)=-\log(x),
\quad
\log(x^n)=n\log(x).
\]
\end{example}

\begin{definition}
        \index{Kernel}
        Let $f\colon G\to H$ be a group homomorphism. The \emph{kernel} of $f$
        is the set 
        $\ker f=\{x\in G:f(x)=1\}$.
\end{definition}

The following property of the kernel is crucial: If $f\colon G\to H$
is a group homomorphism, then 
$f(x)=f(y)$ if and only if $x=yk$ for some $k\in\ker f$.

\begin{example}
Let $f\colon\mathcal{U}(\Z/21)\to\mathcal{U}(\Z/21)$, $f(x)=x^3$. 
Then $f$ is a group homomorphism and 
$\ker f=\{1,4,16\}$ and $f(\mathcal{U}(\Z/21))=\{1,8,13,20\}$.
\end{example}

\begin{exercise}
\label{xca:affine}
Let
\[
\Aff(\R)=\left\{\begin{pmatrix}a&b\\0&1\end{pmatrix}:a\in\R^\times,\,b\in\R\right\}\leq\GL_2(\R).
\]
Prove that the map
\[
f\colon \Aff(\R)\to\R^\times,\quad
\begin{pmatrix}
a&b\\
0&1
\end{pmatrix}
\mapsto a
\]
is a group homomorphism such that 
\[
\ker f=\left\{\begin{pmatrix}
1&b\\
0&1
\end{pmatrix}
\right\}.
\]
Show that $g\colon\Aff(\R)\to\R$, $\begin{pmatrix}a&b\\0&1\end{pmatrix}\mapsto b$,
is not a group homomorphism.
\end{exercise}

% (note that $f(x)=\det(x)$ for all $x\in\Aff(\R)$)


\begin{example}
Sea $f\colon\R\to\C^\times$, $f(x)=\cos x+i\sin x$. Then 
\[
\ker f=\{2\pi k:k\in\Z\}=2\pi\Z.
\]
\end{example}

\begin{definition}
\index{Image}
The \emph{image} of a group homomorphism $f\colon G\to H$
is the set 
\[
f(G)=\{f(x):x\in G\}.
\]
\end{definition}

\begin{proposition}
    Let $f\colon G\to H$ be a group homomorphism.
    The following properties hold: 
        \begin{enumerate}
                \item $\ker f$ is a normal subgroup of $G$.
                \item $f(G)$ is a subgroup of $H$.
        \end{enumerate}
\end{proposition}

\begin{proof}
    We only prove the first claim. We first need to show that 
    $\ker f$ is a subgroup of $G$. Note that
    $1\in \ker f$. If $x,y\in\ker f$, then
    $xy^{-1}\in\ker f$ (because $f$ is a group homomorphism, 
    $f(xy^{-1})=f(x)f(y)^{-1}=1$). Now we prove that $\ker f$ is normal in $G$. Let $x\in\ker f$ and $g\in G$. Then $gxg^{-1}\in\ker f$, as 
    \[
    f(gxg^{-1})=f(g)f(x)f(g)^{-1}=f(g)f(g)^{-1}=1.\qedhere
    \]
\end{proof}

The image of a group homomorphism
is not always a normal subgroup. 

\begin{example}
The inclusion map $\langle (12)\rangle\hookrightarrow\Sym_3$ is a group
homomorphism. Its image is not a normal subgroup of $\Sym_3$.
\end{example}

\begin{example}
Recall that $\mathcal{U}(\Z/21)=\{1,2,4,5,8,10,11,13,16,17,19,20\}$ is an abelian group. The map 
$f\colon\mathcal{U}(\Z/21)\to\mathcal{U}(\Z/21)$, $f(x)=x^3$, is a group
homomorphism. The image of 
$f$ equals $\{1,8,13,20\}$, a subgroup of $\mathcal{U}(\Z/21)$.
\end{example}

\begin{example}
The map $\sgn\colon\Sym_n\to\{-1,1\}$ is a surjective group 
homomorphism such that  
$\ker(\sgn)=\Alt_n$. In particular, $\Alt_n$ is a normal subgroup of $\Sym_n$.
\end{example}

\begin{example}
\index{Canonical homomorphism}
If $N$ is a normal subgroup of $G$, the map $\pi\colon G\to G/N$, $x\mapsto xN$, is a subjective group homomorphism such that 
$\ker\pi=N$. The map $\pi$ is called 
the \emph{canonical homomorphism} $G\to G/N$.
\end{example}

The previous example implies that 
every normal subgroup of a group $G$ is the kernel of a group
homomorphism with domain $G$. 


\begin{exercise}
Let $f\colon G\to H$ be a group homomorphism. 
Prove the following statements:
\begin{enumerate}
\item If $S\leq G$, then $f(S)\leq H$ and $f^{-1}(f(S))=S\ker f$.
\item If $T\leq H$, then $\ker f\leq f^{-1}(T)\leq G$ and $f(f^{-1}(T))=T\cap f(G)$.
\item $f$ is injective if and only if $\ker f=\{1\}$.
\item If $g\in G$ has finite order, then $|f(g)|$ divides $|g|$.
\end{enumerate}
\end{exercise}

If $f\colon G\to H$ is a group isomorphism, then
$f^{-1}\colon H\to G$ is an isomorphism. A group
homomorphism $f\colon G\to H$ is an isomorphism
if and only if there exists a group homomorphism 
$g\colon H\to G$ such that $g\circ f=\id_G$ and $f\circ g=\id_H$.

\begin{example}
$\Sym_2\simeq\Z/2\simeq G_2$.
\end{example}

\begin{example}
$\D_3\simeq\Sym_3$ and an isomorphism is given by $\D_3\to\Sym_3$,
\[
1\mapsto \id,\quad
r\mapsto (123),\quad r^2\mapsto(132),\quad s\mapsto(12),\quad rs\mapsto(13),\quad r^2s\mapsto(23).
\]
\end{example}

\begin{example}
$\Z/2\times\Z/3\simeq\Z/6$ and an isomorphism is given by 
\[
(0,0)\mapsto 0,\quad (1,0)\mapsto 3,\quad
(0,1)\mapsto 4,\quad (1,1)\mapsto 1,\quad (0,2)\mapsto 2,\quad (1,2)\mapsto 5.
\]
\end{example}

\begin{example}
The map $\log\colon\R_{>0}\to\R$ is a group homomorphism. Since
$\log$ is bijective, 
$\R_{>0}\simeq\R$.
\end{example}

If $f\colon G\to H$ is an isomorphism, then
$|g|=|f(g)|$ for all $g\in G$. 

\begin{example}
$\Z/2\times\Z/2\not\simeq\Z/4$, as $\Z/2\times\Z/2$ has no elements of order four. 
\end{example}

\begin{example}
$\Q/\Z\not\simeq\Q$. Both groups are abelian, but they are not isomorphic. To show this, note that every non-trivial element of $\Q$ has infinite order (if $kx=0$ for some $k\in\Z$ and $x\in\Q\setminus\{0\}$, then
$k=0$). However, every non-trivial element of $\Q/\Z$ has finite order. In fact, if $x=r/s\in\Q$, then, since 
\[
s(x+\Z)=sx+\Z=r+\Z=\Z
\]
we conclude that $|x+\Z|\leq s$.
\end{example}

\begin{example}
Note that $\mathcal{U}(\Z/5)\simeq\mathcal{U}(\Z/10)$, as
both groups are cyclic of order four. 
\end{example}

In page \pageref{convention:left-to-right} we mentioned that left-to-right and right-to-left 
conventions for multiplying permutations were equivalent. What does this really mean?

\begin{example}
\label{exa:op}
    For a group $G$, 
    one can define the group $G^{\operatorname{op}}$ as the group structure 
    on the set $G$ given by the \emph{opposite multiplication},  
    \[
    (x,y)\mapsto x\cdot_{\operatorname{op}}y=yx.
    \]
    A routine calculation shows that  
    $G^{\operatorname{op}}$ is a group. We claim that the map 
    \[
    f\colon G\to G^{\op},\quad x\mapsto x^{-1},
    \]
    is an isomorphism of groups. The map $f$ is invertible with inverse $f^{-1}=f$. 
    Let us verify that 
    $f$ is a group homomorphism:
    \[
    f(xy)=(xy)^{-1}=y^{-1}x^{-1}=x^{-1}\cdot_{\operatorname{op}}y^{-1}=f(x)\cdot_{\operatorname{op}}f(y).
    \]
\end{example}

\begin{exercise}
\label{xca:U(Z/10)}
   Prove that $\mathcal{U}(\Z/10)\not\simeq\mathcal{U}(\Z/12)$.
\end{exercise}

\begin{exercise}
Prove that $F=\{\sigma\in\Sym_n:\sigma(n)=n\}\leq\Sym_n$ and $F\simeq\Sym_{n-1}$.
\end{exercise}

If $G$ and $H$ are groups, let 
\[
\Hom(G,H)=\{f\colon G\to H:f\text{ is a group homomorphism}\}.
\]

\begin{example}
We claim that 
$\Hom(\Q,\Z)=\{0\}$. Let $f\in\Hom(\Q,\Z)$ and $p$ be a prime number. 
If $x\in\Q$, then, since 
\[
f(x)=f\left(p(x/p)\right)=pf(x/p),
\]
$p$ divides $f(x)$. It follows that $f(x)=0$ for all
$x\in\Q$, as $p$ is arbitrary. 
\end{example}

\begin{example}
If $G$ is a group, then $\Hom(\Z,G)=\{k\mapsto g^k:g\in G\}$.
For each $g\in G$, the map $\Z\to G$, $k\mapsto g^k$, is a group
homomorphism, as 
$k+l\mapsto g^{k+l}=g^kg^l$. Let $f\in\Hom(\Z,G)$ and  $g=f(1)$. If $k>0$,
\[
f(k)=f(\underbrace{1+\cdots+1}_{k-\text{times}})=f(1)^k=g^k.
\]
If $k<0$, then
\[
f(k)=f(\underbrace{(-1)+\cdots+(-1)}_{|k|-\text{times}})=f(-1)^{-k}=(g^{-1})^{-k}=g^k.
\]
\end{example}

\begin{example}
We claim that $\Hom(\Z/8,\Z/10)$ has exactly two elements (one of these elements being 
the zero homomorphism, that is $\Z/8\to\Z/10$, $x\mapsto 0$).
Let $f\in\Hom(\Z/8,\Z/10)$ be a non-zero homomorphism. 
%be a non-trivial homomorphism. If 
If 
$n=|f(1)|$, then 
$n$ divides $8$, that is $n\in\{1,2,4,8\}$. Since $f(1)\in\Z/10$ and
$f$ is non-trivial,
$n=2$. Thus $f(1)=5$ and $f$ is univocally determined. 
This means that 
$f(k)=5k$ for $k\in\{0,1,\dots,7\}$.
\end{example}

\begin{exercise}
Compute $\Hom(\Z/n,G)$ for any group $G$.
%Demuestre que $\Hom(\Z/n,G)=\{k\mapsto g^k:g\in G,\,|g|\text{ divide a }n\}$.
\end{exercise}

\begin{exercise}
Let $A$, $B$ and $C$ be groups. If $f\in\Hom(A,B)$ and $g\in\Hom(B,C)$,
then $g\circ f\in\Hom(A,C)$.
\end{exercise}

\begin{exercise}
\label{xca:size4}
Prove that $\Z/2\times\Z/2$ and $\Z/4$ are 
the only groups of order four (up to isomorphism). In particular, 
groups of order four are abelian. 
\end{exercise}

The following example is harder than
Exercise \ref{xca:size4}.

\begin{example}
If $G$ is a group of order six, then
either $G\simeq\Sym_3$ or $G$ is cyclic of order six. 
Since $|G|$ is even, there exists an element of $G$ 
that has order two (see Exercise~\ref{xca:orden2}). 
If every element of $G\setminus\{1\}$ has order two, then
$xy=yx$ for all $x,y\in G$. Hence 
\[
\langle x,y\rangle=\{1,x,y,xy\}\leq G,
\]
a contradiction to Lagrange's theorem. Thus there exist 
$x\in G$ of order two 
and $y\in G\setminus\{1\}$ of order $>2$. 
By Lagrange's theorem, $|y|\in\{3,6\}$, 
as the order of $y$ divides $|G|$. If 
$|y|=6$, 
then $G\simeq\Z/6$. It follows that there exists 
$z\in G$ of order three. Thus 
\[
\langle x,z\rangle=\{1,x,z,z^2,xz,xz^2\}=G.
\]
Now we have the group $\langle x,z\rangle$. To ``recognize'' this group,
we need to understand the product $zx$. We know that
$zx\in\{xz,xz^2\}$. If $xz=zx$, then $|xz|=6$ (because
$(xz)^k\ne1$ for all $k\in\{1,\dots,5\}$ and
$(xz)^6=1$). Thus 
$G=\langle xz\rangle\simeq\Z/6$. If, otherwise, 
$zx=xz^2$, then 
\[
G=\langle x,z:x^2=z^3=1,\,xzx^{-1}=z^2\rangle\simeq\D_3.
\]
\end{example}

How many (isomorphism classes of) groups are there? We are now 
ready to collect our results and give a classification 
of isomorphism classes of groups of order $\leq7$; see
Table \ref{tab:grupos<8}. For the classification of groups
of order eight, we need more tools. 

\begin{table}[ht]
    \caption{Groups of order $\leq7$ (up to isomorphism).}

    \begin{tabular}{|c|c|c|}
    \hline
    Order & Number & Group(s)\\
    \hline
        1 & 1 & $\{1\}$ \\
        2 & 1 & $\Z/2$ \\
        3 & 1 & $\Z/3$ \\
        4 & 2 & $\Z/4$ \\
        && $\Z/2\times\Z/2$ \\
        5 & 1 & $\Z/5$ \\
        6 & 2 & $\Z/6$\\
        &&$\Sym_3$ \\
        7 & 1 & $\Z/7$ \\
    \hline
    \end{tabular}
    \label{tab:grupos<8}
\end{table}

\begin{exercise}
\label{xca:size9}
Prove that the groups of order nine are 
$\Z/9$ and $\Z/3\times\Z/3$ (up to isomorphism). In particular, groups of order
nine are abelian. 
\end{exercise}

