\chapter{}

\topic{Permutable subgroups}

\index{Product of subgroups}
If $H$ and $K$ are subgroups of $G$, let 
\[
        HK=\{hk:h\in H,\,k\in K\}.
\]
Note that 
\[
H\cup K\subseteq HK\subseteq\langle H\cup K\rangle.
\]
When $HK$ is a subgroup of $G$? 
Note that $HK\leq G$ if and only if $\langle H\cup K\rangle=HK$.

\begin{proposition}
        Let $H$ and $K$ be subgroups of $G$. Then $HK$ is a subgroup of
        $G$ if and only if $HK=KH$.
\end{proposition}

\begin{proof}
    Assume that $HK=KH$. Since $1\in H\cap K$, $HK\ne\emptyset$. 
    If $h\in H$ and $k\in K$, then $(hk)^{-1}=k^{-1}h^{-1}\in KH=HK$. Moreover, 
    \[
    (HK)(HK)=H(KH)K=H(HK)K=(HH)(KK)=HK.
    \]
    Thus $HK$ is closed under multiplication. 

    Now assume that $HK$ is a subgroup of $G$. Since $H\subseteq HK$,
    $K\subseteq HK$ and $HK$ closed under multiplication,
        $KH\subseteq (HK)(HK)\subseteq HK$. Conversely, let $g\in HK$.
        Since $g^{-1}\in HK$, there exist $h\in H$ and $k\in K$ such that
        $g^{-1}=hk$.
        Thus $HK\subseteq KH$, as 
        $g=k^{-1}h^{-1}\in KH$.
\end{proof}

\begin{exercise}
\label{xca:HK_normal}
Let $H$ and $K$ be subgroups of $G$. Prove that 
if $H$ is normal in $G$, then $HK$ is a subgroup of $G$.
\end{exercise}

\begin{example}
Let $G=\Sym_4$. The subgroups $H=\langle (12)\rangle$ and $K=\langle (34)\rangle$ satisfy that $HK=KH=\{
\id,(12),(34),(12)(34)\}$ is a subgroup of 
$\Sym_4$. Note that not $H$ nor $K$
are normal in $G$.
\end{example}

\begin{exercise}
Demuestre que si $H$ y $K$ son subgrupos normales de $G$, entonces $HK$ es también normal en $G$.
\end{exercise}

\begin{exercise}
Let $G$ be a group and $S$ be a subgroup of $G$. 
If $T\leq N_G(S)$, then $TS$ is a group and $S\leq TS$.
\end{exercise}

\index{Permutable subgroups}
Two subgroups $H$ and $K$ of $G$ are said to be
\textbf{permutable} if $HK=KH$.

\begin{theorem}
\label{thm:|HK|}
    Let $H$ and $K$ be finite subgroups of $G$. 
    Then 
    \[
        |HK|=\frac{|H||K|}{|H\cap K|}.
    \]
\end{theorem}

\begin{proof}
Let $L=H\cap K$.
We decompose $H$ as a disjoint union of left coclases of $L$, say 
$H=\cup_{i=1}^k x_iL$, where $k=(H:L)$. Note that $LK=K$, as $L\subseteq K$. Moreover, $K\subseteq 1K\subseteq LK$.
Then
\[
HK=\bigcup_{i=1}^k x_iLK=\bigcup_{i=1}^k x_iK,
\]
In particular, since the union is disjoint, 
\[
|HK|=\sum_{i=1}^k |x_iK|=k|K|=\frac{|H||K|}{|H\cap K|}.\qedhere
\]
\end{proof}

In the theorem, we do not assume that $HK$ is a subgroup of $G$. 

As an application, the theorem yields a different solution to Exercise~\ref{xca:p_smallest} of page~\pageref{xca:p_smallest}. 
If $\{gHg^{-1}:g\in G\}=\{H\}$, then $H$ is normal in $G$. Assume that
there exists $g\in G$ such that
$H\ne g^{-1}Hg=K$. Since $(H:H\cap K)$ divides $|H|$ 
and all prime divisors of $|G|$ 
are $\geq p$, it follows that $(H:H\cap K)\geq p$. Thus 
\[
|HK|=\frac{|H||K|}{|H\cap K|}\geq p|K|=|G|
\]
as $(G:H)=p$ y $|K|=|H|$. In particular, $HK=G$. Since $K=g^{-1}Hg$, 
$g=h(g^{-1}h_1g)$ for some $h,h_1\in H$. Thus 
\[
1=hg^{-1}h_1\implies h_1h=g\in H\implies H=K,
\]
a contradiction.

\begin{example}
Let $G=\Sym_3$, $H=\langle (12)\rangle$ and $K=\langle (23)\rangle$. Then 
\[
HK=\{\id,(12),(23),(123)\}
\]
is not a subgroup of $G$, as by Lagrange's theorem, 
$G$ cannot have subgroups of four elements. Another way 
to see that $HK$ is not a subgroup of $G$ follows from 
the fact that 
$KH=\{\id,(12),(23),(132)\}\ne HK$.
\end{example}

\begin{example}
Let $G=\Sym_3$, $H=\langle (12)\rangle$ and $K=\langle (123)\rangle$.
Since $K$ is normal in $G$, $HK$ is a subgroup of $G$. By Lagrange's theorem, $|HK|=6$ and hence $G=HK$.
Each $g\in G$ can be written uniquely as $g=hk$ for some $h\in H$ and $k\in K$ (one can prove this either considering all possible cases or 
using the fact that $H\cap K=\{\id\}$). It follows that the map 
\[
H\times K\to G,\quad
(h,k)\mapsto hk,
\]
is bijective. Note that this bijective map is not compatible 
with the operation of $G$, as 
$(h_1k_1)(h_2k_2)\ne (h_1h_2)(k_1k_2)$. 
\end{example}

\topic{Homomorphisms}

