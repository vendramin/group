\RequirePackage{amsmath} 

\documentclass[graybox,envcountsect]{svmono}
%\usepackage{marginnote}
\usepackage[T1]{fontenc}
\usepackage[utf8]{inputenc}
\usepackage{amsmath}
%\usepackage[notref,notcite]{showkeys}
\usepackage{anyfontsize}
\usepackage{fancyhdr}
\usepackage{float}
\usepackage{amssymb}
\usepackage{amstext}
\usepackage{mathtools}
\usepackage{xcolor} 
\usepackage{centernot}
\usepackage{listings}
\usepackage{multicol}
\usepackage{mathptmx}
%\let\openbox\relax
\usepackage{newtxtext,newtxmath}
%\usepackage{txfonts}
\usepackage{datetime}
\usepackage{stmaryrd}
\usepackage{tikz-cd}
    \usetikzlibrary{arrows,positioning,shapes}


\usepackage{helvet}
\usepackage{courier}
\usepackage{type1cm}         
\usepackage{makeidx}        
\usepackage{graphicx}        
\usepackage{multicol}        
\usepackage{hyperref} 
\usepackage{colortbl}
\usepackage{chngcntr}





% Table of contents for lectures and topics
\makeatletter
\newcommand\listtopicsname{List of topics}
\newcommand\listoftopics{
    \section*{\listtopicsname}\@starttoc{top}}
\makeatother

\makeatletter
\newcommand\listlecturesname{Contents}
\newcommand\listoflectures{
    \chapter*{\listlecturesname}\@starttoc{lec}}
\makeatother

\newcommand{\lecture}[1]{
    \chapter{#1}
    \addcontentsline{lec}{chapter}{Lecture \thechapter}
}

\newcommand{\topic}[1]{
    \section{#1}
    \addcontentsline{top}{chapter}{\S\thesection\quad #1}
}


%\usepackage[small,bf]{caption}

\usepackage{tikz}
%\usetikzlibrary{braids}
	
\usepackage[bottom]{footmisc}

% for QED
\let\proof\relax\let\endproof\relax
\let\openbox\relax
\usepackage{amsthm}

\overfullrule=1mm

%%% for Spanish
% \def\abstractname{Resumen}%
% \def\ackname{Agradecimientos}%
% \def\andname{y}%
% \def\bibname{Referencias}%
% \def\lastandname{, y}%
% \def\appendixname{Apéndice}%
\def\chaptername{Lecture}%
% \def\claimname{Afirmación}%
% \def\conjecturename{Conjetura}%
% \def\contentsname{Contenidos}%
% \def\corollaryname{Corolario}%
% \def\definitionname{Definici\'on}%
% \def\emailname{e-mail}%
% \def\examplename{Ejemplo}%
\def\examplesname{Examples}%
% \def\exercisename{Ejercicio}%
\def\figurename{Figure}%
% \def\forewordname{Foreword}%
% \def\keywordname{{\bf Palabras clave:}}%
% \def\indexname{Índice}%
% \def\lemmaname{Lema}%
% \def\listfigurename{Figuras}%
% \def\listtablename{Tablas}%
% \def\notename{Nota}%
% \def\partname{Parte}%
% \def\prefacename{Prefacio}%
\def\problemname{Open problem}%
% \def\proofname{Demostración}%
% \def\propertyname{Propiedad}%
% \def\propositionname{Proposici\'on}%
% \def\questionname{Pregunta}%
% \def\refname{Referencias}%
% \def\remarkname{Observación}%
% \def\seename{see}%
% \def\solutionname{Solución}%
% \def\tablename{Tabla}%
% \def\theoremname{Teorema}
\def\notationname{Notation}
\def\stepsname{Algorithm}
% \def\conventionname{Convención}

% change numbers 
\let\remark\relax
\let\theorem\relax
\let\lemma\relax
\let\definition\relax
\let\proposition\relax
\let\corollary\relax
\let\exercise\relax
\let\example\relax
\let\conjecture\relax

% Numerar con sección y no resetear al cambiar de capítulo
\counterwithout{section}{chapter}
\counterwithout{theorem}{chapter}
\spnewtheorem{theorem}{\theoremname}[section]{\bfseries}{\itshape}

\renewcommand\thetheorem{\thesection.\arabic{theorem}}
\spnewtheorem{lemma}[theorem]{\lemmaname}{\bfseries}{\itshape}
\spnewtheorem{definition}[theorem]{\definitionname}{\bfseries}{\upshape}
\spnewtheorem{proposition}[theorem]{\propositionname}{\bfseries}{\itshape}
\spnewtheorem{corollary}[theorem]{\corollaryname}{\bfseries}{\itshape}
\spnewtheorem{exercise}[theorem]{\exercisename}{\bfseries}{\upshape}
\spnewtheorem{example}[theorem]{\examplename}{\bfseries}{\upshape}
\spnewtheorem{examples}[theorem]{\examplesname}{\bfseries}{\upshape}
\spnewtheorem{remark}[theorem]{\remarkname}{}{\upshape}
\spnewtheorem{conjecture}[theorem]{\conjecturename}{\bfseries}{\upshape}
\spnewtheorem{notation}[theorem]{\notationname}{\bfseries}{\upshape}
\spnewtheorem{steps}[theorem]{\stepsname}{\bfseries}{\upshape}
\spnewtheorem{convention}[theorem]{\conventionname}{\bfseries}{\upshape}

% Numerar con sección y no resetear al cambiar de capítulo
\counterwithout{section}{chapter}

% No sections in TOC
\setcounter{secnumdepth}{1}
\setcounter{tocdepth}{0}

 \usepackage{titlesec}
 \titleformat{\section}
   {\secsize\secstyle}{\S\thesection.}{1em}{}

% para enumerar
\renewcommand{\labelenumi}{\textbf{\arabic{enumi})}}

\makeindex             

\renewcommand{\I}{\operatorname{I}}
\newcommand{\II}{\operatorname{II}}

\newcommand{\GAP}{\textsf{GAP}}
\newcommand{\FK}{\mathcal{E}}
\newcommand{\ad}[1]{\operatorname{ad}\,#1}

%\newcommand{\N}{\mathbb{N}}
\newcommand{\Q}{\mathbb{Q}}
\newcommand{\Z}{\mathbb{Z}}
\newcommand{\F}{\mathbb{F}}
\newcommand{\R}{\mathbb{R}}
\newcommand{\C}{\mathbb{C}}
\renewcommand{\H}{\mathbb{H}}
\newcommand{\A}{\mathbb{A}}
\newcommand{\K}{\mathbb{K}}
\newcommand{\T}{\mathbb{T}}
\renewcommand{\D}{\mathbb{D}}
\newcommand{\B}{\mathbb{B}}
\newcommand{\Fun}{\operatorname{Fun}}
\newcommand{\mpl}{\operatorname{mpl}}
\newcommand{\cL}{\mathcal{L}}
\newcommand{\cE}{\mathcal{E}}
\newcommand{\cH}{\mathcal{H}}

\newcommand{\GF}{\mathsf{GF}}
\newcommand{\MAX}{\operatorname{MAX}}
\newcommand{\MIN}{\operatorname{MIN}}
\newcommand{\cf}{\operatorname{cf}}
\newcommand{\cl}{\operatorname{cl}}
\newcommand{\cd}{\operatorname{cd}}
\newcommand{\bL}{\mathbf{L}}
\newcommand{\bP}{\mathbf{P}}

\newcommand{\Nil}{\operatorname{Nil}}
\newcommand{\rad}{\operatorname{rad}}
\newcommand{\rank}{\operatorname{rank}}

\newcommand{\Aff}{\mathrm{Aff}}
\newcommand{\Ann}{\operatorname{Ann}}
\newcommand{\Der}{\operatorname{Der}}
\newcommand{\Core}{\operatorname{Core}}
\newcommand{\Soc}{\operatorname{Soc}}
\newcommand{\Fix}{\operatorname{Fix}}
\newcommand{\Rad}{\mathrm{rad}}
\newcommand{\Inn}{\mathrm{Inn}}
\newcommand{\dist}{\mathrm{dist}}
\newcommand{\Out}{\mathrm{Out}}
\newcommand{\Ext}{\mathrm{Ext}}
\newcommand{\Img}{\mathrm{im}}
\newcommand{\Hol}{\operatorname{Hol}}
\newcommand{\Hom}{\operatorname{Hom}}
\newcommand{\Alg}{\operatorname{Alg}}
\newcommand{\Bil}{\operatorname{Bil}}
\newcommand{\op}{\operatorname{op}}
\newcommand{\gr}{\operatorname{gr}}
\newcommand{\Syl}{\mathrm{Syl}}
\newcommand{\id}{\operatorname{id}}
\newcommand{\Aut}{\operatorname{Aut}}
\newcommand{\End}{\operatorname{End}}
\newcommand{\Irr}{\operatorname{Irr}}
\newcommand{\Alt}{\mathbb{A}}
\newcommand{\Sym}{\mathbb{S}}
\newcommand{\lcm}{\mathrm{lcm}}
\newcommand{\diag}{\operatorname{diag}}
\newcommand{\spec}{\operatorname{Spec}}
\newcommand{\supp}{\operatorname{supp}}
\newcommand{\trace}{\operatorname{trace}}
\newcommand{\sgn}{\operatorname{sign}}
\newcommand{\ch}{\operatorname{char}}

\newcommand{\inner}{\operatorname{inn}}
\newcommand{\ext}{\operatorname{ext}}
\newcommand{\im}{\operatorname{im}}
\newcommand{\Ret}{\operatorname{Ret}}

\newcommand{\GL}{\mathbf{GL}}
\newcommand{\SL}{\mathbf{SL}}
\newcommand{\PSL}{\mathbf{PSL}}
\newcommand{\PGL}{\mathbf{PGL}}

\newcommand{\legendre}[2]{\left(\frac{#1}{#2}\right)}

\newcommand{\Char}{\operatorname{Char}}

% multiset
\def\multiset#1#2{\ensuremath{\left(\kern-.3em\left(\genfrac{}{}{0pt}{}{#1}{#2}\right)\kern-.3em\right)}}

% column vector
\newcount\colveccount
\newcommand*\colvec[1]{
\global\colveccount#1
\begin{pmatrix}
	\colvecnext
	}
	\def\colvecnext#1{
	#1
	\global\advance\colveccount-1
	\ifnum\colveccount>0
	\\
	\expandafter\colvecnext
	\else
\end{pmatrix}
\fi
}


% numero como secciones
\renewcommand{\thesection}{\arabic{section}}
%\renewcommand{\thesubsection}{\Alph{section}}

% To remove Springer from the title page
\usepackage{etoolbox}
\makeatletter
\patchcmd{\@maketitle}{{\Large Springer\par}}{}{}{}
\def\ps@bchap{%
  \let\@oddhead\@empty\let\@evenhead\@empty
  \def\@oddfoot{\reset@font\small\hfil\thepage\hfil}%
  \let\@evenfoot\@oddfoot
}

% Heading 
\def\ps@headings{%
  \let\@mkboth\markboth
  \def\@oddfoot{\reset@font\small\hfil\thepage\hfil}%
  \let\@evenfoot\@oddfoot
  \def\@evenhead{\runheadsize\runheadstyle\hfil\leftmark}%
  \def\@oddhead{\runheadsize\runheadstyle\rightmark\hfil}%
  \def\chaptermark##1{%
    \markboth{%
      {\if@chapnum Lecture \thechapter\thechapterend\fi ##1}%
    }{%
      {\if@chapnum Lecture \thechapter\thechapterend\fi ##1}}%
    }%
    \def\sectionmark##1{\markright{{\ifnum\c@secnumdepth>\z@
     \S\thesection\seccounterend\hskip\betweenumberspace\fi ##1}}}
}
\makeatother
\pagestyle{headings}

\begin{document}
 
\lstset{language=GAP,
  showstringspaces=false,
  xleftmargin=0.6cm,
  xrightmargin=0.6cm,
  basicstyle=\small\ttfamily,
  frame=single,
  framerule=0pt,
}

\author{Leandro Vendramin}
\title{Group theory}
\subtitle{Notes}
\maketitle

\frontmatter

%\include{dedic}
\preface

The notes correspond to the bachelor 
course \emph{Group Theory} of the 
Vrije Universiteit Brussel, 
Faculty of Sciences, 
Department of Mathematics and Data Sciences. The course
is divided into twelve two-hours lectures. 

The material is somewhat standard. Basic texts on abstract algebra
are for example \cite{MR1129886}, \cite{MR2286236} and \cite{MR600654}. 
Lang's book \cite{MR783636} is also a standard reference, but 
maybe a little bit more advanced. 

We also mention a set of great expository papers by 
Keith Conrad available at 
\url{https://kconrad.math.uconn.edu/blurbs/}. 
The notes are extremely well-written and are useful at  
every stage of a mathematical career. 

% Bibtex information:
% {\footnotesize\begin{verbatim}
% @misc{rings,
%     author={Vendramin, L.},
%     title={Rings and modules},
%     year={2022},
%     note={Available at www.github.com/vendramin/rings},
%     pages={106}
% }
% \end{verbatim}}

%  Thanks go to Wouter Appelmans, Arne van Antwerpen, Ilaria Colazzo, Luca Descheemaeker, {\L}ukas Kubat, Lucas Simons
% and Geoffrey Jassens. 

This version 
was compiled on \today~at~\currenttime. 
Please send comments and corrections to me at \url{Leandro.Vendramin@vub.be}. 


\bigskip
\begin{flushright}
Leandro Vendramin\\Brussels, Belgium\par
\end{flushright}


\tableofcontents 
\listoftopics

\mainmatter

\chapter{}

\topic{Groups}

Before defining groups, we recall that a binary operation on a set $X$
is simply a map 
\[
X\times X\to X,
\quad (x,y)\mapsto xy.
\]
Note that we have used 
juxtaposition to denote this generic binary operation. For example,
$(x,y)\mapsto x-y$ is a binary operation in $\Z$ but not, for example, 
in $\Z_{\geq 0}$.

\begin{definition}
\index{Group}
A \textbf{group} is a non-empty set $G$ with a binary operation
$G\times G\to G$, $(x,y)\mapsto xy$, such that
the following properties hold:
\begin{enumerate}
    \item (Associativity) $(xy)z=x(yz)$ for all $x,y,z\in R$.
    \item (Existence of a neutral element) There exists $e\in G$ such that $xe=ex=x$ for all $x\in G$.
    \item (Existence of inverses) For every $x\in G$ there exists $y\in G$ such that $xy=yx=e$.
\end{enumerate}
\end{definition}

The associativity condition implies
that all ordered products that we can form with 
the elements, say, $x_1,x_2,\dots,x_n$ will be equal. For example,
\[
(x_1x_2)((x_3x_4)x_5)=x_1(x_2(x_3(x_4x_5)))
\]
and hence we can write, without ambiguity (and without using brackets), 
$x_1x_2\cdots x_5$. This fact can be proved by induction; see for example
Lang's book. We will provide an alternative proof as an application
of Cayley's theorem. 

\begin{proposition}
    In a group $G$, every element $x\in G$ admits a unique inverse.
\end{proposition}

\begin{proof}
    Let $y,z\in G$ be inverses of $x\in G$. Then 
    $z=z(xy)=(zx)y=ey=y$.
\end{proof}

\begin{exercise}
    Prove that the neutral element of a group is unique. 
\end{exercise}

In general, when the binary operation is written multiplicatively, one
writes the identity element $e$ of a group as $1_G$ or simply as $1$. The inverse of $x$ will be 
denoted by $x^{-1}$. 

\begin{example}
    Let $n\geq1$. The set $\GL_n(\R)$ of $n\times n$ invertible real 
matrices forms a
    group with the usual matrix multiplication.  
\end{example}

It is a good idea to keep in mind the \emph{group of invertible matrices}. 
With this, the 
the following properties look familiar:
\begin{enumerate}
    \item $(x^{-1})^{-1}=x$ for all $x$.
    \item $(xy)^{-1}=y^{-1}x^{-1}$ for all $x,y$. 
\end{enumerate}

\begin{exercise}
    Prove that in a group, the equation $ax=b$ has 
    a unique solution $x=a^{-1}b$. Similarly, the equation
    $x=ba^{-1}$ is the unique solution of the equation
    $xa=b$. 
\end{exercise}

\begin{definition}
    \index{Group!abelian}
    A group $G$ is \textbf{abelian} if $xy=yx$ for all $x,y\in G$. 
\end{definition}

Most of the time, for abelian groups we will use 
the \emph{additive notation}. This means that the binary operation
of the group will be denoted by $(x,y)\mapsto x+y$, the neutral
element by $0$ and 
the inverse of an element $x$ will be $-x$. 

\begin{definition}
    \index{Group!order}
    The \textbf{order} $|G|$ of a group $G$ is the 
    cardinality of $G$. A group $G$ is said to be
    \textbf{finite} if $|G|$ is finite and \textbf{infinite}
    otherwise. 
\end{definition}

\begin{example}
\label{exa:abelian_groups}
    Let us see some 
    abelian groups: 
    \begin{enumerate}
        \item $\Z$, $\Q$, $\R$ and $\C$ with the usual addition. 
        \item Let $n\geq2$. The set $\Z/n$ of integers modulo $n$ with the usual addition modulo $n$.
        \item $\Q\setminus\{0\}$, $\R\setminus\{0\}$ and $\C\setminus\{0\}$ 
        with the usual multiplication.
        \item Let $p$ be a prime number. The set $(\Z/p)^{\times}=(\Z/p)\setminus\{0\}$ of invertible integers modulo $p$ 
            with the usual multiplication modulo $p$. 
    \end{enumerate}
\end{example}

The groups of the first two items will be written in additive notation. 

The group $\Z/n$ of integers modulo $n$ is a finite group of order $n$. 
The group $(\Z/p)^{\times}$ of units modulo $p$ is a finite
group of order $p-1$. The other groups of Example \ref{exa:abelian_groups} are infinite groups. 

\begin{exercise}
\label{xca:LR}
Let $G$ be a group and $g\in G$. Prove that 
the maps $L_g\colon G\to G$, $x\mapsto gx$, and 
$R_g\colon G\to G$, $x\mapsto xg$, are bijective. 
\end{exercise}






\section{Lecture -- Week 2}

\subsection{Subgroups}

\begin{definition}
\index{Subgroup}
    Let $G$ be a group. 
        A subset $S$ of $G$ is said to be a \emph{subgrup} of $G$ 
        if the following properties are satisfied:
        \begin{enumerate}
                \item $1\in S$,
                \item $x\in S\implies x^{-1}\in S$, and 
                \item $x,y\in S\implies xy\in S$.
        \end{enumerate}
        Notation: $S$ is a subgroup of $G$ if and only if $S\leq G$.
\end{definition}

The first condition of the definition can be replaced by
the following condition: $S\ne\emptyset$. Why? 

\begin{example}
    If $G$ is a group, then 
    $\{1\}$ and $G$ are always subgroups of $G$. 
\end{example}

The subgroup $\{1\}$ is known as the \emph{trivial subgroup} of $G$. 
A subgroup $S$ of $G$ is said to be \emph{proper} if $S\ne G$. 

\begin{example}
Write $2\Z=\{2m:m\in\Z\}$ to denote the set of even integers. Then 
\[
2\Z\leq\Z\leq\Q\leq\R\leq\C
\]
is a chain of subgroups. 
\end{example}

\begin{example}
$S^1=\{z\in\C:|z|=1\}\leq\C^\times=\C\setminus\{0\}$.
\end{example}

Note that one cannot prove that $S^1\leq\C$. Why?

\begin{example}
Let $n\geq1$. Then $G_n=\{z\in\C:z^
n=1\}$ is a subgroup of $\C^\times$. 
Note that
\[
G_n=\{1,\exp(2\pi i/n),\exp(4i\pi/n),\dots,\exp(2(n-1)i\pi/n)\}.
\]
and 
\[
G_n\leq\bigcup_{k\geq1}G_k\leq S^1\leq\C^\times.
\]
\end{example}

Why $\cup_{k\geq1}G_k$ is a group? 

\begin{exercise}
\label{xca:center}
        \index{Center!of a group}
        Let $G$ be a group. Prove that the \emph{center} 
        \[
                Z(G)=\{g\in G:gh=hg\text{ for all $h\in G$}\}
        \]
        of $G$ is a subgroup of $G$. 
\end{exercise}

\index{Centralizer!of an element}
One can prove that, if $G$ is a group and $g\in G$, then 
the \emph{centralizer} 
\[
C_G(g)=\{h\in G:gh=hg\}
\]
is a subgroup of $G$. Moreover, 
$Z(G)=\cap_{g\in G}C_G(g)$. 

\begin{exercise}
        \index{Conjugate of a subgroup}
        \label{xca:conjugate}
        Let $S$ be a subgroup of $G$ and $g\in G$. Prove that
        the \emph{conjugate} $gSg^{-1}$
        of $S$ by $g$ is a subgroup of $G$. 
        Notation: $\prescript{g}{}S=gSg^{-1}$.
\end{exercise}


\begin{exercise}
\label{xca:center_S3}
\index{Center!of $\Sym_3$}
        Prove that $Z(\Sym_3)=\{\id\}$ and compute $C_{\Sym_3}((12))$.
\end{exercise}

The following exercise is useful: 

\begin{exercise}
\label{xca:subgroup}
        Let $G$ be a group and $S$ be a subset of $G$. 
        Prove that $S$ is a subgroup of $G$ if and only if 
        $S\ne\emptyset$ and for all $x,y\in S$ one has
        $xy^{-1}\in S$.
\end{exercise}

Use the previous exercise and
the fact that the determinant is a multiplicative function
to solve the following problem:

\begin{exercise}
\label{xca:SL_subgroup}
Prove that $\SL_n(\R)=\{a\in\GL_n(\R):\det(a)=1\}\leq\GL_n(\R)$. 
\end{exercise}

\begin{exercise}
\label{xca:intersection}
    Prove that the intersection of subgroups is again a subgroup.
\end{exercise}

The previous exercise is easy but crucial. We need it 
to construct subgroups generated 
by a given subset of elements. 

\begin{definition}
        \index{Subgroup!generated by a subset}
        Let $G$ be a group and $X$ a non-empty 
        subset of $G$. The \emph{subgroup
        generated} by $X$ is the smallest subgroup of $G$ that contains
        $X$, that is 
        \[
            \langle X\rangle=\bigcap\{S:S\leq G,X\subseteq S\}.
        \]
\end{definition}

Why this is the smallest subgroup that contains $X$? 
Let $H\leq G$ be such that 
$X\subseteq H$. Since $H$ is one of the subgroups appearing
in the intersection, 
\[
        \langle X\rangle=\bigcap\{S:S\leq G,X\subseteq S\}\subseteq H.        
\]

We will use the following notation:
If $X=\{g_1,\dots,g_k\}$, then 
\[
\langle
X\rangle=\langle\{g_1,\dots,g_k\}\rangle=\langle g_1,\dots,g_k\rangle.
\]

\begin{exercise}
\label{xca:generated}
Prove that 
\[
    \langle X\rangle=\{x_1^{n_1}\cdots x_k^{n_k}:k\geq0,\,x_1,\dots,x_k\in X,\,-1\leq n_1,\dots,n_k\leq 1\}.
\]
\end{exercise}

The previous exercise shows that
the subgroup generated by, say, the elements 
$x_1,\dots,x_n$ is nothing but the 
group formed by (some) words on the letters 
$x_1,\dots,x_n$ and their inverses 
$x_1^{-1},\dots,x_n^{-1}$. 

\begin{example}
Let $n\geq3$. Let 
\[
r=\begin{pmatrix}
\cos(2\pi/n) & -\sin(2\pi/n)\\
\sin(2\pi/n) & \cos(2\pi/n)
\end{pmatrix},
\quad
s=\begin{pmatrix}
        1 & 0\\
        0 & -1
\end{pmatrix}.
\]
The \emph{dihedral group} $\D_n$ is the subgroup of
$\GL_2(\C)$ generated by $r$ and $s$,
that is $\D_n=\langle r,s\rangle$. A direct calculation shows that 
\[
r^n=s^2=\begin{pmatrix}
        1&0\\
        0&1
\end{pmatrix},
\quad
srs=r^{-1}.
\]

An element of $\D_n$ is a finite word of the form 
\[
r^{i_1}s^{j_1}r^{i_2}s^{j_2}\cdots
\]
for some 
 $i_1,i_2,\dots\in\{0,1,\dots,n-1\}$ and 
$j_1,j_2,\dots\in\{0,1\}$. Since $rs=sr^{-1}$, we conclude that
every element of $\D_n$ can be written as $r^is^j$ 
for some $i\in\{0,\dots,n-1\}$ and $j\in\{0,1\}$. Since these elements are all different, we conclude that 
$|\D_n|=2n$.
\end{example}

To understand better the previous example, 
we discuss a particular case. 
% two concrete particular cases. 
% If $n=3$, 
% \[
% r=\begin{pmatrix}
% -1/2 & -\sqrt{3}/2\\
% \sqrt{3}/2 & -1/2
% \end{pmatrix},
% \quad
% s=\begin{pmatrix}
%         1 & 0\\
%         0 & -1
% \end{pmatrix}.
% \]
% and we obtain (another representation of) the group of symmetries of a regular 
% triangle. 
If $n=4$, 
then the elements of $\D_4$ are  
\begin{align*}
r&=\begin{pmatrix}
0 & -1\\
1 & 0
\end{pmatrix},
&
r^2&=\begin{pmatrix}
    -1 & 0\\
    0 & -1
\end{pmatrix},
&r^3&=\begin{pmatrix}
0 & 1\\
-1 & 0
\end{pmatrix},
&
I&=\begin{pmatrix}
        1 & 0\\
        0 & 1
\end{pmatrix},\\
s&=\begin{pmatrix}
1 & 0\\
0 & -1
\end{pmatrix},
&
rs&=\begin{pmatrix}
    0 & 1\\
    1 & 0
\end{pmatrix},
&r^2s&=\begin{pmatrix}
-1 & 0\\
0 & 1
\end{pmatrix},
&
r^3s&=\begin{pmatrix}
    0 & -1\\
    -1 & 0
\end{pmatrix}.
\end{align*}
This is (a representation of) the group of symmetries of the square. 

\begin{exercise}
The group $\D_3$ is generated by 
\[
r=\begin{pmatrix}
-1/2 & -\sqrt{3}/2\\
\sqrt{3}/2 & -1/2
\end{pmatrix},
\quad
s=\begin{pmatrix}
        1 & 0\\
        0 & -1
\end{pmatrix}.
\]
Note that this is (another representation of) the group of symmetries of a regular 
triangle. 
\end{exercise}

\begin{exercise}
\label{xca:union}
        The union of subgroups is not, in general, 
        a subgroup. Can you give an example? 
\end{exercise}

\begin{example}
    Let $Q_8$ be the set of matrices 
    \begin{align*}
        I&=\begin{pmatrix}
            1&0\\
            0&1
        \end{pmatrix},
        &
        -I&=\begin{pmatrix}
            -1&0\\
            0&-1
        \end{pmatrix},
        &
        i&=\begin{pmatrix}
            \sqrt{-1}&0\\
            0&-\sqrt{-1}
        \end{pmatrix},
        &
        -i&=\begin{pmatrix}
            -\sqrt{-1}&0\\
            0&\sqrt{-1}
        \end{pmatrix},\\
        j&=\begin{pmatrix}
            0&1\\
            -1&0
        \end{pmatrix},
        &
        -j&=\begin{pmatrix}
            0&-1\\
            1&0
        \end{pmatrix},
        &
        k&=\begin{pmatrix}
            0&\sqrt{-1}\\
            \sqrt{-1}&0
        \end{pmatrix},
        &
        -k &=\begin{pmatrix}
            0&-\sqrt{-1}\\
            -\sqrt{-1}&0
        \end{pmatrix}.
    \end{align*}
    Then $Q_8$ is a subgroup of $\GL_2(\C)$. It is known as the \emph{quaternion group} of order eight. Sometimes, it is convenient just to write
    \[
    Q_8=\{1,-1,i,-i,j,-j,k,-k\}, 
    \]
    but one needs to remember 
    that 1 is playing the role of identity matrix, that is the neutral element of $Q_8$, 
    -1 commutes with every element of $Q_8$ 
    and that $i^2=j^2=k^2=-1$ and $ijk=-1$. This is enough to compute the multiplication table
    of $Q_8$. For example, to show that $ji=-k$, we proceed as follows: 
    \[
    ijk=-1\implies -jk=-i\implies jk=i\implies -k=ji.
    \]
\end{example}

\begin{exercise}
\label{xca:Q8_table}
    Compute the multiplication table of $Q_8$. 
\end{exercise}

\subsection{Subgroups of $\Z$}

What can we say about the subgroups of $\Z$? 

\begin{theorem}
        \label{thm:Z}
        If $S$ is a subgroup of $\Z$, then
                $S=m\Z=\{mx:x\in \Z\}$
                for some $m\geq0$.
        \end{theorem}
        
        \begin{proof}
                If $S=\{0\}$, take $m=0$. 
                Assume now that $S\ne\{0\}$. Let 
                \[
                m=\min\{s\in S:s>0\}.
                \]
                Why does this $m$ exist?  
                Since $S\ne\{0\}$,  
                it contains 
                an element $n\in S\setminus\{0\}$. 
                There are then two possible cases: 
                $n>0$ or $-n>0$. Since 
                $S$ is a subgroup of $\Z$, $-n\in S$.
        
                We claim that $S=n\Z$.
                If $x\in S$, then $x=my+r$ for $y,r\in\Z$ with 
                $0\leq r<m$. Suppose that $r\ne 0$. Since $x,m\in S$, it follows that 
                $r\in S$,
                a contradiction to the minimality of $n$. Thus $r=0$ 
                and hence $x=my\in
                m\Z$. Conversely, since $n\in S$, it follows that
                 $nk\in S$ for all $k\in\Z$. In fact, if $k=0$, then 
                 $nk=0\in S$. If $k>0$, 
                 then 
                \[
                \underbrace{n+\cdots+n}_{k-\text{times}}\in S.
                \]
                Finally, if $k<0$, 
                then 
                \[
                nk=\underbrace{-n+(-n)+\cdots+(-n)}_{|k|-\text{times}}\in S.\qedhere
                \]
        \end{proof}

The previous theorem has nice applications. 
If $a,b\in\Z$, we say that $a$ \emph{divides} $b$ (or $b$ is divisible by $a$)
if $b=ac$ for some $c\in\Z$. Notation: 
\[
a\mid b\Longleftrightarrow b=ac\text{ for some $c\in\Z$.}
\]
If $a,b\in\Z$ are such that $ab\ne0$, then 
\[
S=a\Z+b\Z=\{m\in\Z:m=ar+bs\text{ for $r,s\in\Z$}\}
\]
is a subgroup of $\Z$ (this is an exercise). 
By Theorem \ref{thm:Z}, $S=d\Z$ for some $d>0$. 
This positive integer $d$
is the \emph{greatest common divisor} of $a$ and $b$, 
that is $d=\gcd(a,b)$. 

\begin{exercise}
Let $a,b\in\Z$ be such that $ab\ne0$ and $d=\gcd(a,b)$. 
Prove the following statements:
\begin{enumerate}
\item $d$ divides $a$ and $b$.
\item If $e\in\Z$ divides $a$ and $b$, then $e$ divides $d$.
\item There are $r,s\in\Z$ such that $d=ar+bs$.
\end{enumerate}
\end{exercise}

Two integers $a$ and $b$ are said to be \emph{coprime} if 
and only if the only positive integer dividing 
$a$ and $b$ is one, that is  
\begin{align*}
a\text{ and }b\text{ are coprime}&\Longleftrightarrow \gcd(a,b)=1\\
&\Longleftrightarrow \Z=a\Z+b\Z\\
&\Longleftrightarrow \text{there exist $r,s\in\Z$ such that $ar+bs=1$.}
\end{align*}

\begin{exercise}
        Let $p$ be a prime and 
        $a,b\in\Z$. Prove that if $p\mid ab$, 
        then $p\mid a$ or $p\mid b$.
\end{exercise}

If $S$ and $T$ are subgroups of $\Z$, then $S\cap T$
is a subgroup of $\Z$.
Let $a,b\in\Z$ be such that $ab\ne 0$. Since $a\Z\cap b\Z$ 
is a non-zero subgroup of $\Z$ (note that it contains $ab\ne 0$), 
we can write  $a\Z\cap b\Z=m\Z$
for some $m\geq1$. The integer $m$
is the \emph{least common multiple} of $a$ and $b$ 
and will be written as $m=\lcm(a,b)$.

\begin{exercise}
Let $a,b\in\Z\setminus\{0\}$ and $m=\lcm(a,b)$. 
Prove the following statements:
\begin{enumerate}
        \item $m$ is divisible by both $a$ and $b$.
        \item If $n$ is divisible by both $a$ and $b$, then 
        $n$ is divisible by $m$.
\end{enumerate}
\end{exercise}

\begin{exercise}
Let $a,b\in\Z_{\geq1}$. Prove that 
$ab=\gcd(a,b)\lcm(a,b)$.
%if $d=\gcd(a,b)$ and $m=\lcm(a,b)$, then 
%$ab=dm$.
\end{exercise}

\subsection{Commutators}

For a group $G$ and $x,y\in G$, the \emph{commutator} 
of $x$ and $y$ is defined as 
\[
[x,y]=xyx^{-1}y^{-1}
\]
Note that $[x,y]yx=xy$ and 
$[x,y]^{-1}=[y,x]$ for all $x,y\in G$. 

\begin{definition}
        \index{Derived subgroup}
        \index{Commutator subgroup}
        The \emph{commutator subgroup}
        $[G,G]$ of $G$ 
        is the subgroup generated by 
        the commutators of $G$, that is 
        $[G,G]=\langle[x,y]: x,y\in G\rangle$. 
\end{definition}

For a group $G$, 
\[
G\text{ is abelian }
\Longleftrightarrow [x,y]=1\text{ for all $x,y\in G$}
\Longleftrightarrow [G,G]=\{1\}.
\]

The commutator subgroup of a group $G$ is also called the \emph{derived subgroup} of~$G$. 
       
\begin{example}
        In $\Z$, the commutator of $x,y\in \Z$ 
        is the integer
        \[
        [x,y]=x+y-x-y=0.
        \]
        This example uses additive notation! 
        Thus $[\Z,\Z]=\{0\}$. 
\end{example}
        
\begin{exercise}
        Prove that $[\Sym_3,\Sym_3]=\{\id,(123),(132)\}$.
\end{exercise}
        
It is natural to ask why we consider the group generated by commutators rather than just the set of commutators. The answer is simple: the set of commutators does not necessarily form a subgroup! But how can we find an example of a group where this happens? This is not so straightforward, but with the help of computers, we can make it accessible to everyone.

\begin{example}
This example is taken from Carmichael's book~\cite{MR0075938}.
        Let $G$ be the subgroup of $\Sym_{16}$ 
        generated by the permutations 
        \begin{align*}
&a = (13)(24),&&
b = (57)(68),\\
&c = (9\,11)(10\,12),&&
d = (13\,15)(14\,16),\\
&e = (13)(57)(9\,11),&&
f = (12)(34)(13\,15),\\
&g = (56)(78)(13\,14)(15\,16),&&
h = (9\,10)(11\,12).
\end{align*}
Then $[G,G]$ has order 16, but the set of commutators of 
$G$ has 15 elements:
% \begin{lstlisting}
% gap> a := (1,3)(2,4);;
% gap> b := (5,7)(6,8);;
% gap> c := (9,11)(10,12);;
% gap> d := (13,15)(14,16);;
% gap> e := (1,3)(5,7)(9,11);;
% gap> f := (1,2)(3,4)(13,15);;
% gap> g := (5,6)(7,8)(13,14)(15,16);;
% gap> h := (9,10)(11,12);;
% gap> G := Group([a,b,c,d,e,f,g,h]);;
% gap> D := DerivedSubgroup(G);;
% gap> Size(D);
% 16
% gap> Size(Set(Cartesian(G, G), x->Comm(x[1], x[2])));
% 15
% gap> c*d in Difference(D, Set(Cartesian(G, G), Comm));
% true
% \end{lstlisting}
% % G<a,b,c,d,e,f,g,h> := PermutationGroup<16 | (1,3)(2,4), (5,7)(6,8),(9,11)(10,12), (13,15)(14,16), (1,3)(5,7)(9,11), (1,2)(3,4)(13,15), (5,6)(7,8)(13,14)(15,16), (9,10)(11,12)>;
% % D := DerivedSubgroup(G);
% % c*d in { x : x in D } diff { u*v*Inverse(u)*Inverse(v) : u in G, v in G};
\begin{lstlisting}
> S16 := Sym(16);
> a := S16 ! (1,3)(2,4);
> b := S16 ! (5,7)(6,8);
> c := S16 ! (9,11)(10,12);
> d := S16 ! (13,15)(14,16);
> e := S16 ! (1,3)(5,7)(9,11);
> f := S16 ! (1,2)(3,4)(13,15);
> g := S16 ! (5,6)(7,8)(13,14)(15,16);
> h := S16 ! (9,10)(11,12);
> G := PermutationGroup< 16 | a,b,c,d,e,f,g,h >;
> D := DerivedSubgroup(G);
> #D;
16
> #{ x*y*Inverse(x)*Inverse(y) : x in G, y in G };
15    
> c*d in { x : x in D } \ 
> diff { u*v*Inverse(u)*Inverse(v) : u in G, v in G };
true
\end{lstlisting}
\end{example}

The following example goes back to Guralnick~\cite{MR673806}. 
It was found by hand when computers were not as popular
in group theory as now. 

\begin{example}
The group 
\[
G=\langle (135)(246)(7\,11\,9)(8\,12\,10),(394\,10)(58)(67)(11\,12)\rangle.
\]
has order 96. The set of commutators 
is different from the commutator subgroup:
% \begin{lstlisting}
% gap> x := (1,3,5)(2,4,6)(7,11,9)(8,12,10);;
% gap> y := (3,9,4,10)(5,8)(6,7)(11,12);;
% gap> G := Group([x,y]);;
% gap> Order(G);
% 96
% gap> D := DerivedSubgroup(G);;
% gap> Order(D);
% 32
% gap> Size(Set(Cartesian(G, G), x->Comm(x[1], x[2])));
% 29
% \end{lstlisting}
\begin{lstlisting}
> a := S12 ! (1,3,5)(2,4,6)(7,11,9)(8,12,10);
> b := S12 ! (3,9,4,10)(5,8)(6,7)(11,12);
> G := PermutationGroup< 12 | a,b >;
> Order(G);
96
> D := DerivedSubgroup(G);
> Order(D);
32
> #{ a*b*Inverse(a)*Inverse(b) : a in G, b in G };
29
\end{lstlisting}
% G<x,y> := PermutationGroup< 12 | (1,3,5)(2,4,6)(7,11,9)(8,12,10), (3 ,9 ,4 ,10)(5 ,8)(6 ,7)(11 ,12) >;
% Order(G);
% D := DerivedSubgroup (G);
% Order(D);
% # { a*b*Inverse(a)*Inverse(b) : a in G, b in G };
Moreover, 
$G$ is the smallest group with the property that  
the set of commutators is not a subgroup. 
\end{example}

\subsection{Cyclic groups}

\begin{definition}
        \index{Cyclic group}
        A group $G$ is said to be \emph{cyclic} if 
        $G=\langle g\rangle$ for some 
        $g\in G$.
\end{definition}

If $G$ is a cyclic group generated by $g$, then 
$G=\langle g\rangle=\{g^k:k\in\Z\}$. Every cyclic group is, 
in particular, an abelian group. 

\begin{example}\
\begin{enumerate}
        \item $\Z=\langle 1\rangle=\langle -1\rangle$.
        \item $\Z/n=\langle 1\rangle$.
        \item $G_n=\langle \exp(2i\pi/n)\rangle$.
\end{enumerate}
\end{example}

\begin{example}
        $\mathcal{U}(\Z/8)\ne\langle 3\rangle$. In fact, $\langle 3\rangle=\{1,3\}\subsetneq\{1,3,5,7\}=\mathcal{U}(\Z/8)$.
\end{example}

\begin{exercise}
\label{xca:subgroups_cyclic}
        Prove that subgroups of a cyclic group are cyclic.
\end{exercise}

\begin{definition}
        \index{Order!of an element}
        Let $G$ be a group and $g\in G$. The \emph{order} of $g$
        is the order of the subgroup generated by $g$. Notation:
        $|g|=|\langle g\rangle|$.
\end{definition}


\begin{theorem}
        Let $G$ be a group and $g\in G$ and $n\geq1$. 
        The following statements are equivalent:
        \begin{enumerate}
                \item $|g|=n$.
                \item $n=\min\{k\in\Z_{\geq1}:g^k=1\}$.
                \item For every $k\in\Z$, $g^k=1\Longleftrightarrow n\mid k$.
                \item $\langle g\rangle=\{1,g,\dots,g^{n-1}\}$ and 
                the elements $1,g,\dots,g^{n-1}$ are all different.
        \end{enumerate}
\end{theorem}


\begin{proof}
        We first prove that $(1)\implies(2)$.
        If $g=1$, then $n=1$. Assume that $g\ne1$. Since $\langle g\rangle=\{g^k:k\in\Z\}$,
        there exist integers $i$ and $j$ with $i>j$ such that $g^i=g^j$, that is $g^{i-j}=1$. In particular,
        the set $\{k\in\Z_{\geq1}:g^k=1\}$ is non-empty and hence has a minimal element, say 
        \[
        d=\min\{k\in\Z_{\geq1}:g^k=1\}.
        \]
        Thus $\langle g\rangle\subseteq\{1,g,\dots,g^{d-1}\}\subseteq\langle g\rangle$. 
        If $g^k\in\langle g\rangle$, then $k=dq+r$ for some $q,r\in\Z$ with $0\leq r<d$. Since $g^d=1$,
        \[
        g^k=g^{dq+r}=(g^d)^qg^r=g^r\in\{1=g^0,g,g^2,\dots,g^{d-1}\}
        \]
        Moreover, $\{1,g,\dots,g^{d-1}\}\subseteq \langle g\rangle$ and 
        $\{1,g,\dots,g^{d-1}\}$ has $d$ elements.

        We now prove that $(2)\implies(3)$. Assume that $g^k=1$. If 
        we write $k=nt+r$ with $0\leq r<n$, then $g^k=g^{nt+r}=g^r$. The minimality of $n$ 
        implies that $r=0$. Hence $n$ divides $k$.
        Conversely, if $k=nt$ for some $t\in\Z$, then $g^k=(g^n)^t=1$.

        Let us prove that $(3)\implies(4)$. Clearly, 
        $\{1,g,\dots,g^{n-1}\}\subseteq\langle g\rangle$. To prove the other 
        inclusion, we write $k=nt+r$ with $0\leq r\leq n-1$. Then 
        \[
                g^k=g^{nt+r}=(g^n)^tg^r=g^r, 
        \]
        as, by assumption, $g^n=1$. To see that the elements 
        $1,g,\dots,g^{n-1}$ are all different, it is enough to show that if $g^k=g^l$ with $0\leq
        k<l\leq n-1$, then, since $g^{l-k}=1$ and $0<l-k\leq n-1$, it follows that 
        $n\leq l-k$ (because by assumption $n$ divides $l-k$, a contradiction).

        Finally, the implication $(4)\implies(1)$ is trivial.
\end{proof}


\section{29/02/2024}

\begin{corollary}
        If $G$ is a group and $g\in G$ has order $n$, 
        then 
        \[
        |g^m|=\frac{n}{\gcd(n,m)}.
        \]
        \end{corollary}
        
\begin{proof}
        Let $k$ be such that $(g^m)^k=1=g^{mk}$. This means that $n$ divides $km$, as $g$ 
        has order $n$. This is also equivalent to the fact that 
        $n/d$ divides $mk/d$, where $d=\gcd(n,m)$. Therefore, since $n/d$ and $m/d$ 
        are coprime, $(g^m)^k=1$ is equivalent to  
        $n/d$ divides $k$, which implies that $g^m$ has order $n/
        d$.
\end{proof}
        
\begin{exercise}
        Let $G$ be a group and $g\in G$. Prove that the following statements are equivalent:
        \begin{enumerate}
        \item $g$ has infinite order. 
        \item The set $\{k\in\Z_{\geq1}:g^k=1\}$ is empty.
        \item If $g^k=1$, then $k=0$.
        \item If $k\ne l$, then $g^k\ne g^l$.
        \end{enumerate}
\end{exercise}

\begin{exercise}
\index{Torsion in abelian groups}
        Let $G$ be an abelian group. Prove that 
        $T(G)=\{g\in G:|g|<\infty\}$ is a subgroup of $G$. Compute $T(\C^\times)$.
\end{exercise}
                
\begin{exercise}
        Let $G=\langle g\rangle$ be a cyclic group. 
        \begin{enumerate}
                \item If $G$ is infinite, only $g$ and $g^{-1}$ generate $G$.
                \item If $G$ is finite of order $n$, then 
                        $G=\langle g^k\rangle$ if and only if $k$ and $n$ are coprime.
        \end{enumerate}
\end{exercise}
                
The following exercise is a particular 
case of Cauchy's theorem; see Theorem~\ref{thm:Cauchy}. 

\begin{exercise}
        \label{xca:orden2}
        Prove that every group of odd order contains
        an element of order two. 
\end{exercise}
                
Let us see some concrete examples: 

\begin{example}
        In $\Sym_3$ we have the following order pattern:
        \[
        |\id|=1,\quad
        |(12)|=|(13)|=|(23)|=2,\quad
        |(123)|=|(132)|=3.
        \]
\end{example}
                        
\begin{example}
        In $\Z$, every non-zero element has 
        infinite order. 
\end{example}
                        
 \begin{example}
        In $\Z\times\Z/6$ there are elements of 
        (in)finite order. For example, $(1,0)$ 
        has infinite order and 
        $(0,1)$ has order six. 
 \end{example}
                        
\begin{example}
        The matrix $\begin{pmatrix}1&1\\0&1\end{pmatrix}\in\GL_2(\R)$ has infinite order.
\end{example}                     

\begin{example}
    Let us compute the orders of $\Z/4=\{0,1,2,3\}$. This is an additive group and $0$ is the neutral element. 
    Thus $|0|=1$. Since we are using additive notations, ``powers'' really mean multiples. A direct calculation
    shows that $|1|=|3|=4$ and $|2|=2$. 
\end{example}
                                
\begin{example}
        The group $G_\infty=\bigcup_{n\geq1}G_n$ is abelian and infinite. Note that every element of 
        $G_\infty$ has finite order. 
\end{example}
          
We conclude the topic with some exercises. 

\begin{exercise}
        Compute the orders of the elements of $\Z/6$.
\end{exercise}       

\begin{exercise}
        Prove that $a=\begin{pmatrix}1&-1\\1&0\end{pmatrix}$ has order four, $b=\
        \begin{pmatrix}0&1\\-1&-1\end{pmatrix}$ has order three and 
        compute the order of $ab$.%=\begin{pmatrix}1&1\\0&1\end{pmatrix}$ tiene orden infinito.
\end{exercise}
                                
\begin{exercise}
        Compute the order of 
        $\begin{pmatrix}1&1\\-1&0\end{pmatrix}\in\GL_2(\R)$.
\end{exercise}
                                
\begin{exercise}
        Prove that in $\D_n$ one has 
        $|r^js|=2$ and $|r^j|=n/\gcd(n,j)$.
\end{exercise}
                                
\begin{exercise}
        Prove that a group with finitely many subgroups
        is finite. 
\end{exercise}




\subsection{Lagrange's theorem}

Let $G$ be a group and $H$ be a subgroup of $G$. We say that the elements $x,y\in
G$ are (left) equivalent modulo $H$ if $x^{-1}y\in H$.
We will use the following notation:
\begin{equation}
\label{eq:modH}
    x\equiv y\bmod
    H\Longleftrightarrow x^{-1}y\in H.
\end{equation}

\begin{exercise}
    Prove that~\eqref{eq:modH} is an equivalence relation, that
    is 
    \begin{enumerate}
    \item $x\equiv x\bmod H$ for all $x$; 
    \item if $x\equiv y\bmod H$, then $y\equiv x\bmod H$; and 
    \item if $x\equiv y\bmod H$ and $y\equiv z\bmod H$, then $x\equiv z\bmod H$.
    \end{enumerate}
\end{exercise}

The equivalence classes of this equivalence relation modulo $H$
are the sets of the form $xH=\{xh:h\in H\}$, as the class 
of an element $x\in G$ is the set 
\[
        \{y\in G:x\equiv y\bmod H\}=\{y\in G:x^{-1}y\in H\}=\{y\in G:y\in xH\}=xH.
\]
The set $xH$ is called 
a \emph{left coset} of $H$ in $G$ and $x$ is 
a \emph{representative} of $xH$.

Having an equivalence relation modulo $H$ in $G$ allows us to
decompose $G$ as a disjoint union of certain subsets related to $H$. 

\begin{proposition}
Let $G$ be a group and $H$ be a subgroup of $G$. 
\begin{enumerate}
\item If $xH\cap yH\ne\emptyset$, then $xH=yH$.
\item The group $G$ decomposes as a disjoint union 
of different left cosets of $H$.
\end{enumerate}
\end{proposition}

\begin{proof}
    Let us prove the first claim. If $g\in xH\cap yH$, we write 
    $g=xh$ for some $h\in H$. Then 
    \[
    gH=(xh)H=x(hH)=xH.
    \]
    Similarly, $gH=yH$. Hence $xH=yH$.
    The second claim follows from the first one. 
\end{proof}

One can also define right cosets: $x\equiv
y\bmod H$ if and only if $xy^{-1}\in H$. In this case, 
the equivalence classes are 
the sets of the form $Hx$ with $x\in X$. The set $Hx$ 
is called a \emph{right coset}
with \emph{representative} $x$ of $H$ in $G$. 

\begin{proposition}
    If $H$ is a subgroup of $G$, then  $|Hx|=|H|=|xH|$ for all $x\in G$.
\end{proposition}

\begin{proof}
    Let $x\in G$. The map $H\to Hx$, $h\mapsto hx$, is bijective 
    with inverse $hx\mapsto h$. Similarly, the map $H\to xH$,
    $h\mapsto xh$, is bijective.
\end{proof}

The map 
\[
        \{\text{right cosets of $H$ in $G$}\}\to\{\text{left cosets of $H$ in $G$}\}
\]
given by $Hx\mapsto x^{-1}H$ is a bijection, as 
\[
        Hx=Hy
        \Longleftrightarrow xy^{-1}\in H
        \Longleftrightarrow (x^{-1})^{-1}y^{-1}\in H
        \Longleftrightarrow x^{-1}H=y^{-1}H.
\]
In particular, the number of right cosets of $H$ in $G$
equals the number of left cosets of $H$ in $G$. 

\begin{definition}
    \index{Index}
    If $H$ is a subgroup of $G$, the \emph{index} of $H$ in $G$
    is the number $(G:H)$ of left (or right) cosets of $H$ in $G$. 
\end{definition}

\begin{example}
    If $G=\Z$ and $S=n\Z$, then 
    \[
    a+S=\{a+nq:q\in\Z\}=\{k\in\Z:k\equiv a\bmod n\}.
    \]
\end{example}
    
\begin{example}
    The subgroups of $\Sym_3$ are $\{\id\}$, the order-two subgroups 
    $\Sym_3$, $\langle(12)\rangle$, 
    $\langle(13)\rangle$ and $\langle(23)\rangle$, and 
    the order-three subgroup $\langle(123)\rangle=\{\id,(123),(132)\}$.  
    If $H=\langle(12)\rangle=\{\id,(12)\}$, then 
    \begin{align*}
    &H=(12)H=\{\id,(12)\},\\
    &(123)H=(13)H=\{(13),(123)\},\\
    &(132)H=(23)H=\{(23),(132)\}.
    \end{align*}
    Note that our group decomposes as 
    \[
    \Sym_3=H\cup (123)H\cup (132)H\quad\text{(disjoint union)}.
    \]
    \end{example}

    \begin{example}
        Let $G=\R^2$ with the usual addition 
        and $v\in\R^2$. The line 
        \[
        L=\{\lambda v:\lambda\in\R\}
        \]
        is a subgroup of $G$. For each 
        $p\in\R^2$, the coset $p+L$ 
        is the line parallel to $L$ that 
        passes through $p$.
    \end{example}

The following important theorem will be used extensively. 

\begin{theorem}[Lagrange]
\index{Lagrange's theorem}
    If $G$ is a finite group and $H$ is a subgroup of $G$, 
    then $|G|=|H|(G:H)$. In particular, $|H|$ divides $|G|$.
\end{theorem}

\begin{proof}
    We decompose $G$ into equivalence classes modulo $H$, that is 
    \[
    G=\bigcup_{i=1}^n x_iH\quad\text{(disjoint union)}
    \]
    for some $x_1,\dots,x_n\in G$, where $n=(G:H)$. 
    Since each of these equivalence classes has 
    exactly 
    $|H|$ elements,
    \[
            |G|=\sum_{i=1}^n|x_iH|=\sum_{i=1}^n|H|=|H|(G:H).\qedhere
    \]
\end{proof}

Let us discuss some corollaries. 

\begin{corollary}
    If $G$ is a finite group and $g\in G$, then $g^{|G|}=1$.
\end{corollary}

\begin{proof}
    By definition. $|g|=|\langle g\rangle|$. Apply Lagrange's theorem 
    to the subgroup $H=\langle g\rangle$ to obtain that 
    \[
            g^{|G|}=g^{|H|(G:H)}=(g^{|H|})^{(G:H)}=1.\qedhere
    \]
\end{proof}

\begin{corollary}
    If $G$ has prime order, then $G$ is cyclic. 
\end{corollary}

\begin{proof}
    Let $g\in G\setminus\{1\}$ and $H=\langle g\rangle$. By Lagrange's theorem, 
    $|H|$ divides $|G|$. Thus $|H|=|G|$, as $|G|$ is prime. Therefore 
    $G=H=\langle g\rangle$.
\end{proof}

\begin{corollary}
\label{cor:coprime_orders}
    If $G$ is a finite abelian group and $g,h\in G$ are elements of finite coprime orders, 
    then 
    $|gh|=|g||h|$.
\end{corollary}

\begin{proof}
    Let $n=|g|$, $m=|h|$ and $l=|gh|$. Since $G$ is abelian,
    \[
    (gh)^{nm}=(g^n)^m(h^m)^n=1. 
    \]
    Thus $l$ divides $nm$. Since $(gh)^l=1$,
    $g^l=h^{-l}\in \langle g\rangle\cap\langle h\rangle=\{1\}$ 
    (because $|\langle g\rangle|=n$ and $|\langle h\rangle|=m$ are coprime, 
    $nm$ divides $l$ by Lagrange's theorem).
\end{proof}

Fermat's little theorem is a particular case of Lagrange's theorem. 

\begin{exercise}[Fermat's little theorem]
    \index{Fermat's little theorem}
    Let $p$ be a prime number. Prove that 
    \[a^{p-1}\equiv1\bmod p
    \]
    for all $a\in\{1,2,\dots,p-1\}$.
\end{exercise}

For the next corollary, we need Euler's totient function. 
Recall that 
$\varphi(n)$ is the number of positive integers $k\in\{1,\dots,n\}$
coprime with $n$. The group of units of 
$\Z/n$ has $\varphi(n)$ elements (because $x\in\Z/n$ is invertible
if and only if $x$ and $n$ are coprime).

\begin{exercise}[Euler's theorem]
        \index{Euler's theorem }
        Let $a$ and $n$ be coprime integers. Prove that 
        \[
        a^{\varphi(n)}\equiv1\bmod n.
        \]
\end{exercise}

The converse of Lagrange's theorem does not hold.   

\begin{example}
\index{Alternating group $\Alt_4$}
\label{exa:A4}
Consider the group 
    \begin{multline*}
\Alt_4=\{\id,(234),(243),(12)(34),(123),(124),(132),(134),(13)(24),(142),(143),(14)(23)\}.
\end{multline*}
This is an important subgroup of $\Sym_4$ known as the 
\emph{alternating group} in four symbols. 

We claim that $\Alt_4$ has no subgroups of order six. If $H\leq\Alt_4$ is such that 
$|H|=6$, then, since $(\Alt_4:H)=2$, for every $x\not\in H$ we can decompose $\Alt_4$ as 
as disjoint union 
$\Alt_4=H\cup xH$.

For each $g\in\Alt_4$ we have that $g^2\in H$ (if $g\not\in H$, then, since $g^2\in\Alt_4=H\cup
gH$, it follows that $g^2\in H$). In particular, since 
$(ijk)=(ikj)^2$, order-three elements of $\Alt_4$ belong to $H$, a contradiction, 
because $\Alt_4$ has eight elements of order three. 
\end{example}

We all need a favorite group. Mine is $\SL_2(3)$,
the group of $2\times2$ matrices with coefficients in $\Z/3$
and determinant one. 

\begin{bonus}
Prove that    \[
    \SL_2(3)=\left\{\begin{pmatrix}a&b\\c&d\end{pmatrix}:ad-bc=1,\,a,b,c,d\in\Z/3\right\}
    \]
    has order 24 and does not contain subgroups of order 12.
\end{bonus}


\section{Lecture: 07/03/2024}

\subsection{The symmetric group}

\index{Cycle}
Let $\sigma\in\Sym_n$. We say that the permutation $\sigma$ is an $r$-cycle 
if there are $a_1,\dots,a_r\in\{1,\dots,n\}$ such that 
$\sigma(j)=j$ for all $j\not\in\{a_1,\dots,a_r\}$ and 
\[
\sigma(a_i)=\begin{cases}
a_{i+1} & \text{if $i<r$},\\
a_1 & \text{if $i=r$}.
\end{cases}
\]

For example, $(12)$, $(13)$ and $(23)$ are 2-cycles of 
$\Sym_3$. Note that 
2-cycles are called \emph{transpositions}.
The permutations $(123)$ and $(132)$ are 3-cycles of $\Sym_3$.

\index{Disjoint permutations}
We say that the permutations $\sigma,\tau\in\Sym_n$ 
are \emph{disjoint} if for all 
$j\in\{1,\dots,n\}$
one has $\sigma(j)=j$ or $\tau(j)=j$. For example, 
$(134)$ and $(25)$ are disjoint. The permutations $(134)$ and 
$(24)$ are not disjoint. 

If $\sigma\in\Sym_n$ and $j$ is such that 
$\sigma(j)=j$, then $j$ is a fixed point of $\sigma$. The elements 
$j$ such that 
$\sigma(j)\ne j$ are the points moved by 
$\sigma$.

\begin{claim}
Disjoint permutations commute. 
\end{claim}


We now prove that every permutation can be written 
as product of disjoint cycles. 
The decomposition is unique up to the order of the factors. 
We start with a lemma (used to prove the uniqueness of the decomposition).

\begin{lemma}
        Let $\sigma=\alpha\beta\in\Sym_n$ with $\alpha$ and $\beta$ disjoint permutations. If $\alpha(
i)\ne i$, then $\sigma^k(i)=\alpha^k(i)$ for all $k\geq0$.
\end{lemma}

\begin{proof}
    Without loss of generality, we may assume that $k>0$. 
    Let $i\in\{1,\dots,n\}$. Then \[
    \sigma^k(i)=(\alpha\beta)^k(i)=\alpha^k(\beta^k(i))=\alpha^k(i).\qedhere
    \]
\end{proof}

\begin{theorem}
Each $\sigma\in\Sym_n\setminus\{\id\}$ can be written as a product
of disjoint cycles of length 
 $\geq2$. The decomposition is unique up to 
 the order of the factors. 
 \end{theorem}

\begin{proof}
    We proceed by induction on the number $k$ 
    of elements of $\{1,\dots,n\}$ moved by $\sigma$. If $k=2$, 
    the result is trivial. Assume that the result 
    holds for all permutations moving $<k$ points. Let
    $\sigma$ be a permutation that moves $k\geq2$ points and 
    $i_1\in\{1,\dots,n\}$ be such that $\sigma(i_1)\ne i_1$. We 
    consider the cycle that contains $i_1$. So let 
        $i_2=\sigma(i_1)$, $i_3=\sigma(i_2)$... We know that 
        there exists $r$ such that $\sigma(i_r)=i_1$
        (otherwise, if $\sigma(i_r)=i_j$ for some 
        $j\geq2$, then 
        \[
        \sigma(i_{j-1})=i_j=\sigma(i_r),
        \]
        a contradiction to the bijectivity of $\sigma$, as $i_{j-1}\ne i_r$). 
        Let $\sigma_1=(i_1\cdots i_r)$. By the inductive hypothesis, since 
        $\sigma_1^{-1}\sigma$ moves $<k$ points (because 
        the $i_j$ are fixed points of $\sigma_1^{-1}\sigma$), 
        we can write \[
        \sigma_1^{-1}\sigma=\sigma_2\cdots\sigma_s,
        \]
        where 
        $\sigma_2,\dots,\sigma_s$ are disjoint cycles. 
        This implies that $\sigma=\sigma_1\sigma_2\cdots\sigma_s$.

        We now prove the uniqueness of the decomposition. 
        Assume that 
        \[
        \sigma=\sigma_1\cdots\sigma_s=\tau_1\cdots\tau
_t
\]
with $s>0$. Let $i_1\in\{1,\dots,n\}$ be such that 
        $\sigma_1(i_1)\ne i_1$. By the previous lemma, 
        \[
        \sigma^k(i_1)=\sigma_1^k(i_1)
        \]
        for all $k\geq0$.
        There exists $j\in\{1,\dots,t\}$ such that 
        $\tau_j(i_1)\ne i_1$. Since the $t_k$'s commute, 
        without loss of generality, we may assume that $j=
1$. Thus $\sigma^k(i_1)=\tau_1^k(i_1)$ for all $k\geq0$.  
This implies that 
        $\sigma_1=\tau_1$, as $\sigma_1$ and $\tau_1$ are cycles. 
        Thus $\sigma_2\cdots\sigma_s=\tau_2\cdots\tau_t$. Repeating
        this procedure, we obtain that $s=t$. Therefore 
        $\sigma_j=\tau_j$ for all $j$.
\end{proof}

\begin{corollary}\
\label{cor:generation}
        \begin{enumerate}
                \item $\Sym_n=\langle (ij):i<j\rangle$.
                \item $\Sym_n=\langle (12),(13),\dots,(1n)\rangle$.
                \item $\Sym_n=\langle (12),(23),\dots,(n-1\,n)\rangle$.
                \item $\Sym_n=\langle (12),(12\cdots n)\rangle$.
        \end{enumerate}
\end{corollary}

\begin{proof}
        The first claim follows from the previous theorem, as 
        \[
        (a_1\cdots a_r)=(a_1a_r)(a_1a_{r-1})\cdots(a_1a_2).
        \]
        If we write $\sigma\in\Sym_n$ as a product of disjoint cycles, 
        the previous formula implies 
        that 
        \[
        \Sym_n\subseteq\langle (ij):i<j\rangle.
        \]
        The other 
        inclusion is trivial. 

        For the second claim, one uses the first claim and the
        formulas 
        \[
        (1i)(1j)(1i)=(ij), 
        \]
        where $i\ne j$.

        To prove the third claim, write $\sigma$ as a product 
        of transpositions and 
        note that 
        \[
        (13)=(12)(23)(12),\quad
        (1\,k+1)=(k\,k+1)(1k)(k\,k+1)
        \]
        for all $k\geq3$.

        Finally, the fourth claim follows from 
        the third claim and 
        the formula 
        \[
        (12\cdots n)^{k-1}(12)(12\cdots n)^{1-k}=(k\,k+1),
        \]
        where $k\geq1$.
\end{proof}

Here is an alternative proof of
the first claim of Corollary 
\ref{cor:generation}. We must show that every 
permutation can be written as a product of transpositions. 
Let us assume that $n$ persons are invited to a concert. They sit
in the first row without following  
the seat number on their tickets. How can we put each person in 
the right seat? First, we locate the person that should be seated 
in the first place. Then we ask this person to 
interchange seats with the person seated in the first place. 
Then we identify the person 
that should be seated in the second spot. We then ask this person
to interchange seats with the person 
seated in the second spot. We do the same with the third spot, the fourth
spot... Once the process is finished, 
we have decomposed 
our permutation into a product of transpositions. 

\begin{exercise}
    Following the tricks of the proof of 
    Corollary \ref{cor:generation}, find the different
    decompositions of the permutation
    $(1324)(56)(789)\in\Sym_9$. 
\end{exercise}

Every permutation yields a permutation matrix. For example, 
the matrix corresponding to $\sigma=\id\in\Sym_3$ 
is the $3\times 3$ identity matrix. The permutation
$\sigma=(123)$ yields the matrix 
\[
P_\sigma=\begin{pmatrix}0&0&1\\1&0&0\\0&1&0\end{pmatrix}.
\]
If $e_1,e_2,e_3$ is the standard basis of $\R^{3\times1}$, then
\[
P_{\sigma}(e_1)=e_2,
\quad 
P_{\sigma}(e_2)=e_3,
\quad 
P_{\sigma}(e_3)=e_1.
\]
We can write $P_\sigma$ as a sum
of elementary matrices: 
\[
P_\sigma=\begin{pmatrix}
    0&0&0\\
    1&0&0\\
    0&0&0
\end{pmatrix}
+\begin{pmatrix}
    0&0&1\\
    0&0&0\\
    0&0&0
\end{pmatrix}
+\begin{pmatrix}
    0&0&0\\
    0&0&0\\
    0&1&0
\end{pmatrix}.
\]

In general, the permutation matrix
$P_\sigma$ associated with a permutation 
$\sigma\in\Sym_n$, permutes the elements of the standard basis
of $\R^{n\times1}$ in the way $\sigma$ permutes 
the elements of $\{1,2,\dots,n\}$.

\index{Elementary matrix}
Recall that the 
\emph{elementary matrix} 
$E_{i,j}$ is the matrix with a one in position
$(i,j)$ and zero in all other entries. Recall the
following formulas: 
\begin{align*}
E_{i,j}E_{k,l}&=\begin{cases}
E_{i,l} & \text{if $j=k$},\\
0 & \text{if $j\ne k$}
\end{cases}
\end{align*}

\begin{exercise}
\label{xca:permutation_matrix}
Let $\sigma\in\Sym_n$. Prove that
\[
P_\sigma=\sum_{i=1}^n E_{\sigma(i),i}.
\]
\end{exercise}

The determinant of a permutation matrix equals 
$\pm1$. Why? 

\begin{proposition}
If $\sigma,\tau\in\Sym_n$, then $P_{\sigma\tau}=P_\sigma P_\tau$.
\end{proposition}

\begin{proof}
We compute 
\begin{align*}
P_\sigma P_\tau &=\left(\sum_{i=1}^n E_{\sigma(i),i}\right)\left(\sum_{j=1}^nE_{\tau{(j)},j}\right)\\
&=\sum_{i=1}^n\sum_{j=1}^n E_{\sigma(i),i}E_{\tau(j),j}\\
&=\sum_{j=1}^n E_{\sigma(\tau(j)),j}\\
&=P_{\sigma\tau},
\end{align*}
where the double sum is zero unless $i=\tau(j)$.
\end{proof}


\begin{definition}
\index{Permutation!even}
\index{Permutation!odd}
\index{Permutation!sign}
    The \emph{sign} of a permutation $\sigma\in\Sym_n$ 
    is the 
    determinant of the matrix 
    $P_\sigma$, that is $\sgn(\sigma)=\det P_\sigma$.
    A permutation $\sigma$ is said to be \emph{even} if $\sgn(\sigma)=1$ and \emph{odd} if $\sgn(\sigma)=-1$.
\end{definition}

The identity is an even permutation. Every 3-cycle is
an even permutation. Each transposition is an odd permutation. 

\begin{proposition}
If $\sigma,\tau\in\Sym_n$, then $\sgn(\sigma\tau)=(\sgn\sigma)(\sgn\tau)$.
\end{proposition}

\begin{proof}
        We compute 
        \[
        \sgn(\sigma\tau)=\det(P_\sigma P_\tau)=(\det P_\sigma)(\det P_\tau)=\sgn(\sigma)\sgn(\tau).\qedhere
        \]
\end{proof}

Each permutation can be written as a product of transpositions. 
There is no uniqueness of this decomposition. For example, 
\[
(13)=(12)(23)(12)=(12)(23)(12)
\]

However, the following result holds: If 
$\sigma=\sigma_1\cdots\sigma_s$ is a product of transpositions, 
then $\sgn(\sigma)=(-1)^s$.
In particular, $\sigma$ is even if and only if
$s$ is even. 


\begin{example}
\index{Center!of $\Sym_n$}
We claim that if $n\geq3$ then $Z(\Sym_n)=\{\id\}$.
Assume that $Z(\Sym_n)\ne\{\id\}$. Let 
$\sigma\in Z(\Sym_n)$ be such that $\sigma(i)=j$ for some $i\ne j$. 
Since $n\geq3$, there exists an element $k\in\{1,\dots,n
\}\setminus\{i,j\}$. Thus 
$\tau=(jk)\in\Sym_n$. Since $\sigma$ is central, 
\[
j=\sigma(i)=\tau\sigma\tau^{-1}(i)=\tau(\sigma(i))=\tau(j)=k,
\]
a contradiction.
\end{example}

\begin{definition}   
\index{Alternating group}
The \emph{alternating group}
\[
\Alt_n=\{\sigma\in\Sym_n:\sgn(\sigma)=1\}
\]
is the subgroup of $\Sym_n$ formed by even permutations. 
\end{definition}

\begin{proposition}
\index{Order!of the alternating group}
$|\Alt_n|=n!/2$.
\end{proposition}

\begin{proof}
Let $\sigma=(12)\not\in\Alt_n$. We claim that 
$\Sym_n=\Alt_n\cup\Alt_n\sigma$ (disjoint union), where 
\[
\Alt_n\sigma=\{\tau\sigma:\tau\in\Alt_n\}
\]
is the right coset of $\Alt_n$ in $\Sym_n$ with 
representative $\sigma$. (We could have used, of course, left cosets.)
If 
$\tau\in\Sym_n$ is such that $\tau\not\in\Alt_n$, then 
\[
\sgn(\tau\sigma)=(\sgn\tau)(\sgn\sigma)=1.
\]
Thus 
$\tau\sigma\in\Alt_n$. Therefore $\tau\in\Alt_n\sigma$. Since  $|\Alt_n\sigma|=|\Alt_
n|$ (because the map $\Alt_n\to\Alt_n\sigma$, $x\mapsto x\sigma$, is bijective), we conclude that 
$n!=|\Sym_n|=2|\Alt_n|$.
\end{proof}

A direct calculation shows that  
\[
\Alt_3=\{\id,(123),(132)\}.
\]
The group $\Alt_3$ is abelian.
In Example \ref{exa:A4}, we used the alternating group 
\begin{multline*}
\Alt_4=\{\id,(234),(243),(12)(34),\\(123),(124),(132),(134),(13)(24),(142),(143),(14)(23)\}.
\end{multline*}
If $n\geq4$, then $\Alt_n$ is non-abelian. For example, 
$(123)$ and $(124)$ do not commute. 

\begin{proposition}
\label{pro:A_n3cycles}
$\Alt_n=\langle\{\text{3-cycles}\}\rangle$.
\end{proposition}

\begin{proof}
Each 3-cycle is an even permutation, as $(ijk)=(ik)(ij)$. 
To prove the other inclusion, let $\sigma\in\Alt_n$.
Write $\sigma=\sigma_1\cdots\sigma_s$ for some even integer $s$ 
and transpositions $\sigma_1,\dots,\sigma_s$. 
Now the claim follows from the formulas 
\[
(kl)(ij)=(kl)(ki)(ki)(ij)=(kil)(ijk),\quad
(ik)(ij)=(ijk).\qedhere
\]
 \end{proof}

Proposition \ref{pro:A_n3cycles} has several 
important applications. 

\begin{exercise}
\label{xca:commutator_A4}
    Prove that $[\Alt_4:\Alt_4]=\{\id,(12)(34),(13)(24),(14)(23)\}$. 
\end{exercise}

\begin{example}
\index{Commutator!of $\Alt_n$}
If $n\geq5$, then $[\Alt_n,\Alt_n]=\Alt_n$. To prove the non-trivial
inclusion, it is enough to note that $\Alt_n$ is generated by 
3-cycles and that, since $n\geq5$, each 3-cycle 
is a product of commutators: 
\[
(abc)=[(acd),(ade)][(ade),(abd)],
\]
where $\#\{a,b,c,d,e\}=5$.
\end{example}

\begin{example}
\index{Commutator!of $\Sym_n$}
If $n\geq3$, then $[\Sym_n,\Sym_n]=\Alt_n$. First, we prove that $[\Sym_n,\Sym_n]\subseteq\Alt_n$. If 
$\sigma\in[\Sym_n,\Sym_n]$,
say $\sigma=[\sigma_1,\tau_1][\sigma_2,\tau_2]\cdots[\sigma_k,\tau_k]$, then
\[
\sgn(\sigma)=\sgn([\sigma_1,\tau_1])\cdots\sgn([\sigma_k,\tau_k])=1.
\]
Conversely, if $\sigma\in\Alt_n$, by the previous proposition, 
we can write $\sigma$ as a product of 3-cycles. 
From this, the claim follows, as each 3-cycle is a commutator: 
\[
(abc)=(ab)(ac)(ab)(ac)=[(ab),(ac)]\in[\Sym_n,\Sym_n].\qedhere
\]
\end{example}
\section{14/03/2024}

\subsection{Quotients}

If $G$ is a group and $N$ is a subgroup of $G$, we want to know
when the set $G/N$ of left cosets 
of $N$ in $G$ is a group with 
the operation 
\begin{equation}
\label{eq:operation}
G/N\times G/N\to G/N,\quad 
(xN,yN)\mapsto xyN,
\end{equation}
that is, when this operation 
is well-defined. 
What does this mean? We need to check that
\eqref{eq:operation} is indeed a function. 
For that purpose, we need to prove that
\eqref{eq:operation} does not depend on the representatives of left
cosets used. Thus we need to show that 
$xN=x_1N$ and $yN=y_1N$, then 
$xyN=x_1y_1N$. 

Let us try to understand this condition. If $x^{-1}x_1\in N$ and 
$y^{-1}y_1\in N$, then $x_1=xn$ and $y_1=ym$ for some 
$m,n\in N$. Thus 
\[
(xy)^{-1}(x_1y_1)=y^{-1}x^{-1}x_1y_1=y^{-1}nym\in N
\]
if and only if $y^{-1}ny\in N$.

\begin{example}
If $G=\Sym_3$ and $H=\langle (12)\rangle$, then $(xN,yN)\mapsto xyN$ is not a function. Recall that 
$G/N=\{N,(123)N,(132)N\}$, where 
$N=(12)N$, $(123)N=(13)N$ and $(132)N=(23)N$. Then 
\[
(132)N=(13)(23)N=(13)N(23)N=(123)N(132)N=N,
\]
a contradiction.
\end{example}

\begin{definition}
    \index{Subgroup!normal}
    Let $G$ be a group. 
    A subgroup $N$ of $G$ is said to be \emph{normal} if $gNg^{-1}\subseteq N$ for all $g\in G$.
    Notation: If $N$ is normal in $G$, then $N\unlhd G$.
\end{definition}

In an abelian group, every subgroup is normal. 

\begin{proposition}
\label{pro:normalidad}
Let $N$ be a subgroup of $G$. 
The following statements are equivalent:
\begin{enumerate}
        \item $gNg^{-1}\subseteq N$ for all $g\in G$.
        \item $gNg^{-1}=N$ for all $g\in G$.
        \item $gN=Ng$ for all $g\in G$.
\end{enumerate}
\end{proposition}

\begin{proof}
We only prove that $1)\implies 2)$, as the other implications are trivial. If $n\in N$ and $g\in G$, then 
$n=g(g^{-1}ng)g^{-1}\in gNg^{-1}$.
\end{proof}

\begin{proposition}
    Let $N$ be a subgroup of $G$. The following statements
    are equivalent: 
    \begin{enumerate}
        \item $N$ is normal in $G$.
        \item $(gN)(hN)=(gh)N$ for all $g,h\in G$.
    \end{enumerate}
\end{proposition}

\begin{proof}
   We first prove that $1)\implies 2)$. Let $g\in G$. Since
   $gNg^{-1}=N$, 
   \[
   (gN)(hN)=g(Nh)N=g(hN)N=(gh)N.
   \]
   
   We now prove that $2)\implies1)$. If $g\in G$, then
    \[
    gNg^{-1}\subseteq (gN)(g^{-1}N)=(gg^{-1})N=N.\qedhere
    \]
\end{proof}

If $G$ is a group, then 
$\{1\}$ and $G$ are always normal subgroups. 

\begin{example}
\index{Center!of a group}
If $G$ is a group, then $Z(G)$ is a normal subgroup of $G$. Moreover, 
if $N\leq Z(G)$, then $N\unlhd
G$.
\end{example}

\begin{example}
If $G$ is a group, then  $[G,G]$ is a normal subgroup of $G$. If 
$x\in [G,G]$ and $g\in G$, then
$gxg^{-1}=(gxg^{-1}x^{-1})x=[g,x]x\in [G,G]$. Alternatively,
\[
g\left(\prod_{i=1}^k[x_i,y_i]\right)g^{-1}=\prod_{i=1}^k [gx_ig^{-1},gy_ig^{-1}]
\]
for all $g,x_1,\dots,x_k,y_1,\dots,y_k\in G$.
\end{example}

\begin{example}
Let $n\geq2$. Then 
$\Alt_n$ is a normal subgroup of $\Sym_n$.
If $\sigma\in\Alt_n$ and $\tau\in\Sym_n$, then $\tau\sigma\tau^{-1}\in\Alt_n$, as 
\[
\sgn(\tau\sigma\tau^{-1})=\sgn(\sigma)=1.
\]
\end{example}

\begin{example}
If $N$ is a subgroup of $G$ such that $(G:N)=2$, then $N$ is normal in $G$. We need to show that $gN=Ng$ for all $g\in G$. Let $g\in G$. 
If $g\in N$, then $gN=Ng$. If $g\not\in N$, then
$gN\ne N$. Since $(G:N)=2$, we can decompose $G$ as 
the disjoint union $G=N\cup gN$. Hence 
$gN=G\setminus N$. Similarly, 
$Ng=G\setminus N$ and therefore $gN=Ng$.
\end{example}

\begin{example}
As a particular case of the previous example, 
\[
\langle (123)\rangle=\{\id,(123),(132)\}\unlhd\Sym_3.
\]
Note that
$\langle (12)\rangle=\{\id,(12)\}$ is not normal in $\Sym_3$.  
For example, $(13)(12)(13)=(23)\not\in\langle(12)\rangle$.
\end{example}

\begin{example}
The subgroup $\SL_n(\R)$ is normal in $\GL_n(\R)$. If $g\in\GL_n(\R)$ and $x\in\SL_n(\R)$, then $\det(gxg^{-1})=(\det g)(\det x)(\det g)^{-1}=1$.
\end{example}

\begin{example}
\index{Klein group}
The Klein group $K=\{\id,(12)(34),(13)(24),(14)(23)\}$ is normal in
$\Sym_4$. We need to show that 
$\sigma K\sigma^{-1}\subseteq K$ for all $\sigma\in\Sym_4$. Do we need to check this for every element of $\Sym_4$? No. One always has tricks! 
Recall that $\Sym_4$ is generated by $(12)$ and $(1234)$. Since
every element of  $\Sym_4$ is a word on $(12)$ and 
$(1234)$,
it is enough to see that
$\sigma K\sigma^{-1}\subseteq K$ for all $\sigma\in\{(12),(1234)\}$. We left as an exercise to show that 
\[
(12)K(12)^{-1}\subseteq K,\quad
(1234)K(1234)^{-1}\subseteq K.
\]
\end{example}

\begin{exercise}
\index{Semi-direct product}
Let $G=\R\times\R^\times$ with the operation 
\[
(x,y)(u,v)=(x+yu,yv).
\]
Prove that $\{(x,1):x\in\R\}$ is normal in $G$ and that
$\{(0,y):y\in\R^\times\}$ is not. 
\end{exercise}

Let us compute the list of normal subgroups of $\Alt_4$.

\begin{example}
\index{Normal subgroups of $\Alt_4$}
We claim that 
$\{\id\}$, $K=\{\id,(12)(34),(13)(24),(14)(23)\}$ and $\Alt_4$ 
are the normal subgroups of $\Alt_4$.

Since $\Alt_4=\{\text{3-cycles}\}\cup K$, $K$ is the only subgroup
of $\Alt_4$ of order four. This implies that $K$ is normal in $\Alt_4$ (because every conjugate $gKg^{-1}$ of $K$ is a subgroup of 
$\Alt_4$ of order four). Let $N\ne\{\id\}$ be a normal subgroup of 
$\Alt_4$. 

If $N$ contains a 3-cycle, say 
$(abc)\in N$, then 
\[
(acd)=(bcd)(abc)(bcd)^{-1}\in N
\]
and hence $N=\Alt_4$ (because $N$ contains every 3-cycle). 

Assume that $N$ does not contain 3-cycles. 
Then some non-trivial element of $K$ belongs to $N$, say 
$(ab)(cd)\in N$. Hence 
\[
(ac)(bd)=(bcd)(ab)(cd)(bcd)^{-1}\in N,\quad
(ad)(bc)=(ab)(cd)(ac)(bd)\in N
\]
and therefore $N=K$.
\end{example}

Normality is not transitive. 

\begin{exercise}
\index{Dihedral group}
Let $G=\D_4$ be the dihedral group of order eight. Let $N=\langle s,r^2\rangle$ and $H=\langle s\rangle$.
Prove that $H$ is normal in $N$, $N$ is normal in $G$ but $H$ is not normal in $G$. 
\end{exercise}

\begin{example}
\index{Normal subgroups!of $\Sym_4$}
We claim that $\{\id\}$, $K$, $\Alt_4$ and $\Sym_4$ are the normal subgroups of $\Sym_4$.

Let $N$ be a normal subgroup of $\Sym_4$. If $N\subseteq\Alt_4$, then
$N$ is normal in $\Alt_4$ and hence either $N=\{\id\}$,
$N=K$ or $N=\Alt_4$. Assume that $N\not\subseteq\Alt_4$, that is
$N$ contains an odd permutation. If $\sigma\in\Sym_4$ is odd, then 
$\sigma$ is either a transposition or a 4-cycle. 
 
If $N$ contains a transposition, then all transpositions 
belong to $N$, as 
\[
\tau(ij)\tau^{-1}=(\tau(i)\,\tau(j))
\]
for all $\tau\in\Sym_4$. In this case, $N=\Sym_4$ because 
the transpositions generate $\Sym_4$. 


If $N$ contains a 4-cycle, all 4-cycles belong to $N$, as 
\[
\tau(ijkl)\tau^{-1}=(\tau(i)\,\tau(j)\,\tau(k)\,\tau(l))
\]
for all $\tau\in\Sym_4$ and $K\subseteq N$ because 
\[
(ac)(bd)=(abcd)^2.
\]
This implies that $|N|\geq10$. Since $K\subseteq N$,  $|N\cap\Alt_4|\geq 5$. Moreover, $N\cap\Alt_4$ is a normal subgroup of $\Alt_4$.
Hence $N\cap\Alt_4=\Alt_4\subseteq N$. Since $N\ne \Alt_4$, $|N|>12$ and hence $N=\Sym_4$.
\end{example}

The following theorem is crucial.

\begin{theorem}
\index{Quotient group}
\label{thm:quotient}
If $N$ is a normal subgroup of $G$, then
$G/N$ is a group with the operation 
$(xN)(yN)=(xy)N$.
\end{theorem}

\begin{exercise}
    Prove Theorem \ref{thm:quotient}.   
\end{exercise}


We will see examples of quotient groups later. 

\begin{exercise}
\label{xca:commutator}
Let $H$ be a normal subgroup of $G$. Prove that
$G/H$ is abelian if and only if $[G,G]\subseteq H$.
\end{exercise}

As an application, 
we compute the commutator subgroup of $\Alt_4$. 

\begin{example}
\index{Commutator!of $\Alt_4$}
$[\Alt_4,\Alt_4]=K=\{\id,(12)(34),(13)(24),(14)(23)\}$.
We know that $K$ is normal in $\Alt_4$. Since $\Alt_4/K$ has order three, 
it is abelian. Then $[\Alt_4,\Alt_4]\subseteq K$. Since 
\[
(ab)(cd)=[(abc),(cda)],
\]
we conclude that $K\subseteq[\Alt_4,\Alt_4]$.
\end{example}

\begin{exercise}
\label{xca:G/Z(G)}
If $G/Z(G)$ is cyclic, then $G$ is abelian.
\end{exercise}

\begin{exercise}
\label{xca:normalizer}
\index{Normalizer}
If $S$ is a subgroup of $G$, the \emph{normalizer} of $S$ in $G$
is the set  
\[
N_G(S)=\{g\in G:gSg^{-1}=S\}.
\]
Prove the following statements:
\begin{enumerate}
\item $N_G(S)\leq G$. 
\item $S\unlhd N_G(S)$.
\item If $S\leq T\leq G$ and $S\unlhd T$, then $T\leq N_G(S)$.
\end{enumerate}
\end{exercise}

The normalizer of a subgroup  $S$ in $G$ is the largest 
subgroup of $G$ that contains $S$ as a normal subgroup. 

\begin{definition}
\index{Group!simple}
A group $G$ is \emph{simple} if $G\ne\{1\}$ and 
$G$ and $\{1\}$ are the only normal subgroups of $G$. 
\end{definition}

If $p$ is a prime number, then Lagrange's theorem implies that 
$\Z/p$ is a simple group. 
For $n\geq5$, the alternating group $\Alt_n$ is simple. However, 
we will not
prove this in this course. 

\begin{bonus}
\label{xca:index_p}
Let $H$ be a subgroup of $G$ such that $p=(G:H)$ is a prime number. 
Prove that the following statements are equivalent:
\begin{enumerate}
\item $H$ is normal in $G$.
\item If $g\in G\setminus H$, then $g^p\in H$.
\item If $g\in G\setminus H$, then $g^n\in H$ for some $n$ with no prime divisors $<p$.
\item If $g\in G\setminus H$, then $g^k\not\in H$ for all $k\in\{2,\dots,p-1\}$.
\end{enumerate}
\end{bonus}

We now present two applications of the previous exercise. 

\begin{bonus}
\label{xca:p_smallest}
    Let $G$ be a finite group and $p$ be the smallest prime number dividing $G$. Prove that if $H$ is a subgroup 
    of $G$ with $(G:H)=p$, then $H$ is normal in $G$. 
\end{bonus}

\begin{bonus}
Let $p$ be a prime number and $G$ be a group such that
every element of $G$ has order a power of $p$. 
If 
$H$ is a subgroup of $G$ of index $p$, then $H$ is normal in $G$.
\end{bonus}


\chapter{}

\topic{Permutable subgroups}

\index{Product of subgroups}
If $H$ and $K$ are subgroups of $G$, let 
\[
        HK=\{hk:h\in H,\,k\in K\}.
\]
Note that 
\[
H\cup K\subseteq HK\subseteq\langle H\cup K\rangle.
\]
When $HK$ is a subgroup of $G$? 
Note that $HK\leq G$ if and only if $\langle H\cup K\rangle=HK$.

\begin{proposition}
        Let $H$ and $K$ be subgroups of $G$. Then $HK$ is a subgroup of
        $G$ if and only if $HK=KH$.
\end{proposition}

\begin{proof}
    Assume that $HK=KH$. Since $1\in H\cap K$, $HK\ne\emptyset$. 
    If $h\in H$ and $k\in K$, then $(hk)^{-1}=k^{-1}h^{-1}\in KH=HK$. Moreover, 
    \[
    (HK)(HK)=H(KH)K=H(HK)K=(HH)(KK)=HK.
    \]
    Thus $HK$ is closed under multiplication. 

    Now assume that $HK$ is a subgroup of $G$. Since $H\subseteq HK$,
    $K\subseteq HK$ and $HK$ closed under multiplication,
        $KH\subseteq (HK)(HK)\subseteq HK$. Conversely, let $g\in HK$.
        Since $g^{-1}\in HK$, there exist $h\in H$ and $k\in K$ such that
        $g^{-1}=hk$.
        Thus $HK\subseteq KH$, as 
        $g=k^{-1}h^{-1}\in KH$.
\end{proof}

\begin{exercise}
\label{xca:HK_normal}
Let $H$ and $K$ be subgroups of $G$. Prove that 
if $H$ is normal in $G$, then $HK$ is a subgroup of $G$.
\end{exercise}

\begin{example}
Let $G=\Sym_4$. The subgroups $H=\langle (12)\rangle$ and $K=\langle (34)\rangle$ satisfy that $HK=KH=\{
\id,(12),(34),(12)(34)\}$ is a subgroup of 
$\Sym_4$. Note that not $H$ nor $K$
are normal in $G$.
\end{example}

\begin{exercise}
Demuestre que si $H$ y $K$ son subgrupos normales de $G$, entonces $HK$ es también normal en $G$.
\end{exercise}

\begin{exercise}
Let $G$ be a group and $S$ be a subgroup of $G$. 
If $T\leq N_G(S)$, then $TS$ is a group and $S\leq TS$.
\end{exercise}

\index{Permutable subgroups}
Two subgroups $H$ and $K$ of $G$ are said to be
\textbf{permutable} if $HK=KH$.

\begin{theorem}
\label{thm:|HK|}
    Let $H$ and $K$ be finite subgroups of $G$. 
    Then 
    \[
        |HK|=\frac{|H||K|}{|H\cap K|}.
    \]
\end{theorem}

\begin{proof}
Let $L=H\cap K$.
We decompose $H$ as a disjoint union of left coclases of $L$, say 
$H=\cup_{i=1}^k x_iL$, where $k=(H:L)$. Note that $LK=K$, as $L\subseteq K$. Moreover, $K\subseteq 1K\subseteq LK$.
Then
\[
HK=\bigcup_{i=1}^k x_iLK=\bigcup_{i=1}^k x_iK,
\]
In particular, since the union is disjoint, 
\[
|HK|=\sum_{i=1}^k |x_iK|=k|K|=\frac{|H||K|}{|H\cap K|}.\qedhere
\]
\end{proof}

In the theorem, we do not assume that $HK$ is a subgroup of $G$. 

As an application, the theorem yields a different solution to Exercise~\ref{xca:p_smallest} of page~\pageref{xca:p_smallest}. 
If $\{gHg^{-1}:g\in G\}=\{H\}$, then $H$ is normal in $G$. Assume that
there exists $g\in G$ such that
$H\ne g^{-1}Hg=K$. Since $(H:H\cap K)$ divides $|H|$ 
and all prime divisors of $|G|$ 
are $\geq p$, it follows that $(H:H\cap K)\geq p$. Thus 
\[
|HK|=\frac{|H||K|}{|H\cap K|}\geq p|K|=|G|
\]
as $(G:H)=p$ y $|K|=|H|$. In particular, $HK=G$. Since $K=g^{-1}Hg$, 
$g=h(g^{-1}h_1g)$ for some $h,h_1\in H$. Thus 
\[
1=hg^{-1}h_1\implies h_1h=g\in H\implies H=K,
\]
a contradiction.

\begin{example}
Let $G=\Sym_3$, $H=\langle (12)\rangle$ and $K=\langle (23)\rangle$. Then 
\[
HK=\{\id,(12),(23),(123)\}
\]
is not a subgroup of $G$, as by Lagrange's theorem, 
$G$ cannot have subgroups of four elements. Another way 
to see that $HK$ is not a subgroup of $G$ follows from 
the fact that 
$KH=\{\id,(12),(23),(132)\}\ne HK$.
\end{example}

\begin{example}
Let $G=\Sym_3$, $H=\langle (12)\rangle$ and $K=\langle (123)\rangle$.
Since $K$ is normal in $G$, $HK$ is a subgroup of $G$. By Lagrange's theorem, $|HK|=6$ and hence $G=HK$.
Each $g\in G$ can be written uniquely as $g=hk$ for some $h\in H$ and $k\in K$ (one can prove this either considering all possible cases or 
using the fact that $H\cap K=\{\id\}$). It follows that the map 
\[
H\times K\to G,\quad
(h,k)\mapsto hk,
\]
is bijective. Note that this bijective map is not compatible 
with the operation of $G$, as 
$(h_1k_1)(h_2k_2)\ne (h_1h_2)(k_1k_2)$. 
\end{example}

\topic{Homomorphisms}


\section{28/03/2024}

\subsection{Isomorphism theorems}

The following theorem is fundamental. For example, it allows us to recognize quotient groups. 

\begin{theorem}[First isomorphism theorem]
\index{First isomorphism theorem}
If $f\colon G\to H$ is group homomorphism, then $G/\ker f\simeq f(G)$.
\end{theorem}

\begin{proof}
    Let $K=\ker f$ and $\varphi\colon G/K\to f(G)$, $xK\mapsto f(x)$. We need to show that $\varphi$ is well-defined. This means that
    we need to show that if $xK=yK$, then $f(x)=f(y)$. If $xK=yK$, then, since $y^{-1}x\in K$, 
    \[
        f(y)^{-1}f(x)=f(y^{-1}x)\in f(K)=\{1\}.
    \]
    Thus $f(x)=f(y)$.

    We now show that $\varphi$ is a group homomorphism: 
        \[
        \varphi(xKyK)=\varphi(xyK)=f(xy)=f(x)f(y)=\varphi(xK)\varphi(yK).
        \]
    To compute $\ker\varphi$ we proceed as follows: 
        \[
        \pi(x)=xK\in\ker\varphi\Longleftrightarrow \varphi(xK)=1
        \Longleftrightarrow f(x)=1
        \Longleftrightarrow x\in K.
        \]
    Therefore  $\ker\varphi$ is trivial and
    $\varphi$ is injective. Since $\varphi\colon G/K\to f(G)$ is surjective, 
    we conclude that $G/K\simeq f(G)$.
\end{proof}

If $G$ is a group, then $G/\{1\}\simeq G$ and $G/G\simeq\{1\}$.

\begin{example}
Since $f\colon\Z\to\Z/n$, $x\mapsto x\bmod n$, is a group homomorphism
with $\ker f=n\Z$, it follows that 
$\Z/n\Z\simeq\Z/n$.
\end{example}

\begin{example}
Let $G$ be an infinite cyclic group, say $G=\langle g\rangle$. The map 
$f \colon\Z\to G$, $k\mapsto g^k$,
is a group isomorphism. Thus $G\simeq\Z$ by the first isomorphism theorem. In particular, $G=\langle g^k\rangle$ if and only if 
 $k\in\{-1,1\}$.
\end{example}

\begin{example}
We claim that $\Z/n\Z\simeq G_n$. Let 
\[
f\colon\Z\to G_n,\quad
f(k)=\exp(2i\pi k/n).
\]
Then $f$ is a surjective group homomorphism and
$\ker f=n\Z$. By the first isomorphism theorem, the claim follows. 
\end{example}

\begin{example}
Note that $2\Z\simeq 3\Z$, as both groups are infinite (alternatively, one can also consider the map $2k\mapsto 3k$). Moreover, 
\[
\Z/2\simeq\Z/2\Z\not\simeq\Z/3\Z\simeq\Z/3.
\]
\end{example}

\begin{example}
Since 
\[
f\colon\C^\times\to\C^\times,
\quad
f(z)=\frac{z}{|z|},
\]
is a group homomorphism with $\ker f=\R_{>0}$ and
$f(\C^\times)=S^1$, the first isomorphism theorem 
implies that 
$\C^\times/\R_{>0}\simeq S^1$.
\end{example}

\begin{example}
If we apply the first isomorphism theorem to the
map $f\colon S^1\to S^1$, $f(z)=z^2$, we obtain that 
$S^1/\{\pm1\}\simeq S^1$, as 
$\ker f=\{-1,1\}$ and $f(S^1)=S^1$.
\end{example}

\begin{example}
Let $f\colon\C^\times\to\C^\times$, $f(z)=|z|$. Since $\ker f=S^1$ and $f(\C^\times)=\R_{>0}$, the first isomorphism theorem 
implies that $\C^\times/S^1\simeq\R_{>0}$.
\end{example}

\begin{example}
We claim that 
$(\Z\times\Z)/\langle (1,3)\rangle\simeq\Z$. We consider
the surjective group homomorphism 
$f\colon\Z\times\Z\to\Z$, $f(x,y)=3x-y$. Since 
\[
\ker f=\{(x,3x):x\in\Z\}=\langle (1,3)\rangle,
\]
the first isomorphism theorem implies that 
$(\Z\times\Z)/\langle (1,3)\rangle\simeq\Z$.
\end{example}

\begin{exercise}
Prove that $\R/\Z\simeq S^1$.
\end{exercise}

% $t\mapsto \exp(2\pi it)$
% $z\mapsto z^2$


\begin{exercise}
Prove that $\Q/\Z\simeq\cup_{n\geq1}G_n$.
\end{exercise}

% x\mapsto cos (2\pi x)+i\sin (2\pi x)

\begin{exercise}
Prove that 
$(\Z\times\Z)/\langle (6,3)\rangle\simeq\Z\times(\Z/3)$.
\end{exercise}
% (x,y)\mapsto (2y-x,y\mod 3)

Let us see another application that shows that
the first isomorphism theorem is quite familiar. 

\begin{example}
Let $V$ be a vector space and $W$ be a subspace of $V$. 
In particular, 
$V$ is an abelian group and 
$W$ is a normal subgroup of $V$. The abelian group 
$V/W$ is then a vector space with 
\[
\lambda(v+W)=(\lambda v)+W,\quad \lambda\in\R,\,v\in V,
\]
and the canonical homomorphism
$\pi\colon V\to V/W$ is also a linear map. As an exercise, 
the reader needs to show that 
$\dim (V/W)=\dim V-\dim W$
if $\dim V<\infty$.

If $f\colon V\to U$ is a linear map, then
$V/\ker f\simeq f(V)$ as abelian groups (by the first
isomorphism theorem). The map realizing this
isomorphism is linear, so 
$V/\ker f\simeq f(V)$ as vector spaces. In particular, 
if $\dim V<\infty$, then
\[
\dim V-\dim\ker f=\dim f(V).
\]
\end{example}

\begin{exercise}
\label{xca:quotients}
Let $f\colon G\to H$ be a group homomorphism and 
$K$ a normal subgroup of $G$ such that $K\subseteq\ker f$. 
Prove that there exists a unique group homomorphism 
\[
\varphi\colon G/K\to H
\]
such that the diagram 
\[\begin{tikzcd}
        G & H \\
        {G/K}
        \arrow["f", from=1-1, to=1-2]
        \arrow["\pi"', from=1-1, to=2-1]
        \arrow["\varphi"', dashed, from=2-1, to=1-2]
\end{tikzcd}
\]
commutes. The commutativity of the diagram means that  $\varphi\circ\pi=f$, where $\pi\colon G\to G/K$ is the canonical group
homomorphism. Moreover, 
$\ker\varphi=\pi(\ker f)$ and $\varphi(G/K)=f(G)$.
In particular, $\varphi$ is injective if and only if 
$\ker f=K$ and $\varphi$ is surjective if and only if $f$ is surjective. 
\end{exercise}

We now discuss the second isomorphism theorem. As a rule to remember
what the theorem is about, one has the following diagram:
\[\begin{tikzcd}
        & NT \\
        N && T \\
        & {N\cap T}
        \arrow[no head, from=1-2, to=2-3]
        \arrow[no head, from=1-2, to=2-1]
        \arrow[no head, from=2-1, to=3-2]
        \arrow[no head, from=2-3, to=3-2]
\end{tikzcd}\]

\begin{exercise}[Second isomorphism theorem]
If $N$ is a normal subgroup of $G$ and 
$T$ is a subgroup of $G$, then $N\cap T$ is normal in $T$
and 
\[
T/(N\cap T)\simeq NT/N.
\]
\end{exercise}


% \begin{proof}
% Sea $\pi\colon G\to G/N$ el morfismo canónico. Ya vimos que la restricción $\pi|_T\colon T\to G/N$ es
%  un morfismo de grupos con núcleo
% $\ker(\pi|_T)=T\cap N$. En particular, $T\cap N$ es normal en $T$. Al aplicar el primer
% teorema de isomorfismos, $T/(T\cap N)\simeq \pi(T)$. Como $N$ es normal en $G$,
% $NT$ es un subgrupo de $G$ que contiene a $N$.
% La restricción $\pi|_{NT}$ es entonces un morfismo de grupos con núcleo $NT\cap N=N$.
% Al aplicar el primer teorema de isomorfismos a $\pi|_{NT}$ obtenemos
% $NT/N\simeq \pi(NT)=\pi(T)$.
% \end{proof}

\begin{exercise}
Let $N$ be a normal subgroup of $G$ and
$\pi\colon G\to G/N$ the canonical homomorphism. Prove that
if $L$ is a subgroup of $G$, then $\pi^{-1}(\pi(L))=NL$.
%%%
%%%Por otro lado, si $L$ es un subgrupo de $G$, entonces $\pi^{-1}(\pi(L))=NL$. En efecto,
%%%si $x\in \pi^{-1}(\pi(L))$, entonces $\pi(x)\in \pi(L)$ y luego
%%%$\pi(x)=\pi(l)$ for some $l\in L$. Como entonces $xl^{-1}\in \ker\pi=N$,
%%%se tiene que $x=(xl^{-1})l\in KL$. Recíprocamente, si $x=kl$ con $k\in K$ y $l\in L$, entonces
%%%$\pi(x)=\pi(kl)=\pi(l)\in \pi(L)$ y luego $x\in\pi^{-1}(\pi(L))$. Observemos que
%%%si $L$ es un subgrupo de $G$, entonces $NL$ es un subgrupo de $G$ que contiene a $N$.
\end{exercise}

The following example uses additive groups. 

\begin{example}
Let $G=\Z/24$, $H=\langle 4\rangle$ and $N=\langle 6\rangle$. Since $G$ is abelian, $H$ and $K$ are normal in $G$. Then 
$H+N=\langle 2\rangle$ and $H\cap N=\{0,12\}$. Let us compute
the left cosets of $N$ in $H+N$:
\[
0+N=\{0,6,12,18\},
\quad
2+N=\{2,8,14,20\},
\quad
4+N=\{4,10,16,22\}.
\]
The left cosets of $H\cap N$ in $H$ are 
\[
0+(H\cap N)=\{0,12\},
\quad
4+(H\cap N)=\{4,16\},
\quad
8+(H\cap N)=\{8,20\}.
\]
By the second isomorphism theorem, $(H+N)/N\simeq H/H\cap N$. The
isomorphism is given by 
$f\colon H/(H\cap N)\to (H+N)/N$, $h+(H\cap N)\mapsto h+N$.
In our particular case, 
\begin{align*}
&f(0+(H\cap N))=0+N,\\
&f(4+(H\cap N))=4+N,\\
&f(8+(H\cap N))=8+N=2+N.
\end{align*}
\end{example}

Let us discuss some applications. 

\begin{example}
Let $a,b\in\Z\setminus\{0\}$. Then $a\Z+b\Z=\gcd(a,b)\Z$ and $a\Z\cap b\Z=\lcm(a,b)\Z$. By the second isomorphism theorem,
\[
\frac{\gcd(a,b)\Z}{b\Z}=\frac{a\Z+b\Z}{b\Z}\simeq
\frac{a\Z}{a\Z\cap b\Z}=\frac{a\Z}{\lcm(a,b)\Z}.
\]
Since the formula involves finite groups, 
computing orders yields 
\[
ab=\gcd(a,b)\lcm(a,b).
\]
\end{example}


\index{Group!meta-abelian}
A group $G$ is said to be \emph{meta-abelian}
if it contains an abelian normal subgroup $N$ and $G/N$
is abelian. Abelian groups are meta-abelian. However, the group $\Sym_3$ is meta-abelian and not abelian.  The following exercise
present another application of the second isomorphism theorem. 

\begin{exercise}
Prove that if $G$ is a mete-abelian group and $H$ is a subgroup of
$G$, then $H$ is meta-abelian. 
\end{exercise}

% \begin{proof}
% Como $G$ es meta-abeliano, existe
% un subgrupo normal $N$ de $G$ tal que $N$ y $G/N$ son ambos abelianos.
% El subgrupo abeliano $H\cap N$ es normal en $H$. Gracias al segundo teorema de isomorfismos,
% \[
% H/(H\cap N)\simeq HN/N
% \]
% es un grupo abeliano pues $HN/N$ es un subgrupo del grupo abeliano $G/N$.
% \end{proof}

There is a third isomorphism theorem. 

% \begin{exercise}
%         \label{xca:para_3er}
%         Sea $f\colon G\to H$ un morfismo de grupos y sean $U\unlhd G$ y $V\unlhd H$. Demuestre que ex
% iste
%         un morfismo de grupos $g\colon G/U\to H/V$ tal que el diagrama
% \[
% \begin{tikzcd}
%         G & H \\
%         {G/U} & {H/V}
%         \arrow["f", from=1-1, to=1-2]
%         \arrow["{\pi_U}"', from=1-1, to=2-1]
%         \arrow["g"', dashed, from=2-1, to=2-2]
%         \arrow["{\pi_V}", from=1-2, to=2-2]
% \end{tikzcd}
% \]
%         es conmutativo si y sólo si $f(U)\subseteq V$, donde $\pi_U\colon G\to G/U$ y $\pi_V\colon H\
% to H/V$ son los morfismos canónicos. Además, en este caso,
%         \begin{enumerate}
%         \item Si $f$ es sobreyectiva, entonces $g$ es sobreyectiva.
%         \item Si $U=f^{-1}(V)$, entonces $g$ es inyectiva.
%         \end{enumerate}
% \end{exercise}
% % todo: escribir bien la demo del tercero, ese lema es horrible
% %Un caso particular del lema nos permite demostrar elegantemente el tercer teorema de isomorfismos.

\begin{bonus}[Third isomorphism theorem]
\label{xca:third}
Let $S$ and $T$ be normal subgroups of $G$ such that 
$S\subseteq T$. Prove that $S$ is normal in $T$, 
$T/S$ is normal in $G/S$ and 
\[
\frac{G/S}{T/S}\simeq G/T,
\]
where $T/S=\{tS:t\in T\}$.
\end{bonus}

The following example helps to visualize the third isomorphism
theorem. 

\begin{example}
If $m$ divides $n$, then $n\Z\leq m\Z\leq\Z$. Thus 
\[
\frac{\Z/n\Z}{m\Z/n\Z}\simeq\Z/m\Z.
\]
\end{example}

The following theorem is known as the \emph{correspondence theorem}.
It is powerful and essential. It helps to have in mind the following diagram: 
\[
\begin{tikzcd}
        && G \\
        & L && {f(G)} \\
        N && Y \\
        & {\{1\}}
        \arrow[no head, from=1-3, to=2-4]
        \arrow[no head, from=1-3, to=2-2]
        \arrow[no head, from=2-2, to=3-1]
        \arrow[no head, from=3-1, to=4-2]
        \arrow[no head, from=2-2, to=3-3]
        \arrow[no head, from=3-3, to=4-2]
        \arrow[no head, from=2-4, to=3-3]
\end{tikzcd}
\]

\begin{theorem}[Correspondence theorem]
Let $f\colon G\to H$ be a group homomorphism and $K=\ker f$. There exists
a bijective correspondence 
\[\begin{tikzcd}
        {\mathcal{A}=\{L:K\leq L\leq G\}} & {\{Y:Y\leq f(G)\}=\mathcal{B}}
        \arrow["\sigma", shift left=1, from=1-1, to=1-2]
        \arrow["\tau", shift left=1, from=1-2, to=1-1]
\end{tikzcd}
\]
The correspondence is given by $\sigma(L)=f(L)$ and $\tau(Y)=f^{-1}(Y)$. Moreover, the following statements hold: 
\begin{enumerate}
\item $L_1\leq L_2$ if and only if $\sigma(L_1)\leq \sigma(L_2)$.
\item $L\unlhd G$ if and only if $\sigma(L)\unlhd f(G)$.
\end{enumerate}
\end{theorem}

\begin{proof}
    Note that $\sigma$ and $\tau$ are well-defined, as 
    $f(L)\leq f(G)$ and $K\leq f^{-1}(Y)\leq G$.

    Let us prove that $\tau\circ\sigma=\id_\mathcal{A}$. We need to show that $\tau(\sigma(L))=L$ for all $L\in\mathcal{A}$. If $x\in f^{-1}(f(L))$, then $f(x)\in f(L)$. Thus $f(x)=f(l)$ for some
    $l\in L$. Hence $xl^{-1}\in K$ and therefore 
    $x\in Kl\subseteq L$, as $K\subseteq L$.
    Conversely, if $l\in L$, then $f(l)\in f(L)$. Thus $l\in f^{-1}(f(L))$.

    We now prove that $\sigma\circ\tau=\id_\mathcal{B}$. If  $Y\in\mathcal{B}$, then $\sigma(\tau(Y))=Y$. If $y\in Y\subseteq f(G)$, then $y=f(x)$ for some $x\in G$, that is $x\in f^{-1}(y)$. This implies that $y=f(x)\in f(f^{-1}(Y))$. Conversely, if $y\in f(f^{-1}(Y))$, then $y=f(x)$ for some $x\in f^{-1}(Y)$. This implies that $y=f(x)\in Y$.

    It is an exercise to show that 
    $X\leq Y$ if and only if $f(X)\leq f(Y)$.

    We now show that $L\unlhd G$ if and only if 
    $f(L)\unlhd f(G)$. If $L\unlhd G$ and $x\in G$,
    then $xLx^{-1}=L$. This implies that $f(L)=f(xLx^{-1})=f(x)f(L)f(x)^{-1}$, that is
    to say that $f(L)$ is normal in $f(G)$. Conversely, if
    $f(L)\unlhd f(G)$ and $x\in G$, then 
        \[
        f(xLx^{-1})=f(x)f(L)f(x)^{-1}=f(L).
        \]
    This implies that $xLx^{-1}\subseteq LK\subseteq L$. Thus 
    $xLx^{-1}\subseteq L$, which means that $L$ is normal in~$G$. 
\end{proof}

\begin{proposition}
    If $f\colon G\to f(G)$ is a surjective group homomorphism 
    and $H\leq G$ is such that $K=\ker f
\subseteq H$, then
        $(G:H)=(f(G):f(H))$.
\end{proposition}

\begin{proof}
% By the correspondence theorem, 
% there is a bijective correspondence 
% \[
% \begin{tikzcd}
%         {\{L:K\leq L\leq G\}} & {\{Y:Y\leq f(G)\}}
%         \arrow[shift left=1, from=1-1, to=1-2]
%         \arrow[shift left=1, from=1-2, to=1-1]
% \end{tikzcd}
% \]
% given by $H\mapsto f(H)$ and 
% $f^{-1}(T)\mapsfrom T$. 
Let $H\leq G$ be such that
$\ker f\subseteq H$ and 
\[
\alpha\colon G/H\to f(G)/f(H),\quad \alpha(gH)=f(g)f(H).
\]
It is an exercise to show that $\alpha$ is well-defined. 
We need to show that $\alpha$ is bijective, as then 
\[
(G:H)=|G/H|=|f(G)/f(H)|=(f(G):f(H)).
\]

First, we show that $\alpha$ is surjective. If $yf(H)\in f(G)/f(H)$, 
then
$y=f(g)$ for some $g\in G$ (because $f$ is surjective).Thus
\[
yf(H)=f(g)f(H)=f(gH)=\alpha(gH).
\]

We now show that $\alpha$ is injective. If $\alpha(gH)=\alpha(g_1H)$, then, 
\[
f(g)^{-1}f(g_1)=f(h)\in f(H)
\]
for some $h\in H$, that is 
\[
f(g_1)=f(g)f(h)=f(gh)
\]
for some $h\in H$. 
This implies that $g_1=ghk$ for some $k\in\ker f\subset
eq H$ and hence 
$g_1=gh_1$ for some $h_1\in H$, that is $g_1H=gH$.
\end{proof}

In the case of the canonical homomorphism $\pi\colon
 G\to G/N$, the previous result reads as follows. 
If $N$ is a normal subgroup of $G$, then
$K\mapsto K/N$ is a bijection between the set of 
(normal) subgroups of $G$ containing $N$ and the 
set of (normal) subgroups of $G/N$.
If $H$ is a subgroup of $G$, then
\[
\pi(H)=HN/N.
\]
\begin{example}
\index{Quaternion group}
Let us show that every subgroup of 
\[
Q_8=\{1,-1,i,-i,j,-j,k,-k\}
\]
is normal $Q_8$. Let $N=\{-1,1\}$. Then $N$ is normal in $Q_8$, as $N\subseteq Z(Q_8)$. Since $|Q_8/N|=4$, 
$Q_8/N$ is an abelian group. 

We claim that $N$ is included in every non-trivial 
subgroup of $Q_8$. If 
$K$ is a non-trivial subgroup of $Q_8$, then $-1\in K$ (because, for example, if $-i\in K$, then
 $-1=(-i)^2\in K$).
Then every subgroup of $Q_8$ corresponds to a subgroup of $Q_8/N$. 
Since $Q_8/N$ is abelian, every subgroup of $Q_8/N$ is normal. Thus 
if $S\leq Q_8$, then 
$\pi(S)$ is normal in $Q_8/N$.
Since $N\subseteq S$, it follows that
$S=\pi^{-1}(\pi(S))$. Hence $S$ is normal in $Q_8$.
\end{example}

% En el ejemplo anterior, podríamos haber demostrado que
% $G/N\simeq\Z/2\times\Z/2$, ya que como sabemos que $|G/N|=4$, hubiera alcanzado con calcular el orden
%  de cada uno de los elementos de $G/N$.

\begin{example}
Let $f\colon\Z/12\to\Z/6$ be the group homomorphism given by 
$1\mapsto 1$. Then $K=\ker f=\{0,6\}$.
The subgroups of $\Z/12$ containing $K$ are
\[
\langle 1\rangle=\{0,1,\dots,11\},
\quad
\langle 2\rangle=\{0,2,4,6,8,10\},
\quad
\langle 3\rangle=\{0,3,6,9\},
\quad
\langle 6\rangle=\{0,6\}.
\]
These subgroups correspond via $f$ to
the subgroups
\[
\langle 1\rangle=\{0,1,\dots,5\},
\quad
\langle 2\rangle=\{0,2,4\},
\quad
\langle 3\rangle=\{0,3\},
\quad
\{0\}
\]
of $\Z/6$, respectively. 
For example, 
\[
\begin{tikzcd}
        && \Z/12 \\
        & \langle 2\rangle && {\Z/6} \\
        \{0,6\} && \langle 2\rangle \\
        & {\{0\}}
        \arrow[no head, from=1-3, to=2-4]
        \arrow[no head, from=1-3, to=2-2]
        \arrow[no head, from=2-2, to=3-1]
        \arrow[no head, from=3-1, to=4-2]
        \arrow[no head, from=2-2, to=3-3]
        \arrow[no head, from=3-3, to=4-2]
        \arrow[no head, from=2-4, to=3-3]
\end{tikzcd}
\]
\end{example}

The correspondence theorem helps to transport  
properties from the image of a group homomorphism 
to the domain. Let us discuss a concrete example. 

\begin{example}
Let $G$ be a finite group and $N$ be a normal subgroup of
$G$ such that $N\simeq\Z/5$ and $G/N\simeq\Sym_4$. The following statements hold: 
\begin{enumerate}
\item $|G|=120$
\item $G$ contains a normal subgroup of order 20.
\item $G$ contains three subgroups of order 15, none of them normal in $G$.
\end{enumerate}

To prove the first claim we note that 
Lagrange's theorem implies that 
\[
24=|G/N|=\frac{|G|}{|N|}=|G|/5.
\]

We prove the second claim. Let $K$ be the subgroup of $G/N$ isomorphic
to the Klein group. 
Then 
$K$ is normal in $G/N$ and $|K|=4$. Since 
$(G/N:K)=6$,
the subgroup $K$ of $G/N$ corresponds to a normal subgroup $H$ of $G$ such that $(G:H)=6$. Using Lagrange's theorem
and the correspondence theorem, 
$|H|=20$, as 
\[
6=(G/N:K)=(G:H)=\frac{|G|}{|H|}.
\]

For the third claim, note that
$G/N\simeq\Sym_4$ has four subgroups of order 
 3 (these are the subgroups generated by a 3-cycle),
none normal in $G/N$. By the correspondence theorem, these
subgroups correspond with four non-normal subgroups of $G$, all of order 15. 
\end{example}

If $G$ is a group, $\Sym_G=\{f\colon G\to G:f\text{ bijective}\}$.

\begin{theorem}[Cayley]
\index{Cayley's theorem}
\label{thm:Cayley}
Every group $G$ is isomorphic to a subgroup of $\Sym_G$.
\end{theorem}

\begin{proof}
Let $f\colon G\to\Sym_G$, $g\mapsto L_g$, where $L_g\colon G\to G$, $L_g(x)=gx$. Then $f$ is a group homomorphism, as 
\[
L_{gh}(x)=(gh)x=g(hx)=L_g(hx)=L_gL_h(x)
\]
for all $g,h,x\in G$. Moreover, $f$ is injective (if $f(g)=f(h)$, then $L_g=L_h$, and this implies that 
$gx=L_g(x)=L_h(x)=hx$ for all $x\in G$, which ultimately implies $g=h$).
It follows that $G\simeq f(G)$, which is a subgroup of $\Sym_G$. 
\end{proof}

Every finite group is isomorphic to a subgroup of some 
$\Sym_n$. 
In particular, using permutation matrices, 
we see that every finite group 
isomorphic to a subgroup of $\GL_n(\Z)$ for some $n$.
Those groups are known as \emph{linear groups}.

\begin{proposition}
Every finite simple group $G$ is contained in some $\Alt_n$.
\end{proposition}

\begin{proof}
If $|G|=2$, the result is trivial, as 
\[
G\simeq\langle (12)(34)\rangle\subseteq\Alt_2.
\]
Assume 
that $|G|>2$.
Let $f\colon G\to\Sym_n$ by the injective group homomorphism
given by Cayley's theorem. If $H=f(G)$, then 
$G\simeq H$ by the first isomorphism theorem. 
We claim that $H\subseteq\Alt_n$. If
$H$ is not a subgroup of $\Alt_n$, there exists $h\in H$ such that  $h\not\in\Alt_n$. Write $h=f(g)$ for some 
$g\in G$. Since $h\not\in\Alt_n$,
$\sgn(f(g))=\sgn(h)=-1$, that is 
$g\not\in\ker(\sgn\circ f)$.
Let $K=\ker(\sgn\circ f)$. Then $K=\{1\}$, as $G$ is simple. Moreover, $\sgn\circ f$ is a bijective map, as
$\sgn(f(1))=1$ and $\sgn(f(g))=-1$. Therefore 
$G\simeq G/K\simeq\Z/2$, by the first isomorphism theorem.  
In particular, $|G|=2$, a contradiction. Thus $H\subseteq\Alt_n$.
\end{proof}

Let us briefly discuss another application of Cayley's theorem. 
We use the theorem to show that in a group, 
no product needs parenthesis 
By Cayley's theorem, a group $G$ is (isomorphic to) a 
subgroup of $\Sym_G$. The composition of maps is an associative operation. Moreover, no composition of finitely many maps 
needs parenthesis. Thus 
\[
(f_1\circ\cdots\circ f_n)(g)=f_1(f_2(\cdots f_n(g))\cdots).
\]

\section{Lecture: 18/05/2024}

\subsection{Semi-direct products}

We first start with an exercise of \emph{direct products}. Let $G$ be a group and $H$ and $K$ be normal subgroups with trivial intersection, that is $H\cap K=\{1\}$. 
Then $hk=kh$ for all $h\in H$ and $k\in K$. In fact, 
\[
[h,k]=hkh^{-1}k^{-1}\in H\cap K=\{1\}, 
\]
since $hkh^{-1}\in K$ because $K$ is normal in $G$ and 
$kh^{-1}k^{-1}\in H$ because $H$ is normal in $G$. 

\begin{exercise}
\index{Direct product of groups}
\label{xca:direct_product}
Let $G$ be a group and $H$ and $K$ be normal subgroups of $G$.
If $G=HK$ and $H\cap K=\{1\}$, then $G\simeq H\times K$.
\end{exercise}

\begin{exercise}
\label{xca:direct_product:quotient}
Let $A$ be a normal subgroup of $H$, and $B$ be a normal subgroup of $K$. 
Prove that $A\times B$ is a normal subgroup of
$H\times K$ and 
\[
\frac{H\times K}{A\times B}\simeq(H/A)\times(K/B).
\]
\end{exercise}

\index{Exact factorization of groups}
We say that a group $G$ admits an exact factorization through 
the subgroups $H$ and $K$ if $G=HK$ and 
$H\cap K=\{1\}$. By Exercise \ref{xca:direct_product}, 
if $G$ admits an exact factorization through two normal subgroups, then 
it is isomorphic to the direct product of these subgroups. 

\begin{exercise}
\label{xca:uniqueness}
    Let $G$ be a group that admits an exact factorization through
    the subgroups $H$ and $K$. Prove that every $g\in G$ can be written 
    uniquely as $g=hk$ for some $h\in H$ and $k\in K$. 
\end{exercise}

\begin{example}
Let $G=\Sym_3$, $H=\langle (123)\rangle\unlhd G$ and $K=\langle (12)\rangle$. Since
$K$ is not normal in $G$, we cannot apply Exercise \ref{xca:direct_product}. 
We do have $G=HK$ and $H\cap K=\{\id\}$, but 
\[
H\times K\simeq\Z/3\times\Z/2\not\simeq\Sym_3,
\]
as $\Z/3\times\Z/2$ is abelian and $\Sym_3$ is not. 
\end{example}

In the next example, we present a group of \emph{affine transformations}. 

\begin{example}
\label{exa:affine}
\index{Group!of affine transformations}
    Let 
    \[ 
    G=\{f\colon\R\to\R, f(x)=ax+b\text{ for some $a,b\in\R$ with $a\ne 0$}\}.
    \]
    Then $G$ is a group with the usual composition of functions. The identity map is an element of $G$. 
    If $f\in G$, say $f(x)=ax+b$, then the inverse is an element of $G$, as 
    $f^{-1}(x)=(1/a)x-b/a$. Finally, compositions of functions of $G$ are elements of $G$: 
    if $f(x)=ax+b$ and $g(x)=cx+d$, then 
    \[
    f(g(x))=f(cx+d)=a(xc+d)+b=(ac)x+(ad+b).
    \]
    
    Note that $K=\{f\in G:f(x)=x+b\text{ for some $b\in\R$}\}$ is a normal subgroup 
    of $G$ isomorphic to the additive group $\R$. In fact, let $f(x)=x+b$ be an element of $K$ 
    and $g(x)=cx+d$ be an element of $G$. Then $gfg^{-1}\in K$, as 
    \begin{align*}
    (gfg^{-1})(x)&=g(f((1/c)x-d/c))\\
    &=g((1/c)x-d/c+b)\\
    &=c( (1/c)x-d/c+b)+d\\
    &=x+(bc).
    \end{align*}
    
    Finally, 
    $Q=\{f\in G:f(x)=ax\text{ for some $a\ne0$}\}$ is a subgroup of $G$ isomorphic to the multiplicative group $\R^{\times}$. 
    Then $G=KQ$ and $K\cap Q=\{\id\}$.
\end{example}

\begin{definition}
\index{Complement}
Let $G$ be a group, $K$ a normal subgroup of $G$, and $Q$ a subgroup of $G$. We say
that $Q$ \emph{complements} $K$ in $G$ if $K\cap Q=\{1\}$ and $G=KQ$.
\end{definition}

\begin{example}
Let $G=\Sym_3$ and $K=\langle (123)\rangle\unlhd G$. The subgroups
$\langle (12)\rangle$, $\langle
(13)\rangle$ and $\langle (23)\rangle$ complement $K$ in $G$.
\end{example}

The previous example shows that complements are not unique. However, 
complements are unique under isomorphism, as 
\[
G/K= KQ/K\simeq Q/K\cap Q=Q/\{1\}\simeq Q.
\]

We now present a generalization of the (internal) direct product of
Exercise~\ref{xca:direct_product}. 

\begin{definition}
\index{Semi-direct product}
We say that a group $G$ is a \emph{semi-direct product} of $Q$ and $K$ if $K$ 
is normal in $G$ and 
$K$ admits a complement in $G$ isomorphic to $Q$. Notation: $G=K\rtimes Q$.
\end{definition}

The symbol $\rtimes$ is not symmetric, but points to remind us which subgroup is normal. 

Let $G=K\rtimes Q$. Then $G=KQ$ with $K$ normal in $G$. 
Let $g=ax$ and $g_1=a_1x_1$ with $a,a_1\in K$ and $x,x_1\in Q$. 
Then 
\[
gg_1=(ax)(a_1x_1)=(axa_1x^{-1})(xx_1)\in KQ,
\]
because $K$ is normal in $G$. 

\begin{theorem}
Let $K$ be a normal subgroup of $G$. The following statements are equivalent:
\begin{enumerate}
\item $K$ admits a complement in $G$.
\item There exists a subgroup $Q$ of $G$ such that each $g\in G$ can be written uniquely 
as $g=xy$ for some 
$x\in K$ and $y\in Q$.
\item There is a group homomorphism $s\colon G/K\to G$ such that $\pi\circ s=\id_{G/K}$, where $\pi\colon G\to G/K
$, $g\mapsto Kg$, is the canonical homomorphism.
\item There exists a group homomorphism $\rho\colon G\to G$ such that $\ker\rho=K$ and the restriction $\rho|_{\rho(G)}$ equals the identity. 
\end{enumerate}
\end{theorem}

\begin{proof}
We first prove that $(1)\implies(2)$. If $Q$ complements $K$, then $G=KQ$ and
$K\cap Q=\{1\}$. In particular, if $g\in G$, then $g=xy$ for some $x\in K$ and $y\in Q$. To show that
the decomposition is unique, suppose that 
$g=x_1y_1$ with $x_1\in K$ and $y_1\in Q$. Then $x_1^{-1}x=yy_1^{-1}\in K\cap Q=\{1\}$ 
and hence $x=x_1$ and $y=y_1$.

%It is an exercise to show that $(2)\implies(1)$.

We now prove that $(2)\implies(3)$. Let $s\colon G/K\to G$, $s(Kg)=y$ if 
$g=xy$ with $x\in K$ and $y\in Q$. (Note that here we use right cosets, as it is more convenient.)
Let us check that $s$ is well-defined. 
For that purpose, we must show that $Kg=Kg_1$ implies $s(Kg)=s(Kg_1)$. Let $g=xy$ 
and $g_1=x_1y_1$ with $x,x_1\in K$ and $y,y_1\in Q$, then, since $Kg=Kg_1$, 
$xyy_1^{-1}x_1^{-1}=gg_1^{-1}\in K$, that is $yy_1^{-1}\in x^{-1}Kx_1=K$
 because $x,x_1\in K$. Hence $yy_1^{-1}\in K\cap Q=\{1\}$ and thus $y=y_1$. 
 
 We now show that $\pi\circ
 s=\id_{G/K}$:
\[
(\pi\circ s)(Kg)=\pi(y)=Ky=Kxy=Kg.
\]

Finally, the map 
$s$ is a group homomorphism, as 
\[
s((Kg)(Kg_1))=s(Kgg_1)=s(Kx(yx_1y^{-1})yy_1)=yy_1=s(Kg)s(Kg_1),
\]
since $yx_1y^{-1}\in K$. 


We now prove that $(3)\implies(4)$. Let $\rho=s\circ\pi$. Then $\rho$ is a group homomorphism (because it is the composition of homomorphisms). 
To prove that $\rho|_{\rho(G)}=\id$, we need to show that $\rho(\rho(g))=\rho(g)$ for all $g\in G$. 
We compute: 
\[
\rho(\rho(g))=(s\circ(\pi\circ s)\circ\pi)(g)=(s\circ\pi)(g)=\rho(g).
\]
We now compute $\ker\rho$. If $g\in\ker\rho$, then $s(\pi(g))=\rho(g)=1$. Thus 
\[
\pi(g)=\pi(s(\pi(g)))=\pi(1)=1_{G/K},
\]
that is $g\in\ker\pi=K$. Conversely, if $x\in K$, then
\[
\rho(x)=\rho(s(Kx))=\rho(s(K))=\rho(1)=1
\]
and hence $x\in\ker\rho$.

Finally, we prove that $(4)\implies(1)$. We claim that $Q=\rho(G)$ complements $K$ in
 $G$. We first show that $K\cap Q=\{1\}$: if $x\in K\cap Q$, then $x=\rho(g)$ for some $g\in G$. Moreover, 
\[
1=\rho(x)=\rho(\rho(g))=\rho(g).
\]
Hence $g\in\ker\rho=K$ and $x=1$. We now prove that $G=KQ$. For the inclusion $G\subseteq KQ$, 
\[
g=(g\rho(g^{-1}))\rho(g)
\]
and $g\rho(g^{-1})\in K=\ker\rho$, as $\rho(g\rho(g^{-1}))=  \rho(g)\rho(g^{-1})=1$.
\end{proof}

\begin{example}
$\Sym_n=\Alt_n\rtimes\langle(12)\rangle$, as $Q=\langle (12)\rangle\simeq\Z/2$ complements 
$\Alt_n$ in $\Sym_n$.
\end{example}

For a group $G$, 
the set 
\[
\Aut(G)=\{f\colon G\to G:f\text{ bijective group homomorphism}\}
\]
is a group with the composition of maps. It is called 
the \emph{automorphism group} of $G$. 
Examples of automorphism groups are the identity map and 
conjugations homomorphisms. 

\begin{example}
$\Aut(\Z)\simeq\Z/2$, as $\Aut(\Z)=\{\id,-\id\}$.
\end{example}

\begin{example}
Let $G$ be a group and $g\in G$. The conjugation 
map $\gamma_g\colon G\to G$, $x\mapsto gxg^{-1}$,
is an automorphism of $G$, as 
\[
\gamma_g(xy)=g(xy)g^{-1}=(gxg^{-1})(gyg^{-1})=\gamma_g(x)\gamma_g(y).
\]
Moreover, $\gamma\colon G\to\Aut(G)$, $g\mapsto\gamma_g$, is a group
homomorphism:
\[
\gamma_{gh}(x)=(gh)x(gh)^{-1}=g(\gamma_h(x))g^{-1}=\gamma_g(\gamma_h(x))=(\gamma_g\circ\gamma_h)(x).
\]

\index{Inner automorphisms}
The group of \emph{inner automorphisms} of $G$ is the 
group 
$\Inn(G)=\gamma(G)$. Note that $\ker\gamma=Z(G)$, as
if $g\in G$ is such that $\gamma_g=\id$, then 
\[
\gamma_g(x)=gxg^{-1}=x
\]
for all $x\in G$. By the first isomorphism theorem, 
\[
G/Z(G)\simeq \gamma(G)=\Inn(G).
\]
\end{example}

\begin{exercise}
\label{xca:aut(S3)}
    Prove that $\Aut(\Sym_3)\simeq\Sym_3$. 
\end{exercise}

\begin{exercise}
\label{xca:inner_is_normal}
    Let $G$ be a group. Prove that $\Inn(G)$ is a normal subgroup of $\Aut(G)$. 
\end{exercise} 

For a group $G$, the quotient $\Aut(G)/\Inn(G)$ is called 
the group of \emph{outer automorphisms} of $G$. 
Note that 
\[
\Inn(G)\text{ is cyclic}\Longleftrightarrow
|\Inn(G)|=1\Longleftrightarrow
G\text{ is abelian.}
\]

\begin{exercise}
\label{xca:aut_cyclic}
Let $G$ be a group. 
Prove that if $\Aut(G)$ is cyclic, then 
$G$ is abelian. 
\end{exercise}

\begin{exercise}
\label{xca:aut_finite}
    Prove that if $G$ is finite, then $\Aut(G)$ is finite.
\end{exercise}

% \begin{exercise}
% \label{xca:autgeq2}
%         Si $G$ es un grupo tal que $|G|\geq3$, entonces $|\Aut(G)|\geq2$.
% \end{exercise}


% \begin{exercise}
% \label{xca:aut_impar}
%         No existe un grupo finito cuyo grupo de automorfismos sea no trivial cíclico y de orden impar.
% \end{exercise}

% \begin{exercise}
% \label{xca:p2dividesAut}
%         Sea $p$ un número primo.
%         Si $G$ es un $p$-grupo no abeliano, entonces $p^2$ divide a $|\Aut(G)|$.
% \end{exercise}

% \begin{exercise}
%         \label{xca:Aut(H)}
%         Si $G=H\times K$, entonces $\Aut(H)$ es isomorfo a un subgrupo de $\Aut(G)$.
% \end{exercise}

% \begin{exercise}
% \label{xca:aut_trivial_center}
%         Si $G$ tiene centro trivial, entonces $\Aut(G)$ también.
% \end{exercise}

\index{Action!on groups}
\index{Action!by automorphisms}
Let us discuss how groups acts on groups. An action of a group on a group is 
a group homomorphism 
$\theta\colon Q\to\Aut(K)$, $x\mapsto\theta_x$. 
This is nothing but a way in which $Q$ permutes the elements of $K$ 
in a way that is compatible with both group structures. Typically, in this setting, 
one says that $Q$ acts on $K$ by automorphisms. 

What does it mean that $\theta$ is a group homomorphism? 
For $x\in Q$, write $\theta_x\colon K\to K$, $a\mapsto x\cdot a$. 
Then $\theta$ is a well-defined group homomorphism if and only if 
the following properties hold: 
\begin{enumerate}
    \item $1\cdot a=a$ for all $a\in K$. 
    \item $x\cdot (y\cdot a)=(xy)\cdot a$ for all $x,y\in Q$ and $a\in K$.
    \item $x\cdot 1=1$ for all $x\in Q$. 
    \item $x\cdot (ab)=(x\cdot a)(x\cdot b)$ for all $x\in Q$ and $a,b\in K$. 
\end{enumerate}

\begin{example}
\label{xca:GL2onR2}
    The group $\GL_2(\R)$ acts 
    on the additive group $\R^{2\times1}$ 
    by automorphisms via the formula 
    \begin{equation}
    \label{eq:action}
    \begin{pmatrix}
        a&b\\
        c&d
        \end{pmatrix}\cdot\colvec{2}{x}{y}=\colvec{2}{ax+by}{cx+dy}.
    \end{equation}
    
    Before doing calculations, note that 
    this formula 
    is nothing but the usual 
    left multiplication of matrices 
    by vectors! In fact
    $\alpha\in\GL_2(\R)$ and $v\in\R^{2\times1}$, 
    the action of \eqref{eq:action} is 
    just 
    \[
    \alpha\cdot v=\alpha v.
    \]
    
    To show that we have an action by automorphisms, there are four properties to verify. 
    First, it is trivial to check the first property, as 
    \[
    \begin{pmatrix}1&0\\0&1\end{pmatrix}\colvec{2}{x}{y}=\colvec{2}{x}{y}.
    \]
    For the second property, a direct calculation shows that 
    \[
    \alpha\cdot (\beta\cdot v)=(\alpha\beta)\cdot v
    \]
    for all $\alpha,\beta\in\GL_2(\R)$ and $v\in\R^{2\times1}$. 
    The third property is trivial:
    \[
    \begin{pmatrix}
        a&b\\
        c&d
    \end{pmatrix}\colvec{2}{0}{0}=\colvec{2}{0}{0}.
    \]
    Finally, the fourth property is just the left distributivity:
    \[
    \alpha\cdot (v+w)=x(v+w)=\alpha v+\alpha w=\alpha\cdot v+\alpha\cdot w
    \]
    for all $\alpha\in\GL_2(\R)$ and $v,w\in\R^{2\times1}$. 
\end{example}

%The following exercise constructs semi-direct products. 

\begin{exercise}
\label{xca:semi-direct}
Let $K$ and $Q$ be groups and $\theta\colon Q\to\Aut(K)$, $x\mapsto\theta_x$, a group
homomorphism. Prove that $K\times Q$ 
with 
\[
(a,x)(b,y)=(a\theta_x(b),xy)
\]
is a group. This group will be written as $K\rtimes_\theta Q$.
\end{exercise}

% \begin{proof}[Bosquejo de la demostración]
% Dejamos como ejercicio verificar que la operación es asociativa. Hay que verificar además que el elem
% ento neutro de $K\rtimes_\theta Q$ será $(1,1)$ y
% que el inverso de $(a,x)\in K\rtimes_\theta Q$ será $(\theta_{x^-1}(a^{-1}),x^{-1})$.
% \end{proof}

The group of Exercise \ref{xca:semi-direct} is the
semi-direct product of the subgroups 
\begin{align*}
K\times\{1\}=\{(a,1):a\in K\}\simeq K,&&
\{1\}\times Q=\{(1,x):x\in Q\}\simeq Q
\end{align*}
of $K\rtimes_\theta Q$. Note that $K\times\{1\}$ is normal in $K\rtimes_\theta Q$. 
We can identity $K\rtimes\{1\}$ with $K$ 
and $\{1\}\rtimes Q$ with $Q$. This means 
that for $a\in K$ and $x\in Q$, 
instead of writing $(a,1)$ one simply writes $a$ and 
$(1,x)$ is replaced by $x$. Moreover,  
\begin{gather*}
ax=(a,1)(1,x)=(a,x)
\shortintertext{and}
xa=(1,x)(a,1)=(\theta_x(a),x)=(\theta_x(a),1)(1,x)=\theta_x(a)x.
\end{gather*}
Thus we can write 
\[
\theta_x(a)=xax^{-1}
\]
for all $x\in Q$ and $a\in K$, that is 
\[
\theta_x(a)=(\theta_x(a),1)=(1,x)(a,1)(1,x)^{-1}=xax^{-1}.
\]

\begin{exercise}
Prove that if $G$ is a semi-direct product of the normal subgroup 
$K$ with the subgroup $Q$, there exists a group homomorphism
$\theta\colon Q\to\Aut(K)$
such that $G\simeq K\rtimes_\theta Q$.
\end{exercise}

% \begin{proof}[Bosquejo de la demostración]
% Para $x\in Q$ sea $\theta_x\colon K\to K$, $\theta_x(a)=xax^{-1}$. Ya vimos que $\theta_x\in\Aut(K)$
% y que $Q\to\Aut(K)$, $x\mapsto\theta_x$ es un morfismo de grupos. Queda verificar que
% la función $K\rtimes_\theta Q\to G$, $(a,x)\mapsto ax$, es un morfismo biyectivo de grupos.
% \end{proof}

Let us discuss some examples. 

\begin{example}
    The multiplicative group $Q=\R^{\times}$ acts on the 
    additive group $K=\R$ by multiplication: If $x\in Q$ and $a\in K$, 
    then $x\cdot a=xa$. This is an action by automorphisms,~as 
    \[
        1\cdot a=a,\quad 
        x\cdot (y\cdot a)=x(ya)=(xy)a=(xy)\cdot a,\quad 
        x\cdot 0=0,\quad 
        x\cdot (a+b)=xa+xb
    \]
    for all $a,b\in\R$ and $x,y\in\R^{\times}$. 
    
    Hence one can construct the semi-direct product $K\rtimes Q$. The operation of 
    this group is 
    \[ 
        (d,c)(b,a)=(cb+d,ca). 
    \]
    This group is isomorphic to the group of affine transformations of
    Example~\ref{exa:affine}. In fact, one can easily show that 
    \[ 
        K\rtimes Q\to \{f\colon\R\to\R:f(x)=ax+b\text{ for some $a\in\R^{\times}$ and $b\in\R$}\},
        \quad 
        (b,a)\mapsto f(x)=ax+b,
    \]
    is a bijective group homomorphism. 
\end{example}



\begin{example}
Let $N\simeq \Z/n$ and $H\simeq\Z/2=\{0,1\}$. The map $\theta\colon H\to\Aut(N)$, 
$1\mapsto (x\mapsto x^{-1})$, is a group homomorphism. Let $G=N\rtimes_\theta H$.
Then $G\simeq\D_n$, the dihedral group of order $2n$.

Recall that 
\[
\D_n=\langle r,s:r^n=s^2=1, srs^{-1}=r^{-1}\rangle.
\]
Assume that $N=\langle x\rangle$ and $H=\langle y\rangle$. Then $|(x,1)|=n$ and $|(1,y)|=2$. 
Moreover,
\begin{align*}
(1,y)(x,1)(1,y)^{-1} &= (\theta_y(x),y)(1,y)\\
&=(\theta_y(x),y^2)\\
&=(\theta_y(x),1)\\
&=(x^{-1},1)\\
&=(x,1)^{-1}.
\end{align*}
If $u=(x,1)$ and $v=(1,y)$, then $u^n=v^2=(1,1)$ and $vuv^{-1}=u^{-1}$. Thus there exists 
a surjective group homomorphism 
$\D_n\to G$ (because $G$ is generated by $u$ and $v$). Moreover, $|G|=|N||H|=2n$. Hence 
$G$ has order $2n$ and therefore $G\simeq\D_n$.
\end{example}

\begin{example}
Let $K=\{\id,(12)(34),(13)(24),(14)(23)\}$. Then $K$ is normal in $\Alt_4$. 
Let $H=\langle (123)\rangle\simeq\Z/3$. Since $K\cap H$ is a subgroup of $H$ and $K$ and, moreover, 
$K$ and $H$ have coprime orders, $H\cap K=\{\id\}$. Hence $\Alt_4=K\rtimes H$.
\end{example}

\begin{example}
Let 
\[
K=\{\id,(12)(34),(13)(24),(14)(23)\},\quad  
H=\{\sigma\in\Sym_4:\sigma(4)=4\}.
\]
Note that $H$ is a subgroup of 
$\Sym_4$ isomorphic to $\Sym_3$. Then $H\cap K=\{\id\}$ and hence 
$\Sym_4=K\rtimes H$.
\end{example}

Let $n\geq5$. 
Using the fact that $\Alt_n$ is a simple group,  
one proves that $\Alt_n$ cannot be written as a semi-direct product of proper subgroups. 

\begin{example}
Let $K=\Z/3$ and $Q=\Z/4$. Since $\Hom(Q,\Aut(K))=\{1,\tau\}$, where 
\[
\tau\colon\Z/4\to\Aut(\Z/3)=\{\id,\rho\}\simeq\Z/2,\quad 1\mapsto\rho,
\]
the semi-direct product $T=K\rtimes_\tau Q$ is a non-abelian group of order 12. Moreover,
$T\not\simeq\Alt
_4$ as $|(2,2)|=6$ and $\Alt_4$ has no elements of order six.
\end{example}


\begin{exercise}
    \label{xca:affine_semidirect}
    Prove that the group of Exercise~\ref{xca:affine} is a semi-direct product. 
\end{exercise}

\section{25/04/2024}

% todo: El grupo Aff(R) es un producto semidirecto
% todo: Necesitamos más ejemplos de producto semidirecto
\subsection{Actions}

\begin{definition}
\index{Action}
    Let $G$ be a group and $X$ a set. 
    A (left) \textbf{action} of $G$ on $X$ is a map 
    $G\times X\to X$, $(g,x)\mapsto g\cdot x$, such that 
    \begin{enumerate}
        \item $1\cdot x=x$ for all $x\in X$, and 
        \item $g\cdot (h\cdot x)=(gh)\cdot x$ for all $g,h\in G$ and $x\in X$.
    \end{enumerate}
\end{definition}

If a group $G$ acts on a set $X$, we also say that
$X$ is a $G$-set.

\begin{example}
\index{Action!trivial}
Every group $G$ acts on $G$ trivially: $g\cdot h=h$ for all $g,h\in G$.
\end{example}

\begin{example}
\index{Action!left multiplication}
Every group $G$ acts on $G$ by left multiplication, that is
$g\cdot h=gh$ for all $g,h\in G$.
\end{example}

\begin{example}
\index{Action!conjugation}
Every group $G$ acts on $G$ by conjugation, that is
$g\cdot h=ghg^{-1}$ for all $g,h\in G$. More generally, 
if $N$ is a normal subgroup of $G$, then $G$ acts on
$N$ by conjugation: $g\cdot x=gx
g^{-1}$ for all $g\in G$ and $x\in N$. 
\end{example}

\begin{example}
\index{Action!on left cosets}
Let $G$ be a group $H$ be a subgroup of $G$. Then $G$ 
acts on the set of left cosets $G/H$ by left multiplication, that 
is $g\cdot (xH)=(gx)H$ for all $g,x\in G$.
\end{example}

There is a bijective correspondence between 
left actions of a group $G$ on a set $X$ and
group homomorphisms 
$\rho\colon G\to\Sym_X$. The correspondence is given by
the formula 
\[
\rho(g)(x)=g\cdot x,\quad g\in G,x\in X.
\]
We will write $\rho_g=\rho(g)$.

As an example, if $G\times X\to X$, $(g,x)\mapsto x$, is 
an action of $G$ on $X$, then
each $\rho_g\colon X\to X$ is a bijective map with inverse 
$(\rho_g)^{-1}=\rho_{g^{-1}}$. Moreover, 
 $\rho$ is a group homomorphism, as 
\[
\rho(gh)(x)=(gh)\cdot x=g\cdot (h\cdot x)=\rho_g(h\cdot x)=\rho_g(\rho_h(x))
\]
for all $g,h\in G$ and $x\in X$.

\begin{example}
Let $G=\Sym_3$ and 
\[
H=\langle (123)\rangle=\{\id,(123),(132)\}.
\]
Let $G$ act on the set $X=G/H=\{H,(12)H\}$ of left cosets of $H$ 
by left multiplication. Write 
$x_1=H$ and $x_2=(12)H$. Then
\begin{align*}
&(12)\cdot x_1=x_2,
&&(12)\cdot x_2=x_1,
&&(123)\cdot x_1=x_1,
&&(123)\cdot x_2=x_2.
\end{align*}
Since $G=\langle (12),(123)\rangle$, one has the group 
homomorphism 
$\rho\colon G\to\Sym_{X}\simeq\Sym_2$ given by
$(12)\mapsto (12)$, $(123)\mapsto\id$.
\end{example}

\begin{example}
As before, let $G=\Sym_3$ and $H=\langle (12)\rangle=\{\id,(12)\}$. Let $G$ act on the set $X=G/H=\{H,(123
)H,(132)H\}$ of left cosets of $H$ by left multiplication. Write 
$x_1=H$, $x_2=(123)H$ and $x_3=(132)H$. Then
\begin{align*}
(12)\cdot x_1=x_1,&& (12)\cdot x_2=x_3, && (12)\cdot x_3=x_2,\\
(123)\cdot x_1=x_2, && (123)\cdot x_2=x_3, &&(123)\cdot x_3=x_1.
\end{align*}
Since $G=\langle (12),(123)\rangle$, one has the group
homomorphism 
$\rho\colon G\to\Sym_{X}\simeq\Sym_3$ given by $(12)\mapsto (23)$, $(123)\mapsto (123)$.
\end{example}


\begin{example}
Let $G=Q_8=\{1,-1,i,-i,j,-j,k,-k\}$ and $N=\{1,-1,i,-i\}$. Since
$N$ is normal in $G$, $G$
acts by conjugation on $X=N$.
If $x_1=1$, $x_2=-1$, $x_3=i$ and $x_4=-i$, then
$i\cdot x=x$ for all $x\in N$. Moreover, 
\begin{align*}
j\cdot x_1=x_1, && j\cdot x_2=x_2, && j\cdot x_3=x_4, && j\cdot x_4=x_3.
\end{align*}
Since $G=\langle i,j\rangle$, a group homomorphism $\rho\colon G\to\Sym_X\simeq\Sym_4$ is determined by 
$\rho_i=\id$ and $\rho_j=(34)$.
\end{example}

The following example is important. 

\begin{example}
    The group $\Sym_n$ acts on $\R^n$ by
    \[
    \sigma\cdot (x_1,\dots,x_n)=(x_{\sigma^{-1}(1)},\dots,x_{\sigma^{-1}(n)}).
    \]
    It is very important to use $\sigma^{-1}$ and not 
    $\sigma$, as we need to permute the elements of the standard basis of $\R^3$.

    As a concrete example, let us see that the operation 
    \[
    \sigma\cdot (x_1,x_2,x_3)=(x_{\sigma(1)},x_{\sigma(2)},x_{\sigma(3)})
    \]
    is not an action of $\Sym_3$ on $\R^3$.
    If $\sigma=(12)$ and $\tau=(23)$, then $\sigma\tau=(123)$. Since 
    \begin{align*}
    &(123)\cdot (5,6,7)=(6,7,5),\\
    &(12)\cdot ((23)\cdot (5,6,7))=(1,2)\cdot (5,7,6)=(7,5,6),
    \end{align*}
    this does not define an action. If we compute 
    \begin{align*}
        \sigma\cdot (\tau\cdot (x_1,\dots,x_n))
        =\sigma\cdot (x_{\tau(1)},\dots,x_{\tau(n)})
    \end{align*}
    and for each $i\in\{1,\dots,n\}$ we set $y_i=x_{\tau(i)}$, then 
    \[
    \sigma\cdot (\tau\cdot (x_1,\dots,x_n))=\sigma\cdot (y_1,\dots,y_n)=(y_{\sigma(1)},\dots,y_{\sigma(n)})
    =(x_{\tau\sigma(1)},\dots,x_{\tau\sigma(n)}),
    \]
    even if $\sigma$ and $\tau$ do not commute.

    Now we show that by using inverses, we do have an action. 
    For $j\in\{1,\dots,n\}$, let $y_j=x_{\tau(j)}$,
    that is 
    \[
    (y_1,y_2,\dots,y_n)=\tau\cdot (x_1,x_2,\cdots,x_n)=(x_{\tau^{-1}(1)},x_{\tau^{-1}(2)},\dots,x_{\tau^{-1}(n)}).
    \]
    Then 
    \begin{align*}
        \sigma\cdot (\tau\cdot (x_1,x_2,\dots,x_n))&=\sigma\cdot (y_1,y_2,\dots,y_n)\\
        &=\left(y_{\sigma^{-1}(1)},y_{\sigma^{-1}(2)},\dots,y_{\sigma^{-1}(n)}\right)\\
        &=\left(x_{\tau^{-1}(\sigma^{-1}(1))},x_{\tau^{-1}(\sigma^{-1}(2))},\dots,x_{\tau^{-1}(\sigma
^{-1}(n))}\right)\\
        &=\left(x_{(\sigma\tau)^{-1}(1))},x_{(\sigma\tau)^{-1}(2))},\dots,x_{(\sigma\tau)^{-1}(n))}\right).
    \end{align*}
\end{example}

The following example is also important: 

\begin{example}
    The group $\Sym_n$ acts on the set of polynomials on
    $n$ variables $X_1,\dots,X_n$
    by permitting the variables. For example, for three variables, if 
    $\sigma=(123)$ and $f=X_2X_3-X_1+5X_2X_3^2X_1$, then 
    $\sigma\cdot f=X_2^2X_3-X_1+5X_2X_3^2X_1$.

    Restricting the action, we see that 
    $\Sym_n$ acts on the set 
    \[
    \{\lambda_1X_1+\cdots\lambda_nX_n:\lambda_1,\dots,\lambda_n\in\R\}.
    \]
    Then 
    \begin{align*}
    \sigma \cdot (\lambda_1X_1+\cdots+\lambda_nX_n) &= (\lambda_1X_{\sigma(1)}+\cdots+\lambda_nX_{\sigma(n)})
    =(\lambda_{\sigma(1)}X_1+\cdots+\lambda_{\sigma(n)}X_n).
    \end{align*}
\end{example}

It is relevant to compute the kernel of the action homomorphism. 

\begin{example}
Let $G$ be a group and $H$ be a subgroup of $G$. Then $G$ 
acts on $G/H$ by left multiplication, that is 
$g\cdot (xH)=(gx)H$ for all $g,x\in G$. Let $\rho\colon G\to\Sym_{G/H}$ be the group homomorphism induced by the action. 

We claim that $\ker\rho=\cap_{x\in G}xHx^{-1}$. 
We first prove $\supseteq$. If $g\in xHx^{-1}$ for all 
$x\in G$, then, for a fixed $x\in G$,
 \[
 \rho(g)(xH)=g\cdot (xH)=(gx)H=(xhx^{-1})xH=(xh)H=xH
 \]
because $g=xhx^{-1}$ for some $h\in H$. Thus $\rho(g)=\id$ 
and hence $g\in\ker\rho$. We now prove 
$\subseteq$. If $g\in\ker\rho$, then
 $\rho(g)=\id$. So for all $x\in G$,
 \begin{align*}
\rho(g)(xH)=xH
\Longleftrightarrow (gx)H=xH
\Longleftrightarrow x^{-1}gx\in H
\Longleftrightarrow g\in xHx^{-1}.
 \end{align*}
It is an exercise to show that
$\ker\rho$ is the largest normal subgroup of $G$ 
contained in~$H$.
\end{example}

With these results, we can provide a third 
solution to Exercise~\ref{xca:p_smallest} of 
page~\pageref{xca:p_smallest}.
We let  $G$ act on $G/H$ by left multiplication. 
The induced group homomorphism  $\rho\colon G\to\Sym_p$ has 
kernel 
\[
K=\ker\rho=\bigcap_{x\in G}xHx^{-1}\subseteq H.
\]
By the first isomorphism theorem, 
$G/K\simeq\rho(G)\lesssim\Sym_p$ (this means that 
$\rho(G)$ is isomorphic to a subgroup of $\Sym_p$). 
Thus $|G/K|$ divides $p!$.
Let $m=(H:K)$. By Lagrange's theorem,
\[
(G:K)=(G:H)(H:K)=pm
\]
and hence $pm$ divides $p!$. This implies that $m$ divides $(p-1)!$. If $q$ a prime number dividing 
 $m$, then $q\geq p$, by the minimality of $p$. Moreover, 
 every prime factor of $(p-1)!$ is 
 $<p$. Hence $m=1$ and therefore $H=K$.

\begin{exercise}
    Let a group $G$ act on a set $X$. 
    On $X$, we define the following relation: $x\sim y$ if and only if
    there exists $g\in G$ such that $g\cdot x=y$. Prove 
    that this is an equivalence relation on $X$. 
\end{exercise}

\begin{definition}
\index{Orbit}
Let $G$ be a group acting on a set $X$. If $x\in X$, the
orbit of $x$ is the set
\[
G\cdot x=\{g\cdot x:g\in G\}.
\]
\end{definition}

The orbits of the action of $G$ on $X$ are 
the equivalence classes of the equivalence 
relation induced by the action. In particular, 
every two orbits will be either disjoint or equal. Moreover, 
$X$ decomposes as a disjoint union of orbits. 

\begin{definition}
        \index{Stabilizer of a point}
        Let $G$ be a group that acts on $X$. If $x\in X$, the \textbf{stabilizer} of $x$ in $G$
        is the set   
        \[
        G_x=\{g\in G:g\cdot x=x\}.
        \]
\end{definition}

The reader must prove that the stabilizer is a subgroup. 

\begin{definition}
\index{Action!transitive}
We say that an action of a group $G$ on a set $X$
is \textbf{transitive} if for any $x,y\in X$ there exists $g\in G$ such that $g\cdot x=y$.
\end{definition}

\begin{example}
    Let $G$ be a group and $H$ a subgroup of $G$. Let $G$ act
    on $G/H$ by left multiplication. The action is transitive: if 
    $xH,yH\in G/H$, there exists $g\in G$ such that
    $(gx)H=yH$ (take for example $g=yx^{-1}$). 
\end{example}

\begin{example}
Por evaluación,
el grupo simétrico $\Sym_n$ actúa transitivamente en el conjunto $\{1,\dots,n\}$.
\end{example}

In the definition of a transitive action, there is no assumption
on the number of elements $g$ such that $g\cdot x=y$.

\begin{definition}
\index{Action!faithful}
We say that an action of a group $G$ on a set 
$X$ is \textbf{faithful} if 
\[
\{g\in G:g\cdot x=x\text{ for all $x\in X$}\}=\{1\}.
\]
\end{definition}

The definition is equivalent to the injectivity of 
the group homomorphism induced
by the action. 

\begin{theorem}[Fundamental counting principle]
\index{Theorem!fundamental counting principle}
\label{thm:fundamental}
Let $G$ be a finite group acting on a finite set $X$. If 
$x\in X$, then $|G\cdot x|=(G:G_x)$.
\end{theorem}

\begin{proof}
    Let $\varphi\colon G/G_x\to G\cdot x$, $gG_x\mapsto g\cdot x$. Then $\varphi$ is well-defined, as 
        \[
        gG_x=hG_x\implies h^{-1}g\in G_x
        \implies h^{-1}g\cdot x=x\implies g\cdot x=h\cdot x.
        \]
    Moreover, $\varphi$ is injective: 
        \[
        \varphi(gG_x)=\varphi(hG_x)\implies
        g\cdot x=h\cdot x\implies
        h^{-1}g\in G_x\implies gG_x=hG_x.
        \]
    Finally, $\varphi$ is surjective. Hence 
    $|G/G_x|=|G\cdot x|$.
\end{proof}

Theorem \ref{thm:fundamental} is also known 
as the orbit--stabilizer theorem. 

If $G$ is a group and $X$ and $Y$ are $G$-sets, 
we say that a map $\varphi\colon X\to Y$ is a 
\textbf{homomorphism} of $G$-sets if $\varphi(g\cdot x)=g\cdot \varphi(x)$ for all $g\in G$ and $x\in X$. The bijection 
$\varphi$ constructed in the proof of Theorem \ref{thm:fundamental}
is a homomorphism of $G$-sets, where
$G$ acts on $G/G_x$ by left multiplication: 
\[
\varphi(g\cdot hG_x)=\varphi((gh)G_x)=(gh)\cdot x=g\cdot (h\cdot x)=g\cdot\varphi(hG_x).
\]
Thus $G\cdot x\simeq G/G_x$ as $G$-sets.

\begin{example}
    If$G$ acts on $G$ by conjugation, that is $g\cdot x=gxg^{-1}$, the orbits of this action are called the \textbf{conjugacy classes} 
    of $G$. They are sets of the form
        \[
        G\cdot x=\{gxg^{-1}:g\in G\}.
        \]
    In particular, $G$ decomposes as a disjoint union of conjugacy classes. Moreover, 
    the stabilizers are the centralizers:
        \[
        G_x=\{g\in G:g\cdot x=x\}=\{g\in G:gxg^{-1}=x\}=C_G(x).
        \]
    In particular, $|G\cdot x|=(G:C_G(x))$.
\end{example}

\begin{example}
    Let $H$ be a subgroup of $G$ and $X$ the set of subsets of $G$. Let $G$ act on 
    $X$ by conjugation, that is $S\in X$. Then
        $g\cdot S=gSg^{-1}$. The orbit of $H$ 
        is 
        \[
        G\cdot H=\{g\cdot H:g\in G\}=\{gHg^{-1}:g\in G\},
        \]
        the set of conjugates of $H$. The stabilizer of $H$ in $G$ 
        is 
        \[
        G_H=\{g\in G:g\cdot H=H\}=\{g\in G:gHg^{-1}=H\}=N_G(H),
        \]
        the normalizer of $H$ in $G$. It follows that
        $H$ has exactly $(G:N_G(H))$ conjugates in $G$. In particular,
        if $G$ is finite, 
        the number of conjugates of $H$ divides $|G|$. 
\end{example}

As an application, we provide an alternative proof
of Theorem~\ref{thm:|HK|}. 

\begin{example}
\label{exa:for_HK}
Let $G$ be a group and $H$ and $K$ be subgroups of $G$. 
The group $L=H\times K$ acts on $X=HK$ by 
\[
(h,k)\cdot x=hxk^{-1},\quad x\in X,\,h\in H,\,k\in K.
\]
Note that $1\in HK$ and the action of $L$ on $X$ is transitive, as 
$(h,k^{-1})\cdot 1 = hk$. Since 
\[
L_1=\{(h,k)\in H\times K: (h,k)\cdot 1=1\}=\{(h,k)\in H\times K:h=k\},
\]
it follows that $|L_1|=|H\cap K|$ because there exists a bijection
between $L_1$ and 
$H\cap K$. By the fundamental counting principle, 
\[
|HK|=(L:L_1)=\frac{|H\times K|}{|H\cap K|}=\frac{|H||K|}{|H\cap K|}.
\]
\end{example}

The idea used in Example \ref{exa:for_HK} can be generalized. 

\begin{example}
\index{Double coset}
Let $G$ be a group and $H$ and $K$ be subgroups of $G$. Let the group $L=H\times K$ act on $G$ by
\[
(h,k)\cdot g=hgk^{-1}.
\]
The orbtis are sets of the form 
\[
HgK=\{hgk:h\in H,\,k\in K\}.
\]
These sets are called \textbf{double $(H,K)$-coset}. 
In particular, two double cosets are either disjoint 
or equal. Moreover, $G$ admits a decomposition 
as a disjoint union of double cosets, that is 
\[
G=\bigcup_{i\in I}Hg_iK,
\]
for some set $I$. Now we compute 
\[
L_g=\{(h,k)\in H\times K:hgk^{-1}=g\}=\{(h,g^{-1}hg)\in H\times K\}.
\]
Then $|L_g|=|H\cap gKg^{-1}|$, because there is a bijection between $L_g$ and
$H\cap gKg^{-1}$. By the fundamental counting principle (Theorem~\ref{thm:fundamental}), 
\[
|HgK|=(L:L_g)=\frac{|H\times K|}{|H\cap gKg^{-1}|}=\frac{|H||K|}{|H\cap gKg^{-1}|}.
\]
\end{example}

As another application, we compute the
order of the group $\GL_n(p)$ para $n\geq1$ and 
a prime number $p$. 
The argument also works for
the group $\GL_n(q)$ in the case where
$q$ is a power of the prime number $p$.

\begin{example}
Let $K=\Z/p$.
We claim that 
\[
|\GL_n(p)|=(p^n-1)p^{n-1}|\GL_{n-1}(p)|,
\]
and hence 
\[
|\GL_n(p)|=(p^n-1)(p^n-p)\cdots (p^n-p^{n-1}).
\]
The formula is valid if $n\in\{1,2\}$. 
Assume that it holds for $n-1\geq1$.
The group $G=\GL_{n}(p)$ acts on
$K^{n}$ by left multiplication. There are two orbits, so 
\[
X=\{0\}\cup (K^{n}\setminus\{0\}),
\]
as if $v,w\in K^{n}\setminus\{0\}$, then there exists 
$g\in G$ such that $gv=w$.
By the fundamental counting principle,  
\[
p^{n+1}-1=|K^{n+1}\setminus\{0\}|=(G:G_{e_1}),
\]
where $e_1=(1,0,\dots,0)^T$. If $g=(g_{ij})\in G$ is such that
$ge_1=e_1$, then 
\[
g=
\begin{pmatrix}
1 & g_{12} & \cdots & g_{1n}\\
0 & g_{22} & \cdots & g_{2n}\\
\vdots & \vdots & \ddots &\vdots\\
0 & g_{n1} & \cdots & g_{nn}
\end{pmatrix}.
\]
Therefore $|G_{e_1}|=p^{n-1}|\GL_{n-1}(p)|$, as the submatrix 
$(g_{ij})_{2\leq i,j\leq n}$ is invertible and the 
$g_{1j}$'s can be chosen 
arbitrarily for all $j\in\{2,\dots,n\}$.
Hence 
\[
p^{n}-1=\frac{|G|}{|G_{e_1}|}=\frac{|\GL_n(p)|}{p^{n-1}|\GL_{n-1}(p)|},
\]
which implies the formula we wanted to prove.
\end{example}




\subsection{$p$-groups}

\index{Action!fixed points}
\index{Class equation}
Let $G$ be a group and
$X$ be a finite $G$-set. Then $X$ decomposes as a disjoint 
union of orbits. Let 
\[
\Fix(X)=\{x\in X:g\cdot x=x\text{ for all $g\in G$}\}
\]
be the set of \textbf{fixed points} of $X$. Gather  
the one-element orbits and apply cardinality. By the 
fundamental counting principle, 
\begin{equation}
\label{eq:clases}
|X|=|\Fix(X)|+\sum_{i=1}^k|G\cdot x_i|
=|\Fix(X)|+\sum_{i=1}^k(G:G_{x_i}),
\end{equation}
where the $x_j$'s are the representatives
of the orbits of size $\geq2$. Equality~\eqref{eq:clases} is extremely 
useful and is called the 
\textbf{class equation}.

\begin{example}
Let a finite group $G$ act on $G$ by conjugation. 
Then $\Fix(G)=Z(G)$ and 
\[
|G|=|Z(G)|+\sum_{i=1}^k(G:C_G(x_i)),
\]
for some $x_1,\dots,x_k\in G$ such that 
$(G:C_G(x_i))\geq2$ for all $i\in\{1,\dots,k\}$.
\end{example}

\begin{definition}
\index{$p$-group}
Let $p$ be a prime number. We say 
that $G$ is a \textbf{$p$-group} if $|G|=p^m$ for some $m\geq0$.
\end{definition}

\begin{theorem}
Let $p$ be a prime number and 
$G$ a $p$-group. If $\{1\}\ne N\unlhd G$, then
$N\cap Z(G)\ne\{1\}$.
\end{theorem}

\begin{proof}
Since $N$ is normal in $G$, $G$ acts on $N$ by conjugation. 
By the fundamental counting principle,
each orbit has prime-power size. Write 
\[
N=\underbrace{\mathcal{O}_1\cup\cdots\cup \mathcal{O}_k}_{\text{one-element orbits}}\cup\underbrace{\mathcal{O}_{k+1}\cup\cdots\cup\mathcal{O}_m}_{\text{orbits of size $>1$}},
\]
Since $N\cap Z(G)=\mathcal{O}_1\cup\cdots\cup\mathcal{O}_k$, 
the integers $k=|N\cap Z(G)|$ and $|N\setminus(N\cap Z(G))|$ are divisible by $p$. Thus 
\[
|N|\equiv|N\cap Z(G)|\bmod p.
\]
Since $1\in N\cap Z(G)$, then
$|N\cap Z(G)|>1$. In particular, $N\
cap Z(G)\ne\{1\}$.
\end{proof}

The following corollary follows immediately: 

\begin{corollary}
Let $p$ be a prime number and 
$G$ a $p$-group. Then 
$Z(G)\ne\{1\}$.
\end{corollary}

In Exercises \ref{xca:size4} and \ref{xca:size9}
we proved that groups of order four and nine are always abelian. 

\begin{corollary}
    Let $p$ be a prime number. If $G$ is a group of order $p^2$, 
    then $G$ is abelian. 
\end{corollary}

\begin{proof}
By Lagrange's theorem, $|Z(G)|\in\{1,p,p^2\}$. Since $G$ 
is a $p$-group, $Z(G)\ne\{1\}$. If $|Z(G)|=p$, then $G/Z(G)$ 
is cyclic. By Exercise \ref{xca:G/Z(G)}, 
$G$ is abelian, a contradiction. 
Thus $|Z(G)|=p^2$ and hence $G=Z(G)$.
\end{proof}

\subsection{Cauchy's theorem}

\begin{theorem}[Cauchy]
\index{Cauchy's theorem}
\label{thm:Cauchy}
Let $G$ be a finite group, and $p$ be a prime number
that divides $|G|$. 
Then there exists $g\in G$ of order $p$.
\end{theorem}

\begin{proof}
Let $C=\Z/p$ and 
\[
X=\{(x_1,\dots,x_p)\in G\times\cdots\times G:x_1\cdots x_p=1\}.
\]
Then $C$ acts on $X$ by $k\cdot (x_1,\dots,x_p)=(x_{k+1},\dots,x_{k+p})$, where the indices are taken modulo $p$. To see that
this is an action, note that 
\[
x_{i_1}\cdots x_{i_p}=1
\implies (x_{i_1}^{-1}x_{i_1})x_{i_2}\cdots x_{i_p}=x_{i_1}^{-1}
\implies x_{i_2}\cdots x_{i_p}x_{i_1}=1.
\]
If $x_1,\dots,x_{p-1}$ are fixed, then 
$x_p=x_{p-1}^{-1}\cdots x_{1}^{-1}$. Thus $|X|=|G|^{p-1}$. Each $C$-orbit
has either one or $p$ elements, as $|C|=p$. Write
\[
X=\underbrace{\mathcal{O}_1\cup\cdots\cup \mathcal{O}_k}_{\text{one-element orbits}}\cup\underbrace{\mathcal{O}_{k+1}\cup\cdots\cup\mathcal{O}_m}_{\text{orbits of size $p$}}.
\]
Hence $0\equiv |G|^{p-1}=|X|\equiv k\bmod p$, that is $p$ divides $k$. Since $(1,1,\dots,1)\in X$, $k\geq 1$. Therefore $p\leq k$. In particular,
there exists $x\in G\setminus\{1\}$ such that $(x,x,\dots,x)\in X$. Hence $|x|=p$.
\end{proof}

\begin{corollary}
    Let $p$ be a prime number and $G$ be a finite group. 
    Then $G$ is a $p$-group if and only if 
    every element of $G$ has order a power of $p$. 
\end{corollary}

\begin{proof}
    If $G$ is a $p$-group, then, by Lagrange's theorem, 
    every element has order a power of $p$. Conversely, 
    if $q$ is a prime divisor of $|G|$, by 
    Cauchy's theorem, there exists $g\in G$ of order $q$. Thus $q=p$.
\end{proof}

\begin{corollary}
    Let $p>2$ be a prime number 
    and $G$ be a group of order $2p$. 
    Then either $G\simeq\Z/2p$ or $G\simeq\D_p$.
\end{corollary}

\begin{proof}
    By Cauchy's theorem, there exist $r,s\in G$ such that
    $|r|=p$ and $|s|=2$. Let $H=\langle r\rangle$. Then
    $(G:H)=2$ and $H\unlhd G$. We can decompose $G$ as 
    $G=H\cup Hs$ (disjoint union), 
    as $s\not\in H$. In particular, 
\[
G=\{1,r,\dots,r^{p-1},s,rs,\dots,r^{p-1}s\}.
\]
Since $srs^{-1}\in H$, it follows that $srs^{-1}=r^k$ for some
$k\in\{0,1,\dots,p-1\}$. Since $s^2=1$,
\[
r=s^2rs^{-2}=s(srs^{-1})s^{-1}=sr^ks^{-1}=r^{k^2}.
\]
Thus $k^2\equiv 1\bmod p$ and either 
$k\equiv 1\bmod p$ or $k\equiv-1\bmod p$. 
If $k\equiv -1\bmod p$, then $srs^{-1}=r^{-1}$ and hence $G\simeq\D_p$.
If $k\equiv 1\bmod p$, then $rs=sr$ and hence, since $G$ is abelian, $G\simeq\Z/{2p}$.
\end{proof}

\begin{theorem}
    A group of order $p^m$ has a normal subgroup of order
    $p^n$ for every $n\leq m$.
\end{theorem}

\begin{proof}
    We proceed by induction on $m$. The case where $m=1$ is trivial. Assume that the result holds for groups of order $p^m$. Let
    $G$ be a group of order $p^{m+1}$.
    We claim that if $n\leq m$, $G$ contains a normal subgroup of
    order $p^n$. Since $Z(G)\ne\{1\}$, there exists
    $g\in Z(G)\setminus\{1\}$ of order $p$. Let
$N=\langle g\rangle\unlhd G$. The quotient 
group $G/N$ has order $p^m$. By the inductive hypothesis, 
there exists a normal subgroup $Y$ of $G/N$ of order $p^n$. By the correspondence theorem, $G$ contains a normal subgroup $K$ of $G$ 
that contains $N$, that is $N\leq K\leq G$. In fact, 
$Y=\pi(K)$ and 
$(G:K)=(\pi(G):\pi(K))=p^{m-n}$. Hence $|K|=p^n$.
\end{proof}
\section{Lecture -- Week 10}


The idea used in Example \ref{exa:for_HK} can be generalized. 

\begin{optional}
\begin{example}
\index{Double coset}
Let $G$ be a group and $H$ and $K$ be subgroups of $G$. Let the group $L=H\times K$ act on $G$ by
\[
(h,k)\cdot g=hgk^{-1}.
\]
The orbtis are sets of the form 
\[
HgK=\{hgk:h\in H,\,k\in K\}.
\]
These sets are called \emph{double $(H,K)$-coset}. 
In particular, two double cosets are either disjoint 
or equal. Moreover, $G$ admits a decomposition 
as a disjoint union of double cosets, that is 
\[
G=\bigcup_{i\in I}Hg_iK,
\]
for some set $I$. Now we compute 
\[
L_g=\{(h,k)\in H\times K:hgk^{-1}=g\}=\{(h,g^{-1}hg)\in H\times K\}.
\]
Then $|L_g|=|H\cap gKg^{-1}|$, because there is a bijection between $L_g$ and
$H\cap gKg^{-1}$. By the fundamental counting principle (Theorem~\ref{thm:fundamental}), 
\[
|HgK|=(L:L_g)=\frac{|H\times K|}{|H\cap gKg^{-1}|}=\frac{|H||K|}{|H\cap gKg^{-1}|}.
\]
\end{example}
\end{optional}

As another application, we compute the
order of the group $\GL_n(p)$ for $n\geq1$ and 
a prime number $p$. 
The argument also works for
the group $\GL_n(q)$ in the case where
$q$ is a power of the prime number $p$.

\begin{example}
Let $p$ be a prime number and $K=\Z/p$.
We claim that 
\[
|\GL_n(p)|=(p^n-1)p^{n-1}|\GL_{n-1}(p)|,
\]
and hence 
\[
|\GL_n(p)|=(p^n-1)(p^n-p)\cdots (p^n-p^{n-1}).
\]

The formula is valid if $n\in\{1,2\}$. 
Assume that it holds for $n-1\geq1$.
The group $G=\GL_{n}(p)$ acts on
$K^{n\times 1}$ by left multiplication. 
How are the orbits? 
Since for every 
non-zero $v,w\in K^{n\times 1}$, then there exists 
$g\in G$ such that $gv=w$. Thus there are only two orbits.
One orbit is the one-element orbit 
of the zero column vector of $K^{n\times1}$, and 
the other orbit is the set $\mathcal{O}$ of non-zero vectors of $K^{n\times1}$. 
By the fundamental counting principle,  
\[
p^{n+1}-1=|\mathcal{O}|=(G:G_{v}),
\]
for every $v\in\mathcal{O}$, that is every $v\in K^{n\times 1}$. 

To compute the stabilizer $G_v$ easily, take 
\[
v=\colvec{4}{1}{0}{\vdots}{0}\in\mathcal{O}. 
\]
If $g=(g_{ij})\in G$ is such that
$gv=v$, then 
\[
g=
\begin{pmatrix}
1 & g_{12} & \cdots & g_{1n}\\
0 & g_{22} & \cdots & g_{2n}\\
\vdots & \vdots & \ddots &\vdots\\
0 & g_{n1} & \cdots & g_{nn}
\end{pmatrix}.
\]
Therefore $|G_{v}|=p^{n-1}|\GL_{n-1}(p)|$, as the submatrix 
$(g_{ij})_{2\leq i,j\leq n}$ is invertible and the 
$g_{1j}$'s can be chosen 
arbitrarily for all $j\in\{2,\dots,n\}$.
Hence 
\[
p^{n}-1=\frac{|G|}{|G_{v}|}=\frac{|\GL_n(p)|}{p^{n-1}|\GL_{n-1}(p)|},
\]
which implies the formula we wanted to prove.
\end{example}

\subsection{$p$-groups}

\index{Action!fixed points}
\index{Class equation}
Let $G$ be a finite group acting on a finite 
set $X$. Let 
\[
\Fix(X)=\{x\in X:g\cdot x=x\text{ for all $g\in G$}\}
\]
be the set of \emph{fixed points} of $X$, that is the set of one-elements 
orbits. We know that $X$ decomposes as a disjoint 
union of orbits. In particular, 
\[
X=\Fix(X)\cup \mathcal{O}_1\cup\cdots\mathcal{O}_k,
\]
where $\mathcal{O}_1,\dots,\mathcal{O}_k$ are orbits such that 
$|\mathcal{O}_j|\geq2$ for all $j\in\{1,\dots,k\}$. 
If we apply cardinality and use the  
fundamental counting principle, 
\begin{equation}
\label{eq:clases}
|X|=|\Fix(X)|+\sum_{i=1}^k|\mathcal{O}_i|
=|\Fix(X)|+\sum_{i=1}^k(G:G_{x_i}),
\end{equation}
where $x_j\in\mathcal{O}_j$ and 
$(G:G_{x_i})\geq2$ for all $j\in\{1,\dots,k\}$. 
Equality~\eqref{eq:clases} is extremely 
useful and is called the 
\emph{class equation}.

\begin{example}
Let a finite group $G$ act on $G$ by conjugation. 
Then $\Fix(G)=Z(G)$ and 
\[
|G|=|Z(G)|+\sum_{i=1}^k(G:C_G(x_i)),
\]
for some $x_1,\dots,x_k\in G$ such that 
$(G:C_G(x_i))\geq2$ for all $i\in\{1,\dots,k\}$.
\end{example}

\begin{definition}
\index{$p$-group}
Let $p$ be a prime number. We say 
that $G$ is a \emph{$p$-group} if $|G|=p^m$ for some $m\geq0$.
\end{definition}

\begin{theorem}
Let $p$ be a prime number and 
$G$ be a $p$-group. If $N$ is a non-trivial normal subgroup of $G$, then
$N\cap Z(G)\ne\{1\}$.
\end{theorem}

\begin{proof}
Since $N$ is normal in $G$, $G$ acts on $N$ by conjugation. 
By the fundamental counting principle,
each orbit has prime-power size. Write 
\[
N=\underbrace{\mathcal{O}_1\cup\cdots\cup \mathcal{O}_k}_{\text{one-element orbits}}\cup\underbrace{\mathcal{O}_{k+1}\cup\cdots\cup\mathcal{O}_m}_{\text{orbits of size $>1$}},
\]
Since $N\cap Z(G)=\mathcal{O}_1\cup\cdots\cup\mathcal{O}_k$, 
the integers $k=|N\cap Z(G)|$ and $|N\setminus(N\cap Z(G))|$ are divisible by $p$. Thus 
\[
|N|\equiv|N\cap Z(G)|\bmod p.
\]
Since $1\in N\cap Z(G)$, then
$|N\cap Z(G)|>1$. In particular, $N\cap Z(G)\ne\{1\}$.
\end{proof}

The following corollary follows immediately: 

\begin{corollary}
Let $p$ be a prime number and 
$G$ a $p$-group. Then 
$Z(G)\ne\{1\}$.
\end{corollary}

In Exercises \ref{xca:size4} and \ref{xca:size9}
we proved that groups of order four and nine are always abelian. 

\begin{corollary}
    Let $p$ be a prime number. If $G$ is a group of order $p^2$, 
    then $G$ is abelian. 
\end{corollary}

\begin{proof}
By Lagrange's theorem, $|Z(G)|\in\{1,p,p^2\}$. Since $G$ 
is a $p$-group, $Z(G)\ne\{1\}$. If $|Z(G)|=p$, then $G/Z(G)$ 
is cyclic. By Exercise \ref{xca:G/Z(G)}, 
$G$ is abelian, a contradiction. 
Thus $|Z(G)|=p^2$ and hence $G=Z(G)$.
\end{proof}

\subsection{Cauchy's theorem}

\begin{theorem}[Cauchy]
\index{Cauchy's theorem}
\label{thm:Cauchy}
Let $G$ be a finite group, and $p$ be a prime number
that divides $|G|$. 
Then there exists $g\in G$ of order $p$.
\end{theorem}

\begin{proof}
Let $C=\Z/p$ and 
\[
X=\{(x_1,\dots,x_p)\in G\times\cdots\times G:x_1\cdots x_p=1\}.
\]
Then $C$ acts on $X$ by $k\cdot (x_1,\dots,x_p)=(x_{k+1},\dots,x_{k+p})$, where the indices are taken modulo $p$. To see that
this is an action, note that 
\[
x_{i_1}\cdots x_{i_p}=1
\implies (x_{i_1}^{-1}x_{i_1})x_{i_2}\cdots x_{i_p}=x_{i_1}^{-1}
\implies x_{i_2}\cdots x_{i_p}x_{i_1}=1.
\]
If $x_1,\dots,x_{p-1}$ are fixed, then 
$x_p=x_{p-1}^{-1}\cdots x_{1}^{-1}$. Thus $|X|=|G|^{p-1}$. Each $C$-orbit
has either one or $p$ elements, as $|C|=p$. Write
\[
X=\underbrace{\mathcal{O}_1\cup\cdots\cup \mathcal{O}_k}_{\text{one-element orbits}}\cup\underbrace{\mathcal{O}_{k+1}\cup\cdots\cup\mathcal{O}_m}_{\text{orbits of size $p$}}.
\]
Hence $0\equiv |G|^{p-1}=|X|\equiv k\bmod p$, that is $p$ divides $k$. Since $(1,1,\dots,1)\in X$, $k\geq 1$. Therefore $p\leq k$. In particular,
there exists $x\in G\setminus\{1\}$ such that $(x,x,\dots,x)\in X$. Hence $|x|=p$.
\end{proof}

\begin{exercise}
\label{xca:p_groups}
    Let $p$ be a prime number and $G$ be a finite group. 
    Then $G$ is a $p$-group if and only if 
    every element of $G$ has order a power of $p$. 
\end{exercise}

\begin{corollary}
    Let $p>2$ be a prime number 
    and $G$ be a group of order $2p$. 
    Then either $G\simeq\Z/2p$ or $G\simeq\D_p$.
\end{corollary}

\begin{proof}
    By Cauchy's theorem, there exist $r,s\in G$ such that
    $|r|=p$ and $|s|=2$. Let $H=\langle r\rangle$. Then
    $(G:H)=2$ and $H\unlhd G$. We can decompose $G$ as 
    $G=H\cup Hs$ (disjoint union), 
    as $s\not\in H$. In particular, 
\[
G=\{1,r,\dots,r^{p-1},s,rs,\dots,r^{p-1}s\}.
\]
Since $srs^{-1}\in H$, it follows that $srs^{-1}=r^k$ for some
$k\in\{0,1,\dots,p-1\}$. Since $s^2=1$,
\[
r=s^2rs^{-2}=s(srs^{-1})s^{-1}=sr^ks^{-1}=r^{k^2}.
\]
Thus $k^2\equiv 1\bmod p$ and either 
$k\equiv 1\bmod p$ or $k\equiv-1\bmod p$. 
If $k\equiv -1\bmod p$, then $srs^{-1}=r^{-1}$ and hence $G\simeq\D_p$.
If $k\equiv 1\bmod p$, then $rs=sr$ and hence, since $G$ is abelian, $G\simeq\Z/{2p}$.
\end{proof}

\begin{theorem}
    Let $p$ be a prime number. 
    A group of order $p^m$ has a normal subgroup of order
    $p^n$ for every $n\leq m$.
\end{theorem}

\begin{proof}
    We proceed by induction on $m$. The case where $m=1$ is trivial. So let $m\geq1$ and 
    assume the result holds for groups of order $p^m$. Let
    $G$ be a group of order $p^{m+1}$.
    We claim that if $n\leq m$, $G$ contains a normal subgroup of
    order $p^n$. Since $Z(G)\ne\{1\}$, there exists
    $g\in Z(G)\setminus\{1\}$ of order $p$. Let
$N=\langle g\rangle\unlhd G$. The quotient 
group $G/N$ has order $p^m$. By the inductive hypothesis, 
there exists a normal subgroup $Y$ of $G/N$ of order $p^n$. Let 
$\pi\colon G\to G/N$ be the canonical map. 
By the correspondence theorem, $G$ contains a normal subgroup $K$ of $G$ 
that contains $N$, that is $N\leq K\leq G$. In fact, 
$Y=\pi(K)$ and 
$(G:K)=(\pi(G):\pi(K))=p^{m-n}$. Hence $|K|=p^{n+1}$.
\end{proof}

% The following exercise uses group actions to 
% prove Fermat's theorem on the representation 
% of primes as sum of two squares. 
% Elsholtz, C. (2010). A Combinatorial Approach to Sums of Two Squares and Related Problems. Additive Number Theory, 115–140. doi:10.1007/978-0-387-68361-4_8 
% \begin{bonus}

%     \label{xca:twosquares}
%     Let 
%     \[
%     X_1=\begin{pmatrix}
%        0&1&0\\
%        1&0&0\\
%        0&0&-1
%     \end{pmatrix},\quad 
%     X_2=\begin{pmatrix}
%         0&1&0\\
%         1&0&0\\
%         0&0&1
%     \end{pmatrix},\quad 
%     X_3=\begin{pmatrix}
%         1&-1&1\\
%         0&1&0\\
%         0&2&-1
%     \end{pmatrix}.
%     \]
%     Let 
%     \begin{align*}
%         S &= \{(x,y,z)\in\Z^3:p=4xy+z^2,\quad x,y>0\},\\
%         T &= \{(x,y,z)\in S:z>0\},\\
%         U &= \{(x,y,z)\in S:x+z>y\}.
%     \end{align*}
%     Prove the following statements:
%     \begin{enumerate}
%         \item $X_1^2=X_2^2=X_3^2=I$, the identity matrix.
%         \item $X_1$ maps $S$ to $S$, $X_2$ maps $T$ to $T$ and $X_3$ maps $U$ to $U$.  
%         \item $|T|=|X_1(T)|$. 
%         \item %$S=T\cup X_1(T)$ (disjoint union). Thus 
%         $|S|=2|T|=2|U|$. 
%         \item The map $X_3$ acting on $U$ has exactly one orbit of length one and since all other orbits have length two. In particular, $|U|$ is odd. 
%         \item The action of $X_2$ on $T$ has an orbit of length one. Conclude that $p=4x^2+y^2$. 
%     \end{enumerate}
% \end{bonus}
\section{16/05/2024}


\subsection{Sylow's theorems}

\begin{definition}
\index{Sylow subgroup}
Let $G$ be a group of $p^\alpha m$, where $p$ is a prime number
coprime with $m$. A subgroup $S$ of $G$ is said to be a \textbf{Sylow $p$-subgroup} of $G$ if $|S|=p^\alpha$.
\end{definition}

A subgroup $S$ of $G$ is a Sylow 
$p$-subgroup of $G$ if and only if $S$ is a $p$-group and
the prime $p$ does not divide $(G:S)$.

\begin{example}\
\begin{enumerate}
\item If $p$ does not divide $|G|$, then $\{1\}$ is a 
Sylow $p$-subgroup of $G$.
\item If $G$ is a $p$-group, then $G$ is a Sylow
$p$-subgroup of $G$.
\end{enumerate}
\end{example}

\begin{example}
Let $G=\Sym_3$. Then $\langle (12)\rangle$, $\langle (13)\rangle$ and $\langle (23)\rangle$ are the Sylow $2$-subgroups of $G$. Moreover, 
$\langle (123)\rangle$ is the only Sylow $3$-subgroup of $G$.
\end{example}

\begin{example}
Let $G=\Sym_4$. The subgroup $\langle (1234),(13)\rangle$ is a Sylow $2$-subgroup of $G$ and 
$\langle (123)\rangle$ is a Sylow $3$-subgroup of $G$.
\end{example}

\begin{example}
Let $G=\Z/18$. The subgroup 
$\langle 2\rangle =\{0,2,4,6,8,10,12,14,16\}$ is the only Sylow $3$-subgroup of $G$ and $\langle 9\rangle=\{0,9\}$ is the only
Sylow $2$-subgroup of $G$.
\end{example}

\begin{example}
Let $p$ be a prime number and 
$G=\GL_n(p)$. Since 
\begin{align*}
|\GL_n(p)|&=(p^n-1)(p^n-p)\cdots (p^n-p^{n-1})\\
&=p^{1+2+\cdots+n}(p^n-1)(p^{n-1}-1)\cdots (p-1),
\end{align*}
we can write $|\GL_n(p)|=p^\alpha m$, where $\alpha=1+2+\cdots+n$ and $m$
is an integer not divisible by $p$. 
The subgroup of matrices of the form 
\[
\begin{pmatrix}
1 & * & \cdots & *\\
0 & 1 & \cdots & *\\
\vdots & \vdots & \ddots & \vdots\\
0 & 0 & \cdots & 1
\end{pmatrix},
\]
that is the set of matrices $(g_{ij})$ with 
\[
g_{ij}=\begin{cases}
1 & \text{si $i=j$},\\
0 & \text{si $i>j$},
\end{cases}
\]
has order $p^\alpha$. Thus it is a Sylow $p$-subgroup of 
$\GL_n(p)$.
\end{example}

We will prove three crucial theorems that go back to Sylow. 
The first one guarantees the existence of Sylow subgroups. 
We shall need a lemma. 

\begin{lemma}
    If $p$ is a prime number, 
    $\alpha\geq0$ and $m\geq 1$, then 
    \[
        \binom{p^\alpha m}{p^\alpha}\equiv m\bmod p.
    \]
\end{lemma}

\begin{proof}
    By the binomial theorem,
        \[
        (1+X)^p=\sum_{j=0}^p\binom{p}{j}X^{p-j}\equiv 1+X^p\bmod p,
        \]
    because $\binom{p}{j}$ is divisible by $p$ for all
    $j\in\{1,\dots,p-1\}$.
    By using induction, one proves that 
        \begin{gather*}
        (1+X)^{p^j}\equiv 1+X^{p^j}\bmod p      \\
        \shortintertext{holds for all $j$. Thus}
        (1+X)^{p^\alpha m}\equiv (1+X^{p^\alpha})^m\bmod p.
        \end{gather*}
Comparing the coefficient of $X^{p^\alpha}$ in the previous
formula, we get the result we wanted to prove.
\end{proof}

\begin{theorem}[Sylow's first theorem]
\index{Theorem!Sylow, I}
\label{thm:Sylow1}
Let $G$ be a finite group and 
$p$ a prime number. Then there exists a Sylow $p$-subgroup of $G$. 
\end{theorem}

\begin{proof}
Write $|G|=p^\alpha m$ with $\gcd(p,m)=1$ and $\alpha\geq1$. Let 
\[
X=\{S:S\subseteq G\text{ subsets of size $p^\alpha$}\}.
\]
Let $G$ act on $X$ by left multiplication, as $|g\cdot S|=|gS|=|S|$ for all $g\in G$ and $S\in X$. Decompose $X$
into $G$-orbits and note that the previous lemma
implies that 
\[
|X|= \binom{p^\alpha m}{p^\alpha}\equiv m\not\equiv 0\bmod p.
\]
Thus there exists an orbit $\mathcal{O}$ of size not divisible by $p$. 
If $S\in\mathcal{O}$, let $G_S$ be the stabilizer of $S$ in $G$. Since
$|\mathcal{O}|=(G:G_S)$ and $|\mathcal{O}|$ is not divisible by $p$, we obtain that $p^\alpha$ divide a $|
G_S|$. In particular, $p^\alpha\leq |G_S|$. If $g\in G_S$, then
$gS=S$. If $x\in S$, then $G_Sx\subseteq S$. Thus 
\[
|G_S|=|G_Sx|\leq |S|=p^\alpha
\]
as $S\in X$. Therefore $G_S$ is a Sylow $p$-subgroup of $G$.
\end{proof}

If $G$ is a finite group and
$p$ is a prime divisor of $|G|$, 
let 
\[
\Syl_p(G)=\{\text{Sylow $p$-subgroups of $G$}\}.
\]
The first Sylow's theorem states that 
$\Syl_p(G)$ is non-empty. 

Before proving Sylow's second theorem, 
we state and prove a slightly more technical result. 

\begin{theorem}
\label{thm:Sylow_auxiliar}
Let $G$ be a finite group. 
If $P$ is a $p$-subgroup of $G$ and $S\in\Syl_p(G)$, then $P\subseteq gSg^{-1}$ for some $g\in G$.
\end{theorem}

\begin{proof}
    Let $X=\{xS:x\in G\}$ be the set of left cosets of $S$ in $G$. Then $|X|=(G:S)$ is not divisible by $p$. Let $G$ act on $X$ 
    by left multiplication. In particular, 
    $P$ also acts on $X$ by left multiplication. 
    Decompose $X$ into 
    $P$-orbits. There exists a $P$-orbit $\mathcal{O}$ of size
    not divisible by $p$, as $|X|$ is not divisible by $p$. Since  $|\mathcal{O}|$ divides $|P|$ and $p$ does not divide 
    $|\mathcal{O}|$, it follows that $|\mathcal{O}|=1$,
    that is $\mathcal{O}=\{gS\}$ for some
    $g\in G$. Since $P(gS)=gS$, in particular, $xg\in gS$ 
    for all $x\in P$. This means that if $x\in P$, then 
    $x\in gSg^{-1}$. Hence $P\subseteq gSg^{-1}$.
\end{proof}

An application:

\begin{corollary}
    Let $p$ be a prime number. 
    If $G$ is a finite group and 
    $P$ is a $p$-subgroup of $G$, then $P$ 
    is contained in some Sylow $p$-subgroup of $G$.
\end{corollary}

\begin{proof}
If $S\in\Syl_p(G)$, then $gSg^{-1}\in\Syl_p(G)$, as $|gSg^{-1}|=|S|$. 
By the previous theorem, 
$P\subseteq gSg^{-1}$ for some $g\in G$. Thus the claim follows. 
\end{proof}

Theorem \ref{thm:Sylow2} states 
that any two Sylow $p$-subgroups are conjugate, that 
is, $G$ acts transitively 
by conjugation on $\Syl_p(G)$. 

\begin{theorem}[Sylow's second theorem]
\index{Theorem!Sylow, II}
\label{thm:Sylow2}
Let $G$ be a finite group and $p$ 
a prime number. If 
$S,T\in\Syl_p(G)$, then there exists 
$g\in G$ such that $gSg^{-1}=T$.
\end{theorem}

\begin{proof}
Use the previous theorem with $P=T$. Then
$T\subseteq gSg^{-1}$ for some $g\in G$. Since 
$|S|=|T|$ and 
$|T|\leq |gSg^{-1}|=|S|$, we conclude that
$T=gSg^{-1}$.
\end{proof}

\begin{corollary}
Let $G$ be a finite group, $p$ a prime number and
$S\in\Syl_p(G)$. If $S$ is normal in $G$, then 
$\Syl_p(G)=\{S\}$.
\end{corollary}

\begin{proof}
If $T\in\Syl_p(G)$, then $T=gSg^{-1}=S$ for some $g\in G$.
\end{proof}

For the next theorem, we need some notation. If $p$
is a prime number and $G$ is a finite group of order
$p^\alpha m$ with $\gcd(p,m)=1$, then
$n_p(G)=|\Syl_p(G)|$. Note that 
\[
n_p(G)=(G:N_G(P))
\]
for all $P\in\Syl_p(G)$. We will prove that
$n_p(G)$ divides $m$.

\begin{theorem}[Sylow's third theorem]
\index{Theorem!Sylow, III}
\label{thm:Sylow3}
Let $G$ be a finite group and $p$ a prime number. 
Then $n_p(G)\equiv 1\bmod p$.
\end{theorem}

\begin{proof}
    Assume that $|G|=p^{\alpha}m$ with $m$ not divisible by $p$.
    By Sylow's first theorem, $\Syl_p(G)$ is non-empty. Let $P\in\Syl_p(G)$ and $n=n_p(G)$. We consider the set  
    
        \[
        X=\{gPg^{-1}:g\in G\}=\{P=P_1,P_2,\dots,P_n\}.
        \]
    By Sylow's second theorem, $|X|=n$.

    Let $G$ act on $X$ by conjugation. Then $P$ also acts on 
    $X$ by conjugation. Each $P$-orbit has size a power of $p$. 
    
    We claim that $\{P\}$ is the only $P$-orbit of size one.  
    Since $xPx^{-1}=P$ if $x\in P$, it follows that
    $\{P\}$ is a $P$-orbit. Let $\{P_i\}$ be a $P$-orbit
    of size one. Then 
        $xP_ix^{-1}=P_i$ for all 
        $x\in P$. Thus $P\subseteq N_G(P_i)$. The group 
        $N_G(P_i)/P_i$ has order not divisible by $p$, as $P_i\in\Syl_p(G)$.
        If $xP_i\in N_G(P_i)/P_i$ con $x\in P$, then $xP_i=P_i$. That is
        $x\in P_i$, since 
        $(xP_i)^{p^{\alpha}}=x^{p^{\alpha}}P_i=P_i$ implies that 
        $|xP_i|$ divides $p^{\alpha}$. Hence $|xP_i|=1$,
        as $N_G(P_i)/P_i$ has order coprime with $p$. Therefore 
        $x\in P_i$.
        This implies that 
        $P\subseteq P_i$. Hence $P=P_i$, as both sets 
        have size $p^{\alpha}$. Now 
        \[
        X=\{P\}\cup \underbrace{\mathcal{O}_1\cup\mathcal{O}_2\cup\cdots\cup\mathcal{O}_k}_{\text{of size $>1$ divisible by $p$}}.
        \]
        Thus $n_p(G)=|X|\equiv 1\bmod p$.
\end{proof}

We now discuss some applications of Sylow's theorems. 

\begin{example}
If $G$ is a group of order 15, then $G$ is cyclic.

Let $n_3=n_3(G)$ and $n_5=n_5(G)$. Then $n_3\equiv1\bmod 3$ and $n_3$ divides 5. Thus $n_3=1$ and hence there exists a unique
$H\in\Syl_3(G)$. This group is then normal in $G$ and isomorphic to
$\Z/3$. Similarly, 
 $n_5=1$ and there is a unique subgroup $K\in\Syl_5(G)$ such that
 $K\unlhd G$ and $K\simeq\Z/5$. Since $H\cap K=\{1\}$ by Lagrange's theorem, 
 \[
|HK|=\frac{|H||K|}{|H\cap K|}=|H||K|=15=|G|.
\]
Hence $G=HK\simeq H\times K\simeq \Z/3\times\Z/5\simeq\Z/15$.
\end{example}

\begin{example}
If $G$ is a group of order 455, then $G$ is cyclic.

For every prime number $p$ dividing $|G|$, let $n_p=n_p(G)$. 
Since $n_5$ divides $7\times 13$ and
$n_5\equiv 1\bmod 5$, then $n_5\in\{1.91\}$. A direct 
calculation shows that $n_7=n_{13}=1$. Let $P\in\Syl_7(G)$ and  $Q\in\Syl_{13}(G)$, both normal subgroups of $G$. Since $P$ and $Q$ 
have coprime orders, Lagrange's theorem implies that 
$P\cap Q=\{1\}$.
We now apply Sylow's theorems to the quotients $G/P$ and $G/Q$.
Let $m_5=n_5(G/P)$ and $m_{13}=n_{13}(G/P)$. Since $m_5$ divides 13 and  $m_5\equiv1\bmod 5$, it follows that 
$m_5=1$. Similarly, $m_{13}=1$ and
hence $G/P\simeq\Z/5\times\Z/13$. The same argument shows that 
$G/Q\simeq\Z/5\times\Z/7$. Thus both $G/P$ and $G/Q$ are abelian. 
This means that 
$[G,G]\subseteq P\cap Q=\{1\}$. Hence $G$ is also abelian. 
In particular, $n_5=1$ and 
\[
G\simeq\Z/5\times\Z/7\times\Z/13\simeq\Z/455.
\]
\end{example}

\begin{example}
If $G$ is a group of order 21, then either 
\[
G\simeq\Z/21
\quad\text{or}\quad 
G\simeq\langle x,y:x^7=y^3=1,\,yx=x^2y\rangle.
\]

Let $n_3=n_3(G)$ and $n_7=n_7(G)$. Since $n_7\equiv1\bmod 7$ and $n_3$ divides $3$, it follows that $n_7=1$. There is a unique
$H\in\Syl_7(G)$. This subgroup $H$ is such that
$H\unlhd G$ and $H\simeq\Z/7$. Thus $H=\langle x\rangle$, where
$x^7=1$.
Let $K\in\Syl_3(G)$. Since $n_3$ divides 7 and
$n_3\equiv1\bmod 3$, it follows that $n_3\in\{1,7\}$. 
Hence $K\simeq\Z/3$ and thus 
$K=\langle y\rangle$ where $y^3=1$. By Lagrange's theorem, 
$H\cap K=\{1\}$ and $G=HK$. In particular,
\[
G=\{x^iy^j:0\leq i\leq 6,\,0\leq j\leq 2\}.
\]
Since $H$ is normal in $G$, $yxy^{-1}\in H$. That is 
$yxy^{-1}=x^i$ for some $i\in\{1,\dots,6\}$. Therefore 
$x^7=y^3=1$ and  $yx=x^iy$ for some $i\in\{1,\dots,6\}$. What can we say about this $i$? We note that
\[
x=y^3xy^{-3}=y^2(yxy^{-1})y^{-2}=y^2x^iy^{-2}=y(x^i)^2y^{-1}=(x^i)^3.
\]
Then $i^3\equiv 1\bmod 7$, that is $i\in\{1,2,4\}$. There are
three cases: 
\begin{enumerate}
        \item[(a)] If $yxy^{-1}=x$, then $xy=yx$. Thus $K\unlhd G$ and $G\simeq H \times K\simeq\Z/21$.
        \item[(b)] If $yxy^{-1}=x^2$, then we can compute the table of $G$. In particular, $G$ can be obtained as a certain subgroup of  $\GL_2(\Z/7)$, that is 
        \[
        x=\begin{pmatrix}
        1&1\\
        0&1\end{pmatrix},
        \quad
        y=\begin{pmatrix}
        2&0\\
        0&1\end{pmatrix},
        \quad
        G\simeq\langle x,y\rangle\leq \GL_2(\Z/7).
\]
\item[(c)] If $yxy^{-1}=x^4$, then $y^2xy^{-2}=x^2$. Since $|y^2|=|y|=3$, if $z=y^2$, then $H=\langle y\rangle=\langle z\rangle$. 
So we are in the previous case. 
\end{enumerate}
\end{example}

\begin{example}
If $G$ is a group of order $5\cdot 7\cdot 17$, then $G$ is cyclic.

For $p\in\{5,7,17\}$, let $n_p=n_p(G)$. Since $n_5\equiv 1\bmod 5$ and $n_5$ divides $7\cdot 17$, it follows that 
$n_5=1$. Let $H\in\Syl_5(G)$. This is the only Sylow $5$-subgrop of $G$, 
so $H$ is normal in $G$. Let $K\in\Syl_7(G)$ and $L\in\Syl_{17}(G)$. Since $H$ is normal in $G$, $HK$ is a subgroup of $G$. By Lagrange's theorem, $H\cap K=\{1\}$ because $H$ and $K$ have coprime orders.
Thus  $|HK|=5\cdot 7$.
We now apply Sylow's theorems to the group $HK$. If $m_7=n_7(HK)$, then $m_7=1$. In particular, $K\in\Syl_7(HK)$ and $K$ is normal in $HK$. 
Thus $HK\subseteq N_G(K)$ and $|HK|\leq |N_G(K)|$. Since 
\[
n_7=(G:N_G(K))=\frac{|G|}{|N_G(K)|}\leq \frac{|G|}{|HK|}=\frac{5\cdot 7\cdot 17}{5\cdot 7}=17
\]
and $n_7\in\{1,5\cdot 17\}$, we conclude that
$n_7=1$. The same argument shows that
$n_{17}=1$. Therefore both $K$ and $L$ are normal in $G$. By
Lagrange's theorem, 
\[
L\cap H=H\cap K=L\cap K=\{1\}
\]
It follows that
\[
L\cap (HK)=H\cap (LK)=K\cap (LH)=\{1\}.
\]
Hence $G=HKL\simeq\Z/5\times\Z/7\times\Z/17\simeq\Z/(5\cdot 7\cdot 17)$.
\end{example}

\begin{example}
If $G$ is a group of order 12 such that 
$n_3(G)\ne1$, then $G\simeq\Alt_4$.

Let $P\in\Syl_3(G)$ and $n_3=n_3(G)=4$. Then $P$ is not normal in
$G$. Let $G$ act on the set $G/P$ by left multiplication. 
This induces a group homomorphism 
\[
\rho\colon G\to\Sym_{G/P}\simeq\Sym_4.
\]
We claim that $\rho$ is injective. Note that
$\ker\rho\subseteq P$, as 
\[
x\in\ker\rho\implies
\rho_x=\id\implies
xP\subseteq P\implies
x\in P.
\]
Since $P$ is not normal in $G$, $P\ne \ker\rho$. Thus $\ker\rho$ is a proper subgroup of $P$. Hence 
$\ker\rho=\{1\}$ since $|P|=3$.
Let $S,T\in\Syl_3(G)$. Since $S\simeq T\simeq\Z/3$, Lagrange's theorem implies that $S\cap T=\{1\}$. Thus $G$ 
contains exactly eight elements of order three. 
Since the elements of order three of $\Sym_4$ belong to $\Alt_4$, the subgroup $\rho(G)\cap\Alt_4$ of $\Sym_4$ contains at least
eight elements. Therefore $G\simeq\rho(G)\simeq\Alt_4$.
\end{example}

Sylow's theorems can be used to detect non-simple groups. 

\begin{example}
If $G$ is a group of order 36, then $G$ is not simple.

Assume that $G$ is simple. Then $n_3=n_3(G)=4$. 
Let $P\in\Syl_3(G)$ and let $G$ act on $X=\{gPg^{-1}:g\in G\}$ by conjugation. This induces a group homomorphism 
\[
\rho\colon G\to\Sym_X\simeq\Sym_4.
\]
Since $G$ is simple, either $\ker\rho=\{1\}$ or $\ker\rho=G$. If $\ker\rho=G$, $P$ is normal in $G$, a
contradiction. Thus $\ker\rho=\{1\}$ and hence
$\rho$ is injective. In particular, 
by the first isomorphism theorem, 
\[
G\simeq G/\ker\rho\simeq\rho(G)\lesssim\Sym_4.
\]
This implies that 36 divides 24, a contradiction. 
\end{example}

\begin{example}
If $G$ is a group of order 30, then $G$ is not simple.

For every prime number $p$ dividing 30, let $n_p=n_p(G)$. Assume that
$n_2>1$, $n_3>1$ and $n_5>1$.
 Then $n_3=10$. There are ten 
Sylow $3$-subgroups, any two of them with trivial intersection. 
In fact, if $P,Q\in\Syl_3(G)$ are such that 
 $P\ne Q$, then $P\cap Q\leq P$ and hence 
$|P\cap Q|\in\{1,3\}$. If $|P\cap Q|=3$, then $P\cap Q=P$ and $P=Q$, a contradiction. Similarly, 
there are six Sylow $5$-subgroups of $G$, any two of them with 
trivial intersection. In conclusion,
\[
|G|\geq 1+10\times 2+6\times 4>30,
\]
a contradiction.
\end{example}

% Al terminar la demostración del primer teorema de Sylow, usamos coclases dobles para demostrar que
% si $H$ es un subgrupo de un grupo finito $G$ y
% $P\in\Syl_p(G)$, entonces $g\in G$ tal que $gPg^{-1}\cap H\in\Syl_p(H)$. Otra demostración
% puede obtenerse al considerar la acción de $H$ en $G/P$ por multiplicación a izquierda.

\subsection{More about Sylow's theorems}

\begin{theorem}
    Let $N$ be a normal subgroup of a finite group 
    $G$ and $P\in\Syl_p(N)$. Then 
    $P\cap N\in\Syl_p(N)$. Moreover, every Sylow 
    $p$-subgroup of $N$  
    can be obtained this way. 
\end{theorem}

\begin{proof}
    Since $N$ is normal,
    by Theorem \ref{thm:Sylow_auxiliar}, there exists 
    $g\in G$ such that 
        \[
                g(P\cap N)g^{-1}=gPg^{-1}\cap gNg^{-1}=gPg^{-1}\cap N\in\Syl_p(N).
        \]
    Then $P\cap N$ is a Sylow $p$-subgroup of $g^{-1}Ng=N$.

    Let $Q\in\Syl_pN$ and $P\in\Syl_p(G)$ be such that
    $Q\subseteq P$. Then
    $Q\subseteq P\cap N$. Hence 
    $Q=P\cap N$, as $P\cap N$ is a Sylow $p$-subgroup of $N$. 
\end{proof}

As a corollary, if $G$ is a finite group and 
$N$ is a normal subgroup of $G$, then 
$n_p(N)\leq n_p(G)$.

\begin{theorem}
    Let $p$ be a prime number, $G$ be a finite group, $P\in\Syl_p(G)$, 
    and $N$ be a normal subgroup of $G$. Let $\pi\colon G\to G/N$ be the canonical homomorphism. Then 
    $\pi(G)\in\Syl_p(G/N)$ and every Sylow 
    $p$-subgroup of $G/N$ can be obtained this way. 
\end{theorem}

\begin{proof}
    Since $\pi(P)=\pi|_{P}(P)\simeq P/N\cap P$,
    the second isomorphism theorem 
    implies that 
    $\pi(P)$ is a $p$-group. Since
    $|PN|=|P||N|/|P\cap N|$,
        \[
                (G/N:\pi(P))=(G:PN)
        \]
        is not divisible by $p$. Thus $\pi(P)\in\Syl_p(G/N)$.

        If $Q\in\Syl_p(G/N)$, then $Q=\pi(H)$ for some
        subgroup $H$ of $G$
        with $N\subseteq H$. In particular,
        \[
                |Q|=|\pi(H)|=\frac{|H|}{|H\cap N|}=\frac{|H|}{|N|}.
        \]
        Thus 
        \[
                (G:H)=\frac{|G|/|N|}{|H|/|N|}=(G/N:Q)
        \]
        is not divisible by $p$. Thus $X\in\Syl_p(H)$ and
        hence   
        $X\in\Syl_p(G)$. Therefore   
        \[
        \pi(X)\subseteq\pi(H)=Q
        \]
        and 
        $\pi(X)=Q$, as 
        $\pi(X)\in\Syl_p(G/N)$.
\end{proof}

As a corollary, if $G$ is a finite group and
$N$ is a normal subgroup of $G$, then 
$n_p(G/N)\leq n_p(G)$.

\begin{corollary}
Let $G$ be a finite group. Assume that 
$G$ contains only one Sylow $p$-subgroup. Then
every subgroup and every quotient of 
$G$ 
contains only one Sylow 
$p$-subgroup. 
\end{corollary}

\begin{proof}
    If $H$ is a subgroup of $G$, then
    $n_p(H)\leq n_p(G)=1$. If $N$ is a normal subgroup of 
    $G$, then $n_p(G/N)\leq n_p(G)=1$.
\end{proof}

%Veamos una aplicación.
%
%\begin{theorem}[Wilson]
%       Sea $n\in\N$. Entonces $n$ es primo si y sólo si
%       $(n-1)!\equiv -1\bmod n$.
%\end{theorem}
%
%\begin{proof}
%       Sea $p$ un número primo.
%       El grupo $\Sym_p$ tiene $(p-1)!$ elementos de orden $p$. Cada
%       $p$-subgroupo de Sylow de $\Sym_p$ está generado por un $p$-ciclo, y luego
%       $n_p\equiv (p-2)!$. Por el tercer teorema de Sylow,
%       $(p-2)!=n_p\equiv 1\bmod p$. Al multiplicar por $p-1$, tenemos
%       $(p-1)!\equiv -1\bmod p$.
%\end{proof}
\section{23/05/2024}

\subsection{More about Sylow's theorems}

\begin{theorem}
    Let $N$ be a normal subgroup of a finite group 
    $G$ and $P\in\Syl_p(G)$. Then 
    $P\cap N\in\Syl_p(N)$. Moreover, every Sylow 
    $p$-subgroup of $N$  
    can be obtained this way.
\end{theorem}

\begin{proof}
    Since $N$ is normal,
    by Theorem \ref{thm:Sylow_auxiliar} applied to the group $N$, 
    there exists 
    $g\in G$ such that 
        \[
                g(P\cap N)g^{-1}=gPg^{-1}\cap gNg^{-1}=gPg^{-1}\cap N\in\Syl_p(N).
        \]
    Then $P\cap N$ is a Sylow $p$-subgroup of $g^{-1}Ng=N$.

    Let $Q\in\Syl_pN$ and $P\in\Syl_p(G)$ be such that
    $Q\subseteq P$. Then
    $Q\subseteq P\cap N$. Hence 
    $Q=P\cap N$, as $P\cap N$ is a Sylow $p$-subgroup of $N$. 
\end{proof}

As a corollary, if $G$ is a finite group and 
$N$ is a normal subgroup of $G$, then 
$n_p(N)\leq n_p(G)$.

\begin{theorem}
    Let $p$ be a prime number, $G$ be a finite group, $P\in\Syl_p(G)$, 
    and $N$ be a normal subgroup of $G$. Let $\pi\colon G\to G/N$ be the canonical homomorphism. Then 
    $\pi(P)\in\Syl_p(G/N)$ and every Sylow 
    $p$-subgroup of $G/N$ can be obtained this way. 
\end{theorem}

\begin{proof}
    Since $\pi(P)=(\pi|_{P})(P)\simeq P/N\cap P$,
    the second isomorphism theorem 
    implies that 
    $\pi(P)$ is a $p$-group. Since
    $|PN|=|P||N|/|P\cap N|$,
        \[
                (G/N:\pi(P))=(G:PN)
        \]
        is not divisible by $p$. Thus $\pi(P)\in\Syl_p(G/N)$.

        If $Q\in\Syl_p(G/N)$, then $Q=\pi(H)$ for some
        subgroup $H$ of $G$
        with $N\subseteq H$. In particular,
        \[
                |Q|=|\pi(H)|=\frac{|H|}{|H\cap N|}=\frac{|H|}{|N|}.
        \]
        Thus 
        \[
                (G:H)=\frac{|G|/|N|}{|H|/|N|}=(G/N:Q)
        \]
        is not divisible by $p$. 
        
        Let $X\in\Syl_p(H)$. Since $(G:H)$ is not divisible by $p$, 
        $X\in\Syl_p(G)$. Hence 
        \[
        \pi(X)\subseteq\pi(H)=Q. 
        \]
        Thus 
        $\pi(X)=Q$, as 
        $\pi(X)\in\Syl_p(G/N)$.
\end{proof}

As a corollary, if $G$ is a finite group and
$N$ is a normal subgroup of $G$, then 
$n_p(G/N)\leq n_p(G)$.

\begin{corollary}
Let $G$ be a finite group. Assume that 
$G$ contains only one Sylow $p$-subgroup. Then
every subgroup and every quotient of 
$G$ 
contains only one Sylow 
$p$-subgroup. 
\end{corollary}

\begin{proof}
    If $H$ is a subgroup of $G$, then
    $n_p(H)\leq n_p(G)=1$. If $N$ is a normal subgroup of 
    $G$, then $n_p(G/N)\leq n_p(G)=1$.
\end{proof}

%Veamos una aplicación.
%
%\begin{theorem}[Wilson]
%       Sea $n\in\N$. Entonces $n$ es primo si y sólo si
%       $(n-1)!\equiv -1\bmod n$.
%\end{theorem}
%
%\begin{proof}
%       Sea $p$ un número primo.
%       El grupo $\Sym_p$ tiene $(p-1)!$ elementos de orden $p$. Cada
%       $p$-subgroupo de Sylow de $\Sym_p$ está generado por un $p$-ciclo, y luego
%       $n_p\equiv (p-2)!$. Por el tercer teorema de Sylow,
%       $(p-2)!=n_p\equiv 1\bmod p$. Al multiplicar por $p-1$, tenemos
%       $(p-1)!\equiv -1\bmod p$.
%\end{proof}

\subsection{Abelian groups}

Let $A$ be an abelian group, written additively, and $x_1,\dots,x_k\in A$. 
The subgroup $\langle x_1,\dots,x_k\rangle$ 
generated by $\{x_1,\dots,x_k\}$
is the set of integer linear combinations of the elements 
$x_1,\dots,x_k$, that is
\[ 
\langle x_1,\dots,x_k\rangle=\left\{ 
\sum_{i=1}^k m_ix_i: m_1,\dots,m_k\in\Z\right\}.
\]
We say that $\{x_1,\dots,x_k\}$ \emph{generates} $A$ if 
$A=\langle x_1,\dots,x_k\rangle$. And we say that the set 
$\{x_1,\dots,x_k\}$ is 
\emph{linearly independent} if 
\[
\sum_{i=1}^k m_ix_i=0\implies m_1x_1=\cdots=m_kx_k=0.
\]

Note that our definition of linearly independence for abelian groups is slightly 
different from that of linear algebra. For example, in the group $\Z/5$, 
one has $5x=0$ for all $x$. Thus there will be no linearly independent sets with 
the standard linear algebra definition. 

\begin{exercise}
    Let $A$ be an abelian group and $X$ and $Y$ be subgroups of $A$. 
    We say that $A$ is the \emph{direct sum} of $X$ and $Y$ 
    if $A=X+Y$ and $X\cap Y=\{0\}$. In this case, we write $A=X\oplus Y$. Prove that 
    every element $a\in X\oplus Y$ can be
    written uniquely as $a=x+y$ for $x\in X$ and $y\in Y$.     
\end{exercise}

A subset $\{x_1,\dots,x_k\}$ is a \emph{basis} of $A$ 
if $A=\langle x_1\rangle\oplus\cdots\oplus \langle x_k\rangle$, that is
if $\{x_1,\dots,x_k\}$ is a 
linearly independent set of generators of $A$. 


\begin{theorem}
\label{thm:fundamental_abelian}
    Every finitely generated abelian group has a basis. In particular, 
    it is a finite direct sum of cyclic groups. 
\end{theorem}

Before proving the theorem, we need a lemma.

\begin{lemma}
\label{lem:trick_abelian}
    Let $A=\langle x_1,\dots,x_n\rangle$ be a finitely generated 
    abelian group and $c_1,\dots,c_n\in\Z_{>0}$ be such that 
    $\gcd(c_1,\dots,c_n)=1$. Then there exist $y_1,\dots,y_n\in A$ 
    such that 
    $A=\langle y_1,\dots,y_n\rangle$ and 
    \[ 
    y_1=c_1x_1+\cdots+c_nx_n.
    \]
\end{lemma}

\begin{proof}
    We proceed by induction on $s=c_1+\cdots+c_n$. The case $s=1$ is trivial. 
    So assume that $s\geq2$. Without loss of generality, we may assume that 
    $c_1\geq c_2>0$. Then 
    \[ 
    (c_1-c_2)+c_2+c_3+\cdots+c_n=c_1+c_3+\cdots+c_n<s.
    \]
    Moreover, $\gcd(c_1-c_2,c_2,\dots,c_n)=1$. Since $A=\langle x_1,x_1+x_2,x_3,\dots,x_n\rangle$, the inductive hypothesis implies that 
    there exist $y_1,\dots,y_n\in A$ such that 
    $A=\langle y_1,\dots,y_n\rangle$ and 
    \[ 
        y_1=(c_1-c_2)x_1+c_2(x_1+x_2)+c_3x_3+\cdots+c_nx_n
        =c_1x_1+c_2x_2+\cdots+c_nx_n.\qedhere
    \]
\end{proof}

Now we are ready to prove the main theorem of this section. 

\begin{proof}[Proof of Theorem \ref{thm:fundamental_abelian}]
    We proceed by induction on the number $n$ of generators. The case $n=1$ is trivial. So assume that the result holds for $n-1$ generators. 
    Among the generating sets $\{x_1,\dots,x_n\}$ 
    with $n$ elements, there is one 
    for which the order $|x_1|$ of $x_1$ is the smallest possible. By the inductive hypothesis, 
    the theorem will be proved if we can show that 
    \begin{equation}
    \label{eq:decomposition}
    A=\langle x_1\rangle\oplus \langle x_2,\dots,x_n\rangle
    \end{equation}
    holds. 
    Assume that \eqref{eq:decomposition} does not hold. 
    Note that 
    $A=\langle x_1\rangle+\langle x_2,\dots,x_n\rangle$, as 
    $\{x_1,\dots,x_n\}$ is a generating set of $A$. Since
    the decomposition  
    \eqref{eq:decomposition} does not hold, 
    $\langle x_1\rangle\cap \langle x_2,\dots,x_n\rangle\ne\{0\}$. Let  
    $\xi\in \langle x_1\rangle\cap \langle x_2,\dots,x_n\rangle$ be a non-zero element. Then 
    $\xi=m_1x_1=m_2x_2+\cdots+m_nx_n$ for some integer 
    $m_1x_1\ne 0$ and $m_2,\dots,m_n\in\Z$ not all zero.  Thus 
    \[ 
    (-m_1)x_1+m_2+\cdots+m_nx_n=0.
    \]
    After changing the sign of some of the generators, 
    we produce a generating set $\{z_1,\dots,z_k\}$ of $A$ 
    such that our linear combination becomes 
    \[ 
    \lambda_1z_1+\lambda_2z_2+\cdots+\lambda_nz_n=0, 
    \]
    where $\lambda_1,\dots,\lambda_n\in\Z_{\geq 0}$ 
    and $0<\lambda_1<|z_1|$. Let $d=\gcd(\lambda_1,\dots,\lambda_n)$ and
    for each $i\in\{1,\dots,n\}$, let 
    $c_i=\lambda_i/d$. 
    By Lemma~\ref{lem:trick_abelian}, there exist 
    $y_1,\dots,y_n\in A$ such that 
    $A=\langle y_1,\dots,y_n\rangle$ and 
    \[ 
    y_1=c_1z_1+\cdots+c_nz_n.
    \]
    But $dy_1=\lambda_1z_1+\cdots+\lambda_nz_n=0$ and 
    $d\leq \lambda_1<|x_1|$. We have found a generating set 
    $\{y_1,\dots,y_n\}$ in which the element
    $y_1$ has order smaller than $|x_1|$, a contradiction. 
\end{proof}

The previous theorem 
translates into the following result.

\begin{theorem}
\label{thm:abelian_factors}
    Let $A$ be a non-zero finitely generated 
    abelian group. Then 
    \[
    A\simeq (\Z/{n_1})\times\cdots\times (\Z/{n_k})\times\Z^r,
    \]
    for integers $n_1,\dots,n_k\geq2$ and 
    $r\geq0$. The integers
    $n_1,\dots,n_k$ 
    can be chosen so that $n_1\geq2$ and 
    $n_j$ divides $n_{j+1}$ for all $j\in\{1,\dots,k-1\}$. 
\end{theorem}

The integer $r$ 
in Theorem \ref{thm:abelian_factors}
is uniquely determined by $A$ and 
is called the \emph{rank} of $A$. The integers $n_1,\dots,n_k$ 
in Theorem \ref{thm:abelian_factors}
are called the \emph{invariant factors} of $A$ and 
are uniquely determined by $A$. 

In these notes, we will not prove that the rank and the invariant factors are uniquely determined by the group. Additionally, we will not prove the existence 
of the invariant factors. Instead, we will explain how to obtain them with some concrete examples.

\begin{example}
    Let $A=(\Z/6)\times(\Z/100)\times(\Z/45)$. We use 
    the fact that 
    \[
    (\Z/a)\times(\Z/b)\simeq\Z/ab
    \]
    whenever $\gcd(a,b)=1$
    to decompose $A$ as follows:
    \[
    A\simeq (\Z/2\times\Z/3)\times(\Z/4\times\Z/25)\times(\Z/5\times\Z/9).
    \]
    Let us order the prime powers: 2, 4, 3, 9, 5, 25. 
    Now we collect the highest prime powers appearing in our decomposition: 4 is the highest power of 2,
    9 is the highest power of 3, and 25 is the highest power of 5. Thus 
    $s_2=4\times 9\times 25=900$ is the highest invariant factor. Now 
    2 is the highest remaining power of 2, 
    3 is the highest power of 3 and 5 is the highest power of 5. Thus 
    $s_1=2\times 3\times 5=30$ is the second invariant factor. 
    Thus 
    \[ 
    A\simeq (\Z/30)\times(\Z/900).
    \]
\end{example}

\begin{example}
    Let $A=(\Z/10)\times(\Z/15)\times(\Z/20)\times(\Z/25)$. 
    As we did in the previous example, we decompose each factor: 
    \[ 
    A\simeq (\Z/2)\times(\Z/5)\times(\Z/3)\times (\Z/5)\times(\Z/4)\times (\Z/5)\times(\Z/25). 
    \]
    The numbers we see are 2, 4, 3, 5, 5, 25. The invariant factors are
    then $s_3=4\times 3\times 25=300$, $s_2=10$, $s_3=5$ and $s_4=5$. 
    Hence 
    \[ 
    A\simeq (\Z/5)\times(\Z/5)\times(\Z/10)\times(\Z/300).
    \]
\end{example}

\begin{exercise}
\label{xca:factors:24,12,4,2}
    Find the invariant factors 
    of the group $(\Z/4)\times(\Z/6)\times(\Z/8)\times(\Z/12)$. 
\end{exercise}




    

% \subsection{Double cosets (optional)}

% The idea used in Example \ref{exa:for_HK} can be generalized. 

% \begin{example}
% \index{Coclase!doble}
% Sea $G$ un grupo y sean $H$ y $K$ subgrupos de $G$. Hacemos
% que el grupo $L=H\times K$ actúe en $G$ por
% \[
% (h,k)\cdot g=hgk^{-1}.
% \]
% Las órbitas son los conjuntos de la forma
% \[
% HgK=\{hgk:h\in H,\,k\in K\},
% \]
% estos conjuntos se llaman $(H,K)$-coclases dobles.
% En particular, dos $(H,K)$-coclases dobles son disjuntas o iguales. Más aún,
% $G$ se descompone como unión disjunta
% \[
% G=\bigcup_{i\in I}Hg_iK,
% \]
% para algún conjunto $I$, es decir $G$
% es unión disjunta de $(H,K)$-coclases dobles.
% Calculamos ahora
% \[
% L_g=\{(h,k)\in H\times K:hgk^{-1}=g\}=\{(h,g^{-1}hg)\in H\times K\}
% \]
% y vemos que $|L_g|=|H\cap gKg^{-1}|$, pues los conjuntos $L_g$ y $H\cap gKg^{-1}$ están en biyección.

% Luego,
% gracias al principio fundamental del conteo,
% \[
% |HgK|=(L:L_g)=\frac{|H\times K|}{|H\cap gKg^{-1}|}=\frac{|H||K|}{|H\cap gKg^{-1}|}.
% \]
% \end{example}

% El ejemplo anterior puede generalizarse, lo que nos da una descomposición
% de un grupo como unión disjunta de \emph{coclases dobles}. Veremos más adelante
% demostraciones alternativas de los teoremas de Sylow basadas en
% coclases dobles.

% \begin{quote}
% Veamos una demostración alternativa del primer teorema de Sylow
% que utiliza coclases dobles.
% Primero demostraremos un resultado auxiliar. Si $P\in\Syl_p(G)$ y $H\leq G$, entonces
% existe un $g\in G$ tal que $H\cap gPg^{-1}\in\Syl_p(H)$. En efecto, supongamos que
% $|H|=p^\beta t$ con $p$ coprimo con $t$.
% Si descomponemos a $G$ en $(H,P)$-coclases dobles,
% \[
% |G|=\sum_{i=1}^k \frac{|H||P|}{|H\cap x_iPx_i^{-1}|}.
% \]
% Al simplificar $|P|=p^\alpha$, tenemos que $m=\sum_{i=1}^k(H:H\cap x_iPx_i^{-1})$, lo que nos dice
% que existe $i\in\{1,\dots,k\}$ tal que
% $(H:H\cap x_iPx_i^{-1})$ no es divisible por $p$. Esto significa que que
% $p^\beta$ divide a $|H\cap x_iPx_i^{-1}|$ y en consecuencia $H\cap x_iPx_i^{-1}\in\Syl_p(H)$.
% Por el teorema de Cayley podemos suponer que nuestro subgrupo $G$ es un subgrupo
% de $\GL_n(p)$ para algún $n$ y algún primo $p$. Sea $P$ un subgrupo de Sylow
% del grupo $\GL_n(P)$. La observación que demostramos
% aplicada al grupo $G$ nos dice que existe $g\in \GL_n(p)$ tal que $G\cap gPg^{-1}$ es un subgrupo de
% Sylow de $G$.
% \end{quote}

% \begin{quote}
% Veamos una
% demostración alternativa del segundo teorema de Sylow que usa coclases dobles.
% Si $P,Q\in\Syl_p(G)$ y
% descomponemos a $G$ en $(P,Q)$-coclases dobles, tenemos
% \[
% p^\alpha m=\sum_{i=1}^k\frac{|P||Q|}{|P\cap x_iQx_i^{-1}|}
% \implies
% m=\sum_{i=1}^k\frac{|P|}{|P\cap x_iQx_i^{-1}|}
% \]
% para ciertos $x_1,\dots,x_k\in G$.
% Como $m$ no es divisible por $p$, existe algún $i\in\{1,\dots,k\}$ tal que $|P|=|P\cap x_iQx_i^{-1}|$
% , lo que
% implica que $P=x_iQx_i^{-1}$ para algún $i\in\{1,\dots,k\}$.
% \end{quote}

% \begin{quote}
% Una demostración alternativa del tercer teorema de Sylow basada en coclases dobles.
% Sean $P\in\Syl_p(G)$ y $N=N_G(P)$. Recordemos
% que $n_p(G)=(G:N)$. Si descomponemos
% a $G$ en $(P,N)$-coclases dobles,
% \[
% |G|=\sum_{i=1}^k\frac{|P||N|}{|N\cap x_iPx_i^{-1}|}
% \]
% para ciertos $x_1,\dots,x_k\in G$. Sin perder generalidad podemos suponer que $x_1=1$, entonces la fó
% rmula anterior queda
% \[
% n_p(G)=1+\sum_{i=2}^k\frac{|P|}{|N\cap x_iPx_i^{-1}|},
% \]
% pues $(G:N)=n_p(G)$.
% El teorema quedará demostrado si vemos que la suma del miembro de la derecha es divisible por $p$. Si
%  esto no pasa,
% es decir si existe $i\in\{2,\dots,k\}$ tal que $|N\cap x_iPx_i^{-1}|=|P|$, entonces
% $x_iPx_i^{-1}=N\cap x_iPx_i^{-1}\subseteq N$. Como entonces $P$ y también $x_iPx_i^{-1}$ son ambos $p
% $-subgrupos de Sylow de $N$,
% el segundo teorema de Sylow afirma que estos subgrupos tienen que ser conjugados en $N$. Por definici
% ón del normalizador, $P$ es normal en $N$.
% En consecuencia, $x_iPx_i=P$, es decir $x_i\in N$, una contradicción pues como $i>1$ se tiene que
% $Px_iN$ y $Px_1N=PN$ son coclases dobles disjuntas.
% \end{quote}

\backmatter

%\addcontentsline{lec}{chapter}{Some hints}
%\include{exercises}
%\include{hints}

%\addcontentsline{lec}{chapter}{Some solutions}
\section*{Some solutions}

\fancyhf{}
\fancyfoot[R]{\thepage}
\fancyhead[L]{\course}
\fancyhead[R]{Some solutions}
\setlength{\headheight}{14pt}

% \pagestyle{plain}
% \fancyhf{}
% \fancyhead[LE,RO]{Rings and modules}
% \fancyhead[RE,LO]{Some solutions}
% \fancyfoot[CE,CO]{\leftmark}
% \fancyfoot[LE,RO]{\thepage}
% \addcontentsline{toc}{chapter}{Some solutions}
\begin{sol}{xca:neutral}
If $e$ and $e_1$ are both neutral elements, then $e=ee_1=e_1$. 
\end{sol}

\begin{sol}{xca:ax=b}
    If $ax=b$, after multiplying on the left by $a^{-1}$ we
    obtain that $x=a^{-1}b$. Similarly, the equation $xa=b$ 
    has $x=ba^{-1}$ as its unique solution. 
\end{sol}

\begin{sol}{xca:LR}
    For $g\in G$, the map $L_g\colon G\to G$, $x\mapsto gx$, is invertible 
    with inverse $L_{g^{-1}}$, as
    \[
    (L_g\circ L_{g^{-1}})(x)=g(g^{-1}x)=(gg^{-1})x=x
    \]
    for all $x\in G$. Similarly, $L_{g^{-1}}\circ L_g)(x)=x$ for all $x\in G$. 
    
    In the same way, we prove that 
    for each $g\in G$, the map $R_{g^{-1}}$ is the inverse of $R_g$. 
\end{sol}

\begin{sol}{xca:GxH}
    To prove the associativity, let $g,g_1,g_2\in G$ and 
    $h,h_1,h_2\in H$. Since $G$ and $H$ are groups, 
    their multiplications are associative. Then 
    \begin{align*}
        ((g,h)(g_1,h_1))(g_2,h_2) &= (gg_1,hh_1)(g_2,h_2)\\
        &=((gg_1)g_2,(hh_1)h_2)\\
        &= (g(g_1g_2),h(h_1h_2))\\
        &= (g,h)(g_1g_2,h_1h_2)\\
        &= (g,h)((g_1,h_1)(g_2,h_2)).
    \end{align*}
    
    The neutral element of $G\times H$ is $(1,1)$, as $(1,1)(g,h)=(g,h)=(g,h)(1,1)$. 
    
    The inverse
    of $(g,h)$ is $(g,h)^{-1}=(g^{-1},h^{-1})$, as
    \begin{align*}
    (g,h)(g,h)^{-1}&=(g,h)(g^{-1},h^{-1})=(gg^{-1},hh^{-1})=(1,1),\\
    (g,h)^{-1}(g,h)&=(g^{-1},h^{-1})(g,h)=(g^{-1}g,h^{-1}h)=(1,1).
    \end{align*}
\end{sol}

\begin{sol}{xca:center}
    Clearly $1\in Z(G)$. If $x\in Z(G)$, then $xg=gx$ for all $g\in G$. Multiplying by $x^{-1}$ on the left and 
    on the right, we get that $gx^{-1}=x^{-1}$ holds for all $g\in G$. Finally, if $x,y\in Z(G)$. Then
    \[
    (xy)g=x(yg)=x(gy)=(xg)y=(gx)y=g(xy)
    \]
    for all $g\in G$. Hence $xy\in Z(G)$. 
\end{sol}

% \begin{sol}{xca:centralizer}
%     First, $1\in G_G(g)$, as $1g=g1=g$. If $x\in c_G(g)$, then $xg=gx$. Multiplying 
%     on the left and on the right by $x^{-1}$, one gets $x^{-1}g=gx^{-1}$, that is 
%     $x^{-1}\in C_G(g)$. Finally, if $x,y\in C_G(g)$, then 
%     \[
%     $(xy)g=$
%     \]
% \end{sol}

\begin{sol}{xca:conjugate}
    Since $S$ is a subgroup, $1\in S$ and 
    if $x,y\in S$, then $x^{-1}\in S$ and $xy\in S$. Now 
    $1\in gSg^{-1}$, as $1\in S$ and $1=g1g^{-1}$. If $x\in gSg^{-1}$, then 
    $x=gsg^{-1}$ for some $s\in S$. Thus 
    \[
    x^{-1}=(gsg^{-1})^{-1}=gs^{-1}g^{-1}\in gSg^{-1},
    \]
    as $s^{-1}\in S$. Finally, 
    if $x=gsg^{-1}\in gSg^{-1}$ and $y=gtg^{-1}\in gSg^{-1}$ for some $s,t\in S$, then 
    \[
    xy=(gsg^{-1})(gtg^{-1})=g(st)g^{-1}\in gSg^{-1},
    \]
    as $st\in S$. 
\end{sol}

\begin{sol}{xca:center_S3}
    If $\sigma\in Z(\Sym_3)$ and $\sigma\ne\id$, there exists $i\in\{1,2,3\}$ such that 
    $\sigma(i)\ne i$. Let $j=\sigma(i)$ and $k\in\{1,2,3\}\setminus\{i,j\}$. Then 
    $(jk)\sigma$ is a permutation such that $i\mapsto k$, while 
    $\sigma(jk)$ is such that $i\mapsto j$. In particular, $(jk)\sigma\ne\sigma(jk)$, a contradiction.

    The group $\Sym_3$ has six elements: $\id$, $(12)$, $(13)$, $(23)$. $(123)$ and $(132)$. 
    First note that $\id\in C_{\Sym_3}((12))$ and 
    $(12)\in C_{\Sym_3}((12))$. However, 
    the permutations $(23)$, $(13)$, $(123)$ and $(132)$ do not commute with
    $(12)$. For example, 
    \[
    (23)(12)=(132)\ne (123)=(12)(23).
    \]
\end{sol}

\begin{sol}{xca:subgroup}
    Let us prove $\implies$. Since $1\in S$, then $S\ne\emptyset$. If $u,v\in S$, then 
    $v^{-1}\in S$ and $uv^{-1}\in S$. 

    Let us prove now $\impliedby$. If $S\ne\emptyset$, let $u\in S$. Then $1=uu^{-1}\in S$. The assumption 
    Let $u,v\in S$. The assumption with $x=1\in S$ and $y=v$ yields $v^{-1}\in S$. The assumption 
    with $x=u$ and $y=v^{-1}$ yields $uv\in S$. 
\end{sol}

\begin{sol}{xca:SL_subgroup}
    The identity matrix belongs to $\SL_n(\R)$. If $a,b\in\SL_n(\R)$, then 
    $ab^{-1}\in\SL_n(\R)$, as 
    \[
    \det(ab^{-1})=\det(a)\det(b^{-1})=\det(a)\det(b)^{-1}=1.
    \]
    By Exercise \ref{xca:subgroup}, $\SL_n(\R)$ is a subgroup of $\GL_n(\R)$. 
\end{sol}

\begin{sol}{xca:intersection}
    Let $\{H_\lambda:\lambda\in\Lambda\}$ be a collection of subgroups of a group $G$ and 
    $H=\cap_{\lambda\in \Lambda}H_\lambda$. We claim that $H$ is a subgroup of $G$. Since
    $1\in H_\lambda$ for all $\lambda$, $H$ is non-empty. If $x,y\in H$, then $x,y\in H_\lambda$ for all $\lambda$. 
    Since each $H_\lambda$ is a subgroup of $G$, $xy^{-1}\in H_\lambda$ for all $\lambda$. Thus $xy^{-1}\in H$.
\end{sol}

\begin{sol}{xca:generated}
    Let 
    \[
    H=\{x_1^{n_1}\cdots x_k^{n_k}:k\geq0,\,x_1,\dots,x_k\in X,\,-1\leq n_1,\dots,n_k\leq 1\}.
    \]
    To prove that $H\subseteq\langle X\rangle$, let $h=x_1^{n_1}\cdots x_k^{n_k}\in H$. 
    If $S$ is a subgroup of $G$ containing $X$, then $x_j\in S$ for all $j$. This implies that 
    $h=x_1^{n_1}\cdots x_k^{n_k}\in S$. Thus 
    \[
    h\in\bigcap_{\substack{S\leq G\\X\subseteq S}}S.
    \]
    
    To prove that $H\supseteq \langle X\rangle$ we first 
    claim that $H$ is a subgroup of $G$. Note that $H\ne\emptyset$, as $1\in H$ (this is the empty word). If 
    $u=x_1^{n_1}\cdots x_k^{n_k}\in H$ and 
    $v=x_{k+1}^{n_{k+1}}\cdots x_{l}^{n_l}\in H$, then 
    \[
    uv^{-1}=x_1^{n_1}\cdots x_k^{n_k}x_{l}^{-n_{l}}\cdots x_{k+1}^{-n_{k+1}}\in H. 
    \]
    Now note that $H$ is a subgroup of $G$ containing $X$. Thus 
    \[
    \langle X\rangle=\bigcap_{\substack{S\leq G\\X\subseteq S}}S\subseteq H.
    \]
\end{sol}

\begin{sol}{xca:union}
    Let $G=\Sym_3$. Then $H=\{\id,(12)\}$ and 
    $K=\{\id,(23)\}$ are subgroups of $G$. However, 
    $H\cup K=\{\id,(12),(23)\}$ is not a subgroup, as 
    $(12)(23)=(123)\not\in H\cup K$. 
\end{sol}

\begin{sol}{xca:permutation_matrix}
Let $\{e_1,\dots,e_n\}$ be the standard basis of $\R^n$. 
To prove this formula note that
\[
E_{i,j}e_k=\begin{cases}
    e_i&\text{if $j=k$,}\\
    0 & \text{if $j\ne k$}.
\end{cases}
\]
and verify that 
$P_\sigma e_k=\sum_{i=1}^n E_{\sigma(i),i}e_k$ 
for all $k\in\{1,\dots,n\}$. Since $P_\sigma$ and
$\sum_{i=1}^n E_{\sigma(i),i}$ coincide in a basis of $\R^n$, 
they are equal. 
\end{sol} 

\begin{sol}{thm:quotient}
Since $N$ is normal in $G$, the operation is well-defined. 
Routine calculations show that 
the operation is associative, that
$N$ is the neutral element of $G/N$ and that 
the inverse of an element $xN$ is 
$(xN)^{-1}=x^{-1}N$. For example, for the associativity, 
we note that for $x,y,z\in G$ one has 
\begin{align*}
    &((xN)(yN))(zN)=((xy)N)zN=(xy)zN,
\shortintertext{equals}
    &(xN)((yN)(zN))=(xN)((yz)N)=x(yz)N.
\end{align*}
since $x(yz)=(xy)z$.
\end{sol}

\begin{sol}{xca:commutator}
For $x,y\in G$,
\begin{align*}
    (xH)(yH)=(yH)(xH) \Longleftrightarrow (xy)H=(yx)H \Longleftrightarrow x^{-1}y^{-1}xy\in H.
\end{align*}
Thus $G/H$ is abelian if and only if  $[x,y]=xyx^{-1}y^{-1}\in H$ for all $x,y\in G$.
\end{sol}


\begin{sol}{xca:G/Z(G)}
Assume that $G/Z(G)$ is generated by $gZ(G)$. Let $x,y\in G$. Then 
$xZ(G)=g^kZ(G)$ and $yZ(G)=g^lZ(G)$ for some $k,l\in\Z$,  
that is 
$x=g^kz_1$ and $y=g^lz_2$ for some $k,l\in\Z$ y $z_1,z_2\in Z(G)$. Thus $xy=yx$.
\end{sol}

\begin{sol}{xca:index_p}
Lagrange's theorem immediately proves $1)\implies 2)$, 
as $|G/H|=p$. 

The implication $2)\implies 3)$ is trivial, as $p$ is a prime number. 

We now prove that $3)\implies 4)$. If $g^k\in H$ for
some $k\in\{2,\dots,p-1\}$, then, since 
$\gcd(k,n)=1$, there exist $r,s\in\Z$ such that 
$rk+sn=1$. Thus 
\[
g=g^1=g^{rk+sn}=(g^k)^r(g^n)^s\in H,
\]
a contradiction. 

Finally, we prove that $4)\implies 1)$. Let $x\in G\setminus H$ and $h\in H$. We claim that 
$xhx^{-1}\in H$. Let $y=xhx^{-1}$ and assume that 
$y\not\in H$. Then, by assumption, 
$y^k\not\in H$ for all 
$k\in\{1,2,\dots,p-1\}$. In particular,  
the cosets 
\[
H, yH, y^2H, . . . , y^{p-1}H
\]
are all different (because if $y^iH=y^jH$ for some $i<j$, then $y^{j-i}\in H$ and $j-i\leq p-2$). Since 
$y=xhx^{-1}$, 
\[
(yx)H = (xh)H= xH= y^iH
\]
for some $i\in\{0,\dots,p-1\}$. If $i=0$, 
then $yx= xh\in H$ and therefore $x\in H$, a contradiction. Hence $(yx)H= y^iH$ for some 
$i\in\{1,\dots,p-1\}$, which implies 
$xH=y^{i-1}H$. Therefore 
\[
y^iH= xH= y^{i-1}H
\]
for some $i\in\{1,\dots,p-2\}$, which implies that 
$y\in H$, a contradiction. 
\end{sol}

\begin{sol}{xca:p_smallest}
    If $g\in G\setminus H$, then $g^n=1\in H$, where $n=|G|$. Since $p$ is prime, $n$ has no prime divisors $<p$. By Exercise \ref{xca:index_p}, $H$ is normal in $G$.
\end{sol}

\begin{sol}{xca:HK_normal}
We first prove that
$HK\subseteq KH$. If $x=hk\in HK$, then
 $x=k(k^{-1}hk)\in KH$, as $k^{-1}hk\in H$. To prove 
that $HK\supseteq KH$, let $y=kh\in KH$. Then $y=(khk^{-1})k\in HK$, as  $khk^{-1}\in H$. 
\end{sol}

\begin{sol}{xca:U(Z/10)}
Just note that $\mathcal{U}(\Z/12)$ has no elements of order four.
\end{sol}

\begin{sol}{xca:p_groups}
    If $G$ is a $p$-group, then, by Lagrange's theorem, 
    every element has order a power of $p$. Conversely, 
    if $q$ is a prime divisor of $|G|$, by 
    Cauchy's theorem, there exists $g\in G$ of order $q$. Thus $q=p$.
\end{sol}


\begin{sol}{xca:factors:24,12,4,2}
Decompose $A$ as $(\Z/4)\times(\Z/2)\times(\Z/3)\times(\Z/8)\times(\Z/4)\times(\Z/3)$.
We list the highest powers appearing in our decomposition of $A$: 
\[ 
\begin{matrix}
8&3\\
4&3\\
4\\
2
\end{matrix} 
\] 
Then $s_1=2$, $s_2=4$, $s_3=12$ and $s_4=24$. Hence 
$A\simeq (\Z/24)\times(\Z/12)\times(\Z/4)\times(\Z/2)$.
\end{sol}


%\addcontentsline{lec}{chapter}{References}
\bibliographystyle{abbrv}
\bibliography{refs}


\printindex     
%\phantom{Trick}
%\addcontentsline{lec}{chapter}{\indexname}

\end{document}





