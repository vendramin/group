\chapter{}

\topic{Cauchy's theorem}

\index{Action!fixed points}
\index{Class equation}
Let $G$ be a group and
$X$ be a finite $G$-set. Then $X$ decomposes as a disjoint 
union of orbits. Let 
\[
\Fix(X)=\{x\in X:g\cdot x=x\text{ for all $g\in G$}\}
\]
be the set of \textbf{fixed points} of $X$. Gather  
the one-element orbits and apply cardinality. By the 
fundamental counting principle, 
\begin{equation}
\label{eq:clases}
|X|=|\Fix(X)|+\sum_{i=1}^k|G\cdot x_i|
=|\Fix(X)|+\sum_{i=1}^k(G:G_{x_i}),
\end{equation}
where the $x_j$'s are the representatives
of the orbits of size $\geq2$. Equality~\eqref{eq:clases} is extremely 
useful and is called the 
\textbf{class equation}.

\begin{example}
Let a finite group $G$ act on $G$ by conjugation. 
Then $\Fix(G)=Z(G)$ and 
\[
|G|=|Z(G)|+\sum_{i=1}^k(G:C_G(x_i)),
\]
for some $x_1,\dots,x_k\in G$ such that 
$(G:C_G(x_i))\geq2$ for all $i\in\{1,\dots,k\}$.
\end{example}

\begin{definition}
\index{$p$-group}
Let $p$ be a prime number. We say 
that $G$ is a \textbf{$p$-group} if $|G|=p^m$ for some $m\geq0$.
\end{definition}

\begin{theorem}
Let $p$ be a prime number and 
$G$ a $p$-group. If $\{1\}\ne N\unlhd G$, then
$N\cap Z(G)\ne\{1\}$.
\end{theorem}

\begin{proof}
Since $N$ is normal in $G$, $G$ acts on $N$ by conjugation. 
By the fundamental counting principle,
each orbit has prime-power size. Write 
\[
N=\underbrace{\mathcal{O}_1\cup\cdots\cup \mathcal{O}_k}_{\text{one-element orbits}}\cup\underbrace{\mathcal{O}_{k+1}\cup\cdots\cup\mathcal{O}_m}_{\text{orbits of size $>1$}},
\]
Since $N\cap Z(G)=\mathcal{O}_1\cup\cdots\cup\mathcal{O}_k$, 
the integers $k=|N\cap Z(G)|$ and $|N\setminus(N\cap Z(G))|$ are divisible by $p$. Thus 
\[
|N|\equiv|N\cap Z(G)|\bmod p.
\]
Since $1\in N\cap Z(G)$, then
$|N\cap Z(G)|>1$. In particular, $N\
cap Z(G)\ne\{1\}$.
\end{proof}

The following corollary follows immediately: 

\begin{corollary}
Let $p$ be a prime number and 
$G$ a $p$-group. Then 
$Z(G)\ne\{1\}$.
\end{corollary}

