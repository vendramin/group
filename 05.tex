\section{12/03/2024}

\subsection{Quotients}

If $G$ is a group and $N$ is a subgroup of $G$, we want to know
when the set $G/N$ of left cosets 
of $N$ in $G$ is a group with 
the operation 
\begin{equation}
\label{eq:operation}
G/N\times G/N\to G/N,\quad 
(xN,yN)\mapsto xyN,
\end{equation}
that is, when this operation 
is well-defined. 
What does this mean? We need to check that
\eqref{eq:operation} is indeed a function. 
For that purpose, we need to prove that
\eqref{eq:operation} does not depend on the representatives of left
cosets used. Thus we need to show that 
$xN=x_1N$ and $yN=y_1N$, then 
$xyN=x_1y_1N$. 

Let us try to understand this condition. If $x^{-1}x_1\in N$ and 
$y^{-1}y_1\in N$, then $x_1=xn$ and $y_1=ym$ for some 
$m,n\in N$. Thus 
\[
(xy)^{-1}(x_1y_1)=y^{-1}x^{-1}x_1y_1=y^{-1}nym\in N
\]
if and only if $y^{-1}ny\in N$.

\begin{example}
If $G=\Sym_3$ and $H=\langle (12)\rangle$, then $(xN,yN)\mapsto xyN$ is not a function. Recall that 
$G/H=\{H,(123)H,(132)H\}$, where 
$H=(12)H$, $(123)H=(13)H$ and $(132)H=(23)H$. Then 
\[
(132)N=(13)(23)N=(13)N(23)N=(123)N(132)N=N,
\]
a contradiction.
\end{example}

\begin{definition}
    \index{Subgroup!normal}
    Let $G$ be a group. 
    A subgroup $N$ of $G$ is said to be \textbf{normal} if $gNg^{-1}\subseteq N$ for all $g\in G$.
    Notation: If $N$ is normal in $G$, then $N\unlhd G$.
\end{definition}

In an abelian group, every subgroup is normal. 

\begin{proposition}
\label{pro:normalidad}
Let $N$ be a subgroup of $G$. 
The following statements are equivalent:
\begin{enumerate}
        \item $gNg^{-1}\subseteq N$ for all $g\in G$.
        \item $gNg^{-1}=N$ for all $g\in G$.
        \item $gN=Ng$ for all $g\in G$.
\end{enumerate}
\end{proposition}

\begin{proof}
We only prove that $1)\implies 2)$, as the other implications are trivial. If $n\in N$ and $g\in G$, then 
$n=g(g^{-1}ng)g^{-1}\in gNg^{-1}$.
\end{proof}

\begin{proposition}
    Let $N$ be a subgroup of $G$. The following statements
    are equivalent: 
    \begin{enumerate}
        \item $N$ is normal in $G$.
        \item $(gN)(hN)=(gh)N$ for all $g,h\in G$.
    \end{enumerate}
\end{proposition}

\begin{proof}
   We first prove that $1)\implies 2)$. Let $g\in G$. Since
   $gNg^{-1}=N$, 
   \[
   (gN)(hN)=g(Nh)N=g(hN)N=(gh)N.
   \]
   
   We now prove that $2)\implies1)$. If $g\in G$, then
    \[
    gNg^{-1}\subseteq (gN)(g^{-1}N)=(gg^{-1})N=N.\qedhere
    \]
\end{proof}

If $G$ is a group, then 
$\{1\}$ and $G$ are always normal subgroups. 

\begin{example}
\index{Center!of a group}
If $G$ is a group, then $Z(G)$ is a normal subgroup of $G$. Moreover, 
if $N\leq Z(G)$, then $N\unlhd
G$.
\end{example}

\begin{example}
If $G$ is a group, then  $[G,G]$ is a normal subgroup of $G$. If 
$x\in [G,G]$ and $g\in G$, then
$gxg^{-1}=(gxg^{-1}x^{-1})x=[g,x]x\in [G,G]$. Alternatively,
\[
g\left(\prod_{i=1}^k[x_i,y_i]\right)g^{-1}=\prod_{i=1}^k [gx_ig^{-1},gy_ig^{-1}]
\]
for all $g,x_1,\dots,x_k,y_1,\dots,y_k\in G$.
\end{example}

\begin{example}
Let $n\geq2$. Then 
$\Alt_n$ is a normal subgroup of $\Sym_n$.
If $\sigma\in\Alt_n$ and $\tau\in\Sym_n$, then $\tau\sigma\tau^{-1}\in\Alt_n$, as 
\[
\sgn(\tau\sigma\tau^{-1})=\sgn(\sigma)=1.
\]
\end{example}

\begin{example}
If $N$ is a subgroup of $G$ such that $(G:N)=2$, then $N$ is normal in $G$. We need to show that $gN=Ng$ for all $g\in G$. Let $g\in G$. 
If $g\in N$, then $gN=Ng$. If $g\not\in N$, then
$gN\ne N$. Since $(G:N)=2$, we can decompose $G$ as 
the disjoint union $G=N\cup gN$. Hence 
$gN=G\setminus N$. Similarly, 
$Ng=G\setminus N$ and therefore $gN=Ng$.
\end{example}

\begin{example}
As a particular case of the previous example, 
$\langle (123)\rangle\unlhd\Sym_3$. Note that
$\langle (12)\rangle$ is not normal in $\Sym_3$.  
For example, $(13)(12)(13)=(23)\not\in\langle(12)\rangle$.
\end{example}

\begin{example}
The subgroup $\SL_n(\R)$ is normal in $\GL_n(\R)$. If $g\in\GL_n(\R)$ and $x\in\SL_n(\R)$, then $\det(gxg^{-1})=(\det g)(\det x)(\det g)^{-1}=1$.
\end{example}

\begin{example}
\index{Klein group}
The Klein group $K=\{\id,(12)(34),(13)(24),(14)(23)\}$ is normal in
$\Sym_4$. We need to show that 
$\sigma K\sigma^{-1}\subseteq K$ for all $\sigma\in\Sym_4$. Do we need to check this for every element of $\Sym_4$? No. One always has tricks! 
Recall that $\Sym_4$ is generated by $(12)$ and $(1234)$. Since
every element of  $\Sym_4$ is a word on $(12)$ and 
$(1234)$,
it is enough to see that
$\sigma K\sigma^{-1}\subseteq K$ for all $\sigma\in\{(12),(1234)\}$. We left as an exercise to show that 
\[
(12)K(12)^{-1}\subseteq K,\quad
(1234)K(1234)^{-1}\subseteq K.
\]
\end{example}

\begin{exercise}
\index{Semi-direct product}
Let $G=\R\times\R^\times$ with the operation 
\[
(x,y)(u,v)=(x+yu,yv).
\]
Prove that $\{(x,1):x\in\R\}$ is normal in $G$ and that
$\{(0,y):y\in\R^\times\}$ is not. 
\end{exercise}

Let us compute the list of normal subgroups of $\Alt_4$.

\begin{example}
\index{Normal subgroups of $\Alt_4$}
We claim that 
$\{\id\}$, $K=\{\id,(12)(34),(13)(24),(14)(23)\}$ and $\Alt_4$ 
are the normal subgroups of $\Alt_4$.

Since $\Alt_4=\{\text{3-cycles}\}\cup K$, $K$ is the only subgroup
of $\Alt_4$ of order four. This implies that $K$ is normal in $\Alt_4$ (because every conjugate $gKg^{-1}$ of $K$ is a subgroup of 
$\Alt_4$ of order four). Let $N\ne\{\id\}$ be a normal subgroup of 
$\Alt_4$. 

If $N$ contains a 3-cycle, say 
$(abc)\in N$, then 
\[
(acd)=(bcd)(abc)(bcd)^{-1}\in N
\]
and hence $N=\Alt_4$ (because $N$ contains every 3-cycle). 

Assume that $N$ does not contain 3-cycles. 
Then some non-trivial element of $K$ belongs to $N$, say 
$(ab)(cd)\in N$. Hence 
\[
(ac)(bd)=(bcd)(ab)(cd)(bcd)^{-1}\in N,\quad
(ad)(bc)=(ab)(cd)(ac)(bd)\in N
\]
and therefore $N=K$.
\end{example}

Normality is not transitive. 

\begin{exercise}
\index{Dihedral group}
Let $G=\D_4$ be the dihedral group of order eight. Let $N=\langle s,r^2\rangle$ and $H=\langle s\rangle$.
Prove that $H$ is normal in $N$, $N$ is normal in $G$ but $H$ is not normal in $G$. 
\end{exercise}

\begin{example}
\index{Normal subgroups!of $\Sym_4$}
We claim that $\{\id\}$, $K$, $\Alt_4$ y $\Sym_4$ are the normal subgroups of $\Sym_4$.

Let $N$ be a normal subgroup of $\Sym_4$. If $N\subseteq\Alt_4$, then
$N$ is normal in $\Alt_4$ and hence either $N=\{\id\}$,
$N=K$ or $N=\Alt_4$. Assume that $N\not\subseteq\Alt_4$, that is
$N$ contains an odd permutation. If $\sigma\in\Sym_4$ is odd, then 
$\sigma$ is either a transposition or a 4-cycle. 
 
If $N$ contains a transposition, then all transpositions 
belong to $N$, as 
\[
\tau(ij)\tau^{-1}=(\tau(i)\,\tau(j))
\]
for all $\tau\in\Sym_4$. In this case, $N=\Sym_4$ because 
the transpositions generate $\Sym_4$. 


If $N$ contains a 4-cycle, all 4-cycles belong to $N$, as 
\[
\tau(ijkl)\tau^{-1}=(\tau(i)\,\tau(j)\,\tau(k)\,\tau(l))
\]
for all $\tau\in\Sym_4$ and $K\subseteq N$ because 
\[
(ac)(bd)=(abcd)^2.
\]
This implies that $|N|\geq10$. Since $K\subseteq N$,  $|N\cap\Alt_4|\geq 5$. Moreover, $N\cap\Alt_4$ is a normal subgroup of $\Alt_4$.
Hence $N\cap\Alt_4=\Alt_4\subseteq N$. Therefore $N=\Sym_4$.
\end{example}

The following theorem is crucial.

\begin{theorem}
\index{Quotient group}
\label{thm:quotient}
If $N$ is a normal subgroup of $G$, then
$G/N$ is a group with the operation 
$(xN)(yN)=(xy)N$.
\end{theorem}

\begin{exercise}
    Prove Theorem \ref{thm:quotient}.   
\end{exercise}


We will see examples of quotient groups later. 

\begin{exercise}
\label{xca:commutator}
Let $H$ be a normal subgroup of $G$. Prove that
$G/H$ is abelian if and only if $[G,G]\subseteq H$.
\end{exercise}

As an application, 
we compute the commutator subgroup of $\Alt_4$. 

\begin{example}
\index{Commutator!of $\Alt_4$}
$[\Alt_4,\Alt_4]=K=\{\id,(12)(34),(13)(24),(14)(23)\}$.
We know that $K$ is normal in $\Alt_4$. Since $\Alt_4/K$ has order three, 
it is abelian. Then $[\Alt_4,\Alt_4]\subseteq K$. Since 
\[
(ab)(cd)=[(abc),(cda)],
\]
we conclude that $K\subseteq[\Alt_4,\Alt_4]$.
\end{example}

\begin{exercise}
\label{xca:G/Z(G)}
If $G/Z(G)$ is cyclic, then $G$ is abelian.
\end{exercise}

\begin{exercise}
\label{xca:normalizer}
\index{Normalizer}
If $S$ is a subgroup of $G$, the \textbf{normalizer} of $S$ in $G$
is the set  
\[
N_G(S)=\{g\in G:gSg^{-1}=S\}.
\]
Prove the following statements:
\begin{enumerate}
\item $N_G(S)\leq G$. 
\item $S\unlhd N_G(S)$.
\item If $S\leq T\leq G$ and $S\unlhd T$, then $T\leq N_G(S)$.
\end{enumerate}
\end{exercise}

The normalizer of a subgroup  $S$ in $G$ is the largest 
subgroup of $G$ that contains $S$ as a normal subgroup. 

\begin{definition}
\index{Group!simple}
A group $G$ is \textbf{simple} if $G\ne\{1\}$ and 
$G$ and $\{1\}$ are the only normal subgroups of $G$. 
\end{definition}

If $p$ is a prime number, then Lagrange's theorem implies that 
$\Z/p$ is a simple group. 
For $n\geq5$, the alternating group $\Alt_n$ is simple. However, 
we will not
prove this in this course. 

\begin{exercise}
\label{xca:index_p}
Let $H$ be a subgroup of $G$ such that $p=(G:H)$ is a prime number. 
Prove that the following statements are equivalent:
\begin{enumerate}
\item $H$ is normal in $G$.
\item If $g\in G\setminus H$, then $g^p\in H$.
\item If $g\in G\setminus H$, then $g^n\in H$ for some $n$ with no prime divisors $<p$.
\item If $g\in G\setminus H$, then $g^k\not\in H$ for all $k\in\{2,\dots,p-1\}$.
\end{enumerate}
\end{exercise}

We now present two applications of the previous exercise. 

\begin{exercise}
\label{xca:p_smallest}
    Let $G$ be a finite group and $p$ be the smallest prime number dividing $G$. Prove that if $H$ is a subgroup 
    of $G$ with $(G:H)=p$, then $H$ is normal in $G$. 
\end{exercise}

\begin{exercise}
Let $p$ be a prime number and $G$ be a group such that
every element of $G$ has order a power of $p$. 
If 
$H$ is a subgroup of $G$ of index $p$, then $H$ is normal in $G$.
\end{exercise}

