\documentclass[12pt]{amsproc}

\newcommand{\course}{Group theory}

\input{mystuff}
\usetikzlibrary{calc}

%\usepackage{mathptmx}
%\usepackage{newtxtext}

\usetikzlibrary{positioning}

\begin{document}

\begin{abstract}
    The notes correspond to the bachelor 
    course \textbf{Group Theory} of the 
    Vrije Universiteit Brussel, 
    Faculty of Sciences, 
    Department of Mathematics and Data Sciences. 
\end{abstract}

\maketitle

\setcounter{tocdepth}{1}
\tableofcontents 

\begin{figure}[h]
    \includegraphics[scale=0.2]{images/VUB.jpg}
\end{figure}

\thispagestyle{plain}

\section*{Introduction}

The notes correspond to the bachelor 
course \textbf{Group Theory} of the 
Vrije Universiteit Brussel, 
Faculty of Sciences, 
Department of Mathematics and Data Sciences. The course
is divided into twelve two-hour lectures. 



The material is somewhat standard. Basic texts on abstract algebra
are for example \cite{MR1129886}, \cite{MR2286236} and \cite{MR600654}. 
Lang's book \cite{MR783636} is also a standard reference, but 
maybe a bit more advanced. 

We also mention a set of 
\href{https://kconrad.math.uconn.edu/blurbs/}{great expository papers} by Keith Conrad. 
The notes are extremely well-written and are useful at  
every stage of a mathematical career. 

The notes include many exercises, some with full detailed solutions. Mandatory exercises have a \colorbox{green!5!white}{green background}, while optional ones (bonus exercises) have a \colorbox{yellow!15!white}{yellow background}.
The notes also include some additional comments. While these are entirely optional, I hope they offer further insight. They are highlighted with a \colorbox{red!5!white}{pink background}.


The notes include Magma code, which we use to verify examples and offer alternative solutions to certain exercises. Magma \cite{zbMATH01077111} is a powerful software tool designed for working with algebraic structures. There is a free \href{https://magma.maths.usyd.edu.au/calc/}{online} version of Magma available.


% Bibtex information:
% {\footnotesize\begin{verbatim}
% @misc{rings,
%     author={Vendramin, L.},
%     title={Rings and modules},
%     year={2022},
%     note={Available at www.github.com/vendramin/rings},
%     pages={106}
% }
% \end{verbatim}}

 Thanks go to Heleen Broodcoorens, 
 Arnaud De Ridder,
 Davide Ferri,
 Daya Huybrechts, 
 and 
 Senne Trappeniers. 
 

This version 
was compiled on \today~at~\currenttime. 
Please send comments and corrections to me at \url{Leandro.Vendramin@vub.be}. 


% \bigskip
% \begin{flushright}
% Leandro Vendramin\\Brussels, Belgium\par
% \end{flushright}

\thispagestyle{plain}
\section*{List of topics}

\contentsline {section}{\tocsubsection {\S }{1.1}{Groups}}{4}{subsection.1.1}%
\contentsline {section}{\tocsubsection {\S }{2.1}{Subgroups}}{9}{subsection.2.1}%
\contentsline {section}{\tocsubsection {\S }{2.2}{Subgroups of $\mathbb {Z}$}}{11}{subsection.2.2}%
\contentsline {section}{\tocsubsection {\S }{2.3}{Commutators}}{13}{subsection.2.3}%
\contentsline {section}{\tocsubsection {\S }{2.4}{Cyclic groups}}{14}{subsection.2.4}%
\contentsline {section}{\tocsubsection {\S }{3.1}{Lagrange's theorem}}{17}{subsection.3.1}%
\contentsline {section}{\tocsubsection {\S }{4.1}{The symmetric group}}{21}{subsection.4.1}%
\contentsline {section}{\tocsubsection {\S }{5.1}{Quotients}}{26}{subsection.5.1}%
\contentsline {section}{\tocsubsection {\S }{6.1}{Permutable subgroups}}{30}{subsection.6.1}%
\contentsline {section}{\tocsubsection {\S }{6.2}{Homomorphisms}}{31}{subsection.6.2}%
\contentsline {section}{\tocsection {\S }{7.1}{Isomorphism theorems}}{36}{subsection.7.1}%
\contentsline {section}{\tocsubsection {\S }{8.1}{Semi-direct products}}{43}{subsection.8.1}%
\contentsline {section}{\tocsubsection {\S }{9.1}{Actions of groups on sets}}{49}{subsection.9.1}%
\contentsline {section}{\tocsubsection {\S }{10.1}{$p$-groups}}{56}{subsection.10.1}%
\contentsline {section}{\tocsubsection {\S }{10.2}{Cauchy's theorem}}{57}{subsection.10.2}%
\contentsline {section}{\tocsubsection {\S }{11.1}{Sylow's theorems}}{59}{subsection.11.1}%
\contentsline {section}{\tocsubsection {\S }{12.1}{More about Sylow's theorems}}{64}{subsection.12.1}%
\contentsline {section}{\tocsubsection {\S }{12.2}{Abelian groups}}{65}{subsection.12.2}%

\tableofcontents 
\bigskip 

\chapter{}

\topic{Groups}

Before defining groups, we recall that a binary operation on a set $X$
is simply a map 
\[
X\times X\to X,
\quad (x,y)\mapsto xy.
\]
Note that we have used 
juxtaposition to denote this generic binary operation. For example,
$(x,y)\mapsto x-y$ is a binary operation in $\Z$ but not, for example, 
in $\Z_{\geq 0}$.

\begin{definition}
\index{Group}
A \textbf{group} is a non-empty set $G$ with a binary operation
$G\times G\to G$, $(x,y)\mapsto xy$, such that
the following properties hold:
\begin{enumerate}
    \item (Associativity) $(xy)z=x(yz)$ for all $x,y,z\in R$.
    \item (Existence of a neutral element) There exists $e\in G$ such that $xe=ex=x$ for all $x\in G$.
    \item (Existence of inverses) For every $x\in G$ there exists $y\in G$ such that $xy=yx=e$.
\end{enumerate}
\end{definition}

The associativity condition implies
that all ordered products that we can form with 
the elements, say, $x_1,x_2,\dots,x_n$ will be equal. For example,
\[
(x_1x_2)((x_3x_4)x_5)=x_1(x_2(x_3(x_4x_5)))
\]
and hence we can write, without ambiguity (and without using brackets), 
$x_1x_2\cdots x_5$. This fact can be proved by induction; see for example
Lang's book. We will provide an alternative proof as an application
of Cayley's theorem. 

\begin{proposition}
    In a group $G$, every element $x\in G$ admits a unique inverse.
\end{proposition}

\begin{proof}
    Let $y,z\in G$ be inverses of $x\in G$. Then 
    $z=z(xy)=(zx)y=ey=y$.
\end{proof}

\begin{exercise}
    Prove that the neutral element of a group is unique. 
\end{exercise}

In general, when the binary operation is written multiplicatively, one
writes the identity element $e$ of a group as $1_G$ or simply as $1$. The inverse of $x$ will be 
denoted by $x^{-1}$. 

\begin{example}
    Let $n\geq1$. The set $\GL_n(\R)$ of $n\times n$ invertible real 
matrices forms a
    group with the usual matrix multiplication.  
\end{example}

It is a good idea to keep in mind the \emph{group of invertible matrices}. 
With this, the 
the following properties look familiar:
\begin{enumerate}
    \item $(x^{-1})^{-1}=x$ for all $x$.
    \item $(xy)^{-1}=y^{-1}x^{-1}$ for all $x,y$. 
\end{enumerate}

\begin{exercise}
    Prove that in a group, the equation $ax=b$ has 
    a unique solution $x=a^{-1}b$. Similarly, the equation
    $x=ba^{-1}$ is the unique solution of the equation
    $xa=b$. 
\end{exercise}

\begin{definition}
    \index{Group!abelian}
    A group $G$ is \textbf{abelian} if $xy=yx$ for all $x,y\in G$. 
\end{definition}

Most of the time, for abelian groups we will use 
the \emph{additive notation}. This means that the binary operation
of the group will be denoted by $(x,y)\mapsto x+y$, the neutral
element by $0$ and 
the inverse of an element $x$ will be $-x$. 

\begin{definition}
    \index{Group!order}
    The \textbf{order} $|G|$ of a group $G$ is the 
    cardinality of $G$. A group $G$ is said to be
    \textbf{finite} if $|G|$ is finite and \textbf{infinite}
    otherwise. 
\end{definition}

\begin{example}
\label{exa:abelian_groups}
    Let us see some 
    abelian groups: 
    \begin{enumerate}
        \item $\Z$, $\Q$, $\R$ and $\C$ with the usual addition. 
        \item Let $n\geq2$. The set $\Z/n$ of integers modulo $n$ with the usual addition modulo $n$.
        \item $\Q\setminus\{0\}$, $\R\setminus\{0\}$ and $\C\setminus\{0\}$ 
        with the usual multiplication.
        \item Let $p$ be a prime number. The set $(\Z/p)^{\times}=(\Z/p)\setminus\{0\}$ of invertible integers modulo $p$ 
            with the usual multiplication modulo $p$. 
    \end{enumerate}
\end{example}

The groups of the first two items will be written in additive notation. 

The group $\Z/n$ of integers modulo $n$ is a finite group of order $n$. 
The group $(\Z/p)^{\times}$ of units modulo $p$ is a finite
group of order $p-1$. The other groups of Example \ref{exa:abelian_groups} are infinite groups. 

\begin{exercise}
\label{xca:LR}
Let $G$ be a group and $g\in G$. Prove that 
the maps $L_g\colon G\to G$, $x\mapsto gx$, and 
$R_g\colon G\to G$, $x\mapsto xg$, are bijective. 
\end{exercise}

\index{Group!table}
Let $G=\{g_1,g_2,\dots,g_n\}$ be a finite group. The \textbf{table} 
of $G$ is the matrix 
that in position $(i,j)$ has the element $g_ig_j$. 
For example, the table of 
the additive group  
$\Z/4$ of integers modulo 4 is the following:
\begin{center}
  \begin{tabular}{l|cccc}
     &0&1&2&3 \\
    \hline
    0 & 0 & 1 & 2 & 3\\
    1 & 1 & 2 & 3 & 0\\
    2 & 2 & 3 & 0 & 1\\
    3 & 3 & 0 & 1 & 2
  \end{tabular}
\end{center}

We know that $\Z$ is a group with the usual addition. 
We now discuss a multiplicative version of this group, as it will be
very important later. We first need a little bit of notation. 
Let $G$ be a group and $g\in G$. For $k\in\Z\setminus\{0\}$, we write 
\begin{align*}
                & g^k=g\cdots g\quad (k-\text{times}) && \text{if $k>0$},\\
                & g^k=g^{-1}\cdots g^{-1}\quad (|k|-\text{times}) && \text{if $k
<0$}.
\end{align*}
By convention, $g^0=1$.
The following facts are left as an exercise: 
        \begin{enumerate}
                \item $(g^k)^l=g^{kl}$ for all $k,l\in\Z$.
                \item If $G$ is abelian, then $(xy)^k=x^ky^k$ for all $x,y\in G$
                        and $k\in\Z$.
        \end{enumerate}

\begin{example}
\label{exa:cyclic}
Fix a formal symbol $g$. Consider the set  
\[
\langle g\rangle=\{g^k:k\in\Z\}
\]
of integers powers of $g$ (with the usual convention $g^0=1$). Then
$\langle g\rangle$
with the operation $g^ig^j=g^{i+j}$ is an abelian group. 
\end{example}

We will see later that $\Z$ and the group of 
Example \ref{exa:cyclic} are ``indistinguishable'' 
as groups, even if they appear to be completely different. 

\begin{example}
Let $n$ be a positive integer. The set 
$G_n=\{z\in\C:z^n=1\}$ is an abelian group with
the usual multiplication. Moreover, the set 
$\cup_{n\geq1}G_n$ is an abelian group. 
\end{example}

\begin{example}
    Let $X$ be a set. The set $\Sym_X$ of bijective maps $X\to X$ 
    is a group with the usual composition of maps. If $|X|\geq3$, the group
    $\mathbb{S}_{X}$ is non-abelian. To prove this, let $a,b,c\in
        X$ be three different elemants. Let $f\colon X\to X$ be such that
        $f(a)=b$, $f(b)=c$ and $f(c)=
a$ and $g\colon X\to
        X$ be such that $g(a)=b$, $g(b)=a$ and $g(x)=x$ for all $x\in
        X\setminus\{a,b\}$.  Then $fg\ne gf$.
\end{example}

\index{Permutation}
\index{Group!symmetric}
If $X=\{1,2,\dots,n\}$, the group $\Sym_X$ will be written as $\Sym_n$. This is
the \textbf{symmetric group} of degree $n$. The elements of $\Sym_n$ are
called \textbf{permutations} of degree $n$. 
Note that $|\Sym_n|=n!$ and $\Sym_n$
is abelian if and only if $n\in\{1,2\}$. Each element of $\Sym_n$ is a 
bijective map 
$f\colon\{1,\dots,n\}\to \{1,\dots,n\}$. To denote permutations, 
we can use the following convention. The symbol  
\[
\binom{12345}{32145}
\]
denotes the map 
$f\colon\{1,2,3,4,5\}\to\{1,2,3,4,5\}$ such that 
\[
f(1)=3,
\quad f(2)=2,
\quad f(3)=1, 
\quad f(4)=4,
\quad f(5)=5.
\]


\begin{example}[Klein group]
\index{Klein group}
The set  
\[
K=\left\{ \mathrm{id},\binom{1234}{2143},\binom{1234}{3412},\binom{1234}{4321}\right\}
\]
together with the usual composition of maps is an abelian group. 
Note that $K$ is included in $\mathbb{S}_{4}$.
Can you compute the table of this group?
\end{example}

Every permutation can be written as a product of disjoint cycles. The fact 
is proved by induction, but is rather intuitive. Let us 
decompose the permutation 
\[
\sigma=\binom{123456789}{638915724}\in\Sym_9
\]
as a product of cycles. We just need to 
draw a picture for $\sigma$:


\begin{example}
\label{exa:S3}
\index{Symmetric group $\Sym_3$}
The set $\Sym_3$ of bijective maps $\{1,2,3\}\to\{1,2,3\}$
together with the composition of maps is a group of order six. 
Its elements are the permutations 
        \[
        \id,\binom{123}{213},\binom{123}{321},\binom{123}{132},\binom{123}{231},
\binom{123}{312}.
        \]
        There is a handy way of writing permutations. It is based on \emph{decomposing
        permutations as a product of disjoint cycles}. 
        In this particular case, the elements of $\Sym_3$ 
        are        
        \[
                \id,(12),(13),(23),(123),(132),
        \]
        where, for example, the symbol $(12)$ represents 
        the map $\{1,2,3\}\to\{1,2,3\}$ such that 
        $1\mapsto 2$, $2\mapsto 1$ and $3\mapsto 3$. Can you construct the table of this group? 
\end{example}

\begin{example}
\label{exa:another_S3}
The set of maps 
\[
G=\left\{x,\frac{1}{x},1-x,\frac{1}{1-x},\frac{x}{x-1},\frac{x-1}{x}\right\}
\]
is a non-abelian group with the usual composition of maps. 
Let $\overline{\R}=\R\cup\{\infty\}$ (here $\infty$ is just a symbol) 
and assume that the following rules hold: 
\[
1/\infty=0,
\quad 1/\infty,
\quad \infty/\infty=1,
\quad 1-\infty=\infty-1=\infty.
\]
Then
$G$ is the set of bijective maps 
$\{0,1,\infty\}\to\{0,1,\infty\}$. For example, 
the map $x\mapsto\frac{1}{x}$ can be identified with the permutation of
the set 
$\{0,1,\infty\}$ that permutes $0$ and $\infty$ and fixes $1$. Similarly, 
$\frac{1}{1-x}$ permutes the elements $\{0,1,\infty\}$ cyclically 
in the following way:
\[
\begin{tikzcd}
        0 \\
        & 1 \\
        \infty
        \arrow[from=1-1, to=2-2]
        \arrow[from=2-2, to=3-1]
        \arrow[from=3-1, to=1-1]
\end{tikzcd}\]
\end{example}

We will see later that the 
groups of Examples \ref{exa:S3} and \ref{exa:another_S3} are
indeed ``indistinguishable'' as groups. 

\begin{example}
Let $n\geq2$. The units of $\Z/n$ form a group 
with the usual multiplication modulo $n$. We will use the following notation:
\[
\mathcal{U}(\Z/n)=\{x\in\Z/n:\gcd(x,n)=1\}.
\]
The order of $\mathcal{U}(\Z/n)$ is $\varphi(n)$, where $\varphi$ 
is the Euler's function, that is 
\[
\varphi(n)=|\{x\in\Z:1\leq x\leq n,\,\gcd(x,n)=1\}|.
\]

Let us discuss a concrete example. The table of 
$\mathcal{U}(\Z/8)=\{1,3,5,7\}$
is
\begin{center}
  \begin{tabular}{l|cccc}
     &1&3&5&7 \\
    \hline
    1 & 1 & 3 & 5 & 7\\
    3 & 3 & 1 & 7 & 5\\
    5 & 5 & 7 & 1 & 3\\
    7 & 7 & 5 & 3 & 1
  \end{tabular}
\end{center}
\end{example}

\begin{exercise}
\label{exa:direct_product}
        \index{Direct product!of groups}
        Let $G$ and $H$ be groups. Prove that 
        the set 
        $G\times H$
        of pairs $(g,h)$, where $g\in G$ and 
        $h\in H$, is a group with
        the operation
        \[
                (g,h)(g_1,h_1)=(gg_1,hh_1).
        \]
        This group is called the 
        \textbf{direct product} of $G$ and $H$.
\end{exercise}

The construction of Example \ref{exa:direct_product}
can be easily generalized to the product of 
three or more groups. 

\section{22/02/2024}

\subsection{Subgroups}

\begin{definition}
\index{Subgroup}
    Let $G$ be a group. 
        A subset $S$ of $G$ is said to be a \textbf{subgrup} of $G$ 
        if the following properties are satisfied:
        \begin{enumerate}
                \item $1\in S$,
                \item $x\in S\implies x^{-1}\in S$, and 
                \item $x,y\in S\implies xy\in S$.
        \end{enumerate}
        Notation: $S$ is a subgroup of $G$ if and only if $S\leq G$.
\end{definition}

The first condition of the definition can be replaced by
the following condition: $S\ne\emptyset$. Why? 

\begin{example}
    If $G$ is a group, then 
    $\{1\}$ and $G$ are always subgroups of $G$. 
\end{example}

The subgroup $\{1\}$ is known as the \textbf{trivial subgroup} of $G$. 
A subgroup $S$ of $G$ is said to be \textbf{proper} if $S\ne G$. 

\begin{example}
Write $2\Z=\{2m:m\in\Z\}$ to denote the set of even integers. Then 
\[
2\Z\leq\Z\leq\Q\leq\R\leq\C
\]
is a chain of subgroups. 
\end{example}

\begin{example}
$S^1=\{z\in\C:|z|=1\}\leq\C^\times=\C\setminus\{0\}$.
\end{example}

\begin{example}
Let $n\geq1$. Then $G_n=\{z\in\C:z^
n=1\}$ is a subgroup of $\C^\times$. 
Note that
\[
G_n=\{1,\exp(2\pi i/n),\exp(4i\pi/n),\dots,\exp(2(n-1)i\pi/n)\}.
\]
and 
\[
G_n\leq\bigcup_{n\geq1}G_n\leq S^1\leq\C^\times.
\]
\end{example}

\begin{exercise}
\label{xca:center}
        \index{Center!of a group}
        Let $G$ be a group. Prove that the \textbf{center} 
        \[
                Z(G)=\{g\in G:gh=hg\text{ for all $h\in G$}\}
        \]
        of $G$ is a subgroup of $G$. 
\end{exercise}

\index{Centralizer!of an element}
One can prove that, if $G$ is a group and $g\in G$, then 
the \textbf{centralizer} 
\[
C_G(g)=\{h\in G:gh=hg\}
\]
is a subgroup of $G$. Moreover, 
$Z(G)=\cap_{g\in G}C_G(g)$. 

\begin{exercise}
        \index{Conjugate of a subgroup}
        \label{xca:conjugate}
        Let $S$ be a subgroup of $G$ and $g\in G$. Prove that
        the \textbf{conjugate} $gSg^{-1}$
        of $S$ by $g$ is a subgroup of $G$. 
        Notation: $\prescript{g}{}S=gSg^{-1}$.
\end{exercise}


\begin{exercise}
\label{xca:center_S3}
\index{Center!of $\Sym_3$}
        Prove that $Z(\Sym_3)=\{\id\}$ and compute $C_{\Sym_3}((12))$.
\end{exercise}

The following exercise is useful: 

\begin{exercise}
\label{xca:subgroup}
        Let $G$ be a group and $S$ be a subset of $G$. 
        Prove that $S$ is a subgroup of $G$ if and only if 
        $S\ne\emptyset$ and for all $x,y\in S$ one has
        $xy^{-1}\in S$.
\end{exercise}

Use the previous exercise and
the fact that the determinant is a multiplicative function
to solve the following problem:

\begin{exercise}
\label{xca:SL_subgroup}
Prove that $\SL_n(\R)=\{a\in\GL_n(\R):\det(a)=1\}\leq\GL_n(\R)$. 
\end{exercise}

\begin{exercise}
\label{xca:intersection}
    Prove that the intersection of subgroups is again a subgroup.
\end{exercise}

The previous exercise is easy but crucial. We need it 
to construct subgroups generated 
by a given subset of elements. 

\begin{definition}
        \index{Subgroup!generated by a subset}
        Let $G$ be a group and $X$ a subset of $G$. The \textbf{subgroup
        generated} by $X$ is the smallest subgroup of $G$ that contains
        $X$, that is 
        \[
            \langle X\rangle=\bigcap\{S:S\leq G,X\subseteq S\}.
        \]
\end{definition}

Why this is the smallest subgroup that contains $X$? 
Let $H\leq G$ be such that 
$X\subseteq H$. Since $H$ is one of the subgroups appearing
in the intersection, 
\[
        \langle X\rangle=\bigcap\{S:S\leq G,X\subseteq S\}\subseteq H.        
\]

We will use the following notation:
If $X=\{g_1,\dots,g_k\}$, then 
\[
\langle
X\rangle=\langle\{g_1,\dots,g_k\}\rangle=\langle g_1,\dots,g_k\rangle.
\]

\begin{exercise}
\label{xca:generated}
Prove that 
\[
    \langle X\rangle=\{x_1^{n_1}\cdots x_k^{n_k}:k\geq0,\,x_1,\dots,x_k\in X,\,-1\leq n_1,\dots,n_k\leq 1\}.
\]
\end{exercise}

The previous exercise shows that
the subgroup generated by, say, the elements 
$x_1,\dots,x_n$ is nothing but the 
group formed by (some) words on the letters 
$x_1,\dots,x_n$ and their inverses 
$x_1^{-1},\dots,x_n^{-1}$. 

\begin{example}
Let $n\geq3$. Let 
\[
r=\begin{pmatrix}
\cos(2\pi/n) & -\sin(2\pi/n)\\
\sin(2\pi/n) & \cos(2\pi/n)
\end{pmatrix},
\quad
s=\begin{pmatrix}
        1 & 0\\
        0 & -1
\end{pmatrix}.
\]
The \textbf{dihedral group} $\D_n$ is the subgroup of
$\GL_2(\C)$ generated by $r$ and $s$,
that is $\D_n=\langle r,s\rangle$. A direct calculation shows that 
\[
r^n=s^2=\begin{pmatrix}
        1&0\\
        0&1
\end{pmatrix},
\quad
srs=r^{-1}.
\]

An element of $\D_n$ is a word of the form 
\[
r^{i_1}s^{j_1}r^{i_2}s^{j_2}\cdots
\]
for some 
 $i_1,i_2,\dots\in\{0,1,\dots,n-1\}$ and 
$j_1,j_2,\dots\in\{0,1\}$. Since $rs=sr^{-1}$, we conclude that
every element of $\D_n$ can be written as $r^is^j$ 
for some $i\in\{0,\dots,n-1\}$ and $j\in\{0,1\}$. In particular, 
$|\D_n|=2n$.
\end{example}

To understand better the previous example, 
we discuss two concrete particular cases. 
If $n=3$, 
\[
r=\begin{pmatrix}
-1/2 & -\sqrt{3}/2\\
\sqrt{3}/2 & -1/2
\end{pmatrix},
\quad
s=\begin{pmatrix}
        1 & 0\\
        0 & -1
\end{pmatrix}.
\]
and we obtain (another representation of) the group of symmetries of a regular 
triangle. 
If 
$n=4$, 
\[
r=\begin{pmatrix}
0 & -1\\
1 & 0
\end{pmatrix},
\quad
s=\begin{pmatrix}
        1 & 0\\
        0 & -1
\end{pmatrix}.
\]
and we obtain (another representation of) the group of symmetries of the square. 

\begin{exercise}
\label{xca:union}
        The union of subgroups is not, in general, 
        a subgroup. Can you give an example? 
\end{exercise}

\subsection{Subgroups of $\Z$}

What can we say about the subgroups of $\Z$? 

\begin{theorem}
        \label{thm:Z}
        If $S$ is a subgroup of $\Z$, then
                $S=m\Z=\{mx:x\in \Z\}$
                for some $m\geq0$.
        \end{theorem}
        
        \begin{proof}
                If $S=\{0\}$, take $m=0$. 
                Assume now that $S\ne\{0\}$. Let 
                $m=\min\{s\in S:s>0\}$. 
                Why does this $m$ exist?  
                Since $S\ne\{0\}$,  
                it contains 
                an element $n\in S\setminus\{0\}$. 
                There are then two possible cases: 
                $n>0$ or $-n>0$. Since 
                $S$ is a subgroup of $\Z$, $-n\in S$.
        
                We claim that $S=n\Z$.
                If $x\in S$, then $x=my+r$ for $y,r\in\Z$ with 
                $0\leq r<m$. Suppose that $r\ne 0$. Since $x,m\in S$, it follows that 
                $r\in S$,
                a contradiction to the minimality of $n$. Thus $r=0$ 
                and hence $x=my\in
                m\Z$. Conversely, since $n\in S$, it follows that
                 $nk\in S$ for all $k\in\Z$. In fact, if $k=0$, then 
                 $nk=0\in S$. If $k>0$, 
                 then 
                \[
                \underbrace{n+\cdots+n}_{k-\text{times}}\in S.
                \]
                Finally, if $k<0$, 
                then 
                \[
                nk=\underbrace{-n+(-n)+\cdots+(-n)}_{|k|-\text{times}}\in S.\qedhere
                \]
        \end{proof}

The previous theorem has nice applications. 
If $a,b\in\Z$, we say that $a$ \textbf{divides} $b$ (or $b$ is divisible by $a$)
if $b=ac$ for some $c\in\Z$. Notation: 
\[
a\mid b\Longleftrightarrow b=ac\text{ for some $c\in\Z$.}
\]
If $a,b\in\Z$ are such that $ab\ne0$, then 
\[
S=a\Z+b\Z=\{m\in\Z:m=ar+bs\text{ for $r,s\in\Z$}\}
\]
is a subgroup of $\Z$ (this is an exercise). 
By Theorem \ref{thm:Z}, $S=d\Z$ for some $d>0$. 
This positive integer $d$
is the \textbf{greatest common divisor} of $a$ and $b$, 
that is $d=\gcd(a,b)$. 

\begin{exercise}
Let $a,b\in\Z$ be such that $ab\ne0$ and $d=\gcd(a,b)$. 
Prove the following statements:
\begin{enumerate}
\item $d$ divides $a$ and $b$.
\item If $e\in\Z$ divides $a$ and $b$, then $e$ divides $d$.
\item There are $r,s\in\Z$ such that $d=ar+bs$.
\end{enumerate}
\end{exercise}

Two integers $a$ and $b$ are said to be \textbf{coprime} if 
and only if the only positive integer dividing 
$a$ and $b$ is one, that is  
\begin{align*}
a\text{ and }b\text{ are coprime}&\Longleftrightarrow \gcd(a,b)=1\\
&\Longleftrightarrow \Z=a\Z+b\Z\\
&\Longleftrightarrow \text{there exist $r,s\in\Z$ such that $ar+bs=1$.}
\end{align*}

\begin{exercise}
        Let $p$ be a prime and 
        $a,b\in\Z$. Prove that if $p\mid ab$, 
        then $p\mid a$ or $p\mid b$.
\end{exercise}

If $S$ and $T$ are subgroups of $\Z$, then $S\cap T$
is a subgroup of $\Z$.
Let $a,b\in\Z$ be such that $ab\ne 0$. Since $a\Z\cap b\Z$ 
is a non-zero subgroup of $\Z$ (note that it contains $ab\ne 0$), 
we can write  $a\Z\cap b\Z=m\Z$
for some $m\geq1$. The integer $m$
is the \textbf{least common multiple} of $a$ and $b$ 
and will be written as $m=\lcm(a,b)$.

\begin{exercise}
Let $a,b\in\Z\setminus\{0\}$ and $m=\lcm(a,b)$. 
Prove the following statements:
\begin{enumerate}
        \item $m$ is divisible by both $a$ and $b$.
        \item If $n$ is divisible by both $a$ and $b$, then 
        $n$ is divisible by $m$.
\end{enumerate}
\end{exercise}

\begin{exercise}
Let $a,b\in\Z_{\geq1}$. Prove that 
$ab=\gcd(a,b)\lcm(a,b)$.
%if $d=\gcd(a,b)$ and $m=\lcm(a,b)$, then 
%$ab=dm$.
\end{exercise}

\subsection{Commutators}

\begin{definition}
        \index{Derived subgroup}
        \index{Commutator subgroup}
        The \textbf{commutator subgroup}
        $[G,G]$ of $G$ 
        is the subgroup generated by 
        the commutators of $G$, that is 
        \[
        [G,G]=\langle[x,y]: x,y\in G\rangle,
        \]
        where $[x,y]=xyx^{-1}y^{-1}$ is the commutator of 
        $x$ and $y$.
\end{definition}
  
In the literature, the commutator subgroup of a group $G$ is also called the \textbf{derived 
subgroup} of~$G$. 
       
\begin{example}
        In $\Z$, the commutator of $x,y\in \Z$ 
        is the integer
        \[
        [x,y]=x+y-x-y=0.
        \]
        This example uses additive notation! 
        Thus $[\Z,\Z]=\{0\}$. 
\end{example}
        
\begin{exercise}
        Prove that $[\Sym_3,\Sym_3]=\{\id,(123),(132)\}$.
\end{exercise}
        
Why do we need to consider the subgroup 
generated by commutators? Because the set of commutators 
is not always a subgroup. With the help of computers, 
one can verify the following examples. 
The first one is taken from Carmichael's book~\cite{MR0075938}.

\begin{example}
        Let $G$ be the subgroup of $\Sym_{16}$ 
        generated by the permutations 
        \begin{align*}
&a = (13)(24),&&
b = (57)(68),\\
&c = (9\,11)(10\,12),&&
d = (13\,15)(14\,16),\\
&e = (13)(57)(9\,11),&&
f = (12)(34)(13\,15),\\
&g = (56)(78)(13\,14)(15\,16),&&
h = (9\,10)(11\,12).
\end{align*}
Then $[G,G]$ has order 16, but the set of commutators of 
$G$ has 15 elements:
\begin{lstlisting}
gap> a := (1,3)(2,4);;
gap> b := (5,7)(6,8);;
gap> c := (9,11)(10,12);;
gap> d := (13,15)(14,16);;
gap> e := (1,3)(5,7)(9,11);;
gap> f := (1,2)(3,4)(13,15);;
gap> g := (5,6)(7,8)(13,14)(15,16);;
gap> h := (9,10)(11,12);;
gap> G := Group([a,b,c,d,e,f,g,h]);;
gap> D := DerivedSubgroup(G);;
gap> Size(D);
16
gap> Size(Set(Cartesian(G, G), x->Comm(x[1], x[2])));
15
gap> c*d in Difference(D, Set(Cartesian(G, G), Comm));
true
\end{lstlisting}
\end{example}

The following example goes back to Guralnick~\cite{MR673806}. 
It was found by hand when computers were not as popular
in group theory as now. 

\begin{example}
The group 
\[
G=\langle (135)(246)(7\,11\,9)(8\,12\,10),(394\,10)(58)(67)(11\,12)\rangle.
\]
has order 96. The set of commutators 
is different from the commutator subgroup:
\begin{lstlisting}
gap> x := (1,3,5)(2,4,6)(7,11,9)(8,12,10);;
gap> y := (3,9,4,10)(5,8)(6,7)(11,12);;
gap> G := Group([x,y]);;
gap> Order(G);
96
gap> D := DerivedSubgroup(G);;
gap> Order(D);
32
gap> Size(Set(Cartesian(G, G), x->Comm(x[1], x[2])));
29
\end{lstlisting}
Moreover, 
$G$ is the smallest group with the property that  
the set of commutators is not a subgroup. 
\end{example}

\subsection{Cyclic groups}

\begin{definition}
        \index{Cyclic group}
        A group $G$ is said to be \textbf{cyclic} if 
        $G=\langle g\rangle$ for some 
        $g\in G$.
\end{definition}

If $G$ is a cyclic group generated by $g$, then 
$G=\langle g\rangle=\{g^k:k\in\Z\}$. Every cyclic group is, 
in particular, an abelian group. 

\begin{example}\
\begin{enumerate}
        \item $\Z=\langle 1\rangle=\langle -1\rangle$.
        \item $\Z/n=\langle 1\rangle$.
        \item $G_n=\langle \exp(2i\pi/n)\rangle$.
\end{enumerate}
\end{example}

\begin{example}
        $\mathcal{U}(\Z/8)\ne\langle 3\rangle$. In fact, $\langle 3\rangle=\{1,3\}\subsetneq\{1,3,5,7\}=\mathcal{U}(\Z/8)$.
\end{example}

\begin{exercise}
\label{xca:subgroups_cyclic}
        Prove that subgroups of a cyclic group are cyclic.
\end{exercise}

\begin{definition}
        \index{Order!of an element}
        Let $G$ be a group and $g\in G$. The \textbf{order} of $g$
        is the order of the subgroup generated by $g$. Notation:
        $|g|=|\langle g\rangle|$.
\end{definition}


\begin{theorem}
        Let $G$ be a group and $g\in G$ and $n\geq1$. 
        The following statements are equivalent:
        \begin{enumerate}
                \item $|g|=n$.
                \item $n=\min\{k\in\Z_{\geq1}:g^k=1\}$.
                \item For every $k\in\Z$, $g^k=1\Longleftrightarrow n\mid k$.
                \item $\langle g\rangle=\{1,g,\dots,g^{n-1}\}$ and 
                the elements $1,g,\dots,g^{n-1}$ are all different.
        \end{enumerate}
\end{theorem}


\begin{proof}
        We first prove that $(1)\implies(2)$.
        If $g=1$, then $n=1$. Assume that $g\ne1$. Since $\langle g\rangle=\{g^k:k\in\Z\}$,
        there exist integers $i$ and $j$ with $i>j$ such that $g^i=g^j$, that is $g^{i-j}=1$. In particular,
        the set $\{k\in\Z_{\geq1}:g^k=1\}$ is non-empty and hence has a minimal element, say 
        \[
        d=\min\{k\in\Z_{\geq1}:g^k=1\}.
        \]
        Thus $\langle g\rangle\subseteq\{1,g,\dots,g^{d-1}\}\subseteq\langle g\rangle$. 
        If $g^k\in\langle g\rangle$, then $k=dq+r$ for some $q,r\in\Z$ with $0\leq r<d$. Since $g^d=1$,
        \[
        g^k=g^{dq+r}=(g^d)^qg^r=g^r\in\{1=g^0,g,g^2,\dots,g^{d-1}\}
        \]
        Moreover, $\{1,g,\dots,g^{d-1}\}\subseteq \langle g\rangle$ and 
        $\{1,g,\dots,g^{d-1}\}$ has $d$ elements.

        We now prove that $(2)\implies(3)$. Assume that $g^k=1$. If 
        we write $k=nt+r$ with $0\leq r<n$, then $g^k=g^{nt+r}=g^r$. The minimality of $n$ 
        implies that $r=0$. Hence $n$ divides $k$.
        Conversely, if $k=nt$ for some $t\in\Z$, then $g^k=(g^n)^t=1$.

        Let us prove that $(3)\implies(4)$. Clearly, 
        $\{1,g,\dots,g^{n-1}\}\subseteq\langle g\rangle$. To prove the other 
        inclusion, we write $k=nt+r$ with $0\leq r\leq n-1$. Then 
        \[
                g^k=g^{nt+r}=(g^n)^tg^r=g^r, 
        \]
        as, by assumption, $g^n=1$. To see that the elements 
        $1,g,\dots,g^{n-1}$ are all different, it is enough to show that if $g^k=g^l$ with $0\leq
        k<l\leq n-1$, then, since $g^{l-k}=1$ and $0<l-k\leq n-1$, it follows that 
        $n\leq l-k$ (because by assumption $n$ divides $l-k$, a contradiction).

        Finally, the implication $(4)\implies(1)$ is trivial.
\end{proof}

\begin{corollary}
        If $G$ is a group and $g\in G$ has order $n$, 
        then 
        \[
        |g^m|=\frac{n}{\gcd(n,m)}.
        \]
        \end{corollary}
        
\begin{proof}
        Let $k$ be such that $(g^m)^k=1=g^{mk}$. This means that $n$ divides $km$, as $g$ 
        has order $n$. This is also equivalent to the fact that 
        $n/d$ divides $mk/d$, where $d=\gcd(n,m)$. Therefore, since $n/d$ and $m/d$ 
        are coprime, $(g^m)^k=1$ is equivalent to  
        $n/d$ divides $k$, which implies that $g^m$ has order $n/
        d$.
\end{proof}
        
\begin{exercise}
        Let $G$ be a group and $g\in G$. Prove that the following statements are equivalent:
        \begin{enumerate}
        \item $g$ has infinite order. 
        \item The set $\{k\in\Z_{\geq1}:g^k=1\}$ is empty.
        \item If $g^k=1$, then $k=0$.
        \item If $k\ne l$, then $g^k\ne g^l$.
        \end{enumerate}
\end{exercise}

\begin{exercise}
\index{Torsion in abelian groups}
        Let $G$ be an abelian group. Prove that 
        $T(G)=\{g\in G:|g|<\infty\}$ is a subgroup of $G$. Compute $T(\C^\times)$.
\end{exercise}
                
\begin{exercise}
        Let $G=\langle g\rangle$ be a cyclic group. 
        \begin{enumerate}
                \item If $G$ is infinite, only $g$ and $g^{-1}$ generate $G$.
                \item If $G$ is finite of order $n$, then 
                        $G=\langle g^k\rangle$ if and only if $k$ and $n$ are coprime.
        \end{enumerate}
\end{exercise}
                
The following exercise is a particular 
case of Cauchy's theorem. 

\begin{exercise}
        \label{xca:orden2}
        Prove that every group of odd order contains
        an element of order two. 
\end{exercise}
                
Let us see some concrete examples: 

\begin{example}
        In $\Sym_3$ we have the following order pattern:
        \[
        |\id|=1,\quad
        |(12)|=|(13)|=|(23)|=2,\quad
        |(123)|=|(132)|=3.
        \]
\end{example}
                        
\begin{example}
        In $\Z$, every non-zero element has 
        infinite order. 
\end{example}
                        
 \begin{example}
        In $\Z\times\Z/6$ there are elements of 
        (in)finite order. For example, $(1,0)$ 
        has infinite order and 
        $(0,1)$ has order six. 
 \end{example}
                        
\begin{example}
        The matrix $\begin{pmatrix}1&1\\0&1\end{pmatrix}\in\GL_2(\R)$ has infinite order.
\end{example}                     
                                
\begin{example}
        The group $G_\infty=\bigcup_{n\geq1}G_n$ is abelian and infinite. Note that every element of 
        $G_\infty$ has finite order. 
\end{example}
          
We conclude the topic with some exercises. 

\begin{exercise}
        Compute the orders of the elements of $\Z/6$.
\end{exercise}       

\begin{exercise}
        Prove that $a=\begin{pmatrix}1&-1\\1&0\end{pmatrix}$ has order four, $b=\
        \begin{pmatrix}0&1\\-1&-1\end{pmatrix}$ has order three and 
        compute the order of $ab$.%=\begin{pmatrix}1&1\\0&1\end{pmatrix}$ tiene orden infinito.
\end{exercise}
                                
\begin{exercise}
        Compute the order of 
        $\begin{pmatrix}1&1\\-1&0\end{pmatrix}\in\GL_2(\R)$.
\end{exercise}
                                
\begin{exercise}
        Prove that in $\D_n$ one has 
        $|r^js|=2$ and $|r^j|=n/\gcd(n,j)$.
\end{exercise}
                                
\begin{exercise}
        Prove that a group with finitely many subgroups
        is finite. 
\end{exercise}



\section{29/02/2024}

\begin{corollary}
        If $G$ is a group and $g\in G$ has order $n$, 
        then 
        \[
        |g^m|=\frac{n}{\gcd(n,m)}.
        \]
        \end{corollary}
        
\begin{proof}
        Let $k$ be such that $(g^m)^k=1=g^{mk}$. This means that $n$ divides $km$, as $g$ 
        has order $n$. This is also equivalent to the fact that 
        $n/d$ divides $mk/d$, where $d=\gcd(n,m)$. Therefore, since $n/d$ and $m/d$ 
        are coprime, $(g^m)^k=1$ is equivalent to  
        $n/d$ divides $k$, which implies that $g^m$ has order $n/
        d$.
\end{proof}
        
\begin{exercise}
        Let $G$ be a group and $g\in G$. Prove that the following statements are equivalent:
        \begin{enumerate}
        \item $g$ has infinite order. 
        \item The set $\{k\in\Z_{\geq1}:g^k=1\}$ is empty.
        \item If $g^k=1$, then $k=0$.
        \item If $k\ne l$, then $g^k\ne g^l$.
        \end{enumerate}
\end{exercise}

\begin{exercise}
\index{Torsion in abelian groups}
        Let $G$ be an abelian group. Prove that 
        $T(G)=\{g\in G:|g|<\infty\}$ is a subgroup of $G$. Compute $T(\C^\times)$.
\end{exercise}
                
\begin{exercise}
        Let $G=\langle g\rangle$ be a cyclic group. 
        \begin{enumerate}
                \item If $G$ is infinite, only $g$ and $g^{-1}$ generate $G$.
                \item If $G$ is finite of order $n$, then 
                        $G=\langle g^k\rangle$ if and only if $k$ and $n$ are coprime.
        \end{enumerate}
\end{exercise}
                
The following exercise is a particular 
case of Cauchy's theorem. 

\begin{exercise}
        \label{xca:orden2}
        Prove that every group of odd order contains
        an element of order two. 
\end{exercise}
                
Let us see some concrete examples: 

\begin{example}
        In $\Sym_3$ we have the following order pattern:
        \[
        |\id|=1,\quad
        |(12)|=|(13)|=|(23)|=2,\quad
        |(123)|=|(132)|=3.
        \]
\end{example}
                        
\begin{example}
        In $\Z$, every non-zero element has 
        infinite order. 
\end{example}
                        
 \begin{example}
        In $\Z\times\Z/6$ there are elements of 
        (in)finite order. For example, $(1,0)$ 
        has infinite order and 
        $(0,1)$ has order six. 
 \end{example}
                        
\begin{example}
        The matrix $\begin{pmatrix}1&1\\0&1\end{pmatrix}\in\GL_2(\R)$ has infinite order.
\end{example}                     
                                
\begin{example}
        The group $G_\infty=\bigcup_{n\geq1}G_n$ is abelian and infinite. Note that every element of 
        $G_\infty$ has finite order. 
\end{example}
          
We conclude the topic with some exercises. 

\begin{exercise}
        Compute the orders of the elements of $\Z/6$.
\end{exercise}       

\begin{exercise}
        Prove that $a=\begin{pmatrix}1&-1\\1&0\end{pmatrix}$ has order four, $b=\
        \begin{pmatrix}0&1\\-1&-1\end{pmatrix}$ has order three and 
        compute the order of $ab$.%=\begin{pmatrix}1&1\\0&1\end{pmatrix}$ tiene orden infinito.
\end{exercise}
                                
\begin{exercise}
        Compute the order of 
        $\begin{pmatrix}1&1\\-1&0\end{pmatrix}\in\GL_2(\R)$.
\end{exercise}
                                
\begin{exercise}
        Prove that in $\D_n$ one has 
        $|r^js|=2$ and $|r^j|=n/\gcd(n,j)$.
\end{exercise}
                                
\begin{exercise}
        Prove that a group with finitely many subgroups
        is finite. 
\end{exercise}




\subsection{Lagrange's theorem}

Let $G$ be a group and $H$ be a subgroup of $G$. We say that the elements $x,y\in
G$ are (left) equivalent modulo $H$ if $x^{-1}y\in H$.
We will use the following notation:
\begin{equation}
\label{eq:modH}
    x\equiv y\bmod
    H\Longleftrightarrow x^{-1}y\in H.
\end{equation}

\begin{exercise}
    Prove that~\eqref{eq:modH} is an equivalence relation, that
    is 
    \begin{enumerate}
    \item $x\equiv x\bmod H$ for all $x$; 
    \item if $x\equiv y\bmod H$, then $y\equiv x\bmod H$; and 
    \item if $x\equiv y\bmod H$ and $y\equiv z\bmod H$, then $x\equiv z\bmod H$.
    \end{enumerate}
\end{exercise}

The equivalence classes of this equivalence relation modulo $H$
are the sets of the form $xH=\{xh:h\in H\}$, as the class 
of an element $x\in G$ is the set 
\[
        \{y\in G:x\equiv y\bmod H\}=\{y\in G:x^{-1}y\in H\}=\{y\in G:y\in xH\}=xH.
\]
The set $xH$ is called 
a \textbf{left coset} of $H$ in $G$ and $x$ is 
a \textbf{representative} of $xH$.

Having an equivalence relation modulo $H$ in $G$ allows us to
decompose $G$ as a disjoint union of certain subsets related to $H$. 

\begin{proposition}
Let $G$ be a group and $H$ be a subgroup of $G$. 
\begin{enumerate}
\item If $xH\cap yH\ne\emptyset$, then $xH=yH$.
\item The group $G$ decomposes as a disjoint union 
of different left cosets of $H$.
\end{enumerate}
\end{proposition}

\begin{proof}
    Let us prove the first claim. If $g\in xH\cap yH$, we write 
    $g=xh$ for some $h\in H$. Then 
    \[
    gH=(xh)H=x(hH)=xH.
    \]
    Similarly, $gH=yH$. Hence $xH=yH$.
    The second claim follows from the first one. 
\end{proof}

One can also define right cosets: $x\equiv
y\bmod H$ if and only if $xy^{-1}\in H$. In this case, 
the equivalence classes are 
the sets of the form $Hx$ with $x\in X$. The set $Hx$ 
is called a \textbf{right coset}
with \textbf{representative} $x$ of $H$ in $G$. 

\begin{proposition}
    If $H$ is a subgroup of $G$, then  $|Hx|=|H|=|xH|$ for all $x\in G$.
\end{proposition}

\begin{proof}
    Let $x\in G$. The map $H\to Hx$, $h\mapsto hx$, is bijective 
    with inverse $hx\mapsto h$. Similarly, the map $H\to xH$,
    $h\mapsto xh$, is bijective.
\end{proof}

The map 
\[
        \{\text{right cosets of $H$ in $G$}\}\to\{\text{left cosets of $H$ in $G$}\}
\]
given by $Hx\mapsto x^{-1}H$ is a bijection, as 
\[
        Hx=Hy
        \Longleftrightarrow xy^{-1}\in H
        \Longleftrightarrow (x^{-1})^{-1}y^{-1}\in H
        \Longleftrightarrow x^{-1}H=y^{-1}H.
\]
In particular, the number of right cosets of $H$ in $G$
equals the number of left cosets of $H$ in $G$. 

\begin{definition}
    \index{Index}
    If $H$ is a subgroup of $G$, the \textbf{index} of $H$ in $G$
    is the number $(G:H)$ of left (or right) cosets of $H$ in $G$. 
\end{definition}

\begin{example}
    If $G=\Z$ and $S=n\Z$, then 
    \[
    a+S=\{a+nq:q\in\Z\}=\{k\in\Z:k\equiv a\bmod n\}.
    \]
\end{example}
    
\begin{example}
    The subgroups of $\Sym_3$ are $\{\id\}$, the order-two subgroups 
    $\Sym_3$, $\langle(12)\rangle$, 
    $\langle(13)\rangle$ and $\langle(23)\rangle$, and 
    the order-three subgroup $\langle(123)\rangle=\{\id,(123),(132)\}$.  
    If $H=\langle(12)\rangle=\{\id,(12)\}$, then 
    \begin{align*}
    &H=(12)H=\{\id,(12)\},\\
    &(123)H=(13)H=\{(13),(123)\},\\
    &(132)H=(23)H=\{(23),(132)\}.
    \end{align*}
    Note that our group decomposes as 
    \[
    \Sym_3=H\cup (123)H\cup (132)H\quad\text{(disjoint union)}.
    \]
    \end{example}

    \begin{example}
        Let $G=\R^2$ with the usual addition 
        and $v\in\R^2$. The line 
        \[
        L=\{\lambda v:\lambda\in\R\}
        \]
        is a subgroup of $G$. For each 
        $p\in R^2$, the coset $p+L$ 
        is the line parallel to $L$ that 
        passes through $p$.
    \end{example}

The following important theorem will be used extensively. 

\begin{theorem}[Lagrange]
\index{Lagrange's theorem}
    If $G$ is a finite group and $H$ is a subgroup of $G$, 
    then $|G|=|H|(G:H)$. In particular, $|H|$ divides $|G|$.
\end{theorem}

\begin{proof}
    We decompose $G$ into equivalence classes modulo $H$, that is 
    \[
    G=\bigcup_{i=1}^n x_iH\quad\text{(disjoint union)}
    \]
    for some $x_1,\dots,x_n\in G$, where $n=(G:H)$. 
    Since each of these equivalence classes has 
    exactly 
    $|H|$ elements,
    \[
            |G|=\sum_{i=1}^n|x_iH|=\sum_{i=1}^n|H|=|H|(G:H).\qedhere
    \]
\end{proof}

Let us discuss some corollaries. 

\begin{corollary}
    If $G$ is a finite group and $g\in G$, then $g^{|G|}=1$.
\end{corollary}

\begin{proof}
    By definition. $|g|=|\langle g\rangle|$. Apply Lagrange's theorem 
    to the subgroup $H=\langle g\rangle$ to obtain that 
    \[
            g^{|G|}=g^{|H|(G:H)}=(g^{|H|})^{(G:H)}=1.\qedhere
    \]
\end{proof}

\begin{corollary}
    If $G$ has prime order, then $G$ is cyclic. 
\end{corollary}

\begin{proof}
    Let $g\in G\setminus\{1\}$ and $H=\langle g\rangle$. By Lagrange's theorem, 
    $|H|$ divides $|G|$. Thus $|H|=|G|$, as $|G|$ is prime. Therefore 
    $G=H=\langle g\rangle$.
\end{proof}

\begin{corollary}
\label{cor:coprime_orders}
    If $G$ is an abelian group and $g,h\in G$ are elements of finite coprime orders, 
    then 
    $|gh|=|g||h|$.
\end{corollary}

\begin{proof}
    Let $n=|g|$, $m=|h|$ and $l=|gh|$. Since $G$ is abelian,
    \[
    (gh)^{nm}=(g^n)^m(h^m)^n=1. 
    \]
    Thus $l$ divides $nm$. Since $(gh)^l=1$,
    $g^l=h^{-l}\in \langle g\rangle\cap\langle h\rangle=\{1\}$ 
    (because $|\langle g\rangle|=n$ and $|\langle h\rangle|=m$ are coprime, 
    $nm$ divides $l$ by Lagrange's theorem).
\end{proof}

Fermat's little theorem is a particular case of Lagrange's theorem. 

\begin{exercise}[Fermat's little theorem]
    \index{Fermat's little theorem}
    Let $p$ be a prime number. Prove that 
    \[a^{p-1}\equiv1\bmod p
    \]
    for all $a\in\{1,2,\dots,p-1\}$.
\end{exercise}

For the next corollary, we need Euler's totient function. 
Recall that 
$\varphi(n)$ is the number of positive integers $k\in\{1,\dots,n\}$
coprime with $n$. The group of units of 
$\Z/n$ has $\varphi(n)$ elements (because $x\in\Z/n$ is invertible
if and only if $x$ and $n$ are coprime).

\begin{exercise}[Euler's theorem]
        \index{Euler's theorem }
        Let $a$ and $n$ be coprime integers. Prove that 
        \[
        a^{\varphi(n)}\equiv1\bmod n.
        \]
\end{exercise}

The converse of Lagrange's theorem does not hold.   

\begin{example}
Consider the \textbf{alternating group}
    \begin{multline*}
\Alt_4=\{\id,(234),(243),(12)(34),(123),(124),(132),(134),(13)(24),(142),(143),(14)(23)\}.
\end{multline*}
We claim that $\Alt_4$ has no subgroups of order six. If $H\leq\Alt_4$ is such that 
$|H|=6$, then, since $(\Alt_4:H)=2$, for every $x\not\in H$ we can decompose $\Alt_4$ as 
as disjoint union 
$\Alt_4=H\cup xH$.

For each $g\in\Alt_4$ we have that $g^2\in H$ (if $g\not\in H$, then, since $g^2\in\Alt_4=H\cup
gH$, it follows that $g^2\in H$). In particular, since 
$(ijk)=(ikj)^2$, order-three elements of $\Alt_4$ belong to $H$, a contradiction, 
because $\Alt_4$ has eight elements of order three. 
\end{example}

We all need a favorite group. Mine is $\SL_2(3)$,
the group of $2\times2$ matrices with coefficients in $\Z/3$
and determinant one. 

\begin{exercise}
Prove that    \[
    \SL_2(3)=\left\{\begin{pmatrix}a&b\\c&d\end{pmatrix}:ad-bc=1,\,a,b,c,d\in\Z/3\right\}
    \]
    has order 24 and does not contain subgroups of order 12.
    \end{exercise}


\section{05/03/2024}

\subsection{The symmetric group}

\index{Cycle}
Let $\sigma\in\Sym_n$. We say that the permutation $\sigma$ is an $r$-cycle 
if there are $a_1,\dots,a_r\in\{1,\dots,n\}$ such that 
$\sigma(j)=j$ for all $j\not\in\{a_1,\dots,a_r\}$ and 
\[
\sigma(a_i)=\begin{cases}
a_{i+1} & \text{if $i<r$},\\
a_1 & \text{if $i=r$}.
\end{cases}
\]

For example, $(12)$, $(13)$ and $(23)$ are 2-cycles of 
$\Sym_3$. We often say that 2-cycles are \textbf{transpositions}.
The permutations $(123)$ and $(132)$ are 3-cycles of $\Sym_3$.

\index{Disjoint permutations}
We say that the permutations $\sigma,\tau\in\Sym_n$ 
are \textbf{disjoint} if for all 
$j\in\{1,\dots,n\}$
one has $\sigma(j)=j$ or $\tau(j)=j$. For example, 
$(134)$ and $(25)$ are disjoint. The permutations $(134)$ and 
$(24)$ are not disjoint. 

If $\sigma\in\Sym_n$ and $j$ is such that 
$\sigma(j)=j$, then $j$ is a fixed point of $\sigma$. The elements 
$j$ such that 
$\sigma(j)\ne j$ are the points moved by 
$\sigma$.

\begin{claim}
Disjoint permutations commute. 
\end{claim}


We now prove that every permutation can be written 
as a product of disjoint cycles. 
The proof of the theorem can be omitted in a first 
lecture. 
We start with a lemma. 

\begin{lemma}
        Let $\sigma=\alpha\beta\in\Sym_n$ with $\alpha$ and $\beta$ disjoint permutations. If $\alpha(
i)\ne i$, then $\sigma^k(i)=\alpha^k(i)$ for all $k\geq0$.
\end{lemma}

\begin{proof}
    Without loss of generality, we may assume that $k>0$. Then \[
    \sigma^k(i)=(\alpha\beta)^k(i)=\alpha^k(\beta^k(i))=\alpha^k(i).\qedhere
    \]
\end{proof}

\begin{theorem}
Each $\sigma\in\Sym_n\setminus\{\id\}$ can be written as a product
of disjoint cycles of length 
 $\geq2$. The decomposition is unique up to 
 the order of the factors. 
 \end{theorem}

\begin{proof}
    We proceed by induction on the number $k$ 
    of elements of $\{1,\dots,n\}$ moved by $\sigma$. If $k=2$, 
    the result is trivial. Assume that the result 
    holds for all permutations moving $<k$ points. Let
    $i_1\in\{1,\dots,n\}$ be such that $\sigma(i_1)\ne i_1$. We 
    consider the cycle that contains $i_1$. So let 
        $i_2=\sigma(i_1)$, $i_3=\sigma(i_2)$... We know that 
        there exists $r$ such that $\sigma(i_r)=i_1$
 (otherwise, if $\sigma(i_r)=i_j$ for some 
        $j\in\{2,\dots,n\}$, then 
        \[
        \sigma(i_{j-1})=i_j=\sigma(i_r),
        \]
        a contradiction to the bijectivity of $\sigma$). 
        Let $\sigma_1=(i_1\cdots i_r)$. By the inductive hypothesis, since 
        $\sigma_1^{-1}\sigma$ moves $<k$ points (because 
        the $i_j$ are fixed points of $\sigma_1^{-1}\sigma$), 
        we can write $\sigma_1^{-1}\sigma=\sigma_2\cdots\sigma_s$, where 
        $\sigma_2,\dots,\sigma_s$ are disjoint cycles. 
        This implies that $\sigma=\sigma_1\sigma_2\cdots\sigma_s$.

        We now prove the uniqueness of the decomposition. 
        Assume that 
        \[
        \sigma=\sigma_1\cdots\sigma_s=\tau_1\cdots\tau
_t
\]
with $s>0$. Let $i_1\in\{1,\dots,n\}$ be such that 
        $\sigma(i_1)\ne i_1$. By the previous lemma, $\sigma^k(i_1)=\sigma_1^k(i_1)$ for all $k\geq0$.
        There exists $j\in\{1,\dots,t\}$ such that 
        $\tau_j(i_1)\ne i_1$. Since the $t_k$'s commute, 
        without loss of generality, we may assume that $j=
1$. Thus $\sigma^k(i_1)=\tau_1^k(i_1)$ for all $k\geq0$.  
This implies that 
        $\sigma_1=\tau_1$, as $\sigma_1$ and $\tau_1$ are cycles. 
        Thus $\sigma_2\cdots\sigma_s=\tau_2\cdots\tau_t$. Repeating
        this procedure, we obtain that $s=t$. Therefore 
        $\sigma_j=\tau_j$ for all $j$.
\end{proof}

\begin{corollary}\
\label{cor:generation}
        \begin{enumerate}
                \item $\Sym_n=\langle (ij):i<j\rangle$.
                \item $\Sym_n=\langle (12),(13),\dots,(1n)\rangle$.
                \item $\Sym_n=\langle (12),(23),\dots,(n-1\,n)\rangle$.
                \item $\Sym_n=\langle (12),(12\cdots n)\rangle$.
        \end{enumerate}
\end{corollary}

\begin{proof}
        The first claim follows from the previous theorem, as 
        \[
        (a_1\cdots a_r)=(a_1a_r)(a_1a_{r-1})\cdots(a_1a_2).
        \]
        If we write $\sigma\in\Sym_n$ as a product of disjoint cycles, 
        the previous formula implies 
        that $\Sym_n\subseteq\langle (ij):i<j\rangle$. The other 
        inclusion is trivial. 

        For the second claim, one uses the first claim and the
        formulas 
        \[
        (1i)(1j)(1i)=(ij), 
        \]
        where $i\ne j$.

        To prove the third claim, write $\sigma$ as a product 
        of transpositions and 
        note that 
        \[
        (13)=(12)(23)(12),\quad
        (1\,k+1)=(k\,k+1)(1k)(k\,k+1)
        \]
        for all $k\geq3$.

        Finally, the fourth claim follows from 
        the third claim and 
        the formula 
        \[
        (12\cdots n)^{k-1}(12)(12\cdots n)^{1-k}=(k\,k+1),
        \]
        where $k\geq1$.
\end{proof}

Here is an alternative proof of
the first claim of Corollary 
\ref{cor:generation}. We must show that every 
permutation can be written as a product of transpositions. 
Let us assume that $n$ persons are invited to a concert. They sit
in the first row without following  
the seat number on their tickets. How can we put each person in 
the right seat? First, we locate the person that should be seated 
in the first place. Then we ask this person to 
interchange seats with the person seated in the first place. 
Then we identify the person 
that should be seated in the second spot. We then ask this person
to interchange seats with the person 
seated in the second spot. We do the same with the third spot, the fourth
spot... Once the process is finished, 
we have decomposed 
our permutation into a product of transpositions. 

\begin{exercise}
    Following the tricks of the proof of 
    Corollary \ref{cor:generation}, find the different
    decompositions of the permutation
    $(1324)(56)(789)\in\Sym_9$. 
\end{exercise}

Every permutation yields a permutation matrix. For example, 
the matrix corresponding to $\sigma=\id\in\Sym_3$ 
is the $3\times 3$ identity matrix. The permutation
$\sigma=(123)$ yields the matrix 
\[
P_\sigma=\begin{pmatrix}0&0&1\\1&0&0\\0&1&0\end{pmatrix}.
\]
If $e_1,e_2,e_3$ is the standard basis of $\R^{3\times1}$, then
\[
P_{\sigma}(e_1)=e_2,
\quad 
P_{\sigma}(e_2)=e_1,
\quad 
P_{\sigma}(e_3)=e_1.
\]
The permutation matrix
$P_\sigma$ associated with 
$\sigma\in\Sym_n$, permutes the elements of the standard basis
of $\R^{n\times1}$ in the way $\sigma$ permutes 
the elements of $\{1,2,\dots,n\}$.

If $\sigma\in\Sym_n$, then
\[
P_\sigma=\sum_{i=1}^n E_{\sigma(i),i},
\]
where $E_{i,j}$ is the matrix with a one in position
$(i,j)$ and zero in all other entries. Recall the
following formulas: 
\begin{equation}
\label{eq:E}
E_{i,j}E_{k,l}=\begin{cases}
E_{i,l} & \text{if $j=k$},\\
0 & \text{if $j\ne k$}.
\end{cases}
\end{equation}

The determinant of a permutation matrix equals 
$\pm1$. Why? 

\begin{proposition}
If $\sigma,\tau\in\Sym_n$, then $P_{\sigma\tau}=P_\sigma P_\tau$.
\end{proposition}

\begin{proof}
We compute 
\begin{align*}
P_\sigma P_\tau &=\left(\sum_{i=1}^n E_{\sigma(i),i}\right)\left(\sum_{j=1}^nE_{\tau{(j)},j}\right)\\
&=\sum_{i=1}^n\sum_{j=1}^n E_{\sigma(i),i}E_{\tau(j),j}\\
&=\sum_{j=1}^n E_{\sigma(\tau(j)),j}\\
&=P_{\sigma\tau},
\end{align*}
where the double sum is zero unless $i=\tau(j)$.
\end{proof}


\begin{definition}
\index{Permutation!even}
\index{Permutation!odd}
\index{Permutation!sign}
    The \textbf{sign} of a permutation $\sigma\in\Sym_n$ 
    is the 
    determinant of the matrix 
    $P_\sigma$, that is $\sgn(\sigma)=\det P_\sigma$.
    A permutation $\sigma$ is said to be \textbf{even} if $\sgn(\sigma)=1$ and \textbf{odd} if $\sgn(\sigma)=-1$.
\end{definition}

The identity is an even permutation. Every 3-cycle is
an even permutation. Each transposition is an odd permutation. 

\begin{proposition}
If $\sigma,\tau\in\Sym_n$, then $\sgn(\sigma\tau)=(\sgn\sigma)(\sgn\tau)$.
\end{proposition}

\begin{proof}
        We compute 
        \[
        \sgn(\sigma\tau)=\det(P_\sigma P_\tau)=(\det P_\sigma)(\det P_\tau)=\sgn(\sigma)\sgn(\tau).\qedhere
        \]
\end{proof}

Each permutation can be written as a product of transpositions. 
There is no uniqueness of this decomposition. For example, 
\[
(13)=(12)(23)(12)=(12)(23)(12)
\]

However, the following result holds: If 
$\sigma=\sigma_1\cdots\sigma_s$ is a product of transpositions, 
then $\sgn(\sigma)=(-1)^s$.
In particular, $\sigma$ is even if and only if
$s$ is even. 


\begin{example}
\index{Center!of $\Sym_n$}
We claim that if $n\geq3$ then $Z(\Sym_n)=\{\id\}$.
Assume that $Z(\Sym_n)\ne\{\id\}$. Let 
$\sigma\in Z(\Sym_n)$ be such that $\sigma(i)=j$ for some $i\ne j$. Since $n\geq3$, there exists $k\in\{1,\dots,n
\}\setminus\{i,j\}$. Thus 
$\tau=(jk)\in\Sym_n$. Since $\sigma$ is central, 
\[
j=\sigma(i)=\tau\sigma\tau^{-1}(i)=\tau(\sigma(i))=\tau(j)=k,
\]
a contradiction.
\end{example}

\index{Alternating group}
The \textbf{alternating group}
\[
\Alt_n=\{\sigma\in\Sym_n:\sgn(\sigma)=1\}
\]
is the subgroup of $\Sym_n$ formed by even permutations. 

\begin{proposition}
\index{Order!of the alternating group}
$|\Alt_n|=n!/2$.
\end{proposition}

\begin{proof}
Let $\sigma=(12)\not\in\Alt_n$. We claim that 
$\Sym_n=\Alt_n\cup\Alt_n\sigma$ (disjoint union), where 
$\Alt_n\sigma=\{\tau\sigma:\tau\in\Alt_n\}$. If 
$\tau\in\Sym_n$ is such that $\tau\not\in\Alt_n$, then 
\[
\sgn(\tau\sigma)=(\sgn\tau)(\sgn\sigma)=1.
\]
Thus 
$\tau\sigma\in\Alt_n$. Therefore $\tau\in\Alt_n\sigma$. Since  $|\Alt_n\sigma|=|\Alt_
n|$ (because the map $\Alt_n\to\Alt_n\sigma$, $x\mapsto x\sigma$, is bijective), we conclude that 
$n!=|\Sym_n|=2|\Alt_n|$.
\end{proof}

\begin{example}
A direct calculation shows that  $\Alt_3=\{\id,(123),(132)\}$. Similarly, 
\begin{multline*}
\Alt_4=\{\id,(234),(243),(12)(34),(123),(124),\\(132),(134),(13)(24),(142),(143),(14)(23)\}
\end{multline*}
\end{example}

The group $\Alt_3$ is abelian.
If $n\geq4$, then $\Alt_n$ is non-abelian. For example, 
$(123)$ and $(124)$ do not commute. 

\begin{proposition}
\label{pro:A_n3cycles}
$\Alt_n=\langle\{\text{3-cycles}\}\rangle$.
\end{proposition}

\begin{proof}
Each 3-cycle is an even permutation, as $(ijk)=(ik)(ij)$. 
To prove the other inclusion, let $\sigma\in\Alt_n$.
Write $\sigma=\sigma_1\cdots\sigma_s$ for some even integer $s$ 
and transpositions $\sigma_1,\dots,\sigma_s$. 
Now the claim follows from the formulas 
\[
(kl)(ij)=(kl)(ki)(ki)(ij)=(kil)(ijk),\quad
(ik)(ij)=(ijk).\qedhere
\]
 \end{proof}

Proposition \ref{pro:A_n3cycles} has several 
important applications. 

\begin{example}
\index{Commutator!of $\Alt_n$}
If $n\geq5$, then $[\Alt_n,\Alt_n]=\Alt_n$. To prove the non-trivial
inclusion, it is enough to note that $\Alt_n$ is generated by 
3-cycles and that, since $n\geq5$, each 3-cycle 
is a product of commutators: 
\[
(abc)=[(acd),(ade)][(ade),(abd)],
\]
where $\#\{a,b,c,d,e\}=5$.
\end{example}

Can you compute the commutator subgroup of $\Alt_4$? 

\begin{example}
\index{Commutator!of $\Sym_n$}
If $n\geq3$, then $[\Sym_n,\Sym_n]=\Alt_n$. First, we prove that $[\Sym_n,\Sym_n]\subseteq\Alt_n$. If 
$\sigma\in[\Sym_n,\Sym_n]$,
say $\sigma=[\sigma_1,\tau_1][\sigma_2,\tau_2]\cdots[\sigma_k,\tau_k]$, then
\[
\sgn(\sigma)=\sgn([\sigma_1,\tau_1])\cdots\sgn([\sigma_k,\tau_k])=1.
\]
Conversely, if $\sigma\in\Alt_n$, by the previous proposition, 
we can write $\sigma$ as a product of 3-cycles. 
From this, the claim follows, as each 3-cycle is a commutator: 
\[
(abc)=(ab)(ac)(ab)(ac)=[(ab),(ac)]\in[\Sym_n,\Sym_n].\qedhere
\]
\end{example}
\chapter{}

\topic{Quotients}

If $G$ is a group and $N$ is a subgroup of $G$, we want to know
when the set $G/N$ of left cosets 
of $N$ in $G$ is a group with 
the operation 
\begin{equation}
\label{eq:operation}
G/N\times G/N\to G/N,\quad 
(xN,yN)\mapsto xyN,
\end{equation}
that is, when this operation 
is well-defined. 
What does this mean? We need to check that
\eqref{eq:operation} is indeed a function. 
For that purpose, we need to prove that
\eqref{eq:operation} does not depend on the representatives of left
cosets used. Thus we need to show that 
$xN=x_1N$ and $yN=y_1N$, then 
$xyN=x_1y_1N$. 

Let us try to understand this condition. If $x^{-1}x_1\in N$ and 
$y^{-1}y_1\in N$, then $x_1=xn$ and $y_1=ym$ for some 
$m,n\in N$. Thus 
\[
(xy)^{-1}(x_1y_1)=y^{-1}x^{-1}x_1y_1=y^{-1}nym\in N
\]
if and only if $y^{-1}ny\in N$.

\begin{example}
If $G=\Sym_3$ and $H=\langle (12)\rangle$, then $(xN,yN)\mapsto xyN$ is not a function. Recall that 
$G/H=\{H,(123)H,(132)H\}$, where 
$H=(12)H$, $(123)H=(13)H$ and $(132)H=(23)H$. Then 
\[
(132)N=(13)(23)N=(13)N(23)N=(123)N(132)N=N,
\]
a contradiction.
\end{example}

\begin{definition}
    \index{Subgroup!normal}
    Let $G$ be a group. 
    A subgroup $N$ of $G$ is said to be \textbf{normal} if $gNg^{-1}\subseteq N$ for all $g\in G$.
    Notation: If $N$ is normal in $G$, then $N\unlhd G$.
\end{definition}

In an abelian group, every subgroup is normal. 

\begin{proposition}
\label{pro:normalidad}
Let $N$ be a subgroup of $G$. 
The following statements are equivalent:
\begin{enumerate}
        \item $gNg^{-1}\subseteq N$ for all $g\in G$.
        \item $gNg^{-1}=N$ for all $g\in G$.
        \item $gN=Ng$ for all $g\in G$.
\end{enumerate}
\end{proposition}

\begin{proof}
We only prove that $1)\implies 2)$, as the other implications are trivial. If $n\in N$ and $g\in G$, then 
$n=g(g^{-1}ng)g^{-1}\in gNg^{-1}$.
\end{proof}

\begin{proposition}
    Let $N$ be a subgroup of $G$. The following statements
    are equivalent: 
    \begin{enumerate}
        \item $N$ is normal in $G$.
        \item $(gN)(hN)=(gh)N$ for all $g,h\in G$.
    \end{enumerate}
\end{proposition}

\begin{proof}
   We first prove that $1)\implies 2)$. Let $g\in G$. Since
   $gNg^{-1}=N$, 
   \[
   (gN)(hN)=g(Nh)N=g(hN)N=(gh)N.
   \]
   
   We now prove that $2)\implies1)$. If $g\in G$, then
    \[
    gNg^{-1}\subseteq (gN)(g^{-1}N)=(gg^{-1})N=N.\qedhere
    \]
\end{proof}

If $G$ is a group, then 
$\{1\}$ and $G$ are always normal subgroups. 

\begin{example}
\index{Center!of a group}
If $G$ is a group, then $Z(G)$ is a normal subgroup of $G$. Moreover, 
if $N\leq Z(G)$, then $N\unlhd
G$.
\end{example}

\begin{example}
If $G$ is a group, then  $[G,G]$ is a normal subgroup of $G$. If 
$x\in [G,G]$ and $g\in G$, then
$gxg^{-1}=(gxg^{-1}x^{-1})x=[g,x]x\in [G,G]$. Alternatively,
\[
g\left(\prod_{i=1}^k[x_i,y_i]\right)g^{-1}=\prod_{i=1}^k [gx_ig^{-1},gy_ig^{-1}]
\]
for all $g,x_1,\dots,x_k,y_1,\dots,y_k\in G$.
\end{example}

\begin{example}
Let $n\geq2$. Then 
$\Alt_n$ is a normal subgroup of $\Sym_n$.
If $\sigma\in\Alt_n$ and $\tau\in\Sym_n$, then $\tau\sigma\tau^{-1}\in\Alt_n$, as 
\[
\sgn(\tau\sigma\tau^{-1})=\sgn(\sigma)=1.
\]
\end{example}

\begin{example}
If $N$ is a subgroup of $G$ such that $(G:N)=2$, then $N$ is normal in $G$. We need to show that $gN=Ng$ for all $g\in G$. Let $g\in G$. 
If $g\in N$, then $gN=Ng$. If $g\not\in N$, then
$gN\ne N$. Since $(G:N)=2$, we can decompose $G$ as 
the disjoint union $G=N\cup gN$. Hence 
$gN=G\setminus N$. Similarly, 
$Ng=G\setminus N$ and therefore $gN=Ng$.
\end{example}

\begin{example}
As a particular case of the previous example, 
$\langle (123)\rangle\unlhd\Sym_3$. Note that
$\langle (12)\rangle$ is not normal in $\Sym_3$.  
For example, $(13)(12)(13)=(23)\not\in\langle(12)\rangle$.
\end{example}

\begin{example}
$\SL_n(\R)$ is normal in $\GL_n(\R)$. If $g\in\GL_n(\R)$ and $x\in\SL_n(\R)$, then $\det(gxg^{-1})=(\det g)(\det x)(\det g)^{-1}=1$.
\end{example}

\begin{example}
\index{Klein group}
The Klein group $K=\{\id,(12)(34),(13)(24),(14)(23)\}$ is normal in
$\Sym_4$. We need to show that 
$\sigma K\sigma^{-1}\subseteq K$ for all $\sigma\in\Sym_4$. Do we need to check this for every element of $\Sym_4$? No. One always has tricks! 
Recall that $\Sym_4$ is generated by $(12)$ and $(1234)$. Since
every element of  $\Sym_4$ is a word on $(12)$ and 
$(1234)$,
it is enough to see that
$\sigma K\sigma^{-1}\subseteq K$ for all $\sigma\in\{(12),(1234)\}$. We left as an exercise to show that 
\[
(12)K(12)^{-1}\subseteq K,\quad
(1234)K(1234)^{-1}\subseteq K.
\]
\end{example}

\begin{exercise}
\index{Semi-direct product}
Let $G=\R\times\R^\times$ with the operation 
\[
(x,y)(u,v)=(x+yu,yv).
\]
Prove that $\{(x,1):x\in\R\}$ is normal in $G$ and that
$\{(0,y):y\in(\R)^\times\}$ is not. 
\end{exercise}

Let us compute the list of normal subgroups of $\Alt_4$.

\begin{example}
\index{Normal subgroups of $\Alt_4$}
We claim that 
$\{\id\}$, $K=\{\id,(12)(34),(13)(24),(14)(23)\}$ and $\Alt_4$ 
are the normal subgroups of $\Alt_4$.

Since $\Alt_4=\{\text{3-ciclos}\}\cup K$, $K$ is the only subgroup
of $\Alt_4$ of order four. This implies that $K$ is normal in $\Alt_4$ (because every conjugate $gKg^{-1}$ of $K$ is a subgroup of 
$\Alt_4$ of order four). Let $N\ne\{\id\}$ be a normal subgroup of 
$\Alt_4$. 

If $N$ contains a 3-cycle, say 
$(abc)\in N$, then 
\[
(acd)=(bcd)(abc)(bcd)^{-1}\in N
\]
and hence $N=\Alt_4$ (because $N$ contains every 3-cycle). 

Assume that $N$ does not contain 3-cycles. 
Then some non-trivial element of $K$ belongs to $N$, say 
$(ab)(cd)\in N$. Hence 
\[
(ac)(bd)=(bcd)(ab)(cd)(bcd)^{-1}\in N,\quad
(ad)(bc)=(ab)(cd)(ac)(bd)\in N
\]
and therefore $N=K$.
\end{example}

Being normal is not transitive. 

\begin{exercise}
\index{Dihedral group}
Let $G=\D_4$ be the dihedral group of order eight. Let $N=\langle s,r^2\rangle$ and $H=\langle s\rangle$.
Prove that $H$ is normal in $N$, $N$ is normal in $G$ but $H$ is not normal in $G$. 
\end{exercise}

\begin{example}
\index{Normal subgroups!of $\Sym_4$}
Vamos a demostrar que $\{\id\}$, $K$, $\Alt_4$ y $\Sym_4$ son los únicos subgrupos normales de $\Sym_
4$.

Sea $N$ un subgrupo normal de $\Sym_4$. Si $N\subseteq\Alt_4$, entonces $N$ es normal en $\Alt_4$ y l
uego, por lo visto en el ejemplo anterior, $N=\{\id\}$,
$N=K$ o bien $N=\Alt_4$. Supongamos entonces que $N\not\subseteq\Alt_4$, es decir $N$ contiene una pe
rmutación impar. Si $\sigma\in\Sym_4$ es una permutación impar, entonces $\sigma$ es una trasposición
 o $\sigma$ es un 4-ciclo.

Si $N$ contiene una trasposición, entonces todas las trasposiciones
también pertenecen a $N$ pues
\[
\tau(ij)\tau^{-1}=(\tau(i)\,\tau(j))
\]
para todo $\tau\in\Sym_4$. En este caso, $N=\Sym_4$ pues $\Sym_4$ está generado por trasposiciones.


Si $N$ contiene un 4-ciclo, todos los 4-ciclos también están en $N$ pues
\[
\tau(ijkl)\tau^{-1}=(\tau(i)\,\tau(j)\,\tau(k)\,\tau(l))
\]
para todo $\tau\in\Sym_4$ y además $K\subseteq N$ pues
\[
(ac)(bd)=(abcd)^2.
\]
Esto nos dice que $|N|\geq10$. Como además $K\subseteq N$, se tiene que $|N\cap\Alt_4|\geq 5$. Por ot
ro lado, $N\cap\Alt_4$ es un subgrupo normal de $\Alt_4$.
Por lo visto en el ejemplo anterior, $N\cap\Alt_4=\Alt_4\subseteq N$. En conclusión, $N=\Sym_4$.
\end{example}

\begin{theorem}
\label{Quotient group}
If $N$ is a normal subgroup of $G$, then
$G/N$ is a group with the operation 
$(xN)(yN)=(xy)N$.
\end{theorem}

\begin{proof}
Since $N$ is normal in $G$, the operation is well-defined. 
It is an exercise to verify that
the operation is associative, that
$N$ is the neutral element of $G/N$ and that 
the inverse of an element $xN$ is 
$(xN)^{-1}=x^{-1}N$. 
\end{proof}

\begin{exercise}
\label{xca:commutator}
Si $H$ es un subgrupo normal de $G$, entonces $G/H$ es abeliano si y sólo si $[G,G]\subseteq H$.
\end{exercise}



Veamos una pequeña aplicación:

\begin{example}
\index{Conmutador!de $\Alt_4$}
$[\Alt_4,\Alt_4]=K=\{\id,(12)(34),(13)(24),(14)(23)\}$.
Sabemos que $K$ es normal en $\Alt_4$. Como $\Alt_4/K$ tiene tres elementos, es abeliano. El ejercici
o anterior, entonces,
nos dice que $[\Alt_4,\Alt_4]\subseteq K$. Por otro lado, como
\[
(ab)(cd)=[(abc),(cda)],
\]
se concluye que $K\subseteq[\Alt_4,\Alt_4]$.
\end{example}

\begin{exercise}
\label{xca:G/Z(G)}
If $G/Z(G)$ is cyclic, then $G$ is abelian.
\end{exercise}


\begin{exercise}
\index{Normalizador!de un subgrupo}
Si $S$ es un subgrupo de $G$, se define el \textbf{normalizador} de $S$ en $G$ al subgrupo
\[
N_G(S)=\{g\in G:gSg^{-1}=S\}.
\]
Demuestre que valen las siguientes afirmaciones:
\begin{enumerate}
\item $S\unlhd N_G(S)$.
\item Si $S\leq T\leq G$ y $S\unlhd T$, entonces $T\leq N_G(S)$.
\end{enumerate}
\end{exercise}

El ejercicio anterior nos dice que el normalizador de un subgrupo $S$ en $G$ es el mayor subgrupo de
$G$ que contiene a $S$ como subgrupo normal.

\begin{definition}
\index{Group!simple}
A group $G$ es \textbf{simple} si $G\ne\{1\}$ y sus únicos subgrupos normales
son $G$ y $\{1\}$.
\end{definition}

Por ahora, nos quedaremos conformes al observar que si $p$ es un número primo, entonces $\Z/p$ es un
grupo simple. Veremos otros ejemplos más adelante.

\begin{exercise}
Sea $p$ un número primo y sea $H$ un subgrupo de $G$. Si $(G:H)=p$, las siguientes afirmaciones son e
quivalentes:
\begin{enumerate}
\item $H$ es normal en $G$.
\item Si $g\in G\setminus H$, entonces $g^p\in H$.
\item Si $g\in G\setminus H$, entonces $g^n\in H$ para algún $n$ sin divisores primos $<p$.
\item Si $g\in G\setminus H$, entonces $g^k\not\in H$ para todo $k\in\{2,\dots,p-1\}$.
\end{enumerate}
\end{exercise}

\begin{exercise}
\label{xca:p_menor}
        Sea $p$ el menor número primo que divide al orden de un grupo finito
        $G$ y sea $H$ es un subgrupo de $G$ índice $p$. Entonces $H$ es normal en $G$.
\end{exercise}

% \begin{proof}
%         Si $g\in G\setminus H$, entonces $g^n=1\in H$, donde $n=|G|$. Como $p$ es primo, $n$ no tiene
%  divisores primos $<p$. El teorema anterior impica entonces que $H$ es normal en $G$.
% \end{proof}

We present two applications of Exercise...

\begin{exercise}
Sea $p$ un número primo y sea $G$ un grupo tal que todo elemento tiene orden una potencia de $p$. Si
$H$ es un subgrupo de $G$ de índice $p$, entonces $H$ es normal en $G$.
\end{exercise}
\chapter{}

\topic{Permutable subgroups}

\index{Product of subgroups}
If $H$ and $K$ are subgroups of $G$, let 
\[
        HK=\{hk:h\in H,\,k\in K\}.
\]
Note that 
\[
H\cup K\subseteq HK\subseteq\langle H\cup K\rangle.
\]
When $HK$ is a subgroup of $G$? 
Note that $HK\leq G$ if and only if $\langle H\cup K\rangle=HK$.

\begin{proposition}
        Let $H$ and $K$ be subgroups of $G$. Then $HK$ is a subgroup of
        $G$ if and only if $HK=KH$.
\end{proposition}

\begin{proof}
    Assume that $HK=KH$. Since $1\in H\cap K$, $HK\ne\emptyset$. 
    If $h\in H$ and $k\in K$, then $(hk)^{-1}=k^{-1}h^{-1}\in KH=HK$. Moreover, 
    \[
    (HK)(HK)=H(KH)K=H(HK)K=(HH)(KK)=HK.
    \]
    Thus $HK$ is closed under multiplication. 

    Now assume that $HK$ is a subgroup of $G$. Since $H\subseteq HK$,
    $K\subseteq HK$ and $HK$ closed under multiplication,
        $KH\subseteq (HK)(HK)\subseteq HK$. Conversely, let $g\in HK$.
        Since $g^{-1}\in HK$, there exist $h\in H$ and $k\in K$ such that
        $g^{-1}=hk$.
        Thus $HK\subseteq KH$, as 
        $g=k^{-1}h^{-1}\in KH$.
\end{proof}

\begin{exercise}
\label{xca:HK_normal}
Let $H$ and $K$ be subgroups of $G$. Prove that 
if $H$ is normal in $G$, then $HK$ is a subgroup of $G$.
\end{exercise}

\begin{example}
Let $G=\Sym_4$. The subgroups $H=\langle (12)\rangle$ and $K=\langle (34)\rangle$ satisfy that $HK=KH=\{
\id,(12),(34),(12)(34)\}$ is a subgroup of 
$\Sym_4$. Note that not $H$ nor $K$
are normal in $G$.
\end{example}

\begin{exercise}
Demuestre que si $H$ y $K$ son subgrupos normales de $G$, entonces $HK$ es también normal en $G$.
\end{exercise}

\begin{exercise}
Let $G$ be a group and $S$ be a subgroup of $G$. 
If $T\leq N_G(S)$, then $TS$ is a group and $S\leq TS$.
\end{exercise}

\index{Permutable subgroups}
Two subgroups $H$ and $K$ of $G$ are said to be
\textbf{permutable} if $HK=KH$.

\begin{theorem}
\label{thm:|HK|}
    Let $H$ and $K$ be finite subgroups of $G$. 
    Then 
    \[
        |HK|=\frac{|H||K|}{|H\cap K|}.
    \]
\end{theorem}

\begin{proof}
Let $L=H\cap K$.
We decompose $H$ as a disjoint union of left coclases of $L$, say 
$H=\cup_{i=1}^k x_iL$, where $k=(H:L)$. Note that $LK=K$, as $L\subseteq K$. Moreover, $K\subseteq 1K\subseteq LK$.
Then
\[
HK=\bigcup_{i=1}^k x_iLK=\bigcup_{i=1}^k x_iK,
\]
In particular, since the union is disjoint, 
\[
|HK|=\sum_{i=1}^k |x_iK|=k|K|=\frac{|H||K|}{|H\cap K|}.\qedhere
\]
\end{proof}

In the theorem, we do not assume that $HK$ is a subgroup of $G$. 

As an application, the theorem yields a different solution to Exercise~\ref{xca:p_smallest} of page~\pageref{xca:p_smallest}. 
If $\{gHg^{-1}:g\in G\}=\{H\}$, then $H$ is normal in $G$. Assume that
there exists $g\in G$ such that
$H\ne g^{-1}Hg=K$. Since $(H:H\cap K)$ divides $|H|$ 
and all prime divisors of $|G|$ 
are $\geq p$, it follows that $(H:H\cap K)\geq p$. Thus 
\[
|HK|=\frac{|H||K|}{|H\cap K|}\geq p|K|=|G|
\]
as $(G:H)=p$ y $|K|=|H|$. In particular, $HK=G$. Since $K=g^{-1}Hg$, 
$g=h(g^{-1}h_1g)$ for some $h,h_1\in H$. Thus 
\[
1=hg^{-1}h_1\implies h_1h=g\in H\implies H=K,
\]
a contradiction.

\begin{example}
Let $G=\Sym_3$, $H=\langle (12)\rangle$ and $K=\langle (23)\rangle$. Then 
\[
HK=\{\id,(12),(23),(123)\}
\]
is not a subgroup of $G$, as by Lagrange's theorem, 
$G$ cannot have subgroups of four elements. Another way 
to see that $HK$ is not a subgroup of $G$ follows from 
the fact that 
$KH=\{\id,(12),(23),(132)\}\ne HK$.
\end{example}

\begin{example}
Let $G=\Sym_3$, $H=\langle (12)\rangle$ and $K=\langle (123)\rangle$.
Since $K$ is normal in $G$, $HK$ is a subgroup of $G$. By Lagrange's theorem, $|HK|=6$ and hence $G=HK$.
Each $g\in G$ can be written uniquely as $g=hk$ for some $h\in H$ and $k\in K$ (one can prove this either considering all possible cases or 
using the fact that $H\cap K=\{\id\}$). It follows that the map 
\[
H\times K\to G,\quad
(h,k)\mapsto hk,
\]
is bijective. Note that this bijective map is not compatible 
with the operation of $G$, as 
$(h_1k_1)(h_2k_2)\ne (h_1h_2)(k_1k_2)$. 
\end{example}

\topic{Homomorphisms}


\chapter{}

\topic{Semi-direct products}


\section{16/04/2024}

\subsection{Actions}

\begin{definition}
\index{Action}
    Let $G$ be a group and $X$ a set. 
    A (left) \textbf{action} of $G$ on $X$ is a map 
    $G\times X\to X$, $(g,x)\mapsto g\cdot x$, such that 
    \begin{enumerate}
        \item $1\cdot x=x$ for all $x\in X$, y
        \item $g\cdot (h\cdot x)=(gh)\cdot x$ for all $g,h\ n G$ and $x\in X$.
    \end{enumerate}
\end{definition}

If a group $G$ acts on a set $X$, we also say that
$X$ is a $G$-set.

\begin{example}
\index{Action!trivial}
Every group $G$ acts on $G$ trivially: $g\cdot h=h$ for all $g,h\in G$.
\end{example}

\begin{example}
\index{Action!left multiplication}
Every group $G$ acts on $G$ by left multiplication, that is
$g\cdot h=gh$ for all $g,h\in G$.
\end{example}

\begin{example}
\index{Action!conjugation}
Every group $G$ acts on $G$ by conjugation, that is
$g\cdot h=hg^{-1}$ for all $g,h\in G$. More generally, 
if $N$ is a normal subgroup of $G$, then $G$ acts on
$N$ by conjugation: $g\cdot x=gx
g^{-1}$ for all $g\in G$ y $x\in N$. 
\end{example}

\begin{example}
\index{Action!on left cosets}
Let $G$ be a group $H$ be a subgroup of $G$. Then $G$ 
acts on the set of left cosets $G/H$ by left multiplication, that 
is $g\cdot (xH)=(gx)H$ for all $g,x\in G$.
\end{example}

There is a bijective correspondence between 
left actions of a group $G$ on a set $X$ and
group homomorphisms 
$\rho\colon G\to\Sym_X$. The correspondence is given by
the formula 
\[
\rho(g)(x)=g\cdot x,\quad g\in G,x\in X.
\]
We will write $\rho_g=\rho(g)$.

As an example, if $G\times X\to X$, $(g,x)\mapsto x$, is 
an action of $G$ on $X$, then
each $\rho_g\colon X\to X$ is a bijective map with inverse 
$(\rho_g)^{-1}=\rho_{g^{-1}}$. Moreover, 
 $\rho$ is a group homomorphism, as 
\[
\rho(gh)(x)=(gh)\cdot x=g\cdot (h\cdot x)=\rho_g(h\cdot x)=\rho_g(\rho_h(x))
\]
for all $g,h\in G$ and $x\in X$.

\begin{example}
Let $G=\Sym_3$ and 
\[
H=\langle (123)\rangle=\{\id,(123),(132)\}.
\]
Let $G$ act on the set $X=G/H=\{H,(12)H\}$ of left cosets of $H$ 
by left multiplication. Write 
$x_1=H$ and $x_2=(12)H$. Then
\begin{align*}
&(12)\cdot x_1=x_2,
&&(12)\cdot x_2=x_1,
&&(123)\cdot x_1=x_1,
&&(123)\cdot x_2=x_2.
\end{align*}
Since $G=\langle (12),(123)\rangle$, one has the group 
homomorphism 
$\rho\colon G\to\Sym_{X}\simeq\Sym_2$ given by
$(12)\mapsto (12)$, $(123)\mapsto\id$.
\end{example}

\begin{example}
As before, let $G=\Sym_3$ and $H=\langle (12)\rangle=\{\id,(12)\}$. Let $G$ act on the set $X=G/H=\{H,(123
)H,(132)H\}$ of left cosets of $H$ by left multiplication. Write 
$x_1=H$, $x_2=(123)H$ and $x_3=(132)H$. Then
\begin{align*}
(12)\cdot x_1=x_1,&& (12)\cdot x_2=x_3, && (12)\cdot x_3=x_2,\\
(123)\cdot x_1=x_2, && (123)\cdot x_2=x_3, &&(123)\cdot x_3=x_1.
\end{align*}
Since $G=\langle (12),(123)\rangle$, one has the group
homomorphism 
$\rho\colon G\to\Sym_{X}\simeq\Sym_3$ given by $(12)\mapsto (23)$, $(123)\mapsto (123)$.
\end{example}


\begin{example}
Let $G=Q_8=\{1,-1,i,-i,j,-j,k,-k\}$ and $N=\{1,-1,i,-i\}$. Since
$N$ is normal in $G$, $G$
acts by conjugation on $X=N$.
If $x_1=1$, $x_2=-1$, $x_3=i$ and $x_4=-i$, then
$i\cdot x=x$ for all $x\in N$. Moreover, 
\begin{align*}
j\cdot x_1=x_1, && j\cdot x_2=x_2, && j\cdot x_3=x_4, && j\cdot x_4=x_3.
\end{align*}
Since $G=\langle i,j\rangle$, a group homomorphism $\rho\colon G\to\Sym_X\simeq\Sym_4$ is determined by 
$\rho_i=\id$ and $\rho_j=(34)$.
\end{example}

The following example is important. 

\begin{example}
    The group $\Sym_n$ acts on $\R^n$ by
    \[
    \sigma\cdot (x_1,\dots,x_n)=(x_{\sigma^{-1}(1)},\dots,x_{\sigma^{-1}(n)}).
    \]
    It is very important to use $\sigma^{-1}$ and not 
    $\sigma$, as we need to permute the elements of the standard basis of $\R^3$.

    As a concrete example, let us see that the operation 
    \[
    \sigma\cdot (x_1,x_2,x_3)=(x_{\sigma(1)},x_{\sigma(2)},x_{\sigma(3)})
    \]
    is not an action of $\Sym_3$ on $\R^3$.
    If $\sigma=(12)$ and $\tau=(23)$, then $\sigma\tau=(123)$. Since 
    \begin{align*}
    &(123)\cdot (5,6,7)=(6,7,5),\\
    &(12)\cdot ((23)\cdot (5,6,7))=(1,2)\cdot (5,7,6)=(7,5,6),
    \end{align*}
    this does not define an action. If we compute 
    \begin{align*}
        \sigma\cdot (\tau\cdot (x_1,\dots,x_n))
        =\sigma\cdot (x_{\tau(1)},\dots,x_{\tau(n)})
    \end{align*}
    and for each $i\in\{1,\dots,n\}$ we set $y_i=x_{\tau(i)}$, then 
    \[
    \sigma\cdot (\tau\cdot (x_1,\dots,x_n))=\sigma\cdot (y_1,\dots,y_n)=(y_{\sigma(1)},\dots,y_{\sigma(n)})
    =(x_{\tau\sigma(1)},\dots,x_{\tau\sigma(n)}),
    \]
    even if $\sigma$ and $\tau$ do not commute.

    Now we show that by using inverses, we do have an action. 
    For $j\in\{1,\dots,n\}$, let $y_j=x_{\tau(j)}$,
    that is 
    \[
    (y_1,y_2,\dots,y_n)=\tau\cdot (x_1,x_2,\cdots,x_n)=(x_{\tau^{-1}(1)},x_{\tau^{-1}(2)},\dots,x_{\tau^{-1}(n)}).
    \]
    Then 
    \begin{align*}
        \sigma\cdot (\tau\cdot (x_1,x_2,\dots,x_n))&=\sigma\cdot (y_1,y_2,\dots,y_n)\\
        &=\left(y_{\sigma^{-1}(1)},y_{\sigma^{-1}(2)},\dots,y_{\sigma^{-1}(n)}\right)\\
        &=\left(x_{\tau^{-1}(\sigma^{-1}(1))},x_{\tau^{-1}(\sigma^{-1}(2))},\dots,x_{\tau^{-1}(\sigma
^{-1}(n))}\right)\\
        &=\left(x_{(\sigma\tau)^{-1}(1))},x_{(\sigma\tau)^{-1}(2))},\dots,x_{(\sigma\tau)^{-1}(n))}\right).
    \end{align*}
\end{example}

The following example is also important: 

\begin{example}
    The group $\Sym_n$ acts on the set of polynomials on
    $n$ variables $X_1,\dots,X_n$
    by permitting the variables. For example, for three variables, if 
    $\sigma=(123)$ y $f=X_2X_3-X_1+5X_2X_3^2X_1$, then 
    $\sigma\cdot f=X_2^2X_3-X_1+5X_2X_3^2X_1$.

    Restricting the action, we see that 
    $\Sym_n$ acts on the set 
    \[
    \{\lambda_1X_1+\cdots\lambda_nX_n:\lambda_1,\dots,\lambda_n\in\R\}.
    \]
    Then 
    \begin{align*}
    \sigma \cdot (\lambda_1X_1+\cdots+\lambda_nX_n) &= (\lambda_1X_{\sigma(1)}+\cdots+\lambda_nX_{\sigma(n)})
    =(\lambda_{\sigma(1)}X_1+\cdots+\lambda_{\sigma(n)}X_n).
    \end{align*}
\end{example}

It is relevant to compute the kernel of the action homomorphism. 

\begin{example}
Let $G$ be a group and $H$ be a subgroup of $G$. Then $G$ 
acts on $G/H$ by left multiplication, that is 
$g\cdot (xH)=(gx)H$ for all $g,x\in G$. Let $\rho\colon G\to\Sym_{G/H}$ be the group homomorphism induced by the action. 

We claim that $\ker\rho=\cap_{x\in G}xHx^{-1}$. 
We first prove $\supseteq$. If $g\in xHx^{-1}$ for all 
$x\in G$, then, for a fixed $x\in G$,
 \[
 \rho(g)(xH)=g\cdot (xH)=(gx)H=(xhx^{-1})xH=(xh)H=xH
 \]
because $g=xhx^{-1}$ for some $h\in H$. Thus $\rho(g)=\id$ 
and hence $g\in\ker\rho$. We now prove 
$\subseteq$. If $g\in\ker\rho$, then
 $\rho(g)=\id$. So for all $x\in G$,
 \begin{align*}
\rho(g)(xH)=xH
\Longleftrightarrow (gx)H=xH
\Longleftrightarrow x^{-1}gx\in H
\Longleftrightarrow g\in xHx^{-1}.
 \end{align*}
It is an exercise to show that
$\ker\rho$ is the largest normal subgroup of $G$ 
contained in~$H$.
\end{example}

With these results, we can provide a third 
solution to Exercise~\ref{xca:p_smallest} of 
page~\pageref{xca:p_smallest}.
We let  $G$ act on $G/H$ by left multiplication. 
The induced group homomorphism  $\rho\colon G\to\Sym_p$ has 
kernel 
\[
K=\ker\rho=\bigcap_{x\in G}xHx^{-1}\subseteq H.
\]
By the first isomorphism theorem, 
$G/K\simeq\rho(G)\lesssim\Sym_p$ (this means that 
$\rho(G)$ is isomorphic to a subgroup of $\Sym_p$). 
Thus $|G/K|$ divides $p!$.
Let $m=(H:K)$. By Lagrange's theorem,
\[
(G:K)=(G:H)(H:K)=pm
\]
and hence $pm$ divides $p!$. This implies that $m$ divides $(p-1)!$. If $q$ a prime number dividing 
 $m$, then $q\geq p$, by the minimality of $p$. Moreover, 
 every prime factor of $(p-1)!$ is 
 $<p$. Hence $m=1$ and therefore $H=K$.

\begin{exercise}
    Let a group $G$ act on a set $X$. 
    On $X$, we define the following relation: $x\sim y$ if and only if
    there exists $g\in G$ such that $g\cdot x=y$. Prove 
    that this is an equivalence relation on $X$. 
\end{exercise}

\begin{definition}
\index{Orbit}
Let $G$ be a group acting on a set $X$. If $x\in X$, the
orbit of $x$ is the set
\[
G\cdot x=\{g\cdot x:g\in G\}.
\]
\end{definition}

The orbits of the action of $G$ on $X$ are 
the equivalence classes of the equivalence 
relation induced by the action. In particular, 
every two orbits will be either disjoint or equal. Moreover, 
$X$ decomposes as a disjoint union of orbits. 

\begin{definition}
        \index{Stabilizer of a point}
        Let $G$ be a group that acts on $X$. If $x\in X$, the \textbf{stabilizer} of $x$ in $G$
        is the set   
        \[
        G_x=\{g\in G:g\cdot x=x\}.
        \]
\end{definition}

The reader must prove that the stabilizer is a subgroup. 

\begin{definition}
\index{Action!transitive}
We say that an action of a group $G$ on a set $X$
is \textbf{transitive} if for any $x,y\in X$ there exists $g\in G$ such that $g\cdot x=y$.
\end{definition}

\begin{example}
    Let $G$ be a group and $H$ a subgroup of $G$. Let $G$ act
    on $G/H$ by left multiplication. The action is transitive: if 
    $xH,yH\in G/H$, there exists $g\in G$ such that
    $(gx)H=yH$ (take for example $g=yx^{-1}$). 
\end{example}

\begin{example}
Por evaluación,
el grupo simétrico $\Sym_n$ actúa transitivamente en el conjunto $\{1,\dots,n\}$.
\end{example}

In the definition of a transitive action, there is no assumption
on the number of elements $g$ such that $g\cdot x=y$.

\begin{definition}
\index{Action!faithful}
We say that an action of a group $G$ on a set 
$X$ is \textbf{faithful} if 
\[
\{g\in G:g\cdot x=x\text{ for all $x\in X$}\}=\{1\}.
\]
\end{definition}

The definition is equivalent to the injectivity of 
the group homomorphism induced
by the action. 

\begin{theorem}[Fundamental counting principle]
\index{Theorem!fundamental counting principle}
\label{thm:fundamental}
Let $G$ be a finite group acting on a finite set $X$. If 
$x\in X$, then $|G\cdot x|=(G:G_x)$.
\end{theorem}

\begin{proof}
    Let $\varphi\colon G/G_x\to G\cdot x$, $gG_x\mapsto g\cdot x$. Then $\varphi$ is well-defined, as 
        \[
        gG_x=hG_x\implies h^{-1}g\in G_x
        \implies h^{-1}g\cdot x=x\implies g\cdot x=h\cdot x.
        \]
    Moreover, $\varphi$ is injective: 
        \[
        \varphi(gG_x)=\varphi(hG_x)\implies
        g\cdot x=h\cdot x\implies
        h^{-1}g\in G_x\implies gG_x=hG_x.
        \]
    Finally, $\varphi$ is surjective. Hence 
    $|G/G_x|=|G\cdot x|$.
\end{proof}

Theorem \ref{thm:fundamental} is also known 
as the orbit--stabilizer theorem. 

If $G$ is a group and $X$ and $Y$ are $G$-sets, 
we say that a map $\varphi\colon X\to Y$ is a 
\textbf{homomorphism} of $G$-sets if $\varphi(g\cdot x)=g\cdot \varphi(x)$ for all $g\in G$ y $x\in X$. The bijection 
$\varphi$ constructed in the proof of Theorem \ref{thm:fundamental}
is a homomorphism of $G$-sets, where
$G$ acts on $G/G_x$ by left multiplication: 
\[
\varphi(g\cdot hG_x)=\varphi((gh)G_x)=(gh)\cdot x=g\cdot (h\cdot x)=g\cdot\varphi(hG_x).
\]
Thus $G\cdot x\simeq G/G_x$ as $G$-sets.

\begin{example}
    If$G$ acts on $G$ by conjugation, that is $g\cdot x=gxg^{-1}$, the orbits of this action are called the \textbf{conjugacy classes} 
    of $G$. They are sets of the form
        \[
        G\cdot x=\{gxg^{-1}:g\in G\}.
        \]
    In particular, $G$ decomposes as a disjoint union of conjugacy classes. Moreover, 
    the stabilizers are the centralizers:
        \[
        G_x=\{g\in G:g\cdot x=x\}=\{g\in G:gxg^{-1}=x\}=C_G(x).
        \]
    In particular, $|G\cdot x|=(G:C_G(x))$.
\end{example}

\begin{example}
    Let $H$ be a subgroup of $G$ and $X$ the set of subsets of $G$. Let $G$ act on 
    $X$ by conjugation, that is $S\in X$. Then
        $g\cdot S=gSg^{-1}$. The orbit of $H$ 
        is 
        \[
        G\cdot H=\{g\cdot H:g\in G\}=\{gHg^{-1}:g\in G\},
        \]
        the set of conjugates of $H$. The stabilizer of $H$ in $G$ 
        is 
        \[
        G_H=\{g\in G:g\cdot H=H\}=\{g\in G:gHg^{-1}=H\}=N_G(H),
        \]
        the normalizer of $H$ in $G$. It follows that
        $H$ has exactly $(G:N_G(H))$ conjugates in $G$. In particular,
        if $G$ is finite, 
        the number of conjugates of $H$ divides $|G|$. 
\end{example}

As an application, we provide an alternative proof
of Theorem~\ref{thm:|HK|}. 

\begin{example}
\label{exa:for_HK}
Let $G$ be a group and $H$ and $K$ be subgroups of $G$. 
The group $L=H\times K$ acts on $X=HK$ by 
\[
(h,k)\cdot x=hxk^{-1},\quad x\in X,\,h\in H,\,k\in K.
\]
Note that $1\in HK$ and the action of $L$ on $X$ is transitive, as 
$(h,k^{-1})\cdot 1 = hk$. Since 
\[
L_1=\{(h,k)\in H\times K: (h,k)\cdot 1=1\}=\{(h,k)\in H\times K:h=k\},
\]
it follows that $|L_1|=|H\cap K|$ because there exists a bijection
between $L_1$ and 
$H\cap K$. By the fundamental counting principle, 
\[
|HK|=(L:L_1)=\frac{|H\times K|}{|H\cap K|}=\frac{|H||K|}{|H\cap K|}.
\]
\end{example}

As another application, we compute the
order of the group $\GL_n(p)$ para $n\geq1$ and 
a prime number $p$. 
The argument also works for
the group $\GL_n(q)$ in the case where
$q$ is a power of the prime number $p$.

\begin{example}
Let $K=\Z/p$.
We claim that 
\[
|\GL_n(p)|=(p^n-1)p^{n-1}|\GL_{n-1}(p)|,
\]
and hence 
\[
|\GL_n(p)|=(p^n-1)(p^n-p)\cdots (p^n-p^{n-1}).
\]
The formula is valid if $n\in\{1,2\}$. 
Assume that it holds for $n-1\geq1$.
The group $G=\GL_{n}(p)$ acts on
$K^{n}$ by left multiplication. There are two orbits, so 
\[
X=\{0\}\cup (K^{n}\setminus\{0\}),
\]
as if $v,w\in K^{n}\setminus\{0\}$, then there exists 
$g\in G$ such that $gv=w$.
By the fundamental counting principle,  
\[
p^{n+1}-1=|K^{n+1}\setminus\{0\}|=(G:G_{e_1}),
\]
where $e_1=(1,0,\dots,0)^T$. If $g=(g_{ij})\in G$ is such that
$ge_1=e_1$, then 
\[
g=
\begin{pmatrix}
1 & g_{12} & \cdots & g_{1n}\\
0 & g_{22} & \cdots & g_{2n}\\
\vdots & \vdots & \ddots &\vdots\\
0 & g_{n1} & \cdots & g_{nn}
\end{pmatrix}.
\]
Therefore $|G_{e_1}|=p^{n-1}|\GL_{n-1}(p)|$, as the submatrix 
$(g_{ij})_{2\leq i,j\leq n}$ is invertible and the 
$g_{1j}$'s can be chosen 
arbitrarily for all $j\in\{2,\dots,n\}$.
Hence 
\[
p^{n}-1=\frac{|G|}{|G_{e_1}|}=\frac{|\GL_n(p)|}{p^{n-1}|\GL_{n-1}(p)|},
\]
which implies the formula we wanted to prove.
\end{example}



\chapter{}

\topic{Cauchy's theorem}

\index{Action!fixed points}
\index{Class equation}
Let $G$ be a group and
$X$ be a finite $G$-set. Then $X$ decomposes as a disjoint 
union of orbits. Let 
\[
\Fix(X)=\{x\in X:g\cdot x=x\text{ for all $g\in G$}\}
\]
be the set of \textbf{fixed points} of $X$. Gather  
the one-element orbits and apply cardinality. By the 
fundamental counting principle, 
\begin{equation}
\label{eq:clases}
|X|=|\Fix(X)|+\sum_{i=1}^k|G\cdot x_i|
=|\Fix(X)|+\sum_{i=1}^k(G:G_{x_i}),
\end{equation}
where the $x_j$'s are the representatives
of the orbits of size $\geq2$. Equality~\eqref{eq:clases} is extremely 
useful and is called the 
\textbf{class equation}.

\begin{example}
Let a finite group $G$ act on $G$ by conjugation. 
Then $\Fix(G)=Z(G)$ and 
\[
|G|=|Z(G)|+\sum_{i=1}^k(G:C_G(x_i)),
\]
for some $x_1,\dots,x_k\in G$ such that 
$(G:C_G(x_i))\geq2$ for all $i\in\{1,\dots,k\}$.
\end{example}

\begin{definition}
\index{$p$-group}
Let $p$ be a prime number. We say 
that $G$ is a \textbf{$p$-group} if $|G|=p^m$ for some $m\geq0$.
\end{definition}

\begin{theorem}
Let $p$ be a prime number and 
$G$ a $p$-group. If $\{1\}\ne N\unlhd G$, then
$N\cap Z(G)\ne\{1\}$.
\end{theorem}

\begin{proof}
Since $N$ is normal in $G$, $G$ acts on $N$ by conjugation. 
By the fundamental counting principle,
each orbit has prime-power size. Write 
\[
N=\underbrace{\mathcal{O}_1\cup\cdots\cup \mathcal{O}_k}_{\text{one-element orbits}}\cup\underbrace{\mathcal{O}_{k+1}\cup\cdots\cup\mathcal{O}_m}_{\text{orbits of size $>1$}},
\]
Since $N\cap Z(G)=\mathcal{O}_1\cup\cdots\cup\mathcal{O}_k$, 
the integers $k=|N\cap Z(G)|$ and $|N\setminus(N\cap Z(G))|$ are divisible by $p$. Thus 
\[
|N|\equiv|N\cap Z(G)|\bmod p.
\]
Since $1\in N\cap Z(G)$, then
$|N\cap Z(G)|>1$. In particular, $N\
cap Z(G)\ne\{1\}$.
\end{proof}

The following corollary follows immediately: 

\begin{corollary}
Let $p$ be a prime number and 
$G$ a $p$-group. Then 
$Z(G)\ne\{1\}$.
\end{corollary}


\section{Lecture -- Week 10}


The idea used in Example \ref{exa:for_HK} can be generalized. 

\begin{optional}
\begin{example}
\index{Double coset}
Let $G$ be a group and $H$ and $K$ be subgroups of $G$. Let the group $L=H\times K$ act on $G$ by
\[
(h,k)\cdot g=hgk^{-1}.
\]
The orbits are sets of the form 
\[
HgK=\{hgk:h\in H,\,k\in K\}.
\]
These sets are called \emph{double $(H,K)$-cosets}. 
In particular, two double cosets are either disjoint 
or equal. Moreover, $G$ admits a decomposition 
as a disjoint union of double cosets, that is 
\[
G=\bigcup_{i\in I}Hg_iK,
\]
for some set $I$. Now we compute 
\[
L_g=\{(h,k)\in H\times K:hgk^{-1}=g\}=\{(h,g^{-1}hg)\in H\times K\}.
\]
Then $|L_g|=|H\cap gKg^{-1}|$, because there is a bijection between $L_g$ and
$H\cap gKg^{-1}$. By the fundamental counting principle (Theorem~\ref{thm:fundamental}), 
\[
|HgK|=(L:L_g)=\frac{|H\times K|}{|H\cap gKg^{-1}|}=\frac{|H||K|}{|H\cap gKg^{-1}|}.
\]
\end{example}
\end{optional}

As another application, we compute the
order of the group $\GL_n(p)$ for $n\geq1$ and 
a prime number $p$. 
The argument also works for
the group $\GL_n(q)$ in the case where
$q$ is a power of the prime number $p$.

\begin{example}
Let $p$ be a prime number and $K=\Z/p$.
We claim that 
\[
|\GL_n(p)|=(p^n-1)p^{n-1}|\GL_{n-1}(p)|,
\]
and hence 
\[
|\GL_n(p)|=(p^n-1)(p^n-p)\cdots (p^n-p^{n-1}).
\]

The formula is valid if $n\in\{1,2\}$. 
Assume that it holds for $n-1\geq1$.
The group $G=\GL_{n}(p)$ acts on
$K^{n\times 1}$ by left multiplication. 
What are the orbits? 
Since for every 
non-zero $v,w\in K^{n\times 1}$, then there exists 
$g\in G$ such that $gv=w$. Thus there are only two orbits.
One orbit is the one-element orbit 
of the zero column vector of $K^{n\times1}$, and 
the other orbit is the set $\mathcal{O}$ of non-zero vectors of $K^{n\times1}$. 
By the fundamental counting principle,  
\[
p^{n+1}-1=|\mathcal{O}|=(G:G_{v}),
\]
for every $v\in\mathcal{O}$, that is every $v\in K^{n\times 1}$. 

To compute the stabilizer $G_v$ easily, take 
\[
v=\colvec{4}{1}{0}{\vdots}{0}\in\mathcal{O}. 
\]
If $g=(g_{ij})\in G$ is such that
$gv=v$, then 
\[
g=
\begin{pmatrix}
1 & g_{12} & \cdots & g_{1n}\\
0 & g_{22} & \cdots & g_{2n}\\
\vdots & \vdots & \ddots &\vdots\\
0 & g_{n1} & \cdots & g_{nn}
\end{pmatrix}.
\]
Therefore $|G_{v}|=p^{n-1}|\GL_{n-1}(p)|$, as the submatrix 
$(g_{ij})_{2\leq i,j\leq n}$ is invertible and the 
$g_{1j}$'s can be chosen 
arbitrarily for all $j\in\{2,\dots,n\}$.
Hence 
\[
p^{n}-1=\frac{|G|}{|G_{v}|}=\frac{|\GL_n(p)|}{p^{n-1}|\GL_{n-1}(p)|},
\]
which implies the formula we wanted to prove.
\end{example}

\subsection{$p$-groups}

\index{Action!fixed points}
\index{Class equation}
Let $G$ be a finite group acting on a finite 
set $X$. Let 
\[
\Fix(X)=\{x\in X:g\cdot x=x\text{ for all $g\in G$}\}
\]
be the set of \emph{fixed points} of $X$, that is the set of one-elements 
orbits. We know that $X$ decomposes as a disjoint 
union of orbits. In particular, 
\[
X=\Fix(X)\cup \mathcal{O}_1\cup\cdots\mathcal{O}_k,
\]
where $\mathcal{O}_1,\dots,\mathcal{O}_k$ are orbits such that 
$|\mathcal{O}_j|\geq2$ for all $j\in\{1,\dots,k\}$. 
If we apply cardinality and use the  
fundamental counting principle, 
\begin{equation}
\label{eq:clases}
|X|=|\Fix(X)|+\sum_{i=1}^k|\mathcal{O}_i|
=|\Fix(X)|+\sum_{i=1}^k(G:G_{x_i}),
\end{equation}
where $x_j\in\mathcal{O}_j$ and 
$(G:G_{x_i})\geq2$ for all $j\in\{1,\dots,k\}$. 
Equality~\eqref{eq:clases} is extremely 
useful and is called the 
\emph{class equation}.

\begin{example}
Let a finite group $G$ act on $G$ by conjugation. 
Then $\Fix(G)=Z(G)$ and 
\[
|G|=|Z(G)|+\sum_{i=1}^k(G:C_G(x_i)),
\]
for some $x_1,\dots,x_k\in G$ such that 
$(G:C_G(x_i))\geq2$ for all $i\in\{1,\dots,k\}$.
\end{example}

\begin{definition}
\index{$p$-group}
Let $p$ be a prime number. We say 
that $G$ is a \emph{$p$-group} if $|G|=p^m$ for some $m\geq0$.
\end{definition}

\begin{theorem}
Let $p$ be a prime number and 
$G$ be a $p$-group. If $N$ is a non-trivial normal subgroup of $G$, then
$N\cap Z(G)\ne\{1\}$.
\end{theorem}

\begin{proof}
Since $N$ is normal in $G$, $G$ acts on $N$ by conjugation. 
By the fundamental counting principle,
each orbit has prime-power size. Write 
\[
N=\underbrace{\mathcal{O}_1\cup\cdots\cup \mathcal{O}_k}_{\text{one-element orbits}}\cup\underbrace{\mathcal{O}_{k+1}\cup\cdots\cup\mathcal{O}_m}_{\text{orbits of size $>1$}},
\]
Since $N\cap Z(G)=\mathcal{O}_1\cup\cdots\cup\mathcal{O}_k$, 
the integers $k=|N\cap Z(G)|$ and $|N\setminus(N\cap Z(G))|$ are divisible by $p$. Thus 
\[
|N|\equiv|N\cap Z(G)|\bmod p.
\]
Since $1\in N\cap Z(G)$, then
$|N\cap Z(G)|>1$. In particular, $N\cap Z(G)\ne\{1\}$.
\end{proof}

The following corollary follows immediately: 

\begin{corollary}
Let $p$ be a prime number and 
$G$ a $p$-group. Then 
$Z(G)\ne\{1\}$.
\end{corollary}

In Exercises \ref{xca:size4} and \ref{xca:size9}
we proved that groups of order four and nine are always abelian. 

\begin{corollary}
    Let $p$ be a prime number. If $G$ is a group of order $p^2$, 
    then $G$ is abelian. 
\end{corollary}

\begin{proof}
By Lagrange's theorem, $|Z(G)|\in\{1,p,p^2\}$. Since $G$ 
is a $p$-group, $Z(G)\ne\{1\}$. If $|Z(G)|=p$, then $G/Z(G)$ 
is cyclic. By Exercise \ref{xca:G/Z(G)}, 
$G$ is abelian, a contradiction. 
Thus $|Z(G)|=p^2$ and hence $G=Z(G)$.
\end{proof}

\subsection{Cauchy's theorem}

\begin{theorem}[Cauchy]
\index{Cauchy's theorem}
\label{thm:Cauchy}
Let $G$ be a finite group, and $p$ be a prime number
that divides $|G|$. 
Then there exists $g\in G$ of order $p$.
\end{theorem}

\begin{proof}
Let $C=\Z/p$ and 
\[
X=\{(x_1,\dots,x_p)\in G\times\cdots\times G:x_1\cdots x_p=1\}.
\]
Then $C$ acts on $X$ by $k\cdot (x_1,\dots,x_p)=(x_{k+1},\dots,x_{k+p})$, where the indices are taken modulo $p$. To see that
this is an action, note that 
\[
x_{i_1}\cdots x_{i_p}=1
\implies (x_{i_1}^{-1}x_{i_1})x_{i_2}\cdots x_{i_p}=x_{i_1}^{-1}
\implies x_{i_2}\cdots x_{i_p}x_{i_1}=1.
\]
If $x_1,\dots,x_{p-1}$ are fixed, then 
$x_p=x_{p-1}^{-1}\cdots x_{1}^{-1}$. Thus $|X|=|G|^{p-1}$. Each $C$-orbit
has either one or $p$ elements, as $|C|=p$. Write
\[
X=\underbrace{\mathcal{O}_1\cup\cdots\cup \mathcal{O}_k}_{\text{one-element orbits}}\cup\underbrace{\mathcal{O}_{k+1}\cup\cdots\cup\mathcal{O}_m}_{\text{orbits of size $p$}}.
\]
Hence $0\equiv |G|^{p-1}=|X|\equiv k\bmod p$, that is $p$ divides $k$. Since $(1,1,\dots,1)\in X$, $k\geq 1$. Therefore $p\leq k$. In particular,
there exists $x\in G\setminus\{1\}$ such that $(x,x,\dots,x)\in X$. Hence $|x|=p$.
\end{proof}

\begin{exercise}
\label{xca:p_groups}
    Let $p$ be a prime number and $G$ be a finite group. 
    Then $G$ is a $p$-group if and only if 
    every element of $G$ has order a power of $p$. 
\end{exercise}

\begin{corollary}
    Let $p>2$ be a prime number 
    and $G$ be a group of order $2p$. 
    Then either $G\simeq\Z/2p$ or $G\simeq\D_p$.
\end{corollary}

\begin{proof}
    By Cauchy's theorem, there exist $r,s\in G$ such that
    $|r|=p$ and $|s|=2$. Let $H=\langle r\rangle$. Then
    $(G:H)=2$ and $H\unlhd G$. We can decompose $G$ as 
    $G=H\cup Hs$ (disjoint union), 
    as $s\not\in H$. In particular, 
\[
G=\{1,r,\dots,r^{p-1},s,rs,\dots,r^{p-1}s\}.
\]
Since $srs^{-1}\in H$, it follows that $srs^{-1}=r^k$ for some
$k\in\{0,1,\dots,p-1\}$. Since $s^2=1$,
\[
r=s^2rs^{-2}=s(srs^{-1})s^{-1}=sr^ks^{-1}=r^{k^2}.
\]
Thus $k^2\equiv 1\bmod p$ and either 
$k\equiv 1\bmod p$ or $k\equiv-1\bmod p$. 
If $k\equiv -1\bmod p$, then $srs^{-1}=r^{-1}$ and hence $G\simeq\D_p$.
If $k\equiv 1\bmod p$, then $rs=sr$ and hence, since $G$ is abelian, $G\simeq\Z/{2p}$.
\end{proof}

\begin{theorem}
    Let $p$ be a prime number. 
    A group of order $p^m$ has a normal subgroup of order
    $p^n$ for every $n\leq m$.
\end{theorem}

\begin{proof}
    We proceed by induction on $m$. The case where $m=1$ is trivial. So let $m\geq1$ and 
    assume the result holds for groups of order $p^m$. Let
    $G$ be a group of order $p^{m+1}$.
    We claim that if $n\leq m$, $G$ contains a normal subgroup of
    order $p^{n+1}$. Since $Z(G)\ne\{1\}$, there exists
    $g\in Z(G)\setminus\{1\}$ of order $p$. Let
$N=\langle g\rangle\unlhd G$. The quotient 
group $G/N$ has order $p^m$. By the inductive hypothesis, 
there exists a normal subgroup $Y$ of $G/N$ of order $p^n$. Let 
$\pi\colon G\to G/N$ be the canonical map. 
By the correspondence theorem, $G$ contains a normal subgroup $K$ of $G$ 
that contains $N$, that is $N\leq K\leq G$. In fact, 
$Y=\pi(K)$ and 
$(G:K)=(\pi(G):\pi(K))=p^{m-n}$. Hence $|K|=p^{n+1}$.
\end{proof}

% The following exercise uses group actions to 
% prove Fermat's theorem on the representation 
% of primes as sum of two squares. 
% Elsholtz, C. (2010). A Combinatorial Approach to Sums of Two Squares and Related Problems. Additive Number Theory, 115–140. doi:10.1007/978-0-387-68361-4_8 
% \begin{bonus}

%     \label{xca:twosquares}
%     Let 
%     \[
%     X_1=\begin{pmatrix}
%        0&1&0\\
%        1&0&0\\
%        0&0&-1
%     \end{pmatrix},\quad 
%     X_2=\begin{pmatrix}
%         0&1&0\\
%         1&0&0\\
%         0&0&1
%     \end{pmatrix},\quad 
%     X_3=\begin{pmatrix}
%         1&-1&1\\
%         0&1&0\\
%         0&2&-1
%     \end{pmatrix}.
%     \]
%     Let 
%     \begin{align*}
%         S &= \{(x,y,z)\in\Z^3:p=4xy+z^2,\quad x,y>0\},\\
%         T &= \{(x,y,z)\in S:z>0\},\\
%         U &= \{(x,y,z)\in S:x+z>y\}.
%     \end{align*}
%     Prove the following statements:
%     \begin{enumerate}
%         \item $X_1^2=X_2^2=X_3^2=I$, the identity matrix.
%         \item $X_1$ maps $S$ to $S$, $X_2$ maps $T$ to $T$ and $X_3$ maps $U$ to $U$.  
%         \item $|T|=|X_1(T)|$. 
%         \item %$S=T\cup X_1(T)$ (disjoint union). Thus 
%         $|S|=2|T|=2|U|$. 
%         \item The map $X_3$ acting on $U$ has exactly one orbit of length one and since all other orbits have length two. In particular, $|U|$ is odd. 
%         \item The action of $X_2$ on $T$ has an orbit of length one. Conclude that $p=4x^2+y^2$. 
%     \end{enumerate}
% \end{bonus}
\section{16/05/2024}

\chapter{}

\topic{Double cosets (optional)}

The idea used in Example \ref{exa:for_HK} can be generalized. 

\begin{example}
\index{Coclase!doble}
Sea $G$ un grupo y sean $H$ y $K$ subgrupos de $G$. Hacemos
que el grupo $L=H\times K$ actúe en $G$ por
\[
(h,k)\cdot g=hgk^{-1}.
\]
Las órbitas son los conjuntos de la forma
\[
HgK=\{hgk:h\in H,\,k\in K\},
\]
estos conjuntos se llaman $(H,K)$-coclases dobles.
En particular, dos $(H,K)$-coclases dobles son disjuntas o iguales. Más aún,
$G$ se descompone como unión disjunta
\[
G=\bigcup_{i\in I}Hg_iK,
\]
para algún conjunto $I$, es decir $G$
es unión disjunta de $(H,K)$-coclases dobles.
Calculamos ahora
\[
L_g=\{(h,k)\in H\times K:hgk^{-1}=g\}=\{(h,g^{-1}hg)\in H\times K\}
\]
y vemos que $|L_g|=|H\cap gKg^{-1}|$, pues los conjuntos $L_g$ y $H\cap gKg^{-1}$ están en biyección.

Luego,
gracias al principio fundamental del conteo,
\[
|HgK|=(L:L_g)=\frac{|H\times K|}{|H\cap gKg^{-1}|}=\frac{|H||K|}{|H\cap gKg^{-1}|}.
\]
\end{example}

El ejemplo anterior puede generalizarse, lo que nos da una descomposición
de un grupo como unión disjunta de \textbf{coclases dobles}. Veremos más adelante
demostraciones alternativas de los teoremas de Sylow basadas en
coclases dobles.

\section*{Some solutions}

\fancyhf{}
\fancyfoot[R]{\thepage}
\fancyhead[L]{\course}
\fancyhead[R]{Some solutions}
\setlength{\headheight}{14pt}

% \pagestyle{plain}
% \fancyhf{}
% \fancyhead[LE,RO]{Rings and modules}
% \fancyhead[RE,LO]{Some solutions}
% \fancyfoot[CE,CO]{\leftmark}
% \fancyfoot[LE,RO]{\thepage}
% \addcontentsline{toc}{chapter}{Some solutions}
\begin{sol}{xca:neutral}
If $e$ and $e_1$ are both neutral elements, then $e=ee_1=e_1$. 
\end{sol}

\begin{sol}{xca:ax=b}
    If $ax=b$, after multiplying on the left by $a^{-1}$ we
    obtain that $x=a^{-1}b$. Similarly, the equation $xa=b$ 
    has $x=ba^{-1}$ as its unique solution. 
\end{sol}

\begin{sol}{xca:LR}
    For $g\in G$, the map $L_g\colon G\to G$, $x\mapsto gx$, is invertible 
    with inverse $L_{g^{-1}}$, as
    \[
    (L_g\circ L_{g^{-1}})(x)=g(g^{-1}x)=(gg^{-1})x=x
    \]
    for all $x\in G$. Similarly, $L_{g^{-1}}\circ L_g)(x)=x$ for all $x\in G$. 
    
    In the same way, we prove that 
    for each $g\in G$, the map $R_{g^{-1}}$ is the inverse of $R_g$. 
\end{sol}

\begin{sol}{xca:GxH}
    To prove the associativity, let $g,g_1,g_2\in G$ and 
    $h,h_1,h_2\in H$. Since $G$ and $H$ are groups, 
    their multiplications are associative. Then 
    \begin{align*}
        ((g,h)(g_1,h_1))(g_2,h_2) &= (gg_1,hh_1)(g_2,h_2)\\
        &=((gg_1)g_2,(hh_1)h_2)\\
        &= (g(g_1g_2),h(h_1h_2))\\
        &= (g,h)(g_1g_2,h_1h_2)\\
        &= (g,h)((g_1,h_1)(g_2,h_2)).
    \end{align*}
    
    The neutral element of $G\times H$ is $(1,1)$, as $(1,1)(g,h)=(g,h)=(g,h)(1,1)$. 
    
    The inverse
    of $(g,h)$ is $(g,h)^{-1}=(g^{-1},h^{-1})$, as
    \begin{align*}
    (g,h)(g,h)^{-1}&=(g,h)(g^{-1},h^{-1})=(gg^{-1},hh^{-1})=(1,1),\\
    (g,h)^{-1}(g,h)&=(g^{-1},h^{-1})(g,h)=(g^{-1}g,h^{-1}h)=(1,1).
    \end{align*}
\end{sol}

\begin{sol}{xca:center}
    Clearly $1\in Z(G)$. If $x\in Z(G)$, then $xg=gx$ for all $g\in G$. Multiplying by $x^{-1}$ on the left and 
    on the right, we get that $gx^{-1}=x^{-1}$ holds for all $g\in G$. Finally, if $x,y\in Z(G)$. Then
    \[
    (xy)g=x(yg)=x(gy)=(xg)y=(gx)y=g(xy)
    \]
    for all $g\in G$. Hence $xy\in Z(G)$. 
\end{sol}

% \begin{sol}{xca:centralizer}
%     First, $1\in G_G(g)$, as $1g=g1=g$. If $x\in c_G(g)$, then $xg=gx$. Multiplying 
%     on the left and on the right by $x^{-1}$, one gets $x^{-1}g=gx^{-1}$, that is 
%     $x^{-1}\in C_G(g)$. Finally, if $x,y\in C_G(g)$, then 
%     \[
%     $(xy)g=$
%     \]
% \end{sol}

\begin{sol}{xca:conjugate}
    Since $S$ is a subgroup, $1\in S$ and 
    if $x,y\in S$, then $x^{-1}\in S$ and $xy\in S$. Now 
    $1\in gSg^{-1}$, as $1\in S$ and $1=g1g^{-1}$. If $x\in gSg^{-1}$, then 
    $x=gsg^{-1}$ for some $s\in S$. Thus 
    \[
    x^{-1}=(gsg^{-1})^{-1}=gs^{-1}g^{-1}\in gSg^{-1},
    \]
    as $s^{-1}\in S$. Finally, 
    if $x=gsg^{-1}\in gSg^{-1}$ and $y=gtg^{-1}\in gSg^{-1}$ for some $s,t\in S$, then 
    \[
    xy=(gsg^{-1})(gtg^{-1})=g(st)g^{-1}\in gSg^{-1},
    \]
    as $st\in S$. 
\end{sol}

\begin{sol}{xca:center_S3}
    If $\sigma\in Z(\Sym_3)$ and $\sigma\ne\id$, there exists $i\in\{1,2,3\}$ such that 
    $\sigma(i)\ne i$. Let $j=\sigma(i)$ and $k\in\{1,2,3\}\setminus\{i,j\}$. Then 
    $(jk)\sigma$ is a permutation such that $i\mapsto k$, while 
    $\sigma(jk)$ is such that $i\mapsto j$. In particular, $(jk)\sigma\ne\sigma(jk)$, a contradiction.

    The group $\Sym_3$ has six elements: $\id$, $(12)$, $(13)$, $(23)$. $(123)$ and $(132)$. 
    First note that $\id\in C_{\Sym_3}((12))$ and 
    $(12)\in C_{\Sym_3}((12))$. However, 
    the permutations $(23)$, $(13)$, $(123)$ and $(132)$ do not commute with
    $(12)$. For example, 
    \[
    (23)(12)=(132)\ne (123)=(12)(23).
    \]
\end{sol}

\begin{sol}{xca:subgroup}
    Let us prove $\implies$. Since $1\in S$, then $S\ne\emptyset$. If $u,v\in S$, then 
    $v^{-1}\in S$ and $uv^{-1}\in S$. 

    Let us prove now $\impliedby$. If $S\ne\emptyset$, let $u\in S$. Then $1=uu^{-1}\in S$. The assumption 
    Let $u,v\in S$. The assumption with $x=1\in S$ and $y=v$ yields $v^{-1}\in S$. The assumption 
    with $x=u$ and $y=v^{-1}$ yields $uv\in S$. 
\end{sol}

\begin{sol}{xca:SL_subgroup}
    The identity matrix belongs to $\SL_n(\R)$. If $a,b\in\SL_n(\R)$, then 
    $ab^{-1}\in\SL_n(\R)$, as 
    \[
    \det(ab^{-1})=\det(a)\det(b^{-1})=\det(a)\det(b)^{-1}=1.
    \]
    By Exercise \ref{xca:subgroup}, $\SL_n(\R)$ is a subgroup of $\GL_n(\R)$. 
\end{sol}

\begin{sol}{xca:intersection}
    Let $\{H_\lambda:\lambda\in\Lambda\}$ be a collection of subgroups of a group $G$ and 
    $H=\cap_{\lambda\in \Lambda}H_\lambda$. We claim that $H$ is a subgroup of $G$. Since
    $1\in H_\lambda$ for all $\lambda$, $H$ is non-empty. If $x,y\in H$, then $x,y\in H_\lambda$ for all $\lambda$. 
    Since each $H_\lambda$ is a subgroup of $G$, $xy^{-1}\in H_\lambda$ for all $\lambda$. Thus $xy^{-1}\in H$.
\end{sol}

\begin{sol}{xca:generated}
    Let 
    \[
    H=\{x_1^{n_1}\cdots x_k^{n_k}:k\geq0,\,x_1,\dots,x_k\in X,\,-1\leq n_1,\dots,n_k\leq 1\}.
    \]
    To prove that $H\subseteq\langle X\rangle$, let $h=x_1^{n_1}\cdots x_k^{n_k}\in H$. 
    If $S$ is a subgroup of $G$ containing $X$, then $x_j\in S$ for all $j$. This implies that 
    $h=x_1^{n_1}\cdots x_k^{n_k}\in S$. Thus 
    \[
    h\in\bigcap_{\substack{S\leq G\\X\subseteq S}}S.
    \]
    
    To prove that $H\supseteq \langle X\rangle$ we first 
    claim that $H$ is a subgroup of $G$. Note that $H\ne\emptyset$, as $1\in H$ (this is the empty word). If 
    $u=x_1^{n_1}\cdots x_k^{n_k}\in H$ and 
    $v=x_{k+1}^{n_{k+1}}\cdots x_{l}^{n_l}\in H$, then 
    \[
    uv^{-1}=x_1^{n_1}\cdots x_k^{n_k}x_{l}^{-n_{l}}\cdots x_{k+1}^{-n_{k+1}}\in H. 
    \]
    Now note that $H$ is a subgroup of $G$ containing $X$. Thus 
    \[
    \langle X\rangle=\bigcap_{\substack{S\leq G\\X\subseteq S}}S\subseteq H.
    \]
\end{sol}

\begin{sol}{xca:union}
    Let $G=\Sym_3$. Then $H=\{\id,(12)\}$ and 
    $K=\{\id,(23)\}$ are subgroups of $G$. However, 
    $H\cup K=\{\id,(12),(23)\}$ is not a subgroup, as 
    $(12)(23)=(123)\not\in H\cup K$. 
\end{sol}

\begin{sol}{xca:permutation_matrix}
Let $\{e_1,\dots,e_n\}$ be the standard basis of $\R^n$. 
To prove this formula note that
\[
E_{i,j}e_k=\begin{cases}
    e_i&\text{if $j=k$,}\\
    0 & \text{if $j\ne k$}.
\end{cases}
\]
and verify that 
$P_\sigma e_k=\sum_{i=1}^n E_{\sigma(i),i}e_k$ 
for all $k\in\{1,\dots,n\}$. Since $P_\sigma$ and
$\sum_{i=1}^n E_{\sigma(i),i}$ coincide in a basis of $\R^n$, 
they are equal. 
\end{sol} 

\begin{sol}{thm:quotient}
Since $N$ is normal in $G$, the operation is well-defined. 
Routine calculations show that 
the operation is associative, that
$N$ is the neutral element of $G/N$ and that 
the inverse of an element $xN$ is 
$(xN)^{-1}=x^{-1}N$. For example, for the associativity, 
we note that for $x,y,z\in G$ one has 
\begin{align*}
    &((xN)(yN))(zN)=((xy)N)zN=(xy)zN,
\shortintertext{equals}
    &(xN)((yN)(zN))=(xN)((yz)N)=x(yz)N.
\end{align*}
since $x(yz)=(xy)z$.
\end{sol}

\begin{sol}{xca:commutator}
For $x,y\in G$,
\begin{align*}
    (xH)(yH)=(yH)(xH) \Longleftrightarrow (xy)H=(yx)H \Longleftrightarrow x^{-1}y^{-1}xy\in H.
\end{align*}
Thus $G/H$ is abelian if and only if  $[x,y]=xyx^{-1}y^{-1}\in H$ for all $x,y\in G$.
\end{sol}


\begin{sol}{xca:G/Z(G)}
Assume that $G/Z(G)$ is generated by $gZ(G)$. Let $x,y\in G$. Then 
$xZ(G)=g^kZ(G)$ and $yZ(G)=g^lZ(G)$ for some $k,l\in\Z$,  
that is 
$x=g^kz_1$ and $y=g^lz_2$ for some $k,l\in\Z$ y $z_1,z_2\in Z(G)$. Thus $xy=yx$.
\end{sol}

\begin{sol}{xca:index_p}
Lagrange's theorem immediately proves $1)\implies 2)$, 
as $|G/H|=p$. 

The implication $2)\implies 3)$ is trivial, as $p$ is a prime number. 

We now prove that $3)\implies 4)$. If $g^k\in H$ for
some $k\in\{2,\dots,p-1\}$, then, since 
$\gcd(k,n)=1$, there exist $r,s\in\Z$ such that 
$rk+sn=1$. Thus 
\[
g=g^1=g^{rk+sn}=(g^k)^r(g^n)^s\in H,
\]
a contradiction. 

Finally, we prove that $4)\implies 1)$. Let $x\in G\setminus H$ and $h\in H$. We claim that 
$xhx^{-1}\in H$. Let $y=xhx^{-1}$ and assume that 
$y\not\in H$. Then, by assumption, 
$y^k\not\in H$ for all 
$k\in\{1,2,\dots,p-1\}$. In particular,  
the cosets 
\[
H, yH, y^2H, . . . , y^{p-1}H
\]
are all different (because if $y^iH=y^jH$ for some $i<j$, then $y^{j-i}\in H$ and $j-i\leq p-2$). Since 
$y=xhx^{-1}$, 
\[
(yx)H = (xh)H= xH= y^iH
\]
for some $i\in\{0,\dots,p-1\}$. If $i=0$, 
then $yx= xh\in H$ and therefore $x\in H$, a contradiction. Hence $(yx)H= y^iH$ for some 
$i\in\{1,\dots,p-1\}$, which implies 
$xH=y^{i-1}H$. Therefore 
\[
y^iH= xH= y^{i-1}H
\]
for some $i\in\{1,\dots,p-2\}$, which implies that 
$y\in H$, a contradiction. 
\end{sol}

\begin{sol}{xca:p_smallest}
    If $g\in G\setminus H$, then $g^n=1\in H$, where $n=|G|$. Since $p$ is prime, $n$ has no prime divisors $<p$. By Exercise \ref{xca:index_p}, $H$ is normal in $G$.
\end{sol}

\begin{sol}{xca:HK_normal}
We first prove that
$HK\subseteq KH$. If $x=hk\in HK$, then
 $x=k(k^{-1}hk)\in KH$, as $k^{-1}hk\in H$. To prove 
that $HK\supseteq KH$, let $y=kh\in KH$. Then $y=(khk^{-1})k\in HK$, as  $khk^{-1}\in H$. 
\end{sol}

\begin{sol}{xca:U(Z/10)}
Just note that $\mathcal{U}(\Z/12)$ has no elements of order four.
\end{sol}

\begin{sol}{xca:p_groups}
    If $G$ is a $p$-group, then, by Lagrange's theorem, 
    every element has order a power of $p$. Conversely, 
    if $q$ is a prime divisor of $|G|$, by 
    Cauchy's theorem, there exists $g\in G$ of order $q$. Thus $q=p$.
\end{sol}


\begin{sol}{xca:factors:24,12,4,2}
Decompose $A$ as $(\Z/4)\times(\Z/2)\times(\Z/3)\times(\Z/8)\times(\Z/4)\times(\Z/3)$.
We list the highest powers appearing in our decomposition of $A$: 
\[ 
\begin{matrix}
8&3\\
4&3\\
4\\
2
\end{matrix} 
\] 
Then $s_1=2$, $s_2=4$, $s_3=12$ and $s_4=24$. Hence 
$A\simeq (\Z/24)\times(\Z/12)\times(\Z/4)\times(\Z/2)$.
\end{sol}


\fancyhf{}
\fancyfoot[R]{\thepage}
\fancyhead[L]{\course}
\setlength{\headheight}{14pt}

\bibliographystyle{abbrv}
\bibliography{refs}

\printindex

\end{document}
