\chapter*{Some solutions}

\pagestyle{plain}
\fancyhf{}
\fancyhead[LE,RO]{Rings and modules}
\fancyhead[RE,LO]{Some solutions}
\fancyfoot[CE,CO]{\leftmark}
\fancyfoot[LE,RO]{\thepage}

\addcontentsline{toc}{chapter}{Some solutions}

\begin{sol}{xca:commutator}
For $x,y\in G$,
\begin{align*}
    (xH)(yH)=(yH)(xH) \Longleftrightarrow (xy)H=(yx)H \Longleftrightarrow x^{-1}y^{-1}xy\in H.
\end{align*}
Thus $G/H$ is abelian if and only if  $[x,y]=xyx^{-1}y^{-1}\in H$ for all $x,y\in G$.
\end{sol}


\begin{sol}{xca:G/Z(G)}
Assume that $G/Z(G)=\langle gZ(G)\rangle$. Let $x,y\in G$. 
Write $xZ(G)=g^kZ(G)$ and $yZ(G)=g^lZ(G)$, 
that is 
$x=g^kz_1$ and $y=g^lz_2$ for some $k,l\in\Z$ y $z_1,z_2\in Z(G)$. Thus $xy=yx$.
\end{sol}

\begin{sol}{xca:p_smallest}
    If $g\in G\setminus H$, then $g^n=1\in H$, where $n=|G|$. Since $p$ is prime, $n$ has no prime divisors $<p$. By Exercise \ref{xca:index_p}, $H$ is normal in $G$.
\end{sol}

\begin{sol}{xca:HK_normal}
We need to show that $HK=KH$. We first prove that
$HK\subseteq KH$. If $x=hk\in HK$, then
 $x=k(k^{-1}hk)\in KH$, as $k^{-1}hk\in H$. To prove 
that $HK\supseteq KH$, let $y=kh\in KH$. Then $y=(khk^{-1})k\in HK$, as  $khk^{-1}\in H$. 
\end{sol}