\section*{Some solutions}

\fancyhf{}
\fancyfoot[R]{\thepage}
\fancyhead[L]{\course}
\fancyhead[R]{Some solutions}
\setlength{\headheight}{14pt}

% \pagestyle{plain}
% \fancyhf{}
% \fancyhead[LE,RO]{Rings and modules}
% \fancyhead[RE,LO]{Some solutions}
% \fancyfoot[CE,CO]{\leftmark}
% \fancyfoot[LE,RO]{\thepage}
% \addcontentsline{toc}{chapter}{Some solutions}
\begin{sol}{xca:neutral}
If $e$ and $e_1$ are both neutral elements, then $e=ee_1=e_1$. 
\end{sol}

\begin{sol}{xca:ax=b}
    If $ax=b$, after multiplying on the left by $a^{-1}$ we
    obtain that $x=a^{-1}b$. Similarly, the equation $xa=b$ 
    has $x=ba^{-1}$ as its unique solution. 
\end{sol}

\begin{sol}{xca:LR}
    For $g\in G$, the map $L_g\colon G\to G$, $x\mapsto gx$, is invertible 
    with inverse $L_{g^{-1}}$, as
    \[
    (L_g\circ L_{g^{-1}})(x)=g(g^{-1}x)=(gg^{-1})x=x
    \]
    for all $x\in G$. Similarly, $L_{g^{-1}}\circ L_g)(x)=x$ for all $x\in G$. 
    
    In the same way, we prove that 
    for each $g\in G$, the map $R_{g^{-1}}$ is the inverse of $R_g$. 
\end{sol}

\begin{sol}{xca:GxH}
    The neutral element of $G\times H$ is $(1,1)$, as $(1,1)(g,h)=(g,h)=(g,h)(1,1)$. The inverse
    of $(g,h)$ is $(g,h)^{-1}=(g^{-1},h^{-1})$, as
    \begin{align*}
    (g,h)(g,h)^{-1}&=(g,h)(g^{-1},h^{-1})=(gg^{-1},hh^{-1})=(1,1),\\
    (g,h)^{-1}(g,h)&=(g^{-1},h^{-1})(g,h)=(g^{-1}g,h^{-1}h)=(1,1).
    \end{align*}
    To prove the associativity, let $g,g_1,g_2\in G$ and 
    $h,h_1,h_2\in H$. Since $G$ and $H$ are groups, 
    their multiplications are associative. Then 
    \begin{align*}
        ((g,h)(g_1,h_1))(g_2,h_2) &= (gg_1,hh_1)(g_2,h_2)
        =((gg_1)g_2,(hh_1)h_2)\\
        &= (g(g_1g_2),h(h_1h_2))
        = (g,h)(g_1g_2,h_1h_2)
        = (g,h)((g_1,h_1)(g_2,h_2)).
    \end{align*}
\end{sol}

\begin{sol}{xca:center}
    Clearly $1\in Z(G)$. If $x\in Z(G)$, then $xg=gx$ for all $g\in G$. Multiplying by $x^{-1}$ on the left and 
    on the right, we get that $gx^{-1}=x^{-1}$ holds for all $g\in G$. Finally, if $x,y\in Z(G)$. Then
    \[
    (xy)g=x(yg)=x(gy)=(xg)y=(gx)y=g(xy)
    \]
    for all $g\in G$. Hence $xy\in Z(G)$. 
\end{sol}

% \begin{sol}{xca:centralizer}
%     First, $1\in G_G(g)$, as $1g=g1=g$. If $x\in c_G(g)$, then $xg=gx$. Multiplying 
%     on the left and on the right by $x^{-1}$, one gets $x^{-1}g=gx^{-1}$, that is 
%     $x^{-1}\in C_G(g)$. Finally, if $x,y\in C_G(g)$, then 
%     \[
%     $(xy)g=$
%     \]
% \end{sol}

\begin{sol}{xca:conjugate}
    Since $S$ is a subgroup, $1\in S$ and 
    if $x,y\in S$, then $x^{-1}\in S$ and $xy\in S$. Now 
    $1\in gSg^{-1}$, as $1\in S$ and $1=g1g^{-1}$. If $x\in gSg^{-1}$, then 
    $x=gsg^{-1}$ for some $s\in S$. Thus 
    \[
    x^{-1}=(gsg^{-1})^{-1}=gs^{-1}g^{-1}\in gSg^{-1},
    \]
    as $s^{-1}\in S$. Finally, 
    if $x=gsg^{-1}\in gSg^{-1}$ and $y=gtg^{-1}\in gSg^{-1}$ for some $s,t\in S$, then 
    \[
    xy=(gsg^{-1})(gtg^{-1})=g(st)g^{-1}\in gSg^{-1},
    \]
    as $st\in S$. 
\end{sol}

\begin{sol}{xca:center_S3}
    If $\sigma\in Z(\Sym_3)$ and $\sigma\ne\id$, there exists $i\in\{1,2,3\}$ such that 
    $\sigma(i)\ne i$. Let $j=\sigma(i)$ and $k\in\{1,2,3\}\setminus\{i,j\}$. Then 
    $(jk)\sigma$ is a permutation such that $i\mapsto k$, while 
    $\sigma(jk)$ is such that $i\mapsto j$. In particular, $(jk)\sigma\ne\sigma(jk)$, a contradiction.

    The group $\Sym_3$ has six elements: $\id$, $(12)$, $(13)$, $(23)$. $(123)$ and $(132)$. 
    First note that $\id\in C_{\Sym_3}((12))$ and 
    $(12)\in C_{\Sym_3}((12))$. However, 
    the permutations $(23)$, $(13)$, $(123)$ and $(132)$ do not commute with
    $(12)$. For example, 
    \[
    (23)(12)=(132)\ne (123)=(12)(23).
    \]
\end{sol}

\begin{sol}{xca:subgroup}
    Let us prove $\implies$. Since $1\in S$, then $S\ne\emptyset$. If $u,v\in S$, then 
    $v^{-1}\in S$ and $uv^{-1}\in S$. 

    Let us prove now $\impliedby$. If $S\ne\emptyset$, let $u\in S$. Then $1=uu^{-1}\in S$. The assumption 
    Let $u,v\in S$. The assumption with $x=1\in S$ and $y=v$ yields $v^{-1}\in S$. The assumption 
    with $x=u$ and $y=v^{-1}$ yields $uv\in S$. 
\end{sol}

\begin{sol}{xca:SL_subgroup}
    The identity matrix belongs to $\SL_n(\R)$. If $a,b\in\SL_n(\R)$, then 
    $ab^{-1}\in\SL_n(\R)$, as 
    \[
    \det(ab^{-1})=\det(a)\det(b^{-1})=\det(a)\det(b)^{-1}=1.
    \]
    By Exercise \ref{xca:subgroup}, $\SL_n(\R)$ is a subgroup of $\GL_n(\R)$. 
\end{sol}

\begin{sol}{xca:intersection}
    Let $\{H_\lambda:\lambda\in\Lambda\}$ be a collection of subgroups of a group $G$ and 
    $H=\cap_{\lambda\in \Lambda}H_\lambda$. We claim that $H$ is a subgroup of $G$. Since
    $1\in H_\lambda$ for all $\lambda$, $H$ is non-empty. If $x,y\in H$, then $x,y\in H_\lambda$ for all $\lambda$. 
    Since each $H_\lambda$ is a subgroup of $G$, $xy^{-1}\in H_\lambda$ for all $\lambda$. Thus $xy^{-1}\in H$.
\end{sol}

\begin{sol}{xca:generated}
    Let 
    \[
    H=\{x_1^{n_1}\cdots x_k^{n_k}:k\geq0,\,x_1,\dots,x_k\in X,\,-1\leq n_1,\dots,n_k\leq 1\}.
    \]
    To prove that $H\subseteq\langle X\rangle$, let $h=x_1^{n_1}\cdots x_k^{n_k}\in H$. 
    If $S$ is a subgroup of $G$ containing $X$, then $x_j\in S$ for all $j$. This implies that 
    $h=x_1^{n_1}\cdots x_k^{n_k}\in S$. Thus 
    \[
    h\in\bigcap_{\substack{S\leq G\\X\subseteq S}}S.
    \]
    
    To prove that $H\supseteq \langle X\rangle$ we first 
    claim that $H$ is a subgroup of $G$. Note that $H\ne\emptyset$, as $1\in H$ (this is the empty word). If 
    $u=x_1^{n_1}\cdots x_k^{n_k}\in H$ and 
    $v=x_{k+1}^{n_{k+1}}\cdots x_{l}^{n_l}\in H$, then 
    \[
    uv^{-1}=x_1^{n_1}\cdots x_k^{n_k}x_{l}^{-n_{l}}\cdots x_{k+1}^{-n_{k+1}}\in H. 
    \]
    Now note that $H$ is a subgroup of $G$ containing $X$. Thus 
    \[
    \langle X\rangle=\bigcap_{\substack{S\leq G\\X\subseteq S}}S\subseteq H.
    \]
\end{sol}

\begin{sol}{xca:union}
    Let $G=\Sym_3$. Then $H=\{\id,(12)\}$ and 
    $K=\{\id,(23)\}$ are subgroups of $G$. However, 
    $H\cup K=\{\id,(12),(23)\}$ is not a subgroup, as 
    $(12)(23)=(123)\not\in H\cup K$. 
\end{sol}

\begin{sol}{xca:permutation_matrix}
Let $\{e_1,\dots,e_n\}$ be the standard basis of $\R^n$. 
To prove this formula note that
\[
E_{i,j}e_k=\begin{cases}
    e_i&\text{if $j=k$,}\\
    0 & \text{if $j\ne k$}.
\end{cases}
\]
and verify that 
$P_\sigma e_k=\sum_{i=1}^n E_{\sigma(i),i}e_k$ 
for all $k\in\{1,\dots,n\}$. Since $P_\sigma$ and
$\sum_{i=1}^n E_{\sigma(i),i}$ coincide in a basis of $\R^n$, 
they are equal. 
\end{sol} 

\begin{sol}{thm:quotient}
Since $N$ is normal in $G$, the operation is well-defined. 
Routine calculations show that 
the operation is associative, that
$N$ is the neutral element of $G/N$ and that 
the inverse of an element $xN$ is 
$(xN)^{-1}=x^{-1}N$. For example, for the associativity, 
we note that for $x,y,z\in G$ one has 
\begin{align*}
    &((xN)(yN))(zN)=((xy)N)zN=(xy)zN,
\shortintertext{equals}
    &(xN)((yN)(zN))=(xN)((yz)N)=x(yz)N.
\end{align*}
since $x(yz)=(xy)z$.
\end{sol}

\begin{sol}{xca:commutator}
For $x,y\in G$,
\begin{align*}
    (xH)(yH)=(yH)(xH) \Longleftrightarrow (xy)H=(yx)H \Longleftrightarrow x^{-1}y^{-1}xy\in H.
\end{align*}
Thus $G/H$ is abelian if and only if  $[x,y]=xyx^{-1}y^{-1}\in H$ for all $x,y\in G$.
\end{sol}


\begin{sol}{xca:G/Z(G)}
Assume that $G/Z(G)$ is generated by $gZ(G)$. Let $x,y\in G$. Then 
$xZ(G)=g^kZ(G)$ and $yZ(G)=g^lZ(G)$ for some $k,l\in\Z$,  
that is 
$x=g^kz_1$ and $y=g^lz_2$ for some $k,l\in\Z$ y $z_1,z_2\in Z(G)$. Thus $xy=yx$.
\end{sol}



\begin{sol}{xca:p_smallest}
    If $g\in G\setminus H$, then $g^n=1\in H$, where $n=|G|$. Since $p$ is prime, $n$ has no prime divisors $<p$. By Exercise \ref{xca:index_p}, $H$ is normal in $G$.
\end{sol}

\begin{sol}{xca:HK_normal}
We need to show that $HK=KH$. We first prove that
$HK\subseteq KH$. If $x=hk\in HK$, then
 $x=k(k^{-1}hk)\in KH$, as $k^{-1}hk\in H$. To prove 
that $HK\supseteq KH$, let $y=kh\in KH$. Then $y=(khk^{-1})k\in HK$, as  $khk^{-1}\in H$. 
\end{sol}

\begin{sol}{xca:U(Z/10)}
Just note that $\mathcal{U}(\Z/12)$ has no elements of order four.
\end{sol}

\begin{sol}{xca:p_groups}
    If $G$ is a $p$-group, then, by Lagrange's theorem, 
    every element has order a power of $p$. Conversely, 
    if $q$ is a prime divisor of $|G|$, by 
    Cauchy's theorem, there exists $g\in G$ of order $q$. Thus $q=p$.
\end{sol}


\begin{sol}{xca:factors:24,12,4,2}
Decompose $A$ as $(\Z/4)\times(\Z/2)\times(\Z/3)\times(\Z/8)\times(\Z/4)\times(\Z/3)$.
We list the highest powers appearing in our decomposition of $A$: 
\[ 
\begin{matrix}
8&3\\
4&3\\
4\\
2
\end{matrix} 
\] 
Then $s_1=2$, $s_2=4$, $s_3=12$ and $s_4=24$. Hence 
$A\simeq (\Z/24)\times(\Z/12)\times(\Z/4)\times(\Z/2)$.
\end{sol}
