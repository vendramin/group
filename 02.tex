\chapter{}

\topic{Subgroups}

\begin{definition}
\index{Subgroup}
    Let $G$ be a group. 
        A subset $S$ of $G$ is said to be a \textbf{subgrup} of $G$ 
        if the following properties are satisfied:
        \begin{enumerate}
                \item $1\in S$,
                \item $x\in S\implies x^{-1}\in S$, and 
                \item $x,y\in S\implies xy\in S$.
        \end{enumerate}
        Notation: $S$ is a subgroup of $G$ if and only if $S\leq G$.
\end{definition}

The first condition of the definition can be replaced by
the following condition: $S\ne\emptyset$. Why? 

\begin{example}
    If $G$ is a group, then 
    $\{1\}$ and $G$ are always subgroups of $G$. 
\end{example}

The subgroup $\{1\}$ is known as the \textbf{trivial subgroup} of $G$. 
A subgroup $S$ of $G$ is said to be \textbf{proper} if $S\ne G$. 

\begin{example}
Write $2\Z=\{2m:m\in\Z\}$ to denote the set of even integers. Then 
$2\Z\leq\Z\leq\Q\leq\R\leq\C$ is a chain of subgroups. 
\end{example}

\begin{example}
$S^1=\{z\in\C:|z|=1\}\leq\C^\times=\C\setminus\{0\}$.
\end{example}

\begin{example}
Let $n\geq1$. Then $G_n=\{z\in\C:z^
n=1\}$ is a subgroup of $\C^\times$. 
Note that
\[
G_n=\{1,\exp(2\pi i/n),\exp(4i\pi/n),\dots,\exp(2(n-1)i\pi/n)\}.
\]
and 
\[
G_n\leq\bigcup_{n\geq1}G_n\leq S^1\leq\C^\times.
\]
\end{example}

\begin{exercise}
        \index{Center!of a group}
        Let $G$ be a group. Prove that the \textbf{center} 
        \[
                Z(G)=\{g\in G:gh=hg\text{ for all $h\in G$}\}
        \]
        of $G$ is a subgroup of $G$. 
\end{exercise}

\begin{exercise}
        \index{Centralizer!of an element}
        Let $G$ be a group and $g\in G$. Prove that
        the \textbf{centralizer}
        \[
                C_G(g)=\{h\in G:gh=hg\}
        \]
        of $g$ in $G$ is a subgroup of $G$. 
\end{exercise}

One can prove that, if $G$ is a group, then
$Z(G)=\cap_{g\in G}C_G(g)$. 

\begin{exercise}
        \index{Conjugate of a subgroup}
        \label{xca:conjugate_subgroup}
        Let $S$ be a subgroup of $G$ and $g\in G$. Prove that
        the \textbf{conjugate} $gSg^{-1}$
        of $S$ by $g$ is a subgroup of $G$. 
        Notation: $\prescript{g}{}S=gSg^{-1}$.
\end{exercise}


\begin{exercise}
\index{Center!of $\Sym_3$}
        Prove that $Z(\Sym_3)=\{\id\}$ and compute $C_{\Sym_3}((12))$.
\end{exercise}

The following exercise is useful: 

\begin{exercise}
        Let $G$ be a group and $S$ be a subset of $G$. 
        Prove that $S$ is a subgroup of $G$ if and only if 
        $S\ne\emptyset$ and for all $x,y\in S$ one has
        $xy^{-1}\in S$.
\end{exercise}

Use the previous exercise and
the fact that the determinant is a multiplicative function
to solve the following problem:

\begin{exercise}
\label{xca:SL_subgroup}
Prove that $\SL_n(\R)=\{a\in\GL_n(\R):\det(a)=1\}\leq\GL_n(\R)$. 
\end{exercise}

\begin{exercise}
\label{xca:intersection}
    Prove that the intersection of subgroups is again a subgroup.
\end{exercise}

The previous exercise is easy but crucial. We need it 
to construct subgroups generated 
by a given subset of elements. 

\begin{definition}
        \index{Subgroup!generated by a subset}
        Let $G$ be a group and $X$ a subset of $G$. The \textbf{subgroup
        generated} by $X$ is the smallest subgroup of $G$ that contains
        $X$, that is 
        \[
            \langle X\rangle=\bigcap\{S:S\leq G,X\subseteq S\}.
        \]
\end{definition}

Why this is the smallest subgroup that contains $X$? 
Let $H\leq G$ be such that 
$X\subseteq H$. Since $H$ is one of the subgroups appearing
in the intersection, 
\[
        \langle X\rangle=\bigcap\{S:S\leq G,X\subseteq S\}\subseteq H.        
\]

We will use the following notation:
If $X=\{g_1,\dots,g_k\}$, then 
\[
\langle
X\rangle=\langle\{g_1,\dots,g_k\}\rangle=\langle g_1,\dots,g_k\rangle.
\]

\begin{exercise}
Prove that 
\[
        \langle X\rangle=\{x_1^{n_1}\cdots x_k^{n_k}:k\geq0,\,x_1,\dots,x_k\in X,\,-1\leq n_1,\dots,n_k\leq 1\}.
\]
\end{exercise}

The previous exercise shows that
the subgroup generated by, say, the elements 
$x_1,\dots,x_n$ is nothing but the 
group formed by (some) words on the letters 
$x_1,\dots,x_n$ and their inverses 
$x_1^{-1},\dots,x_n^{-1}$. 

\begin{example}
Let $n\geq3$. Let 
\[
r=\begin{pmatrix}
\cos(2\pi/n) & -\sin(2\pi/n)\\
\sin(2\pi/n) & \cos(2\pi/n)
\end{pmatrix},
\quad
s=\begin{pmatrix}
        1 & 0\\
        0 & -1
\end{pmatrix}.
\]
The \textbf{dihedral group} $\D_n$ is the subgroup of
$\GL_2(\C)$ generated by $r$ and $s$,
that is $\D_n=\langle r,s\rangle$. A direct calculation shows that 
\[
r^n=s^2=\begin{pmatrix}
        1&0\\
        0&1
\end{pmatrix},
\quad
srs=r^{-1}.
\]

An element of $\D_n$ is a word of the form 
\[
r^{i_1}s^{j_1}r^{i_2}s^{j_2}\cdots
\]
for some 
 $i_1,i_2,\dots\in\{0,1,\dots,n-1\}$ and 
$j_1,j_2,\dots\in\{0,1\}$. Since $rs=sr^{-1}$, we conclude that
every element of $\D_n$ can be written as $r^is^j$ 
for some $i\in\{0,\dots,n-1\}$ and $j\in\{0,1\}$. In particular, 
$|\D_n|=2n$.
\end{example}

To understand better the previous example, 
we discuss two concrete particular cases. 
If $n=3$, 
\[
r=\begin{pmatrix}
-1/2 & -\sqrt{3}/2\\
\sqrt{3}/2 & -1/2
\end{pmatrix},
\quad
s=\begin{pmatrix}
        1 & 0\\
        0 & -1
\end{pmatrix}.
\]
and we obtain (another representation of) the group of symmetries of a regular 
triangle. 
If 
$n=4$, 
\[
r=\begin{pmatrix}
0 & -1\\
1 & 0
\end{pmatrix},
\quad
s=\begin{pmatrix}
        1 & 0\\
        0 & -1
\end{pmatrix}.
\]
and we obtain (another representation of) the group of symmetries of the square. 

\begin{exercise}
        The union of subgroups is not, in general, 
        a subgroup. Can you give an example? 
\end{exercise}

\topic{Subgroups of $\Z$}

It is time for the first theorem. 
What can we say about the subgroups of $\Z$? 

\begin{theorem}
        \label{thm:Z}
        If $S$ is a subgroup of $\Z$, then
                $S=m\Z=\{mx:x\in \Z\}$
                for some $m\geq0$.
        \end{theorem}
        
        \begin{proof}
                If $S=\{0\}$, take $m=0$. 
                Assume now that $S\ne\{0\}$. Let 
                $m=\min\{s\in S:s>0\}$. 
                Why does this $m$ exist?  
                Since $S\ne\{0\}$,  
                it contains 
                an element $n\in S\setminus\{0\}$. 
                There are then two possible cases: 
                $n>0$ or $-n>0$. Since 
                $S$ is a subgroup of $\Z$, $-n\in S$.
        
                We claim that $S=n\Z$.
                If $x\in S$, then $x=my+r$ for $y,r\in\Z$ with 
                $0\leq r<m$. Suppose that $r\ne 0$. Since $x,m\in S$, it follows that 
                $r\in S$,
                a contradiction to the minimality of $n$. Thus $r=0$ 
                and hence $x=my\in
                m\Z$. Conversely, since $n\in S$, it follows that
                 $nk\in S$ for all $k\in\Z$. In fact, if $k=0$, then 
                 $nk=0\in S$. If $k>0$, 
                 then 
                \[
                \underbrace{n+\cdots+n}_{k-\text{times}}\in S.
                \]
                Finally, if $k<0$, 
                then 
                \[
                nk=\underbrace{-n+(-n)+\cdots+(-n)}_{|k|-\text{times}}\in S.\qedhere
                \]
        \end{proof}

The previous theorem has nice applications. 
If $a,b\in\Z$, we say that $a$ \textbf{divides} $b$ (or $b$ is divisible by $a$)
if $b=ac$ for some $c\in\Z$. Notation: 
\[
a\mid b\Longleftrightarrow b=ac\text{ for some $c\in\Z$.}
\]
If $a,b\in\Z$ are such that $ab\ne0$, then 
\[
S=a\Z+b\Z=\{m\in\Z:m=ar+bs\text{ for $r,s\in\Z$}\}
\]
is a subgroup of $\Z$ (this is an exercise). 
By Theorem \ref{thm:Z}, $S=d\Z$ for some $d>0$. 
This positive integer $d$
is the \textbf{greatest common divisor} of $a$ and $b$, 
that is $d=\gcd(a,b)$. 

\begin{exercise}
Let $a,b\in\Z$ be such that $ab\ne0$ and $d=\gcd(a,b)$. 
Prove the following statements:
\begin{enumerate}
\item $d$ divides $a$ and $b$.
\item If $e\in\Z$ divides $a$ and $b$, then $e$ divides $d$.
\item There are $r,s\in\Z$ such that $d=ar+bs$.
\end{enumerate}
\end{exercise}

Two integers $a$ and $b$ are said to be \textbf{coprime} if 
and only if the only positive integer dividing 
$a$ and $b$ is one, that is  
\begin{align*}
a\text{ and }b\text{ are coprime}&\Longleftrightarrow \gcd(a,b)=1\\
&\Longleftrightarrow \Z=a\Z+b\Z\\
&\Longleftrightarrow \text{there exist $r,s\in\Z$ such that $ar+bs=1$.}
\end{align*}

\begin{exercise}
        Let $p$ be a prime and 
        $a,b\in\Z$. Prove that if $p\mid ab$, 
        then $p\mid a$ or $p\mid b$.
\end{exercise}

If $S$ and $T$ are subgroups of $\Z$, then $S\cap T$
is a subgroup of $\Z$.
Let $a,b\in\Z$ be such that $ab\ne 0$. Since $a\Z\cap b\Z$ 
is a non-zero subgroup of $\Z$ (note that it contains $ab\ne 0$), 
we can write  $a\Z\cap b\Z=m\Z$
for some $m\geq1$. The integer $m$
is the \textbf{least common multiple} of $a$ and $b$ 
and will be written as $m=\lcm(a,b)$.

\begin{exercise}
Let $a,b\in\Z\setminus\{0\}$ and $m=\lcm(a,b)$. 
Prove the following statements:
\begin{enumerate}
        \item $m$ is divisible by both $a$ and $b$.
        \item If $n$ is divisible by both $a$ and $b$, then 
        $n$ is divisible by $m$.
\end{enumerate}
\end{exercise}

\begin{exercise}
Let $a,b\in\Z_{\geq1}$. Prove that if $d=\gcd(a,b)$ and $m=\lcm(a,b)$, then 
$ab=dm$.
\end{exercise}

\topic{Commutators}

\begin{definition}
        \index{Derived subgroup}
        \index{Commutator subgroup}
        The \textbf{commutator subgroup}
        $[G,G]$ of $G$ 
        is the subgroup generated by 
        the commutators of $G$, that is 
        \[
        [G,G]=\langle[x,y]\mid x,y\in G\rangle,
        \]
        where $[x,y]=xyx^{-1}y^{-1}$ is the commutator of 
        $x$ and $y$.
\end{definition}
  
In the literature, the commutator subgroup of a group $G$ is also called the \textbf{derived 
subgroup} of $G$. 
       
\begin{example}
        In $\Z$, the commutator of $x,y\in \Z$ 
        is the integer
        \[
        [x,y]=x+y-x-y=0.
        \]
        This example uses additive notation! 
        Thus $[\Z,\Z]=\{0\}$. 
\end{example}
        
\begin{exercise}
        Prove that $[\Sym_3,\Sym_3]=\{\id,(123),(132)\}$.
\end{exercise}
        
Why do we need to consider the subgroup 
generated by commutators? Because the set of commutators 
is not always a subgroup. With the help of computers, 
one can verify the following examples. 
The first one is taken from Carmichael's book
\cite{MR0075938}.

\begin{example}
        Let $G$ be the subgroup of $\Sym_{16}$ 
        generated by the permutations 
        \begin{align*}
&a = (13)(24),&&
b = (57)(68),\\
&c = (9\,11)(10\,12),&&
d = (13\,15)(14\,16),\\
&e = (13)(57)(9\,11),&&
f = (12)(34)(13\,15),\\
&g = (56)(78)(13\,14)(15\,16),&&
h = (9\,10)(11\,12).
\end{align*}
Then $[G,G]$ has order 16, but the set of commutators of 
$G$ has 15 elements. 
\end{example}

The following example goes back to Guralnick~\cite{MR673806}. 
It was found by hand when computers were not as popular
in group theory as now. 

\begin{example}
The group 
\[
G=\langle (135)(246)(7\,11\,9)(8\,12\,10),(394\,10)(58)(67)(11\,12)\rangle.
\]
has order 96. The set of commutators 
is different from the commutator subgroup. Moreover, 
$G$ is the smallest group with the property that  
the set of commutators is not a subgroup. 
\end{example}

\topic{Cyclic groups}

\begin{definition}
        \index{Cyclic group}
        A group $G$ is said to be \textbf{cyclic} if 
        $G=\langle g\rangle$ for some 
        $g\in G$.
\end{definition}

If $G$ is a cyclic group generated by $g$, then 
$G=\langle g\rangle=\{g^k:k\in\Z\}$. Every cyclic group is, 
in particular, an abelian group. 

\begin{examples}\
\begin{enumerate}
        \item $\Z=\langle 1\rangle=\langle -1\rangle$.
        \item $\Z/n=\langle 1\rangle$.
        \item $G_n=\langle \exp(2i\pi/n)\rangle$.
\end{enumerate}
\end{examples}

\begin{example}
        $\mathcal{U}(\Z/8)\ne\langle 3\rangle$. In fact, $\langle 3\rangle=\{1,3\}\subsetneq\{1,3,5,7\}=\mathcal{U}(\Z/8)$.
\end{example}

\begin{exercise}
\label{xca:subgroups_cyclic}
        Prove that subgroups of a cyclic group are cyclic.
\end{exercise}

\begin{definition}
        \index{Order!of an element}
        Let $G$ be a group and $g\in G$. The \textbf{order} of $g$
        is the order of the subgroup generated by $g$. Notation:
        $|g|=|\langle g\rangle|$.
\end{definition}


\begin{theorem}
        Let $G$ be a group and $g\in G$ and $n\geq1$. 
        The following statements are equivalent:
        \begin{enumerate}
                \item $|g|=n$.
                \item $n=\min\{k\in\Z_{\geq1}:g^k=1\}$.
                \item For every $k\in\Z$, $g^k=1\Longleftrightarrow n\mid k$.
                \item $\langle g\rangle=\{1,g,\dots,g^{n-1}\}$ and 
                the elements $1,g,\dots,g^{n-1}$ are all different.
        \end{enumerate}
\end{theorem}


\begin{proof}
        We first prove that $(1)\implies(2)$.
        If $g=1$, then $n=1$. Assume that $g\ne1$. Since $\langle g\rangle=\{g^k:k\in\Z\}$,
        there exist integers $i$ and $j$ with $i>j$ such that $g^i=g^j$, that is $g^{i-j}=1$. In particular,
        the set $\{k\in\Z_{\geq1}:g^k=1\}$ is non-empty and hence has a minimal element, say 
        \[
        d=\min\{k\in\Z_{\geq1}:g^k=1\}.
        \]
        Thus $\langle g\rangle\subseteq\{1,g,\dots,g^{d-1}\}\subseteq\langle g\rangle$. 
        If $g^k\in\langle g\rangle$, then $k=dq+r$ for some $q,r\in\Z$ with $0\leq r<d$. Since $g^d=1$,
        \[
        g^k=g^{dq+r}=(g^d)^qg^r=g^r\in\{1=g^0,g,g^2,\dots,g^{d-1}\}
        \]
        Moreover, $\{1,g,\dots,g^{d-1}\}\subseteq \langle g\rangle$ and 
        $\{1,g,\dots,g^{d-1}\}$ has $d$ elements.

        We now prove that $(2)\implies(3)$. Assume that $g^k=1$. If 
        we write $k=nt+r$ with $0\leq r<n$, then $g^k=g^{nt+r}=g^r$. The minimality of $n$ 
        implies that $r=0$. Hence $n$ divides $k$.
        Conversely, if $k=nt$ for some $t\in\Z$, then $g^k=(g^n)^t=1$.

        Let us prove that $(3)\implies(4)$. Clearly, 
        $\{1,g,\dots,g^{n-1}\}\subseteq\langle g\rangle$. To prove the other 
        inclusion, we write $k=nt+r$ with $0\leq r\leq n-1$. Then 
        \[
                g^k=g^{nt+r}=(g^n)^tg^r=g^r, 
        \]
        as, by assumption, $g^n=1$. To see that the elements 
        $1,g,\dots,g^{n-1}$ are all different, it is enough to show that if $g^k=g^l$ with $0\leq
        k<l\leq n-1$, then, since $g^{l-k}=1$ and $0<l-k\leq n-1$, it follows that 
        $n\leq l-k$ (because by assumption $n$ divides $l-k$, a contradiction).

        Finally, the implication $(4)\implies(1)$ is trivial.
\end{proof}

\begin{corollary}
        If $G$ is a group and $g\in G$ has order $n$, 
        then 
        \[
        |g^m|=\frac{n}{\gcd(n,m)}.
        \]
        \end{corollary}
        
\begin{proof}
        Let $k$ be such that $(g^m)^k=1=g^{mk}$. This means that $n$ divides $km$, as $g$ 
        has order $n$. This is also equivalent to the fact that 
        $n/d$ divides $mk/d$, where $d=\gcd(n,m)$. Therefore, since $n/d$ and $m/d$ 
        are coprime, $(g^m)^k=1$ is equivalent to  
        $n/d$ divides $k$, which implies that $g^m$ has order $n/
        d$.
\end{proof}
        
\begin{exercise}
        Let $G$ be a group and $g\in G$. Prove that the following statements are equivalent:
        \begin{enumerate}
        \item $g$ has infinite order. 
        \item The set $\{k\in\Z_{\geq1}:g^k=1\}$ is empty.
        \item If $g^k=1$, then $k=0$.
        \item If $k\ne l$, then $g^k\ne g^l$.
        \end{enumerate}
\end{exercise}

\begin{exercise}
\index{Torsion in abelian groups}
        Let $G$ be an abelian group. Prove that 
        $T(G)=\{g\in G:|g|<\infty\}$ is a subgroup of $G$. Compute $T(\C^\times)$.
\end{exercise}
                
\begin{exercise}
        Let $G=\langle g\rangle$ be a cyclic group. 
        \begin{enumerate}
                \item If $G$ is infinite, only $g$ and $g^{-1}$ generate $G$.
                \item If $G$ is finite of order $n$, then 
                        $G=\langle g^k\rangle$ if and only if $k$ and $n$ are coprime.
        \end{enumerate}
\end{exercise}
                
The following exercise is a particular 
case of Cauchy's theorem. 

\begin{exercise}
        \label{xca:orden2}
        Prove that every group of odd order contains
        an element of order two. 
\end{exercise}
                
Let us see some concrete examples: 

\begin{example}
        In $\Sym_3$ we have the following order pattern:
        \[
        |\id|=1,\quad
        |(12)|=|(13)|=|(23)|=2,\quad
        |(123)|=|(132)|=3.
        \]
\end{example}
                        
\begin{example}
        In $\Z$, every non-zero element has 
        infinite order. 
\end{example}
                        
 \begin{example}
        In $\Z\times\Z/6$ there are elements of 
        (in)finite order. For example, $(1,0)$ 
        has infinite order and 
        $(0,1)$ has order six. 
 \end{example}
                        
\begin{example}
        The matrix $\begin{pmatrix}1&1\\0&1\end{pmatrix}\in\GL_2(\R)$ has infinite order.
\end{example}                     
                                
\begin{example}
        The group $G_\infty=\bigcup_{n\geq1}G_n$ is abelian and infinite. Note that every element of 
        $G_\infty$ has finite order. 
\end{example}
          
We conclude the topic with some exercises. 

\begin{exercise}
        Compute the orders of the elements of $\Z/6$.
\end{exercise}       

\begin{exercise}
        Prove that $a=\begin{pmatrix}1&-1\\1&0\end{pmatrix}$ has order four, $b=\
        \begin{pmatrix}0&1\\-1&-1\end{pmatrix}$ has order three and 
        compute the order of $ab$.%=\begin{pmatrix}1&1\\0&1\end{pmatrix}$ tiene orden infinito.
\end{exercise}
                                
\begin{exercise}
        Compute the order of 
        $\begin{pmatrix}1&1\\-1&0\end{pmatrix}\in\GL_2(\R)$.
\end{exercise}
                                
\begin{exercise}
        Prove that in $\D_n$ one has 
        $|r^js|=2$ and $|r^j|=n/\gcd(n,j)$.
\end{exercise}
                                
\begin{exercise}
        Prove that a group with finitely many subgroups
        is finite. 
\end{exercise}


