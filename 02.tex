\chapter{}

\topic{Subgroups}


\begin{definition}
\index{Subgroup}
    Let $G$ be a group. 
        A subset $S$ of $G$ is said to be a \textbf{subgrup} of $G$ 
        if the following properties are satisfied:
        \begin{enumerate}
                \item $1\in S$,
                \item $x\in S\implies x^{-1}\in S$, and 
                \item $x,y\in S\implies xy\in S$.
        \end{enumerate}
        Notation: $S$ is a subgroup of $G$ if and only if $S\leq G$.
\end{definition}

The first condition of the definition can be replaced by
the following condition: $S\ne\emptyset$. Why? 

\begin{example}
    If $G$ is a group, then 
    $\{1\}$ and $G$ are always subgroups of $G$. 
\end{example}

The subgroup $\{1\}$ is known as the \textbf{trivial subgroup} of $G$. 
A subgroup $S$ of $G$ is said to be \textbf{proper} if $S\ne G$. 

\begin{example}
Write $2\Z=\{2m:m\in\Z\}$ to denote the set of even integers. Then 
$2\Z\leq\Z\leq\Q\leq\R\leq\C$ is a chain of subgroups. 
\end{example}

\begin{example}
$S^1=\{z\in\C:|z|=1\}\leq\C^\times=\C\setminus\{0\}$.
\end{example}

\begin{example}
$S^1=\{z\in\C:|z|=1\}\leq\C^\times=\C\setminus\{0\}$.
\end{example}

\begin{example}
Let $n\geq1$. Then $G_n=\{z\in\C:z^
n=1\}$ is a subgroup of $\C^\times$. 
Note that
\[
G_n=\{1,\exp(2\pi i/n),\exp(4i\pi/n),\dots,\exp(2(n-1)i\pi/n)\}.
\]
and 
\[
G_n\leq\bigcup_{n\geq1}G_n\leq S^1\leq\C^\times.
\]
\end{example}

\begin{exercise}
        \index{Center!of a group}
        Let $G$ be a group. Prove that the \textbf{center} 
        \[
                Z(G)=\{g\in G:gh=hg\text{ for all $h\in G$}\}
        \]
        of $G$ is a subgroup of $G$. 
\end{exercise}

\begin{exercise}
        \index{Centralizer!of an element}
        Let $G$ be a group and $g\in G$. Prove that
        the \textbf{centralizer}
        \[
                C_G(g)=\{h\in G:gh=hg\}
        \]
        of $g$ in $G$ is a subgroup of $G$. 
\end{exercise}

One can prove that, if $G$ is a group, then
$Z(G)=\cap_{g\in G}C_G(g)$. 

\begin{exercise}
\index{Center!of $\Sym_3$}
Prove that $Z(\Sym_3)=\{\id\}$ and compute $C_{\Sym_3}((12))$.
\end{exercise}

The following exercise is useful: 

\begin{exercise}
        Let $G$ be a group and $S$ be a subset of $G$. 
        Prove that $S$ is a subgroup of $G$ if and only if 
        $S\ne\emptyset$ and for all $x,y\in S$ one has
        $xy^{-1}\in S$.
\end{exercise}

Use the previous exercise and
the fact that the determinant is a multiplicative function
to solve the following problem:

\begin{exercise}
\label{xca:SL_subgroup}
$\SL_n(\R)=\{a\in\GL_n(\R):\det(a)=1\}\leq\GL_n(\R)$. 
%En efecto, la matriz identidad pertenece a $\SL_2(\R)$ y luego $\SL_2(\R)$ es no %vacío. Además si $a,b\in\SL_n(\R)$,
%entonces $ab^{-1}\in\SL_2(\R)$ pues $\det(ab^{-1})=\det(a)\det(b)^{-1}=1        $.
\end{exercise}

\begin{exercise}
\label{xca:intersection}
    Prove that the intersection of subgroups is again a subgroup.
\end{exercise}

The previous exercise is easy but crucial. We 

\begin{definition}
        \index{Subgroup!generated by a subset}
        Let $G$ be a group and $X$ a subset of $G$. The \textbf{subgroup
        generated} by $X$ is the smallest subgroup of $G$ that contains
        $X$, that is 
        \[
            \langle X\rangle=\bigcap\{S:S\leq G,X\subseteq S\}.
        \]
\end{definition}

One can indeed check that if $S\leq G$ is such that $X\subseteq S$, then
$S\subseteq\langle X\rangle$. 



Cuando el conjunto de generadores sea finito, se utilizará la siguiente notación. Si $X=\{g_1,\dots,g_k\}$, entonces $\langle
X\rangle=\langle\{g_1,\dots,g_k\}\rangle=\langle g_1,\dots,g_k\rangle$.

\begin{exercise}
Demuestre que $\langle
X\rangle$ es el menor subgrupo de $G$ que contiene a $X$, es decir que si $H$ es un subgrupo de $G$ tal que $X\subseteq H$, entonces $\langle X\rangle\subseteq H$.
\end{exercise}

\begin{exercise}
Demuestre que
\[
        \langle X\rangle=\{x_1^{n_1}\cdots x_k^{n_k}:k\in\N,\,x_1,\dots,x_k\in X,\,-1\leq n_1,\dots,n_k\leq 1\}.
\]
\end{exercise}

\begin{example}
Let $n\geq3$. Let 
\[
r=\begin{pmatrix}
\cos(2\pi/n) & -\sin(2\pi/n)\\
\sin(2\pi/n) & \cos(2\pi/n)
\end{pmatrix},
\quad
s=\begin{pmatrix}
        1 & 0\\
        0 & -1
\end{pmatrix}.
\]
The \textbf{dihedral group} $\D_n$ is the subgroup of
$\GL_2(\C)$ generated by $r$ and $s$,
that is $\D_n=\langle r,s\rangle$. A direct calculation shows that 
\[
r^n=s^2=\begin{pmatrix}
        1&0\\
        0&1
\end{pmatrix},
\quad
srs=r^{-1}.
\]
Un elemento arbitrario de $\D_n$ es una palabra
de la forma $r^{i_1}s^{j_1}r^{i_2}s^{j_2}\cdots$, donde $i_1,i_2,\dots\in\{0,1,\dots,n-1\}$ y
$j_1,j_2,\dots\in\{0,1\}$. Como $rs=sr^{-1}$, se concluye que
todo elemento de $\D_n$ puede escribirse como $r^is_j$,
donde $i\in\{0,\dots,n-1\}$ y $j\in\{0,1\}$. Luego
$|\D_n|=2n$.
\end{example}