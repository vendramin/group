\chapter{}

\topic{Lagrange's theorem}

Let $G$ be a group and $H$ be a subgroup of $G$. We say that the elements $x,y\in
G$ are (left) equivalent modulo $H$ if $x^{-1}y\in H$.
We will use the following notation:
\begin{equation}
\label{eq:modH}
    x\equiv y\bmod
    H\Longleftrightarrow x^{-1}y\in H.
\end{equation}

\begin{exercise}
    Prove that~\eqref{eq:modH} is an equivalence relation. This means that
    the following properties hold:
    \begin{enumerate}
    \item $x\equiv x\bmod H$ for all $x$.
    \item If $x\equiv y\bmod H$, then $y\equiv x\bmod H$.
    \item If $x\equiv y\bmod H$ and $y\equiv z\bmod H$, then $x\equiv z\bmod H$.
    \end{enumerate}
\end{exercise}

The equivalence classes of this equivalence relation modulo $H$
are the sets of the form $xH=\{xh:h\in H\}$, as the class 
of an element $x\in G$ is the set 
\[
        \{y\in G:x\equiv y\bmod H\}=\{y\in G:x^{-1}y\in H\}=\{y\in G:y\in xH\}=xH.
\]
The set $xH$ is called 
a \textbf{left coset} of $H$ in $G$.

Having an equivalence relation modulo $H$ in $G$ allows us to
decompose $G$ as a disjoint union of certain subsets related to $H$. 

\begin{proposition}
Let $G$ be a group and $H$ be a subgroup of $G$. 
\begin{enumerate}
\item If $xH\cap yH\ne\emptyset$, then $xH=yH$.
\item The group $G$ decomposes as a disjoint union 
of different left cosets of $H$.
\end{enumerate}
\end{proposition}

\begin{proof}
    Let us prove the first claim. If $g\in xH\cap yH$, we write 
    $g=xh$ for some $h\in H$. Then 
    \[
    gH=(xh)H=x(hH)=xH.
    \]
    Similarly, $gH=yH$. Hence $xH=yH$.
    The second claim follows from the first one. 
\end{proof}

One can also define right cosets: $x\equiv
y\bmod H$ if and only if $xy^{-1}\in H$. In this case, 
the equivalence classes are 
the sets of the form $Hx$ with $x\in X$. The set $Hx$ 
is called a \textbf{right coset}
of $H$ in $G$.

\begin{proposition}
    If $H$ is a subgroup of $G$, then  $|Hx|=|H|=|xH|$ for all $x\in G$.
\end{proposition}

\begin{proof}
    Let $x\in G$. The map $H\to Hx$, $h\mapsto hx$, is bijective 
    with inverse $hx\mapsto h$. Similarly, the map $H\to xH$,
    $h\mapsto xh$, is bijective.
\end{proof}

The map 
\[
        \{\text{right cosets of $H$ in $G$}\}\to\{\text{left cosets of $H$ in $G$}\}
\]
given by $Hx\mapsto x^{-1}H$ is a bijection, as 
\[
        Hx=Hy
        \Longleftrightarrow xy^{-1}\in H
        \Longleftrightarrow (x^{-1})^{-1}y^{-1}\in H
        \Longleftrightarrow x^{-1}H=y^{-1}H.
\]
In particular, the number of right cosets of $H$ in $G$
equals the number of left cosets of $H$ in $G$. 


\begin{example}
    If $G=\Z$ and $S=n\Z$, then 
    \[
    a+S=\{a+nq:q\in\Z\}=\{k\in\Z:k\equiv a\bmod n\}.
    \]
\end{example}
    
\begin{example}
    The subgroups of $\Sym_3$ are $\{\id\}$,the order-two subgroups 
    $\Sym_3$, $\langle(12)\rangle$, 
    $\langle(13)\rangle$ and $\langle(23)\rangle$, and 
    the order-three subgroup $\langle(123)\rangle=\{\id,(123),(132)\}$.  
    If $H=\langle(12)\rangle=\{\id,(12)\}$, then 
    \begin{align*}
    &H=(12)H=\{\id,(12)\},\\
    &(123)H=(13)H=\{(13),(123)\},\\
    &(132)H=(23)H=\{(23),(132)\}.
    \end{align*}
    Note that our group decomposes as 
    \[
    \Sym_3=H\cup (123)H\cup (132)H\quad\text{(disjoint union)}.
    \]
    \end{example}

    \begin{example}
        Let $G=\R^2$ with the usual addition 
        and $v\in\R^2$. The line $L=\{\lambda v:\lambda\in\R\}$ 
        is a subgroup of $G$. For each 
        $p\in R^2$, the coset $p+L$ 
        is the line parallel to $L$ that 
        passes through $p$.
    \end{example}


\begin{definition}
    \index{Index}
    If $H$ is a subgroup of $G$, the \textbf{índex} of $H$ in $G$
    is the number $(G:H)$ of left (or right) cosets of $H$ in $G$. 
\end{definition}

The following important theorem will be used extensively. 

\begin{theorem}[Lagrange]
\index{Lagrange's theorem}
    If $G$ is a finite group and $H$ is a subgroup of $G$, 
    then $|G|=|H|(G:H)$. In particular, $|H|$ divides $|G|$.
\end{theorem}

\begin{proof}
    We decompose $G$ into equivalence classes modulo $H$, that is 
    \[
    G=\bigcup_{i=1}^n x_iH\quad\text{(disjoint union)}
    \]
    for some $x_1,\dots,x_n\in G$, where $n=(G:H)$. 
    Since each of these equivalence classes has 
    exactly 
    $|H|$ elements,
    \[
            |G|=\sum_{i=1}^n|x_iH|=\sum_{i=1}^n|H|=|H|(G:H).\qedhere
    \]
\end{proof}

Let us discuss some corollaries. 

\begin{corollary}
    If $G$ is a finite group and $g\in G$, then $g^{|G|}=1$.
\end{corollary}

\begin{proof}
    By definition. $|g|=|\langle g\rangle|$. Apply Lagrange's theorem 
    to the subgroup $H=\langle g\rangle$ to obtain that 
    \[
            g^{|G|}=g^{|H|(G:H)}=(g^{|H|})^{(G:H)}=1.\qedhere
    \]
\end{proof}

\begin{corollary}
    If $G$ has prime order, then $G$ is cyclic. 
\end{corollary}

\begin{proof}
    Let $g\in G\setminus\{1\}$ and $H=\langle g\rangle$. By Lagrange's theorem, 
    $|H|$ divides $|G|$. Thus $|H|=|G|$, as $|G|$ is prime. Therefore 
    $G=H=\langle g\rangle$.
\end{proof}

\begin{corollary}
\label{cor:coprime_orders}
    If $G$ is an abelian group and $g,h\in G$ are elements of finite coprime orders, 
    then 
    $|gh|=|g||h|$.
\end{corollary}

\begin{proof}
    Let $n=|g|$, $m=|h|$ and $l=|gh|$. Since $G$ is abelian,
    \[
    (gh)^{nm}=(g^n)^m(h^m)^n=1. 
    \]
    Thus $l$ divides $nm$. Since $(gh)^l=1$,
    $g^l=h^{-l}\in \langle g\rangle\cap\langle h\rangle=\{1\}$ 
    (because $|\langle g\rangle|=n$ and $|\langle h\rangle|=m$ are coprime, 
    $nm$ divides $l$ by Lagrange's theorem).
\end{proof}

Fermat's little theorem is a particular case of Lagrange's theorem. 

\begin{exercise}[Fermat's little theorem]
    \index{Fermat's little theorem}
    Let $p$ be a prime number. Prove that 
    \[a^{p-1}\equiv1\bmod p
    \]
    for all $a\in\{1,2,\dots,p-1\}$.
\end{exercise}

For the next corollary, we need Euler's totient function. 
Recall that 
$\varphi(n)$ is the number of positive integers $k\in\{1,\dots,n\}$
coprime with $n$. The group of units of 
$\Z/n$ has $\varphi(n)$ elements (because $x\in\Z/n$ is invertible
if and only if $x$ and $n$ are coprime).

\begin{exercise}[Euler's theorem]
        \index{Euler's theorem }
        Let $a$ and $n$ be coprime integers. Prove that 
        $a^{\varphi(n)}\equiv1\bmod n$.
\end{exercise}

The converse of Lagrange's theorem does not hold.   

\begin{example}
Consider the \textbf{alternating group}
    \begin{multline*}
\Alt_4=\{\id,(234),(243),(12)(34),(123),(124),\\(132),(134),(13)(24),(142),(143),(14)(23)\}\leq\Sym_4.
\end{multline*}
We claim that $\Alt_4$ has no subgroups of order six. If $H\leq\Alt_4$ is such that 
$|H|=6$, then, since $(\Alt_4:H)=2$, for every $x\not\in H$ we can decompose $\Alt_4$ as 
as disjoint union 
$\Alt_4=H\cup xH$.

For each $g\in\Alt_4$ we have that $g^2\in H$ (if $g\not\in H$, then, since $g^2\in\Alt_4=H\cup
gH$, it follows that $g^2\in H$). In particular, since 
$(ijk)=(ikj)^2$, order-three elements of $\Alt_4$ belong to $H$, a contradiction, 
because $\Alt_4$ has eight elements of order three. 
\end{example}

We all need a favorite group. Mine is $\SL_2(3)$,
the group of $2\times2$ matrices with coefficients in $\Z/3$
and determinant one. 

\begin{exercise}
Prove that    \[
    \SL_2(3)=\left\{\begin{pmatrix}a&b\\c&d\end{pmatrix}:ad-bc=1,\,a,b,c,d\in\Z/3\right\}
    \]
    has order 24 and does not contain subgroups of order 12.
    \end{exercise}

\topic{The symmetric group}

\index{Cycle}
Let $\sigma\in\Sym_n$. We say that $\sigma$ is an $r$-cycle 
if there are $a_1,\dots,a_r\in\{1,\dots,n\}$ such that 
$\sigma(j)=j$ for all $j\not\in\{a_1,\dots,a_r\}$ and 
\[
\sigma(a_i)=\begin{cases}
a_{i+1} & \text{if $i<r$},\\
a_1 & \text{if $i=r$}.
\end{cases}
\]

For example, $(12)$, $(13)$ and $(23)$ are 2-cycles of 
$\Sym_3$. We often say that 2-cycles are \textbf{transpositions}.
The permutations $(123)$ and $(132)$ are 3-cycles of $\Sym_3$.

\index{Disjoint permutations}
We say that the permutations $\sigma,\tau\in\Sym_n$ 
are \textbf{disjoint} if for all 
$j\in\{1,\dots,n\}$
one has $\sigma(j)=j$ or $\tau(j)=j$. For example, 
$(134)$ and $(25)$ are disjoint. The permutations $(134)$ and 
$(24)$ are not disjoint. 

If $\sigma\in\Sym_n$ and $j$ is such that 
$\sigma(j)=j$, then $j$ is a fixed point of $\sigma$. The elements 
$j$ such that 
$\sigma(j)\ne j$ are the points moved by 
$\sigma$.

\begin{claim}
Disjoint permutations commute. 
\end{claim}


We now prove that every permutation can be written 
as a product of disjoint cycles. 
The proof of the theorem can be omitted in a first 
lecture. 
We start with a lemma. 

\begin{lemma}
        Let $\sigma=\alpha\beta\in\Sym_n$ with $\alpha$ and $\beta$ disjoint permutations. If $\alpha(
i)\ne i$, then $\sigma^k(i)=\alpha^k(i)$ for all $k\geq0$.
\end{lemma}

\begin{proof}
    Without loss of generality, we may assume that $k>0$. Then \[
    \sigma^k(i)=(\alpha\beta)^k(i)=\alpha^k(\beta^k(i))=\alpha^k(i).\qedhere
    \]
\end{proof}

\begin{theorem}
Each $\sigma\in\Sym_n\setminus\{\id\}$ can be written as a product
of disjoint cycles of length 
 $\geq2$. The decomposition is unique up to 
 the order of the factors. 
 \end{theorem}

\begin{proof}
    We proceed by induction on the number $k$ 
    of elements of $\{1,\dots,n\}$ moved by $\sigma$. If $k=2$, 
    the result is trivial. Assume that the result 
    holds for all permutations moving $<k$ points. Let
    $i_1\in\{1,\dots,n\}$ be such that $\sigma(i_1)\ne i_1$. We 
    consider the cycle that contains $i_1$. So let 
        $i_2=\sigma(i_1)$, $i_3=\sigma(i_2)$... We know that 
        there exists $r$ such that $\sigma(i_r)=i_1$
 (otherwise, if $\sigma(i_r)=i_j$ for some 
        $j\in\{2,\dots,n\}$, then 
        \[
        \sigma(i_{j-1})=i_j=\sigma(i_r),
        \]
        a contradiction to the bijectivity of $\sigma$). 
        Let $\sigma_1=(i_1\cdots i_r)$. By the inductive hypothesis, since 
        $\sigma_1^{-1}\sigma$ moves $<k$ points (because 
        the $i_j$ are fixed points of $\sigma_1^{-1}\sigma$), 
        we can write $\sigma_1^{-1}\sigma=\sigma_2\cdots\sigma_s$, where 
        $\sigma_2,\dots,\sigma_s$ are disjoint cycles. 
        This implies that $\sigma=\sigma_1\sigma_2\cdots\sigma_s$.

        We now prove the uniqueness of the decomposition. 
        Assume that 
        \[
        \sigma=\sigma_1\cdots\sigma_s=\tau_1\cdots\tau
_t
\]
with $s>0$. Let $i_1\in\{1,\dots,n\}$ be such that 
        $\sigma(i_1)\ne i_1$. By the previous lemma, $\sigma^k(i_1)=\sigma_1^k(i_1)$ for all $k\geq0$.
        There exists $j\in\{1,\dots,t\}$ such that 
        $\tau_j(i_1)\ne i_1$. Since the $t_k$'s commute, 
        without loss of generality, we may assume that $j=
1$. Thus $\sigma^k(i_1)=\tau_1^k(i_1)$ for all $k\geq0$.  
This implies that 
        $\sigma_1=\tau_1$, as $\sigma_1$ and $\tau_1$ are cycles. 
        Thus $\sigma_2\cdots\sigma_s=\tau_2\cdots\tau_t$. Repeating
        this procedure, we obtain that $s=t$. Therefore 
        $\sigma_j=\tau_j$ for all $j$.
\end{proof}

\begin{corollary}\
\label{cor:generation}
        \begin{enumerate}
                \item $\Sym_n=\langle (ij):i<j\rangle$.
                \item $\Sym_n=\langle (12),(13),\dots,(1n)\rangle$.
                \item $\Sym_n=\langle (12),(23),\dots,(n-1\,n)\rangle$.
                \item $\Sym_n=\langle (12),(12\cdots n)\rangle$.
        \end{enumerate}
\end{corollary}

\begin{proof}
        The first claim follows from the previous theorem, as 
        \[
        (a_1\cdots a_r)=(a_1a_r)(a_1a_{r-1})\cdots(a_1a_2).
        \]
        If we write $\sigma\in\Sym_n$ as a product of disjoint cycles, 
        the previous formula implies 
        that $\Sym_n\subseteq\langle (ij):i<j\rangle$. The other 
        inclusion is trivial. 

        For the second claim, one uses the first claim and the
        formulas 
        \[
        (1i)(1j)(1i)=(ij), 
        \]
        where $i\ne j$.

        To prove the third claim, write $\sigma$ as a product 
        of transpositions and 
        note that 
        \[
        (13)=(12)(23)(12),\quad
        (1\,k+1)=(k\,k+1)(1k)(k\,k+1)
        \]
        for all $k\geq3$.

        Finally, the fourth claim follows from 
        the third claim and 
        the formula 
        \[
        (12\cdots n)^{k-1}(12)(12\cdots n)^{1-k}=(k\,k+1),
        \]
        where $k\geq1$.
\end{proof}

Here is an alternative proof of
the first claim of Corollary 
\ref{cor:generation}. We must show that every 
permutation can be written as a product of transpositions. 
Let us assume that $n$ persons are invited to a concert. They sit
in the first row without following  
the seat number on their tickets. How can we put each person in 
the right seat? First, we locate the person that should be seated 
in the first place. Then we ask this person to 
interchange seats with the person seated in the first place. 
Then we identify the person 
that should be seated in the second spot. We then ask this person
to interchange seats with the person 
seated in the second spot. We do the same with the third spot, the fourth
spot... Once the process is finished, 
we have decomposed 
our permutation into a product of transpositions. 

\begin{exercise}
    Following the tricks of the proof of 
    Corollary \ref{cor:generation}, find the different
    decompositions of the permutation
    $(1324)(56)(789)\in\Sym_9$. 
\end{exercise}

Every permutation yields a permutation matrix. For example, 
the matrix corresponding to $\sigma=\id\in\Sym_3$ 
is the $3\times 3$ identity matrix. The permutation
$\sigma=(123)$ yields the matrix 
\[
P_\sigma=\begin{pmatrix}0&0&1\\1&0&0\\0&1&0\end{pmatrix}.
\]
If $e_1,e_2,e_3$ is the standard basis of $\R^{3\times1}$, then
\[
P_{\sigma}(e_1)=e_2,
\quad 
P_{\sigma}(e_2)=e_1,
\quad 
P_{\sigma}(e_3)=e_1.
\]
The permutation matrix
$P_\sigma$ associated with 
$\sigma\in\Sym_n$, permutes the elements of the standard basis
of $\R^{n\times1}$ in the way $\sigma$ permutes 
the elements of $\{1,2,\dots,n\}$.

If $\sigma\in\Sym_n$, then
\[
P_\sigma=\sum_{i=1}^n E_{\sigma(i),i},
\]
where $E_{i,j}$ is the matrix with a one in position
$(i,j)$ and zero in all other entries. Recall the
following formulas: 
\begin{equation}
\label{eq:E}
E_{i,j}E_{k,l}=\begin{cases}
E_{i,l} & \text{if $j=k$},\\
0 & \text{if $j\ne k$}.
\end{cases}
\end{equation}

The determinant of a permutation matrix equals 
$\pm1$. Why? 

\begin{proposition}
If $\sigma,\tau\in\Sym_n$, then $P_{\sigma\tau}=P_\sigma P_\tau$.
\end{proposition}

\begin{proof}
We compute 
\begin{align*}
P_\sigma P_\tau &=\left(\sum_{i=1}^n E_{\sigma(i),i}\right)\left(\sum_{j=1}^nE_{\tau{(j)},j}\right)\\
&=\sum_{i=1}^n\sum_{j=1}^n E_{\sigma(i),i}E_{\tau(j),j}
=\sum_{j=1}^n E_{\sigma(\tau(j)),j}=P_{\sigma\tau},
\end{align*}
where the double sum is zero unless $i=\tau(j)$.
\end{proof}


\begin{definition}
\index{Permutation!even}
\index{Permutation!odd}
\index{Permutation!sign}
    The \textbf{sign} of a permutation $\sigma\in\Sym_n$ 
    is the 
    determinant of the matrix 
    $P_\sigma$, that is $\sgn(\sigma)=\det P_\sigma$.
    A permutation $\sigma$ is said to be \textbf{even} if $\sgn(\sigma)=1$ and \textbf{odd} if $\sgn(\sigma)=-1$.
\end{definition}

The identity is an even permutation. Every 3-cycle is
an even permutation. Each transposition is an odd permutation. 

\begin{proposition}
If $\sigma,\tau\in\Sym_n$, then $\sgn(\sigma\tau)=(\sgn\sigma)(\sgn\tau)$.
\end{proposition}

\begin{proof}
        We compute 
        \[
        \sgn(\sigma\tau)=\det(P_\sigma P_\tau)=(\det P_\sigma)(\det P_\tau)=\sgn(\sigma)\sgn(\tau).\qedhere
        \]
\end{proof}

Each permutation can be written as a product of transpositions. 
There is no uniqueness of this decomposition. For example, 
\[
(13)=(12)(23)(12)=(12)(23)(12)
\]

However, the following result holds: If 
$\sigma=\sigma_1\cdots\sigma_s$ is a product of transpositions, 
then $\sgn(\sigma)=(-1)^s$.
In particular, $\sigma$ is even if and only if
$s$ is even. 


\begin{example}
\index{Center!of $\Sym_n$}
We claim that if $n\geq3$ then $Z(\Sym_n)=\{\id\}$.
Assume that $Z(\Sym_n)\ne\{\id\}$. Let 
$\sigma\in Z(\Sym_n)$ be such that $\sigma(i)=j$ for some $i\ne j$. Since $n\geq3$, there exists $k\in\{1,\dots,n
\}\setminus\{i,j\}$. Thus 
$\tau=(jk)\in\Sym_n$. Since $\sigma$ is central, 
\[
j=\sigma(i)=\tau\sigma\tau^{-1}(i)=\tau(\sigma(i))=\tau(j)=k,
\]
a contradiction.
\end{example}

\index{Alternating group}
The \textbf{alternating group}
\[
\Alt_n=\{\sigma\in\Sym_n:\sgn(\sigma)=1\}
\]
is the subgroup of $\Sym_n$ formed by even permutations. 

\begin{proposition}
\index{Order!of the alternating group}
$|\Alt_n|=n!/2$.
\end{proposition}

\begin{proof}
Let $\sigma=(12)\not\in\Alt_n$. We claim that 
$\Sym_n=\Alt_n\cup\Alt_n\sigma$ (disjoint union), where 
$\Alt_n\sigma=\{\tau\sigma:\tau\in\Alt_n\}$. If 
$\tau\in\Sym_n$ is such that $\tau\not\in\Alt_n$, then 
\[
\sgn(\tau\sigma)=(\sgn\tau)(\sgn\sigma)=1.
\]
Thus 
$\tau\sigma\in\Alt_n$. Therefore $\tau\in\Alt_n\sigma$. Since  $|\Alt_n\sigma|=|\Alt_
n|$ (because the map $\Alt_n\to\Alt_n\sigma$, $x\mapsto x\sigma$, is bijective), we conclude that 
$n!=|\Sym_n|=2|\Alt_n|$.
\end{proof}

\begin{example}
A direct calculation shows that  $\Alt_3=\{\id,(123),(132)\}$. Similarly, 
\begin{multline*}
\Alt_4=\{\id,(234),(243),(12)(34),(123),(124),\\(132),(134),(13)(24),(142),(143),(14)(23)\}
\end{multline*}
\end{example}

The group $\Alt_3$ is abelian.
If $n\geq4$, then $\Alt_n$ is non-abelian. For example, 
$(123)$ and $(124)$ do not commute. 

\begin{proposition}
\label{pro:A_n3cycles}
$\Alt_n=\langle\{\text{3-cycles}\}\rangle$.
\end{proposition}

\begin{proof}
Each 3-cycle is an even permutation, as $(ijk)=(ik)(ij)$. 
To prove the other inclusion, let $\sigma\in\Alt_n$.
Write $\sigma=\sigma_1\cdots\sigma_s$ for some even integer $s$ 
and transpositions $\sigma_1,\dots,\sigma_s$. 
Now the claim follows from the formulas 
\[
(kl)(ij)=(kl)(ki)(ki)(ij)=(kil)(ijk),\quad
(ik)(ij)=(ijk).\qedhere
\]
 \end{proof}

Proposition \ref{pro:A_n3cycles} has several 
important applications. 

\begin{example}
\index{Commutator!of $\Alt_n$}
If $n\geq5$, then $[\Alt_n,\Alt_n]=\Alt_n$. To prove the non-trivial
inclusion, it is enough to note that $\Alt_n$ is generated by 
3-cycles and that, since $n\geq5$, each 3-cycle 
is a product of commutators: 
\[
(abc)=[(acd),(ade)][(ade),(abd)],
\]
where $\#\{a,b,c,d,e\}=5$.
\end{example}

Can you compute the commutator subgroup of $\Alt_4$? 

\begin{example}
\index{Commutator!of $\Sym_n$}
If $n\geq3$, then $[\Sym_n,\Sym_n]=\Alt_n$. First, we prove that $[\Sym_n,\Sym_n]\subseteq\Alt_n$. If 
$\sigma\in[\Sym_n,\Sym_n]$,
say $\sigma=[\sigma_1,\tau_1][\sigma_2,\tau_2]\cdots[\sigma_k,\tau_k]$, then
\[
\sgn(\sigma)=\sgn([\sigma_1,\tau_1])\cdots\sgn([\sigma_k,\tau_k])=1.
\]
Conversely, if $\sigma\in\Alt_n$, by the previous proposition, 
we can write $\sigma$ as a product of 3-cycles. 
From this the claim follows, as each 3-cycle is a commutator: 
\[
(abc)=(ab)(ac)(ab)(ac)=[(ab),(ac)]\in[\Sym_n,\Sym_n].\qedhere
\]
\end{example}