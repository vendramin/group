\section{29/02/2024}

\begin{corollary}
        If $G$ is a group and $g\in G$ has order $n$, 
        then 
        \[
        |g^m|=\frac{n}{\gcd(n,m)}.
        \]
        \end{corollary}
        
\begin{proof}
        Let $k$ be such that $(g^m)^k=1=g^{mk}$. This means that $n$ divides $km$, as $g$ 
        has order $n$. This is also equivalent to the fact that 
        $n/d$ divides $mk/d$, where $d=\gcd(n,m)$. Therefore, since $n/d$ and $m/d$ 
        are coprime, $(g^m)^k=1$ is equivalent to  
        $n/d$ divides $k$, which implies that $g^m$ has order $n/
        d$.
\end{proof}
        
\begin{exercise}
        Let $G$ be a group and $g\in G$. Prove that the following statements are equivalent:
        \begin{enumerate}
        \item $g$ has infinite order. 
        \item The set $\{k\in\Z_{\geq1}:g^k=1\}$ is empty.
        \item If $g^k=1$, then $k=0$.
        \item If $k\ne l$, then $g^k\ne g^l$.
        \end{enumerate}
\end{exercise}

\begin{exercise}
\index{Torsion in abelian groups}
        Let $G$ be an abelian group. Prove that 
        $T(G)=\{g\in G:|g|<\infty\}$ is a subgroup of $G$. Compute $T(\C^\times)$.
\end{exercise}
                
\begin{exercise}
        Let $G=\langle g\rangle$ be a cyclic group. 
        \begin{enumerate}
                \item If $G$ is infinite, only $g$ and $g^{-1}$ generate $G$.
                \item If $G$ is finite of order $n$, then 
                        $G=\langle g^k\rangle$ if and only if $k$ and $n$ are coprime.
        \end{enumerate}
\end{exercise}
                
The following exercise is a particular 
case of Cauchy's theorem. 

\begin{exercise}
        \label{xca:orden2}
        Prove that every group of odd order contains
        an element of order two. 
\end{exercise}
                
Let us see some concrete examples: 

\begin{example}
        In $\Sym_3$ we have the following order pattern:
        \[
        |\id|=1,\quad
        |(12)|=|(13)|=|(23)|=2,\quad
        |(123)|=|(132)|=3.
        \]
\end{example}
                        
\begin{example}
        In $\Z$, every non-zero element has 
        infinite order. 
\end{example}
                        
 \begin{example}
        In $\Z\times\Z/6$ there are elements of 
        (in)finite order. For example, $(1,0)$ 
        has infinite order and 
        $(0,1)$ has order six. 
 \end{example}
                        
\begin{example}
        The matrix $\begin{pmatrix}1&1\\0&1\end{pmatrix}\in\GL_2(\R)$ has infinite order.
\end{example}                     
                                
\begin{example}
        The group $G_\infty=\bigcup_{n\geq1}G_n$ is abelian and infinite. Note that every element of 
        $G_\infty$ has finite order. 
\end{example}
          
We conclude the topic with some exercises. 

\begin{exercise}
        Compute the orders of the elements of $\Z/6$.
\end{exercise}       

\begin{exercise}
        Prove that $a=\begin{pmatrix}1&-1\\1&0\end{pmatrix}$ has order four, $b=\
        \begin{pmatrix}0&1\\-1&-1\end{pmatrix}$ has order three and 
        compute the order of $ab$.%=\begin{pmatrix}1&1\\0&1\end{pmatrix}$ tiene orden infinito.
\end{exercise}
                                
\begin{exercise}
        Compute the order of 
        $\begin{pmatrix}1&1\\-1&0\end{pmatrix}\in\GL_2(\R)$.
\end{exercise}
                                
\begin{exercise}
        Prove that in $\D_n$ one has 
        $|r^js|=2$ and $|r^j|=n/\gcd(n,j)$.
\end{exercise}
                                
\begin{exercise}
        Prove that a group with finitely many subgroups
        is finite. 
\end{exercise}




\subsection{Lagrange's theorem}

Let $G$ be a group and $H$ be a subgroup of $G$. We say that the elements $x,y\in
G$ are (left) equivalent modulo $H$ if $x^{-1}y\in H$.
We will use the following notation:
\begin{equation}
\label{eq:modH}
    x\equiv y\bmod
    H\Longleftrightarrow x^{-1}y\in H.
\end{equation}

\begin{exercise}
    Prove that~\eqref{eq:modH} is an equivalence relation, that
    is 
    \begin{enumerate}
    \item $x\equiv x\bmod H$ for all $x$; 
    \item if $x\equiv y\bmod H$, then $y\equiv x\bmod H$; and 
    \item if $x\equiv y\bmod H$ and $y\equiv z\bmod H$, then $x\equiv z\bmod H$.
    \end{enumerate}
\end{exercise}

The equivalence classes of this equivalence relation modulo $H$
are the sets of the form $xH=\{xh:h\in H\}$, as the class 
of an element $x\in G$ is the set 
\[
        \{y\in G:x\equiv y\bmod H\}=\{y\in G:x^{-1}y\in H\}=\{y\in G:y\in xH\}=xH.
\]
The set $xH$ is called 
a \textbf{left coset} of $H$ in $G$ and $x$ is 
a \textbf{representative} of $xH$.

Having an equivalence relation modulo $H$ in $G$ allows us to
decompose $G$ as a disjoint union of certain subsets related to $H$. 

\begin{proposition}
Let $G$ be a group and $H$ be a subgroup of $G$. 
\begin{enumerate}
\item If $xH\cap yH\ne\emptyset$, then $xH=yH$.
\item The group $G$ decomposes as a disjoint union 
of different left cosets of $H$.
\end{enumerate}
\end{proposition}

\begin{proof}
    Let us prove the first claim. If $g\in xH\cap yH$, we write 
    $g=xh$ for some $h\in H$. Then 
    \[
    gH=(xh)H=x(hH)=xH.
    \]
    Similarly, $gH=yH$. Hence $xH=yH$.
    The second claim follows from the first one. 
\end{proof}

One can also define right cosets: $x\equiv
y\bmod H$ if and only if $xy^{-1}\in H$. In this case, 
the equivalence classes are 
the sets of the form $Hx$ with $x\in X$. The set $Hx$ 
is called a \textbf{right coset}
with \textbf{representative} $x$ of $H$ in $G$. 

\begin{proposition}
    If $H$ is a subgroup of $G$, then  $|Hx|=|H|=|xH|$ for all $x\in G$.
\end{proposition}

\begin{proof}
    Let $x\in G$. The map $H\to Hx$, $h\mapsto hx$, is bijective 
    with inverse $hx\mapsto h$. Similarly, the map $H\to xH$,
    $h\mapsto xh$, is bijective.
\end{proof}

The map 
\[
        \{\text{right cosets of $H$ in $G$}\}\to\{\text{left cosets of $H$ in $G$}\}
\]
given by $Hx\mapsto x^{-1}H$ is a bijection, as 
\[
        Hx=Hy
        \Longleftrightarrow xy^{-1}\in H
        \Longleftrightarrow (x^{-1})^{-1}y^{-1}\in H
        \Longleftrightarrow x^{-1}H=y^{-1}H.
\]
In particular, the number of right cosets of $H$ in $G$
equals the number of left cosets of $H$ in $G$. 

\begin{definition}
    \index{Index}
    If $H$ is a subgroup of $G$, the \textbf{index} of $H$ in $G$
    is the number $(G:H)$ of left (or right) cosets of $H$ in $G$. 
\end{definition}

\begin{example}
    If $G=\Z$ and $S=n\Z$, then 
    \[
    a+S=\{a+nq:q\in\Z\}=\{k\in\Z:k\equiv a\bmod n\}.
    \]
\end{example}
    
\begin{example}
    The subgroups of $\Sym_3$ are $\{\id\}$, the order-two subgroups 
    $\Sym_3$, $\langle(12)\rangle$, 
    $\langle(13)\rangle$ and $\langle(23)\rangle$, and 
    the order-three subgroup $\langle(123)\rangle=\{\id,(123),(132)\}$.  
    If $H=\langle(12)\rangle=\{\id,(12)\}$, then 
    \begin{align*}
    &H=(12)H=\{\id,(12)\},\\
    &(123)H=(13)H=\{(13),(123)\},\\
    &(132)H=(23)H=\{(23),(132)\}.
    \end{align*}
    Note that our group decomposes as 
    \[
    \Sym_3=H\cup (123)H\cup (132)H\quad\text{(disjoint union)}.
    \]
    \end{example}

    \begin{example}
        Let $G=\R^2$ with the usual addition 
        and $v\in\R^2$. The line 
        \[
        L=\{\lambda v:\lambda\in\R\}
        \]
        is a subgroup of $G$. For each 
        $p\in R^2$, the coset $p+L$ 
        is the line parallel to $L$ that 
        passes through $p$.
    \end{example}

The following important theorem will be used extensively. 

\begin{theorem}[Lagrange]
\index{Lagrange's theorem}
    If $G$ is a finite group and $H$ is a subgroup of $G$, 
    then $|G|=|H|(G:H)$. In particular, $|H|$ divides $|G|$.
\end{theorem}

\begin{proof}
    We decompose $G$ into equivalence classes modulo $H$, that is 
    \[
    G=\bigcup_{i=1}^n x_iH\quad\text{(disjoint union)}
    \]
    for some $x_1,\dots,x_n\in G$, where $n=(G:H)$. 
    Since each of these equivalence classes has 
    exactly 
    $|H|$ elements,
    \[
            |G|=\sum_{i=1}^n|x_iH|=\sum_{i=1}^n|H|=|H|(G:H).\qedhere
    \]
\end{proof}

Let us discuss some corollaries. 

\begin{corollary}
    If $G$ is a finite group and $g\in G$, then $g^{|G|}=1$.
\end{corollary}

\begin{proof}
    By definition. $|g|=|\langle g\rangle|$. Apply Lagrange's theorem 
    to the subgroup $H=\langle g\rangle$ to obtain that 
    \[
            g^{|G|}=g^{|H|(G:H)}=(g^{|H|})^{(G:H)}=1.\qedhere
    \]
\end{proof}

\begin{corollary}
    If $G$ has prime order, then $G$ is cyclic. 
\end{corollary}

\begin{proof}
    Let $g\in G\setminus\{1\}$ and $H=\langle g\rangle$. By Lagrange's theorem, 
    $|H|$ divides $|G|$. Thus $|H|=|G|$, as $|G|$ is prime. Therefore 
    $G=H=\langle g\rangle$.
\end{proof}

\begin{corollary}
\label{cor:coprime_orders}
    If $G$ is an abelian group and $g,h\in G$ are elements of finite coprime orders, 
    then 
    $|gh|=|g||h|$.
\end{corollary}

\begin{proof}
    Let $n=|g|$, $m=|h|$ and $l=|gh|$. Since $G$ is abelian,
    \[
    (gh)^{nm}=(g^n)^m(h^m)^n=1. 
    \]
    Thus $l$ divides $nm$. Since $(gh)^l=1$,
    $g^l=h^{-l}\in \langle g\rangle\cap\langle h\rangle=\{1\}$ 
    (because $|\langle g\rangle|=n$ and $|\langle h\rangle|=m$ are coprime, 
    $nm$ divides $l$ by Lagrange's theorem).
\end{proof}

Fermat's little theorem is a particular case of Lagrange's theorem. 

\begin{exercise}[Fermat's little theorem]
    \index{Fermat's little theorem}
    Let $p$ be a prime number. Prove that 
    \[a^{p-1}\equiv1\bmod p
    \]
    for all $a\in\{1,2,\dots,p-1\}$.
\end{exercise}

For the next corollary, we need Euler's totient function. 
Recall that 
$\varphi(n)$ is the number of positive integers $k\in\{1,\dots,n\}$
coprime with $n$. The group of units of 
$\Z/n$ has $\varphi(n)$ elements (because $x\in\Z/n$ is invertible
if and only if $x$ and $n$ are coprime).

\begin{exercise}[Euler's theorem]
        \index{Euler's theorem }
        Let $a$ and $n$ be coprime integers. Prove that 
        \[
        a^{\varphi(n)}\equiv1\bmod n.
        \]
\end{exercise}

The converse of Lagrange's theorem does not hold.   

\begin{example}
Consider the \textbf{alternating group}
    \begin{multline*}
\Alt_4=\{\id,(234),(243),(12)(34),(123),(124),(132),(134),(13)(24),(142),(143),(14)(23)\}.
\end{multline*}
We claim that $\Alt_4$ has no subgroups of order six. If $H\leq\Alt_4$ is such that 
$|H|=6$, then, since $(\Alt_4:H)=2$, for every $x\not\in H$ we can decompose $\Alt_4$ as 
as disjoint union 
$\Alt_4=H\cup xH$.

For each $g\in\Alt_4$ we have that $g^2\in H$ (if $g\not\in H$, then, since $g^2\in\Alt_4=H\cup
gH$, it follows that $g^2\in H$). In particular, since 
$(ijk)=(ikj)^2$, order-three elements of $\Alt_4$ belong to $H$, a contradiction, 
because $\Alt_4$ has eight elements of order three. 
\end{example}

We all need a favorite group. Mine is $\SL_2(3)$,
the group of $2\times2$ matrices with coefficients in $\Z/3$
and determinant one. 

\begin{exercise}
Prove that    \[
    \SL_2(3)=\left\{\begin{pmatrix}a&b\\c&d\end{pmatrix}:ad-bc=1,\,a,b,c,d\in\Z/3\right\}
    \]
    has order 24 and does not contain subgroups of order 12.
    \end{exercise}

