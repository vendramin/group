\section{28/03/2024}

\subsection{Isomorphism theorems}

The following theorem is fundamental. For example, it allows us to recognize quotient groups. 

\begin{theorem}[First isomorphism theorem]
\index{First isomorphism theorem}
If $f\colon G\to H$ is group homomorphism, then $G/\ker f\simeq f(G)$.
\end{theorem}

\begin{proof}
    Let $K=\ker f$ and $\varphi\colon G/K\to f(G)$, $xK\mapsto f(x)$. We need to show that $\varphi$ is well-defined. This means that
    we need to show that if $xK=yK$, then $f(x)=f(y)$. If $xK=yK$, then, since $y^{-1}x\in K$, 
    \[
        f(y)^{-1}f(x)=f(y^{-1}x)\in f(K)=\{1\}.
    \]
    Thus $f(x)=f(y)$.

    We now show that $\varphi$ is a group homomorphism: 
        \[
        \varphi(xKyK)=\varphi(xyK)=f(xy)=f(x)f(y)=\varphi(xK)\varphi(yK).
        \]
    To compute $\ker\varphi$ we proceed as follows: 
        \[
        \pi(x)=xK\in\ker\varphi\Longleftrightarrow \varphi(xK)=1
        \Longleftrightarrow f(x)=1
        \Longleftrightarrow x\in K.
        \]
    Therefore  $\ker\varphi$ is trivial and
    $\varphi$ is injective. Since $\varphi\colon G/K\to f(G)$ is surjective, 
    we conclude that $G/K\simeq f(G)$.
\end{proof}

If $G$ is a group, then $G/\{1\}\simeq G$ and $G/G\simeq\{1\}$.

\begin{example}
Since $f\colon\Z\to\Z/n$, $x\mapsto x\bmod n$, is a group homomorphism
with $\ker f=n\Z$, it follows that 
$\Z/n\Z\simeq\Z/n$.
\end{example}

\begin{example}
Let $G$ be an infinite cyclic group, say $G=\langle g\rangle$. The map 
$f \colon\Z\to G$, $k\mapsto g^k$,
is a group isomorphism. Thus $G\simeq\Z$ by the first isomorphism theorem. In particular, $G=\langle g^k\rangle$ if and only if 
 $k\in\{-1,1\}$.
\end{example}

\begin{example}
We claim that $\Z/n\Z\simeq G_n$. Let 
\[
f\colon\Z\to G_n,\quad
f(k)=\exp(2i\pi k/n).
\]
Then $f$ is a surjective group homomorphism and
$\ker f=n\Z$. By the first isomorphism theorem, the claim follows. 
\end{example}

\begin{example}
Note that $2\Z\simeq 3\Z$, as both groups are infinite (alternatively, one can also consider the map $2k\mapsto 3k$). Moreover, 
\[
\Z/2\simeq\Z/2\Z\not\simeq\Z/3\Z\simeq\Z/3.
\]
\end{example}

\begin{example}
Since 
\[
f\colon\C^\times\to\C^\times,
\quad
f(z)=\frac{z}{|z|},
\]
is a group homomorphism with $\ker f=\R_{>0}$ and
$f(\C^\times)=S^1$, the first isomorphism theorem 
implies that 
$\C^\times/\R_{>0}\simeq S^1$.
\end{example}

\begin{example}
If we apply the first isomorphism theorem to the
map $f\colon S^1\to S^1$, $f(z)=z^2$, we obtain that 
$S^1/\{\pm1\}\simeq S^1$, as 
$\ker f=\{-1,1\}$ and $f(S^1)=S^1$.
\end{example}

\begin{example}
Let $f\colon\C^\times\to\C^\times$, $f(z)=|z|$. Since $\ker f=S^1$ and $f(\C^\times)=\R_{>0}$, the first isomorphism theorem 
implies that $\C^\times/S^1\simeq\R_{>0}$.
\end{example}

\begin{example}
We claim that 
$(\Z\times\Z)/\langle (1,3)\rangle\simeq\Z$. We consider
the surjective group homomorphism 
$f\colon\Z\times\Z\to\Z$, $f(x,y)=3x-y$. Since 
\[
\ker f=\{(x,3x):x\in\Z\}=\langle (1,3)\rangle,
\]
the first isomorphism theorem implies that 
$(\Z\times\Z)/\langle (1,3)\rangle\simeq\Z$.
\end{example}

\begin{exercise}
Prove that $\R/\Z\simeq S^1$.
\end{exercise}

% $t\mapsto \exp(2\pi it)$
% $z\mapsto z^2$


\begin{exercise}
Prove that $\Q/\Z\simeq\cup_{n\geq1}G_n$.
\end{exercise}

% x\mapsto cos (2\pi x)+i\sin (2\pi x)

\begin{exercise}
Prove that 
$(\Z\times\Z)/\langle (6,3)\rangle\simeq\Z\times(\Z/3)$.
\end{exercise}
% (x,y)\mapsto (2y-x,y\mod 3)

Let us see another application that shows that
the first isomorphism theorem is quite familiar. 

\begin{example}
Let $V$ be a vector space and $W$ be a subspace of $V$. 
In particular, 
$V$ is an abelian group and 
$W$ is a normal subgroup of $V$. The abelian group 
$V/W$ is then a vector space with 
\[
\lambda(v+W)=(\lambda v)+W,\quad \lambda\in\R,\,v\in V,
\]
and the canonical homomorphism
$\pi\colon V\to V/W$ is also a linear map. As an exercise, 
the reader needs to show that 
$\dim (V/W)=\dim V-\dim W$
if $\dim V<\infty$.

If $f\colon V\to U$ is a linear map, then
$V/\ker f\simeq f(V)$ as abelian groups (by the first
isomorphism theorem). The map realizing this
isomorphism is moreover linear, so 
$V/\ker f\simeq f(V)$ as vector spaces. In particular, 
if $\dim V<\infty$, then
\[
\dim V-\dim\ker f=\dim f(V).
\]
\end{example}

\begin{exercise}
\label{xca:quotients}
Let $f\colon G\to H$ be a group homomorphism and 
$K$ a normal subgroup of $G$ such that $K\subseteq\ker f$. 
Prove that there exists a unique group homomorphism 
$\varphi\colon G/K\to H$ such that the diagram 
\[\begin{tikzcd}
        G & H \\
        {G/K}
        \arrow["f", from=1-1, to=1-2]
        \arrow["\pi"', from=1-1, to=2-1]
        \arrow["\varphi"', dashed, from=2-1, to=1-2]
\end{tikzcd}
\]
commutes. The commutativity of the diagram means that  $\varphi\circ\pi=f$, where $\pi\colon G\to G/K$ is the canonical group
homomorphism. Moreover, 
$\ker\varphi=\pi(\ker f)$ and $\varphi(G/K)=f(G)$.
In particular, $\varphi$ is injective if and only if 
$\ker f=K$ and $\varphi$ is surjective if and only if $f$ is surjective. 
\end{exercise}

We now discuss the second isomorphism theorem. As a rule to remember
what the theorem is about, one has the following diagram:
\[\begin{tikzcd}
        & NT \\
        N && T \\
        & {N\cap T}
        \arrow[no head, from=1-2, to=2-3]
        \arrow[no head, from=1-2, to=2-1]
        \arrow[no head, from=2-1, to=3-2]
        \arrow[no head, from=2-3, to=3-2]
\end{tikzcd}\]

\begin{exercise}[Second isomorphism theorem]
If $N$ is a normal subgroup of $G$ and 
$T$ is a subgroup of $G$, then $N\cap T$ is normal in $T$
and 
\[
T/N\cap T\simeq NT/N.
\]
\end{exercise}


% \begin{proof}
% Sea $\pi\colon G\to G/N$ el morfismo canónico. Ya vimos que la restricción $\pi|_T\colon T\to G/N$ es
%  un morfismo de grupos con núcleo
% $\ker(\pi|_T)=T\cap N$. En particular, $T\cap N$ es normal en $T$. Al aplicar el primer
% teorema de isomorfismos, $T/(T\cap N)\simeq \pi(T)$. Como $N$ es normal en $G$,
% $NT$ es un subgrupo de $G$ que contiene a $N$.
% La restricción $\pi|_{NT}$ es entonces un morfismo de grupos con núcleo $NT\cap N=N$.
% Al aplicar el primer teorema de isomorfismos a $\pi|_{NT}$ obtenemos
% $NT/N\simeq \pi(NT)=\pi(T)$.
% \end{proof}

\begin{exercise}
Let $N$ be a normal subgroup of $G$ and
$\pi\colon G\to G/N$ the canonical homomorphism. Prove that
if $L$ is a subgroup of $G$, then $\pi^{-1}(\pi(L))=NL$.
%%%
%%%Por otro lado, si $L$ es un subgrupo de $G$, entonces $\pi^{-1}(\pi(L))=NL$. En efecto,
%%%si $x\in \pi^{-1}(\pi(L))$, entonces $\pi(x)\in \pi(L)$ y luego
%%%$\pi(x)=\pi(l)$ for some $l\in L$. Como entonces $xl^{-1}\in \ker\pi=N$,
%%%se tiene que $x=(xl^{-1})l\in KL$. Recíprocamente, si $x=kl$ con $k\in K$ y $l\in L$, entonces
%%%$\pi(x)=\pi(kl)=\pi(l)\in \pi(L)$ y luego $x\in\pi^{-1}(\pi(L))$. Observemos que
%%%si $L$ es un subgrupo de $G$, entonces $NL$ es un subgrupo de $G$ que contiene a $N$.
\end{exercise}

The following example uses additive groups. 

\begin{example}
Let $G=\Z/24$, $H=\langle 4\rangle$ and $N=\langle 6\rangle$. Since $G$ is abelian, $H$ and $K$ are normal in $G$. Then 
$H+N=\langle 2\rangle$ and $H\cap N=\{0,12\}$. Let us compute
the left cosets of $N$ in $H+N$:
\[
0+N=\{0,6,12,18\},
\quad
2+N=\{2,8,14,20\},
\quad
4+N=\{4,10,16,22\}.
\]
The left cosets of $H\cap N$ in $H$ are 
\[
0+(H\cap N)=\{0,12\},
\quad
4+(H\cap N)=\{4,16\},
\quad
8+(H\cap N)=\{8,20\}.
\]
By the second isomorphism theorem, $(H+N)/N\simeq H/H\cap N$. The
isomorphism is given by 
$f\colon H/(H\cap N)\to (H+N)/N$, $h+(H\cap N)\mapsto h+N$.
In our particular case, 
\begin{align*}
&f(0+(H\cap N))=0+N,\\
&f(4+(H\cap N))=4+N,\\
&f(8+(H\cap N))=8+N=2+N.
\end{align*}
\end{example}

Let us discuss some applications. 

\begin{example}
Let $a,b\in\Z\setminus\{0\}$. Then $a\Z+b\Z=\gcd(a,b)\Z$ and $a\Z\cap b\Z=\lcm(a,b)\Z$. By the second isomorphism theorem,
\[
\frac{\gcd(a,b)\Z}{b\Z}=\frac{a\Z+b\Z}{b\Z}\simeq
\frac{a\Z}{a\Z\cap b\Z}=\frac{a\Z}{\lcm(a,b)\Z}.
\]
Since the formula involves finite groups, 
computing orders yields 
\[
ab=\gcd(a,b)\lcm(a,b).
\]
\end{example}


\index{Group!meta-abelian}
A group $G$ is said to be \textbf{meta-abelian}
if it contains an abelian normal subgroup $N$ and $G/N$
is abelian. Abelian groups are meta-abelian. However, the group $\Sym_3$ is meta-abelian and not abelian.  The following exercise
present another application of the second isomorphism theorem. 

\begin{exercise}
Prove that if $G$ is a mete-abelian group and $H$ is a subgroup of
$G$, then $H$ is meta-abelian. 
\end{exercise}

% \begin{proof}
% Como $G$ es meta-abeliano, existe
% un subgrupo normal $N$ de $G$ tal que $N$ y $G/N$ son ambos abelianos.
% El subgrupo abeliano $H\cap N$ es normal en $H$. Gracias al segundo teorema de isomorfismos,
% \[
% H/(H\cap N)\simeq HN/N
% \]
% es un grupo abeliano pues $HN/N$ es un subgrupo del grupo abeliano $G/N$.
% \end{proof}

There is a third isomorphism theorem. 

% \begin{exercise}
%         \label{xca:para_3er}
%         Sea $f\colon G\to H$ un morfismo de grupos y sean $U\unlhd G$ y $V\unlhd H$. Demuestre que ex
% iste
%         un morfismo de grupos $g\colon G/U\to H/V$ tal que el diagrama
% \[
% \begin{tikzcd}
%         G & H \\
%         {G/U} & {H/V}
%         \arrow["f", from=1-1, to=1-2]
%         \arrow["{\pi_U}"', from=1-1, to=2-1]
%         \arrow["g"', dashed, from=2-1, to=2-2]
%         \arrow["{\pi_V}", from=1-2, to=2-2]
% \end{tikzcd}
% \]
%         es conmutativo si y sólo si $f(U)\subseteq V$, donde $\pi_U\colon G\to G/U$ y $\pi_V\colon H\
% to H/V$ son los morfismos canónicos. Además, en este caso,
%         \begin{enumerate}
%         \item Si $f$ es sobreyectiva, entonces $g$ es sobreyectiva.
%         \item Si $U=f^{-1}(V)$, entonces $g$ es inyectiva.
%         \end{enumerate}
% \end{exercise}
% % todo: escribir bien la demo del tercero, ese lema es horrible
% %Un caso particular del lema nos permite demostrar elegantemente el tercer teorema de isomorfismos.

\begin{exercise}[Third isomorphism theorem]
\label{xca:third}
Let $S$ and $T$ be normal subgroups of $G$ such that 
$S\subseteq T$. Prove that $S$ is normal in $T$, 
$T/S$ is normal in $G/S$ and 
\[
\frac{G/S}{T/S}\simeq G/T,
\]
where $T/S=\{tS:t\in T\}$.
\end{exercise}

The following example helps to visualize the third isomorphism
theorem. 

\begin{example}
If $m$ divides $n$, then $n\Z\leq m\Z\leq\Z$. Thus 
\[
\frac{\Z/n\Z}{m\Z/n\Z}\simeq\Z/m\Z.
\]
\end{example}

The following theorem is known as the \emph{correspondence theorem}.
It is powerful and essential. It helps to have in mind the following diagram: 
\[
\begin{tikzcd}
        && G \\
        & L && {f(G)} \\
        N && Y \\
        & {\{1\}}
        \arrow[no head, from=1-3, to=2-4]
        \arrow[no head, from=1-3, to=2-2]
        \arrow[no head, from=2-2, to=3-1]
        \arrow[no head, from=3-1, to=4-2]
        \arrow[no head, from=2-2, to=3-3]
        \arrow[no head, from=3-3, to=4-2]
        \arrow[no head, from=2-4, to=3-3]
\end{tikzcd}
\]

\begin{theorem}[Correspondence theorem]
Let $f\colon G\to H$ be a group homomorphism and $K=\ker f$. There exists
a bijective correspondence 
\[\begin{tikzcd}
        {\mathcal{A}=\{L:K\leq L\leq G\}} & {\{Y:Y\leq f(G)\}=\mathcal{B}}
        \arrow["\sigma", shift left=1, from=1-1, to=1-2]
        \arrow["\tau", shift left=1, from=1-2, to=1-1]
\end{tikzcd}
\]
The correspondence is given by $\sigma(L)=f(L)$ and $\tau(Y)=f^{-1}(Y)$. Moreover, the following statements hold: 
\begin{enumerate}
\item $L_1\leq L_2$ if and only if $\sigma(L_1)\leq \sigma(L_2)$.
\item $L\unlhd G$ if and only if $\sigma(L)\unlhd f(G)$.
\end{enumerate}
\end{theorem}

\begin{proof}
    Note that $\sigma$ and $\tau$ are well-defined, as 
    $f(L)\leq f(G)$ and $K\leq f^{-1}(Y)\leq G$.

    Let us prove that $\tau\circ\sigma=\id_\mathcal{A}$. We need to show that $\tau(\sigma(L))=L$ for all $L\in\mathcal{A}$. If $x\in f^{-1}(f(L))$, then $f(x)\in f(L)$. Thus $f(x)=f(l)$ for some
    $l\in L$. Hence $xl^{-1}\in K$ and therefore 
    $x\in Kl\subseteq L$, as $K\subseteq L$.
    Conversely, if $l\in L$, then $f(l)\in f(L)$. Thus $l\in f^{-1}(f(L))$.

    We now prove that $\sigma\circ\tau=\id_\mathcal{B}$. If  $Y\in\mathcal{B}$, then $\sigma(\tau(Y))=Y$. If $y\in Y\subseteq f(G)$, then $y=f(x)$ for some $x\in G$, that is $x\in f^{-1}(y)$. This implies that $y=f(x)\in f(f^{-1}(Y))$. Conversely, if $y\in f(f^{-1}(Y))$, then $y=f(x)$ for some $x\in f^{-1}(Y)$. This implies that $y=f(x)\in Y$.

    It is an exercise to show that 
    $X\leq Y$ if and only if $f(X)\leq f(Y)$.

    We now show that $L\unlhd G$ if and only if 
    $f(L)\unlhd f(G)$. If $L\unlhd G$ and $x\in G$,
    then $xLx^{-1}=L$. This implies that $f(L)=f(xLx^{-1})=f(x)f(L)f(x)^{-1}$, that is
    to say that $f(L)$ is normal in $f(G)$. Conversely, if
    $f(L)\unlhd f(G)$ and $x\in G$, then 
        \[
        f(xLx^{-1})=f(x)f(L)f(x)^{-1}=f(L).
        \]
    This implies that $xLx^{-1}\subseteq LK\subseteq L$. Thus 
    $xLx^{-1}\subseteq L$, which means that $L$ is normal in $G$. 
\end{proof}

\begin{proposition}
    If $f\colon G\to f(G)$ is a surjective group homomorphism 
    and $H\leq G$ is such that $K=\ker f
\subseteq H$, then
        $(G:H)=(f(G):f(H))$.
\end{proposition}

\begin{proof}
% By the correspondence theorem, 
% there is a bijective correspondence 
% \[
% \begin{tikzcd}
%         {\{L:K\leq L\leq G\}} & {\{Y:Y\leq f(G)\}}
%         \arrow[shift left=1, from=1-1, to=1-2]
%         \arrow[shift left=1, from=1-2, to=1-1]
% \end{tikzcd}
% \]
% given by $H\mapsto f(H)$ and 
% $f^{-1}(T)\mapsfrom T$. 
Let $H\leq G$ be such that
$\ker f\subseteq H$ and 
\[
\alpha\colon G/H\to f(G)/f(H),\quad \alpha(gH)=f(g)f(H).
\]
It is an exercise to show that $\alpha$ is well-defined. 
We need to show that $\alpha$ is bijective, as then 
\[
(G:H)=|G/H|=|f(G)/f(H)|=(f(G):f(H)).
\]

First, we show that $\alpha$ is surjective. If $yf(H)\in f(G)/f(H)$, 
then
$y=f(g)$ for some $g\in G$ (because $f$ is surjective).Thus
\[
yf(H)=f(g)f(H)=f(gH)=\alpha(gH).
\]

We now show that $\alpha$ is injective. If $\alpha(gH)=\alpha(g_1H)$, then, 
\[
f(g)^{-1}f(g_1)=f(h)\in f(H)
\]
for some $h\in H$, that is 
$f(g_1)=f(g)f(h)=f(gh)$ for some $h\in H$. 
This implies that $g_1=ghk$ for some $k\in\ker f\subset
eq H$ and hence 
$g_1=gh_1$ for some $h_1\in H$, that is $g_1H=gH$.
\end{proof}

In the case of the canonical homomorphism $\pi\colon
 G\to G/N$, the previous result reads as follows. 
If $N$ is a normal subgroup of $G$, then
$K\mapsto K/N$ is a bijection between the set of 
(normal) subgroups of $G$ containing $N$ and the 
set of (normal) subgroups of$G/N$.
If $H$ is a subgroup of $G$, then
\[
\pi(H)=HN/N.
\]
\begin{example}
\index{Quaternion group}
Let us show that every subgroup of 
\[
Q_8=\{1,-1,i,-i,j,-j,k,-k\}
\]
is normal $Q_8$. Let $N=\{-1,1\}$. Then $N$ is normal in $Q_8$, as $N\subseteq Z(Q_8)$). Since $|Q_8/N|=4$, 
$Q_8/N$ is an abelian group. 

We claim that $N$ is included in every non-trivial 
subgroup of $Q_8$. If 
$K$ is a non-trivial subgroup of $Q_8$, then $-1\in K$ (because, for example, if $-i\in K$, then
 $-1=(-i)^2\in K$).
Then every subgroup of $Q_8$ corresponds to a subgroup of $Q_8/N$. 
Since $Q_8/N$ is abelian, every subgroup of $Q_8/N$ is normal. Thus 
if $S\leq Q_8$, then 
$\pi(S)$ is normal in $Q_8/N$.
Since $N\subseteq S$, it follows that
$S=\pi^{-1}(\pi(S))$. Hence $S$ is normal in $Q_8$.
\end{example}

% En el ejemplo anterior, podríamos haber demostrado que
% $G/N\simeq\Z/2\times\Z/2$, ya que como sabemos que $|G/N|=4$, hubiera alcanzado con calcular el orden
%  de cada uno de los elementos de $G/N$.

\begin{example}
Let $f\colon\Z/12\to\Z/6$ be the group homomorphism given by 
$1\mapsto 1$. Then $K=\ker f=\{0,6\}$.
The subgroups of $\Z/12$ containing $K$ are
\[
\langle 1\rangle=\{0,1,\dots,11\},
\quad
\langle 2\rangle=\{0,2,4,6,8,10\},
\quad
\langle 3\rangle=\{0,3,6,9\},
\quad
\langle 6\rangle=\{0,6\}.
\]
These subgroups correspond via $f$ to
the subgroups
\[
\langle 1\rangle=\{0,1,\dots,5\},
\quad
\langle 2\rangle=\{0,2,4\},
\quad
\langle 3\rangle=\{0,3\},
\quad
\{0\}
\]
of $\Z/6$, respectively. 
For example, 
\[
\begin{tikzcd}
        && \Z/12 \\
        & \langle 2\rangle && {\Z/6} \\
        \{0,6\} && \langle 2\rangle \\
        & {\{0\}}
        \arrow[no head, from=1-3, to=2-4]
        \arrow[no head, from=1-3, to=2-2]
        \arrow[no head, from=2-2, to=3-1]
        \arrow[no head, from=3-1, to=4-2]
        \arrow[no head, from=2-2, to=3-3]
        \arrow[no head, from=3-3, to=4-2]
        \arrow[no head, from=2-4, to=3-3]
\end{tikzcd}
\]
\end{example}

The correspondence theorem helps to transport  
properties from the image of a group homomorphism 
to the domain. Let us discuss a concrete example. 

\begin{example}
Let $G$ be a finite group and $N$ be a normal subgroup of
$G$ such that $N\simeq\Z/5$ and $G/N\simeq\Sym_4$. The following statements hold: 
\begin{enumerate}
\item $|G|=120$
\item $G$ contains a normal subgroup of order 20.
\item $G$ contains three subgroups of order 15, none of them normal in $G$.
\end{enumerate}

To prove the first claim we note that 
Lagrange's theorem implies that 
\[
24=|G/N|=\frac{|G|}{|N|}=|G|/5.
\]

We prove the second claim. Let $K$ be the subgroup of $G/N$ isomorphic
to the Klein group. 
Then 
$K$ is normal in $G/N$ and $|K|=4$. Since 
$(G/N:K)=6$,
the subgroup $K$ of $G/N$ corresponds to a normal subgroup $H$ of $G$ such that $(G:H)=6$. Using Lagrange's theorem
and the correspondence theorem, 
$|H|=20$, as 
\[
6=(G/N:K)=(G:H)=\frac{|G|}{|H|}.
\]

For the third claim, note that
$G/N\simeq\Sym_4$ has four subgroups of order 
 3 (these are the subgroups generated by a 3-cycle),
none normal in $G/N$. By the correspondence theorem, these
subgroups correspond with four non-normal subgroups of $G$, all of order 15. 
\end{example}

If $G$ is a group, $\Sym_G=\{f\colon G\to G:f\text{ bijective}\}$.

\begin{theorem}[Cayley]
\index{Cayley's theorem}
\label{thm:Cayley}
Every group $G$ is isomorphic to a subgroup of $\Sym_G$.
\end{theorem}

\begin{proof}
Let $f\colon G\to\Sym_G$, $g\mapsto L_g$, where $L_g\colon G\to G$, $L_g(x)=gx$. Then $f$ is a group homomorphism, as 
\[
L_{gh}(x)=(gh)x=g(hx)=L_g(hx)=L_gL_h(x)
\]
for all $g,h,x\in G$. Moreover, $f$ is injective (if $f(g)=f(h)$, then $L_g=L_h$, and this implies that 
$gx=L_g(x)=L_h(x)=hx$ for all $x\in G$, which ultimately implies $g=h$).
It follows that $G\simeq f(G)$, which is a subgroup of $\Sym_G$. 
\end{proof}

Every finite group is isomorphic to a subgroup of some 
$\Sym_n$. 
In particular, using permutation matrices, 
we see that every finite group 
isomorphic to a subgroup of $\GL_n(\Z)$ for some $n$.
Those groups are known as \textbf{linear groups}.

\begin{proposition}
Every finite simple group $G$ is contained in some $\Alt_n$.
\end{proposition}

\begin{proof}
If $|G|=2$, the result is trivial, as $G\simeq\langle (12)(34)\rangle\subseteq\Alt_2$. Assume 
that $|G|>2$.
Let $f\colon G\to\Sym_n$ by the injective group homomorphism
given by Cayley's theorem. If $H=f(G)$, then 
$G\simeq H$ by the first isomorphism theorem. 
We claim that $H\subseteq\Alt_n$. If
$H$ is not a subgroup of $\Alt_n$, there exists $h\in H$ such that  $h\not\in\Alt_n$. Write $h=f(g)$ for some 
$g\in G$. Since $h\not\in\Alt_n$,
$\sgn(f(g))=\sgn(h)=-1$, that is 
$g\not\in\ker(\sgn\circ f)$.
Let $K=\ker(\sgn\circ f)$. Then $K=\{1\}$, as $G$ is simple. Moreover, $\sgn\circ f$ is a bijective map, as
$\sgn(f(1))=1$ and $\sgn(f(g))=-1$. Therefore 
$G\simeq G/K\simeq\Z/2$, by the first isomorphism theorem.  
In particular, $|G|=2$, a contradiction. Thus $H\subseteq\Alt_n$.
\end{proof}

Let us briefly discuss another application of Cayley's theorem. 
We use the theorem to show that in a group, 
no product needs parenthesis 
By Cayley's theorem, a group $G$ is (isomorphic to) a 
subgroup of $\Sym_G$. The composition of maps is an associative operation. Moreover, no composition of finitely many maps 
needs parenthesis. Thus 
\[
(f_1\circ\cdots\circ f_n)(g)=f_1(f_2(\cdots f_n(g))\cdots).
\]
