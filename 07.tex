\section{}

\subsection{Semi-direct products}

We first start with an exercise of \textbf{direct products}.

\begin{exercise}
\index{Direct product of groups}
\label{xca:direct_product}
Let $G$ be a group and $H$ and $K$ be normal subgroups of $G$.
If $G=HK$ and $H\cap K=\{1\}$, then $G\simeq H\times K$.
\end{exercise}

% \begin{proof}
% Sea $f\colon G\to H\times K$, $f(g)=(h,k)$, donde $h\in H$ y $k\in K$ son únicos tales que $g=hk$. Esto tiene sentido pues si $g\in G$ then $g=hk$ para algún $h\in H$ y $k\in K$; si además $g=h_1h_1$ para $h_1\in H$ y $k_1\in K$, then, como $hk=h_1k_1$, se tiene que $h_1^{-1}h=k_1k^{-1}\in H\cap K=\{1\}$ y luego $h=h_1$ y $k=k_1$.

% Veamos que si $g=hk$ y $g_1=h_1k_1$ para $h,h_1\in H$ y $k,k_1\in K$, then $kh_1=h_1k$. En efecto, $[k,h_1]=kh_1k^{-1}h_1^{-1}\in H\cap K=\{1\}$ pues la normalidad de $H$ y $K$ implican que $ kh_1k^{-1}\in H$ y $h_1k^{-1}h_1^{-1}\in K$.

% La observación anterior nos permite demostrar que $f$ es un morfismo de grupos. Si $g=hk$ y $g_1=h_1k_1$ con $h,h_1\in H$ y $k,k_1\in K$, then, como $f(g)=(h,k)$ y $f(g_1)=(h_1,k_1)$, tenemos que
% \[
% f(gg_1)=f((hk)(h_1k_1))=f(h(kh_1)k_1)=f((hh_1)(kk_1))=(hh_1,kk_1).
% \]
% Queda como ejercicio demostrar que $f$ es biyectiva.
% \end{proof}

\begin{exercise}
Let $A$ be a normal subgroup of $H$, and $B$ be a normal subgroup of $K$. 
Prove that $A\times B$ is a normal subgroup of
$H\times K$ and 
\[
\frac{H\times K}{A\times B}\simeq(H/A)\times(K/B).
\]
\end{exercise}

\index{Exact factorization of groups}
We say that a group $G$ admits an exact factorization through 
the subgroups $H$ and $K$ if $G=HK$ and 
$H\cap K=\{1\}$. By Exercise \ref{xca:direct_product}, 
if $G$ admits an exact factorization through two normal subgroups, then 
it is isomorphic to the direct product of these subgroups. 

\begin{exercise}
    Let $G$ be a group that admits an exact factorization through
    the subgroups $H$ and $K$. Prove that every $g\in G$ can be written 
    uniquely as $g=hk$ for some $h\in H$ and $k\in K$. 
\end{exercise}

\begin{example}
Let $G=\Sym_3$, $H=\langle (123)\rangle\unlhd G$ and $K=\langle (12)\rangle$. Since
$K$ is not normal in $G$, we cannot apply Exercise \ref{xca:direct_product}. 
We do have $G=HK$ and $H\cap K=\{\id\}$, but $H\times K\simeq\Z/3\times\Z/2\not\simeq\Sym_3$, 
as $\Z/3\times\Z/2$ is abelian and $\Sym_3$ is not. 
\end{example}


\begin{definition}
\index{Complement}
Let $G$ be a group, $K$ a normal subgroup of $G$, and $Q$ a subgroup of $G$. We say
that $Q$ \textbf{complements} $K$ in $G$ if $K\cap Q=\{1\}$ y $G=KQ$.
\end{definition}

\begin{example}
Let $G=\Sym_3$ and $K=\langle (123)\rangle\unlhd G$. The subgroups
$\langle (12)\rangle$, $\langle
(13)\rangle$ and $\langle (23)\rangle$ complement $K$ in $G$.
\end{example}

The previous example shows that complements are not unique. However, 
complements are unique under isomorphism, as 
\[
G/K= KQ/K\simeq Q/K\cap Q=Q/\{1\}\simeq Q.
\]

\begin{definition}
\index{Semi-direct product}
We say that a group $G$ is a \textbf{semi-direct product} of $Q$ and $K$ if $K$ 
is normal in $G$ and 
$K$ admits a complement in $G$ isomorphic to $Q$. Notation: $G=K\rtimes Q$.
\end{definition}

Let us discuss some characterizations:

\begin{proposition}
Let $K$ be a normal subgroup of $G$. The following statements are equivalent:
\begin{enumerate}
\item $K$ admits a complement in $G$.
\item There exists a subgroup $Q$ of $G$ such that each $g\in G$ can be written uniquely 
as $g=xy$ for some 
$x\in K$ and $y\in Q$.
\item There exists a group homomorphism $s\colon G/K\to G$ such that $\pi\circ s=\id_{G/K}$, where $\pi\colon G\to G/K
$, $g\mapsto Kg$, is the canonical homomorphism.
\item There exists a group homomorphism $\rho\colon G\to G$ such that $\ker\rho=K$ and the restriction $\rho|_{\rho(G)}$ equals the identity. 
\end{enumerate}
\end{proposition}

\begin{proof}
We first prove that $(1)\implies(2)$. If $Q$ complements $K$, then $G=KQ$ and
$K\cap Q=\{1\}$. In particular, if $g\in G$, then $g=xy$ para $x\in K$ e $y\in Q$. To show that
the decomposition is unique, suppose that 
$g=x_1y_1$ with $x_1\in K$ and $y_1\in Q$. Then $x_1^{-1}x=yy_1^{-1}\in K\cap Q=\{1\}$ 
and hence $x=x_1$ and $y=y_1$.

It is an exercise to show that $(2)\implies(1)$.

We now prove that $(2)\implies(3)$. Let $s\colon G/K\to G$, $s(Kg)=y$ if 
$g=xy$ with $x\in K$ and $y\in Q$. (Note that here we use right cosets, as it is more convenient.)
Let us check that $s$ is well-defined. 
For that purpose, we must show that $Kg=Kg_1$ implies $s(Kg)=s(Kg_1)$. If $g=xy$ 
and $g_1=x_1y_1$ with $x,x_1\in K$ and $y,y_1\in Q$, then, since $Kg=Kg_1$, 
$xyy_1^{-1}x_1^{-1}=gg_1^{-1}\in K$, that is $yy_1^{-1}\in x^{-1}Kx_1=K$
 because $x,x_1\in K$. Hence $yy_1^{-1}\in K\cap Q=\{1\}$ and thus $y=y_1$. 
 We now show that $\pi\circ
 s=\id_{G/K}$. If $g=xy$ with $x\in K$ e $y\in Q$, then
$(\pi\circ s)(Kg)=\pi(y)=Ky=Kxy=Kg$. It is left to the reader to prove that 
$s$ is a group homomorphism. 

We now prove that $(3)\implies(4)$. Let $\rho=s\circ\pi$. Then $\rho$ is a group homomorphism (because it is the composition of homomorphisms). We compute: 
\[
\rho(\rho(g))=\rho( (s\circ\pi)(g))=\rho(s(Kg))=((s\circ\pi)\circ s)(Kg)=s(Kg)=\rho(g).
\]
We now compute $\ker\rho$. If $g\in\ker\rho$, then $s(\pi(g))=\rho(g)=1$. Thus 
\[
\pi(g)=\pi(s(\pi(g)))=\pi(1)=1_{G/K},
\]
that is $g\in\ker\pi=K$. Conversely, if $x\in K$, then
\[
\rho(x)=\rho(s(Kx))=\rho(s(K))=\rho(1)=1
\]
and hence $x\in\ker\rho$.

Finally, we prove that $(4)\implies(1)$. We claim that $Q=\rho(G)$ complements $K$ in
 $G$. We first show that $K\cap Q=\{1\}$: if $x\in K\cap Q$, then $x=\rho(g)$ for some $g\in G$. Moreover, 
\[
1=\rho(x)=\rho(\rho(g))=\rho(g).
\]
Hence $g\in\ker\rho=K$ and $x=1$. We now prove that $G=KQ$. For the inclusion $G\subseteq KQ$, 
\[
g=(g\rho(g^{-1}))\rho(g)
\]
and $g\rho(g^{-1})\in K=\ker\rho$, as $\rho(g\rho(g^{-1}))=  \rho(g)\rho(g^{-1})=1$.
\end{proof}

\begin{example}
$\Sym_n=\Alt_n\rtimes\Z/2$, as $Q=\langle (12)\rangle\simeq\Z/2$ complements the normal subgroup 
$\Alt_n$ of $\Sym_n$.
\end{example}

For a group $G$, 
the set 
\[
\Aut(G)=\{f\colon G\to G:f\text{ bijective}\}
\]
is a group with the composition of maps. It is called 
the \textbf{automorphism group} of $G$. 
Examples of automorphism groups are the identity map and 
conjugations homomorphisms. 

\begin{example}
$\Aut(\Z)\simeq\Z/2$, as $\Aut(\Z)=\{\id,-\id\}$.
\end{example}

\begin{example}
$\Aut(\Z)\simeq\Z/2$, as $\Aut(\Z)=\{\id,-\id\}$.
\end{example}

\begin{example}
Let $G$ be a group and $g\in G$. The conjugation 
map $\gamma_g\colon G\to G$, $x\mapsto gxg^{-1}$,
is an automorphism of $G$, as 
\[
\gamma_g(xy)=g(xy)g^{-1}=(gxg^{-1})(gyg^{-1})=\gamma_g(x)\gamma_g(y).
\]
Moreover, $\gamma\colon G\to\Aut(G)$, $g\mapsto\gamma_g$, is a group
homomorphism:
\[
\gamma_{gh}(x)=(gh)x(gh)^{-1}=g(\gamma_h(x))g^{-1}=\gamma_g(\gamma_h(x))=(\gamma_g\circ\gamma_h)(x).
\]

\index{Inner automorphisms}
The group of \textbf{inner automorphisms} of $G$ is the 
group 
$\Inn(G)=\gamma(G)$. Note that $\ker\gamma=Z(G)$, as
if $g\in G$ is such that $\gamma_g=\id$, then 
\[
\gamma_g(x)=gxg^{-1}=x
\]
for all $x\in G$. By the first isomorphism theorem, 
\[
G/Z(G)\simeq \gamma(G)=\Inn(G).
\]
\end{example}

\begin{exercise}
\label{xca:aut(S3)}
    Prove that $\Aut(\Sym_3)\simeq\Sym_3$. 
\end{exercise}

\begin{exercise}
    Let $G$ be a group. Prove that $\Inn(G)$ is a normal subgroup of $\Aut(G)$. 
\end{exercise} 

For a group $G$, the quotient $\Aut(G)/\Inn(G)$ is called 
the group of \textbf{outer automorphisms} of $G$. 
Note that 
\[
\Inn(G)\text{ is cyclic}\Longleftrightarrow
|\Inn(G)|=1\Longleftrightarrow
G\text{ is abelian.}
\]

\begin{exercise}
Let $G$ be a group. 
Prove that if $\Aut(G)$ is cyclic, then 
$G$ is abelian. 
\end{exercise}

\begin{exercise}
    Prove that if $G$ is finite, then $\Aut(G)$ is finite.
\end{exercise}

% \begin{exercise}
% \label{xca:autgeq2}
%         Si $G$ es un grupo tal que $|G|\geq3$, entonces $|\Aut(G)|\geq2$.
% \end{exercise}


% \begin{exercise}
% \label{xca:aut_impar}
%         No existe un grupo finito cuyo grupo de automorfismos sea no trivial cíclico y de orden impar.
% \end{exercise}

% \begin{exercise}
% \label{xca:p2dividesAut}
%         Sea $p$ un número primo.
%         Si $G$ es un $p$-grupo no abeliano, entonces $p^2$ divide a $|\Aut(G)|$.
% \end{exercise}

% \begin{exercise}
%         \label{xca:Aut(H)}
%         Si $G=H\times K$, entonces $\Aut(H)$ es isomorfo a un subgrupo de $\Aut(G)$.
% \end{exercise}

% \begin{exercise}
% \label{xca:aut_trivial_center}
%         Si $G$ tiene centro trivial, entonces $\Aut(G)$ también.
% \end{exercise}

The following exercise constructs semi-direct products. 

\begin{exercise}
\label{xca:semi-direct}
Let $K$ and $Q$ be groups and $\theta\colon Q\to\Aut(K)$, $x\mapsto\theta_x$, a group
homomorphism. Prove that $K\times Q$ 
with 
\[
(a,x)(b,y)=(a\theta_x(b),xy)
\]
is a group. This group will be written as $K\rtimes_\theta Q$.
\end{exercise}

% \begin{proof}[Bosquejo de la demostración]
% Dejamos como ejercicio verificar que la operación es asociativa. Hay que verificar además que el elem
% ento neutro de $K\rtimes_\theta Q$ será $(1,1)$ y
% que el inverso de $(a,x)\in K\rtimes_\theta Q$ será $(\theta_{x^-1}(a^{-1}),x^{-1})$.
% \end{proof}

The group of Exercise \ref{xca:semi-direct} is the
semi-direct product of the subgroups 
\begin{align*}
K\times\{1\}=\{(a,1):a\in K\}\simeq K,&&
\{1\}\times Q=\{(1,x):x\in Q\}\simeq Q
\end{align*}
of $K\rtimes_\theta Q$. Note that $K\times\{1\}$ is normal in $K\rtimes_\theta Q$. 
We can identity $K\rtimes\{1\}$ with $K$ 
and $\{1\}\rtimes Q$ with $Q$. Thus we can write 
\[
\theta_x(a)=xax^{-1}
\]
for all $x\in Q$ and $a\in K$.


\begin{exercise}
Prove that if $G$ is a semi-direct product of the normal subgroup 
$K$ with the subgroup $Q$, there exists a group homomorphism 
$\theta\colon Q\to\Aut(K)$
such that $G\simeq K\rtimes_\theta Q$.
\end{exercise}

% \begin{proof}[Bosquejo de la demostración]
% Para $x\in Q$ sea $\theta_x\colon K\to K$, $\theta_x(a)=xax^{-1}$. Ya vimos que $\theta_x\in\Aut(K)$
% y que $Q\to\Aut(K)$, $x\mapsto\theta_x$ es un morfismo de grupos. Queda verificar que
% la función $K\rtimes_\theta Q\to G$, $(a,x)\mapsto ax$, es un morfismo biyectivo de grupos.
% \end{proof}

Let us discuss some examples. 

\begin{example}
Let $N\simeq \Z/n$ and $H\simeq\Z/2=\{0,1\}$. The map $\theta\colon H\to\Aut(N)$, 
$1\mapsto (x\mapsto x^{-1})$, is a group homomorphism. Let $G=N\rtimes_\theta H$.
Then $G\simeq\D_n$, the dihedral group of order $2n$.
Recall that 
\[
\D_n=\langle r,s:r^n=s^2=1, srs^{-1}=r^{-1}\rangle.
\]
Assume that $N=\langle x\rangle$ and $H=\langle y\rangle$. Then $|(x,1)|=n$ and $|(1,y)|=2$. 
Moreover,
\begin{align*}
(1,y)(x,1)(1,y)^{-1} &= (\varphi_y(x),y)(1,y)\\
&=(\varphi_y(x),y^2)\\
&=(\varphi_y(x),1)\\
&=(x^{-1},1)\\
&=(x,1)^{-1}.
\end{align*}
If $u=(x,1)$ y $v=(1,y)$, then $u^n=v^2=(1,1)$ and $vuv^{-1}=u^{-1}$. Thus there exists 
a surjective group homomorphism 
$\D_n\to G$ (because $G$ is generated by $u$ and $v$). Moreover, $|G|=|N||H|=2n$. Hence 
$G$ has order $2n$ and therefore $G\simeq\D_n$.
\end{example}

\begin{example}
Let $K=\{\id,(12)(34),(13)(24),(14)(23)\}$. Then $K$ is normal in $\Alt_4$. 
Let $H=\langle (123)\rangle\simeq\Z/3$. Since $K\cap H$ is a subgroup of $H$ and $K$ and, moreover, 
$K$ and $H$ have coprime orders, $H\cap K=\{\id\}$. Hence $\Alt_4=K\rtimes H$.
\end{example}

\begin{example}
Let 
\[
K=\{\id,(12)(34),(13)(24),(14)(23)\},\quad  
H=\{\sigma\in\Sym_4:\sigma(4)=4\}.
\]
Note that $H$ is a subgroup of 
$\Sym_4$ isomorphic to $\Sym_3$. Then $H\cap K=\{\id\}$ and hence 
$\Sym_4=K\rtimes H$.
\end{example}

Let $n\geq5$. 
Using the fact that $\Alt_n$ is a simple group,  
one proves that $\Alt_n$ cannot be written as a semi-direct product of proper subgroups. 

\begin{example}
Let $K=\Z/3$ and $Q=\Z/4$. Since $\Hom(Q,\Aut(K))=\{1,\tau\}$, where 
\[
\tau\colon\Z/4\to\Aut(\Z/3)=\{\id,\rho\}\simeq\Z/2,\quad 1\mapsto\rho,
\]
the semi-direct product $T=K\rtimes_\tau Q$ is a non-abelian group of order 12. Moreover,
$T\not\simeq\Alt
_4$ as $|(2,2)|=6$ and $\Alt_4$ has no elements of order six.
\end{example}



% todo: El grupo Aff(R) es un producto semidirecto
% todo: Necesitamos más ejemplos de producto semidirecto