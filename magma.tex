\section*{Magma code}

\fancyhf{}
\fancyfoot[R]{\thepage}
\fancyhead[L]{\course}
\fancyhead[R]{Magma code}
\setlength{\headheight}{14pt}

\subsection*{Permutations in Magma}

In our calculations with permutations, we have adopted the right-to-left convention, aligning with the rest of the course, where we consider groups acting on the left on sets (see~\pageref{convention:left-to-right}). 
However, some people prefer the left-to-right convention, which corresponds to the way English is read (and to groups acting on the right). This is the convention used by Magma:
\begin{lstlisting}
> S3 := Sym(3);
> a := S3!(1,2);
> b := S3!(2,3);
> a*b;
(1, 3, 2)
\end{lstlisting}
But there's no need to worry: both conventions are equivalent, see Example \ref{exa:op}. 

\subsection*{Example~\ref{exa:Carmichael}}
\label{magma:Carmichael}

A nice subgroup of $\Sym_{16}$ 
in which the set of commutators is not a subgroup: 

\begin{lstlisting}
> S16 := Sym(16);
> a := S16 ! (1,3)(2,4);
> b := S16 ! (5,7)(6,8);
> c := S16 ! (9,11)(10,12);
> d := S16 ! (13,15)(14,16);
> e := S16 ! (1,3)(5,7)(9,11);
> f := S16 ! (1,2)(3,4)(13,15);
> g := S16 ! (5,6)(7,8)(13,14)(15,16);
> h := S16 ! (9,10)(11,12);
> G := PermutationGroup< 16 | a,b,c,d,e,f,g,h >;
> D := DerivedSubgroup(G);
> #D;
16
> #{ x*y*Inverse(x)*Inverse(y) : x in G, y in G };
15    
> c*d in { x : x in D } \ 
> diff { u*v*Inverse(u)*Inverse(v) : u in G, v in G };
true
\end{lstlisting}

\subsection*{Example~\ref{exa:Guralnick}}
\label{magma:Guralnick}

In the example, we claimed that 
this is an example of a group where the set of commutators is not a subgroup. We 
start creating a function that returns the set of commutators:
\begin{lstlisting}
> SetOfCommutators := function(G)
function> return { (x,y) : x,y in G };
function> end function;
\end{lstlisting}
Now we check that in Guralnick's group, the commutator subgroup 
differs from the set of commutators: 
\begin{lstlisting}
> S12 := Sym(12);
> a := S12 ! (1,3,5)(2,4,6)(7,11,9)(8,12,10);
> b := S12 ! (3,9,4,10)(5,8)(6,7)(11,12);
> G := PermutationGroup< 12 | a,b >;
> Order(G);
96
> D := DerivedSubgroup(G);
> Order(D);
32
> #SetOfCommutators(G);
29
\end{lstlisting}
Let us see that no smaller group 
have the desired property: 
\begin{lstlisting}
> { G : G in SmallGroups([1..95]) | not #DerivedSubgroup(G) 
> eq #SetOfCommutators(G) };
{}
\end{lstlisting}

\subsection*{Example~\ref{exa:HKneKH}}


\subsection*{Exercise~\ref{xca:D3}}

The following Magma code uses some more advanced methods to construct elements such as $\sqrt{3}$. These details are not needed for solving the exercise and may be ignored; they are included only to record a few useful tips for certain constructions in Magma. 

\begin{lstlisting}
> Q<x> := PolynomialAlgebra(Rationals());
> f := x^2 - 3;
> K<z> := NumberField(f);
> z^2;
3
> r := Matrix(2, [-1/2,-z/2,z/2,-1/2]);
> s := Matrix(2, [1,0,0,-1]);
> G := MatrixGroup<2,K|r,s>;
> #G
6      
\end{lstlisting}

\subsection*{Example~\ref{exa:Q8}}

This is the code to create the quaternion group $Q_8$ 
in Magma:
\begin{lstlisting}
> K<i> := CyclotomicField(4);
> a := Matrix(2, [i,0,0,-i]);
> b := Matrix(2, [0,1,-1,0]);
> G := MatrixGroup<2,K|a,b>;
> GroupName(G);
Q8      
\end{lstlisting}    

\subsection*{Exercise~\ref{xca:orders_Z6}}

Here is a Magma solution, where for example we see that, as we know,
$1\in\Z/6$ is a generator, 
$2\in\Z/6$ has order three, and $4\in\Z/6$ has order three:
\begin{lstlisting}
> R := Integers(6);
> G, g := AdditiveGroup(R);
> { <g(x),Order(x)> : x in G };
{ <0, 1>, <2, 3>, <5, 6>, <3, 2>, <1, 6>, <4, 3> }   
\end{lstlisting}

\subsection*{Exercise~\ref{xca:order_ab}}

Let us see a Magma solution: 
\begin{lstlisting}
> a := Matrix(2, [1,-1,1,0]);
> b := Matrix(2, [0,1,-1,-1]);
> Order(a);
6
> Order(b);
3
> HasFiniteOrder(a*b);
false    
\end{lstlisting}

\subsection*{Example~\ref{exa:S3_cosets}}

As we discussed, the annoying fact that Magma multiplies permutations in the opposite order from us is not a serious issue. In this particular example, we only need to work with right cosets instead of left cosets.
Why? Because switching from left to right cosets exactly compensates for the reversed convention in permutation multiplication!

\begin{lstlisting}
> G := Sym(3);
> H := sub<G|(1,2)>;
> K := sub<G|(1,2,3)>;
> { h*G!(1,2,3) : h in H };
{
    (1, 3),
    (1, 2, 3)
}
> { h*G!(2,3) : h in H };
{
    (1, 3, 2),
    (2, 3)
}
\end{lstlisting}

\subsection*{Example~\ref{exa:normal_A4}}

For Magma, getting the list of normal subgroups of $\Alt_4$ 
is easy: 
\begin{lstlisting}
> G := Alt(4);
> NormalSubgroups(G);
Conjugacy classes of subgroups
------------------------------

[1]     Order 1            Length 1
        Permutation group acting on a set of cardinality 4
        Order = 1
[2]     Order 4            Length 1
        Permutation group acting on a set of cardinality 4
        Order = 4 = 2^2
            (1, 3)(2, 4)
            (1, 2)(3, 4)
[3]     Order 12           Length 1
        Permutation group acting on a set of cardinality 4
        Order = 12 = 2^2 * 3
            (2, 3, 4)
            (1, 3)(2, 4)
            (1, 2)(3, 4)    
\end{lstlisting}


\subsection*{Exercise~\ref{xca:D4_normal}}

The exercise shows that normality is in general not transitive: 
\begin{lstlisting}
> G<s,r> := DihedralGroup(GrpGPC, 4);
> Order(r);
4
> Order(s);
2
> N := sub<G|s,r^2>;
> H := sub<G|s>;
> IsNormal(N,H);
true
> IsNormal(G,N);
true
> IsNormal(G,H);
false    
\end{lstlisting}

Here is the Magma code:
\begin{lstlisting}
> G := Sym(3);
> H := sub<G|(1,2)>;
> K := sub<G|(2,3)>;
> H*K eq K*H;
false    
\end{lstlisting}

\subsection*{Example~\ref{exa:anotherHK}}

Here is the code to verify some of the statements mentioned: 
\begin{lstlisting}
> G := Sym(3);
> H := sub<G|(1,2)>;
> K := sub<G|(1,2,3)>;
> IsNormal(G,K);
true
> #(H*K);
6
> #(H meet K);
1    
\end{lstlisting}

\subsection*{Example~\ref{exa:U(21)}}

Recall that Magma uses additive notation for abelian groups. Here is the code: 
\begin{lstlisting}
> R := ResidueClassRing(21);
> G, g := UnitGroup(R);    
> { g(x) : x in G };
{ 17, 1, 2, 19, 20, 4, 5, 8, 10, 11, 13, 16 }
> f := hom<G->G|x:->3*x>;
> { g(x) : x in Kernel(f) };
{ 1, 4, 16 }
> { g(y) : y in Image(f) };
{ 1, 13, 8, 20 }
\end{lstlisting}
Alternative code:
\begin{lstlisting}
> U := { x : x in Integers(21) | GCD(x,21) eq 1 };
> U;
{ 17, 1, 2, 19, 20, 4, 5, 8, 10, 11, 13, 16 }
> G := GenericAbelianGroup(U: 
> IdIntrinsic := "Id", 
> AddIntrinsic := "*", 
> InverseIntrinsic := "/");
> f := hom<G->G|x:->3*x>;
> { y : y in Image(f) };
{ 1, 13, 8, 20 }    
\end{lstlisting}

% How to check that a map is a homomorphism of abelian groups? 
% > P, p := PermutationGroup(G);
% > f := hom<P->P|x:->x^3>;
% > IsHomomorphism(f);
% true

\subsection*{Example~\ref{exa:order4}}

The groups are not isomorphic because 
they do not have the same number of elements of order four: 
\begin{lstlisting}
> A := AbelianGroup([4]);
> B := AbelianGroup([2,2]);
> #{ x : x in A | Order(x) eq 4 };
2
> #{ x : x in B | Order(x) eq 4 };
0
\end{lstlisting}

\subsection*{Example~\ref{exa:U(Z/5)}}

Both groups are cyclic of order four: 
\begin{lstlisting}
> A := UnitGroup(Integers(5));
> B := UnitGroup(Integers(10));
> IsCyclic(A);
true
> IsCyclic(B);
true
> #A;
4
> #B;
4    
> IsIsomorphic(A,B);
true
\end{lstlisting}

\subsection*{Exercise~\ref{xca:U(Z/10)}}

The groups are not isomorphic. Why? 
Here is the code: 
\begin{lstlisting}
> A := UnitGroup(Integers(10));
> B := UnitGroup(Integers(12));
> IsIsomorphic(A,B);
false    
> GroupName(A);
C4
> GroupName(B);
C2^2
\end{lstlisting}


\subsection*{Example~\ref{exa:Q8normal}}

Checking in Magma that every subgroup of $Q_8$ 
is normal is very easy. Here is the code:
\begin{lstlisting}
> G := Group("Q8");
> forall(S){S:S in Subgroups(G)|IsNormal(G,S`subgroup)};
true    
\end{lstlisting}

\subsection*{Example~\ref{exa:Z12->Z6}}

Let us use Magma to illustrate the correspondence 
theorem. Instead of using 
additive abelian groups we will 
use their corresponding permutation groups, as 
several algorithms do not work 
on abelian groups. 
\begin{lstlisting} 
> Z12 := AdditiveGroup(Integers(12));;
> Z6 := AdditiveGroup(Integers(6));;
> C12, a := PermutationGroup(Z12);
> C6, b := PermutationGroup(Z6);
> f := hom<C12->C6| <C12.1,C6.1>>;
> K:=Kernel(f);
> IsHomomorphism(f);
true
> for x in AllSubgroups(C12) do
for> if K subset x then
for|if> { a(y) : y in x };
for|if> end if;
for> end for;
\end{lstlisting}