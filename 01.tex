\chapter{}

\topic{Groups}

Before defining groups, we recall that a binary operation on a set $X$
is simply a map 
\[
X\times X\to X,
\quad (x,y)\mapsto xy.
\]
Note that we have used 
juxtaposition to denote this generic binary operation. For example,
$(x,y)\mapsto x-y$ is a binary operation in $\Z$ but not, for example, 
in $\Z_{\geq 0}$.

\begin{definition}
\index{Group}
A \textbf{group} is a non-empty set $G$ with a binary operation
$G\times G\to G$, $(x,y)\mapsto xy$, such that
the following properties hold:
\begin{enumerate}
    \item (Associativity) $(xy)z=x(yz)$ for all $x,y,z\in R$.
    \item (Existence of a neutral element) There exists $e\in G$ such that $xe=ex=x$ for all $x\in G$.
    \item (Existence of inverses) For every $x\in G$ there exists $y\in G$ such that $xy=yx=e$.
\end{enumerate}
\end{definition}

The associativity condition implies
that all ordered products that we can form with 
the elements, say, $x_1,x_2,\dots,x_n$ will be equal. For example,
\[
(x_1x_2)((x_3x_4)x_5)=x_1(x_2(x_3(x_4x_5)))
\]
and hence we can write, without ambiguity (and without using brackets), 
$x_1x_2\cdots x_5$. This fact can be proved by induction; see for example
Lang's book. We will provide an alternative proof as an application
of Cayley's theorem. 

\begin{proposition}
    In a group $G$, every element $x\in G$ admits a unique inverse.
\end{proposition}

\begin{proof}
    Let $y,z\in G$ be inverses of $x\in G$. Then 
    $z=z(xy)=(zx)y=ey=y$.
\end{proof}

\begin{exercise}
    Prove that the neutral element of a group is unique. 
\end{exercise}

In general, when the binary operation is written multiplicatively, one
writes the identity element $e$ of a group as $1_G$ or simply as $1$. The inverse of $x$ will be 
denoted by $x^{-1}$. 

\begin{example}
    Let $n\geq1$. The set $\GL_n(\R)$ of $n\times n$ invertible real 
matrices forms a
    group with the usual matrix multiplication.  
\end{example}

It is a good idea to keep in mind the \emph{group of invertible matrices}. 
With this, the 
the following properties look familiar:
\begin{enumerate}
    \item $(x^{-1})^{-1}=x$ for all $x$.
    \item $(xy)^{-1}=y^{-1}x^{-1}$ for all $x,y$. 
\end{enumerate}

\begin{exercise}
    Prove that in a group, the equation $ax=b$ has 
    a unique solution $x=a^{-1}b$. Similarly, the equation
    $x=ba^{-1}$ is the unique solution of the equation
    $xa=b$. 
\end{exercise}

\begin{definition}
    \index{Group!abelian}
    A group $G$ is \textbf{abelian} if $xy=yx$ for all $x,y\in G$. 
\end{definition}

Most of the time, for abelian groups we will use 
the \emph{additive notation}. This means that the binary operation
of the group will be denoted by $(x,y)\mapsto x+y$, the neutral
element by $0$ and 
the inverse of an element $x$ will be $-x$. 

\begin{definition}
    \index{Group!order}
    The \textbf{order} $|G|$ of a group $G$ is the 
    cardinality of $G$. A group $G$ is said to be
    \textbf{finite} if $|G|$ is finite and \textbf{infinite}
    otherwise. 
\end{definition}

\begin{example}
\label{exa:abelian_groups}
    Let us see some 
    abelian groups: 
    \begin{enumerate}
        \item $\Z$, $\Q$, $\R$ and $\C$ with the usual addition. 
        \item Let $n\geq2$. The set $\Z/n$ of integers modulo $n$ with the usual addition modulo $n$.
        \item $\Q\setminus\{0\}$, $\R\setminus\{0\}$ and $\C\setminus\{0\}$ 
        with the usual multiplication.
        \item Let $p$ be a prime number. The set $(\Z/p)^{\times}=(\Z/p)\setminus\{0\}$ of invertible integers modulo $p$ 
            with the usual multiplication modulo $p$. 
    \end{enumerate}
\end{example}

The groups of the first two items will be written in additive notation. 

The group $\Z/n$ of integers modulo $n$ is a finite group of order $n$. 
The group $(\Z/p)^{\times}$ of units modulo $p$ is a finite
group of order $p-1$. The other groups of Example \ref{exa:abelian_groups} are infinite groups. 

\begin{exercise}
\label{xca:LR}
Let $G$ be a group and $g\in G$. Prove that 
the maps $L_g\colon G\to G$, $x\mapsto gx$, and 
$R_g\colon G\to G$, $x\mapsto xg$, are bijective. 
\end{exercise}





