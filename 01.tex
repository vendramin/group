\section{Lecture -- Week 1} 

\subsection{Groups}

Before defining groups, we recall that a binary operation on a set $X$
is simply a map 
\[
X\times X\to X,
\quad (x,y)\mapsto xy.
\]
We have used 
juxtaposition to denote this generic binary operation. For example,
\[
(x,y)\mapsto x-y
\]
is a binary operation in $\Z$ but not
in $\Z_{\geq 0}$.

\begin{definition}
\index{Group}
A \emph{group} is a non-empty set $G$ with a binary operation
$G\times G\to G$, $(x,y)\mapsto xy$, such that
the following properties hold:
\begin{enumerate}
    \item (Associativity) $(xy)z=x(yz)$ for all $x,y,z\in R$.
    \item (Existence of a neutral element) There exists $e\in G$ such that $xe=ex=x$ for all $x\in G$.
    \item (Existence of inverses) For each $x\in G$ there exists $y\in G$ such that $xy=yx=e$.
\end{enumerate}
\end{definition}

The associativity condition implies
that all ordered products that we can form with, say, 
the elements $x_1,x_2,\dots,x_n$ will be equal. For example,
\[
(x_1x_2)((x_3x_4)x_5)=x_1(x_2(x_3(x_4x_5)))
\]
and hence we can write, without ambiguity and without using brackets, 
\[
x_1x_2\cdots x_5.
\]
This fact can be proved by induction; see for example
Lang's book \cite{MR783636}. We will provide an alternative proof as an application
of Cayley's theorem \ref{thm:Cayley}. 

\begin{proposition}
    In a group $G$, every element $x\in G$ admits a unique inverse.
\end{proposition}

\begin{proof}
    Let $y,z\in G$ be inverses of $x\in G$. Then 
    $z=z(xy)=(zx)y=ey=y$.
\end{proof}

\begin{exercise}
\label{xca:neutral}
    Prove that the neutral element of a group is unique. 
\end{exercise}

In general, when the binary operation is written multiplicatively, one
writes the neutral element $e$ of a group as $1_G$ or simply as $1$. The inverse of $x$ will be 
denoted by $x^{-1}$. 

\begin{example}
    Let $n\geq1$. The set $\GL_n(\R)$ of $n\times n$ invertible real 
matrices forms a
    group with the usual matrix multiplication. The neutral element is the identity matrix. The product 
    of matrices is associative. And, by definition, every element of $\GL_n(\R)$ admits an inverse. 
\end{example}

It is a good idea to keep in mind the \emph{group of invertible matrices}. 
With this, the following properties (valid in every group) look familiar:
\begin{enumerate}
    \item $(x^{-1})^{-1}=x$ for all $x$.
    \item $(xy)^{-1}=y^{-1}x^{-1}$ for all $x,y$. 
\end{enumerate}

\begin{exercise}
\label{xca:ax=b}
    Prove that in a group, the equation $ax=b$ has 
    a unique solution: $x=a^{-1}b$. Similarly, 
    $x=ba^{-1}$ is the unique solution of the equation
    $xa=b$. 
\end{exercise}

\begin{definition}
    \index{Group!abelian}
    A group $G$ is \emph{abelian} if $xy=yx$ for all $x,y\in G$. 
\end{definition}

Most of the time, for abelian groups, we will use 
the \emph{additive notation}. This means that the binary operation
of the group will be denoted by $(x,y)\mapsto x+y$, the neutral
element by $0$ and 
the inverse of an element $x$ will be $-x$. 

\begin{example}
\label{exa:abelian_groups}
    Let us see some examples 
    abelian groups: 
    \begin{enumerate}
        \item $\Z$, $\Q$, $\R$ and $\C$ with the usual addition. 
        \item Let $n\geq2$. The set $\Z/n$ of integers modulo $n$ with the usual addition modulo $n$.
        \item $\Q^\times=\setminus\{0\}$, $\R^\times=\setminus\{0\}$ and $\C^\times=\setminus\{0\}$ 
        with the usual multiplication.
        \item Let $p$ be a prime number. The set $(\Z/p)^{\times}=\Z/p\setminus\{0\}$ of invertible integers modulo $p$ 
            with the usual multiplication modulo $p$. 
    \end{enumerate}
\end{example}

The groups of the first two items will be written in additive notation. 

\index{Group!trivial}
The \emph{trivial group} is the (unique) group containing exactly one element, the neutral element. We can write this group additively, so we have the group $\{0\}$ with the addition $0+0=0$, or multiplicatively as $\{1\}$ with multiplication $1\cdot 1=1$. 

\begin{definition}
    \index{Group!order}
    The \emph{order} $|G|$ of a group $G$ is the 
    cardinality of $G$. A group $G$ is said to be
    \emph{finite} if $|G|$ is finite and \emph{infinite}
    otherwise. 
\end{definition}

The group $\Z/n$ of integers modulo $n$ is a finite group of order $n$. 
The group $(\Z/p)^{\times}$ of units modulo $p$ is a finite
group of order $p-1$. The other groups of Example \ref{exa:abelian_groups} are infinite. 

\begin{exercise}
\label{xca:LR}
Let $G$ be a group and $g\in G$. Prove that 
the maps $L_g\colon G\to G$, $x\mapsto gx$, and 
$R_g\colon G\to G$, $x\mapsto xg$, are bijective. 
\end{exercise}

\index{Group!table}
Let $G=\{g_1,g_2,\dots,g_n\}$ be a finite group. The \emph{table} 
of $G$ is the matrix 
that in position $(i,j)$ has the element $g_ig_j$. 
For example, the table of 
the (additive) group  
$\Z/4$ of integers modulo 4 is the following:
\begin{center}
  \begin{tabular}{l|cccc}
     &0&1&2&3 \\
    \hline
    0 & 0 & 1 & 2 & 3\\
    1 & 1 & 2 & 3 & 0\\
    2 & 2 & 3 & 0 & 1\\
    3 & 3 & 0 & 1 & 2
  \end{tabular}
\end{center}

We know that $\Z$ is a group with the usual addition. 
We now discuss a multiplicative version of this group, as it will be
very important later. We first need a little bit of notation. 
Let $G$ be a group and $g\in G$. For $k\in\Z\setminus\{0\}$, we write 
\begin{align*}
    & g^k=g\cdots g\quad (k-\text{times}) && \text{if $k>0$},\\
    & g^k=g^{-1}\cdots g^{-1}\quad (|k|-\text{times}) && \text{if $k<0$}.
\end{align*}
By convention, $g^0=1$.
The following facts are left as an exercise: 
\begin{enumerate}
    \item $(x^k)^l=x^{kl}$ for all $x\in G$ and $k,l\in\Z$.
    \item If $G$ is abelian, then $(xy)^k=x^ky^k$ for all $x,y\in G$ and $k\in\Z$.
\end{enumerate}

\begin{example}
\label{exa:cyclic}
Fix a symbol $g$. Consider the set  
\[
\langle g\rangle=\{g^k:k\in\Z\}
\]
of integers powers of $g$ (with the usual convention $g^0=1$). Then
$\langle g\rangle$
with the operation $g^ig^j=g^{i+j}$ is an abelian group. 
\end{example}

We will see later that $\Z$ and the group of 
Example \ref{exa:cyclic} are ``indistinguishable'' 
as groups, even if they appear to be completely different. 

\begin{example}
Let $n$ be a positive integer. The set 
$G_n=\{z\in\C:z^n=1\}$ is an abelian group with
the usual multiplication of complex numbers. Moreover, the set 
$\cup_{n\geq1}G_n$ is an abelian group. 
\end{example}

\begin{example}
    Let $X$ be a non-empty set. The set $\Sym_X$ of bijective maps $X\to X$ 
    is a group with the usual composition of maps. If $|X|\geq3$, the group
    $\mathbb{S}_{X}$ is non-abelian. To prove this, let $a,b,c\in
        X$ be three different elements. Let $f\colon X\to X$ be such that
        $f(a)=b$, $f(b)=c$ and $f(c)=
a$ and $g\colon X\to
        X$ be such that $g(a)=b$, $g(b)=a$ and $g(x)=x$ for all $x\in
        X\setminus\{a,b\}$.  Then $fg\ne gf$.
\end{example}

\index{Permutation}
\index{Group!symmetric}
If $X=\{1,2,\dots,n\}$, the group $\Sym_X$ will be written as $\Sym_n$. This is
the \emph{symmetric group} of degree $n$. The elements of $\Sym_n$ are
called \emph{permutations} of degree $n$. 
Note that $|\Sym_n|=n!$ and $\Sym_n$
is abelian if and only if $n\in\{1,2\}$. Each element of $\Sym_n$ is a 
bijective map 
\[
f\colon\{1,\dots,n\}\to \{1,\dots,n\}.
\]
To denote permutations, 
we can use the following convention. The symbol  
\[
\binom{12345}{32145}
\]
denotes the map 
$f\colon\{1,2,3,4,5\}\to\{1,2,3,4,5\}$ such that 
\[
f(1)=3,
\quad f(2)=2,
\quad f(3)=1, 
\quad f(4)=4,
\quad f(5)=5.
\]
Here 2 and 4 are \emph{fixed points} of the permutation $f$. 

As we said, the operation of $\Sym_n$ is the usual
composition of maps. 
For example, 
\[
\binom{12345}{32145}
\binom{12345}{13452}
=
\binom{12345}{32145}
\circ 
\binom{12345}{13452}
=\binom{12345}{31452}.
\]

\begin{example}[Klein group]
\index{Klein group}
The set  
\[
K=\left\{ \mathrm{id},\binom{1234}{2143},\binom{1234}{3412},\binom{1234}{4321}\right\}
\]
together with the usual composition of maps is an abelian group. 
Note that $K$ is included in $\mathbb{S}_{4}$.
Can you compute the table of this group?
\end{example}

Every permutation can be written as a product of disjoint cycles. The fact 
is proved by induction but is relatively intuitive. Let us 
decompose the permutation 
\[
\sigma=\binom{123456789}{638915724}\in\Sym_{9}
\]
as a product of cycles. We need to 
draw a picture for $\sigma$:
\[\begin{tikzcd}
	1 & 6 & 2 & 3 & 4 & \\
	& 5 && 8 & 9 & 7\arrow[loop]        
	\arrow[from=1-1, to=1-2]
	\arrow[from=1-2, to=2-2]
	\arrow[from=2-2, to=1-1]
	\arrow[from=1-3, to=1-4]
	\arrow[from=1-4, to=2-4]
	\arrow[from=2-4, to=1-3]
	\arrow[from=1-5, to=2-5]
	\arrow[from=2-5, to=1-5]
\end{tikzcd}\]
We see that $\sigma$ has two 3-cycles, one 2-cycle
and one loop. 
Therefore 
\[
\sigma=(165)(238)(49)(7).
\]
Generally, 
one omits loops and orders  
the cycles according to 
the length. Thus 
\[
\sigma=(49)(165)(238).
\]

\begin{example}
\label{exa:S3}
\index{Symmetric group $\Sym_3$}
The set $\Sym_3$ of bijective maps $\{1,2,3\}\to\{1,2,3\}$
together with the composition of maps is a group of order six. 
Its elements are the permutations 
        \[
        \id,\binom{123}{213},\binom{123}{321},\binom{123}{132},\binom{123}{231},
\binom{123}{312}.
        \]
        Writing 
        permutations as a product of disjoint cycles,  
        the elements of $\Sym_3$ 
        are then         
        \[
                \id,(12),(13),(23),(123),(132),
        \]
        where, as we know, the symbol $(12)$ represents 
        the map $\{1,2,3\}\to\{1,2,3\}$ such that 
        $1\mapsto 2$, $2\mapsto 1$ and $3\mapsto 3$. 
        Can you construct the table for this group? 
\end{example}

\label{convention:left-to-right}
In our calculations with permutations, we have adopted the right-to-left convention, aligning with the rest of the course, where we consider groups acting on the left on sets. However, some people prefer the left-to-right convention, which corresponds to the way English is read (and to groups acting on the right). This is the convention used Magma:
\begin{lstlisting}
> S3 := Sym(3);
> a := S3!(1,2);
> b := S3!(2,3);
> a*b;
(1, 3, 2)
\end{lstlisting}
But there’s no need to worry: both conventions are equivalent, see Example \ref{ex:op}. 

\begin{example}
\label{exa:another_S3}
Let $\overline{\R}=\R\cup\{\infty\}$ (here $\infty$ is just a symbol) 
and assume that the following rules hold: 
\[
1/\infty=0,
\quad 1/0=\infty,
\quad \infty/\infty=1,
\quad 1-\infty=\infty-1=\infty.
\]
We now consider some maps $\overline{\R}\to\overline{\R}$ such as  
$x\mapsto x$, $x\mapsto 1-x$ and $x\mapsto\frac{1}{x}$. 
We claim that 
the set 
\[
G=\left\{x,\frac{1}{x},1-x,\frac{1}{1-x},\frac{x}{x-1},\frac{x-1}{x}\right\}
\subseteq\{f\colon\overline{\R}\to\overline{\R}:f\text{ is a map}\}
\]
is a non-abelian group with the usual composition of maps. How is this group ``acting'' on 
the set $\{0,1,\infty\}$? 
The group 
$G$ can be identified with the set of bijective maps 
\[
\{0,1,\infty\}\to\{0,1,\infty\}.
\]
For example, 
the map $x\mapsto\frac{1}{x}$ can be identified with the permutation of
the set 
$\{0,1,\infty\}$ that permutes $0$ and $\infty$ and fixes $1$. Similarly, 
$\frac{1}{1-x}$ permutes the elements $\{0,1,\infty\}$ cyclically 
in the following way:
\[
\begin{tikzcd}
        0 \\
        & 1 \\
        \infty
        \arrow[from=1-1, to=2-2]
        \arrow[from=2-2, to=3-1]
        \arrow[from=3-1, to=1-1]
\end{tikzcd}\]
\end{example}
Writing the elements of $G$ as cycles, 
\[
G=\{\id,
(0\;\infty),
(0\:1),
(1\;\infty\;0),
(\infty\;1),
(1\;0\;\infty)\}.
\]

We will see later that the 
groups of Examples \ref{exa:S3} and \ref{exa:another_S3} are
indeed ``indistinguishable'' as groups. 

\begin{example}
Let $n\geq2$. The multiplicative units of $\Z/n$ form a group 
with the usual multiplication modulo $n$. 
We will use the following notation:
\[
\mathcal{U}(\Z/n)=\{x\in\Z/n:\gcd(x,n)=1\}.
\]
The order of $\mathcal{U}(\Z/n)$ is $\varphi(n)$, where $\varphi$ 
is the Euler's function, that is 
\[
\varphi(n)=|\{x\in\Z:1\leq x\leq n,\,\gcd(x,n)=1\}|.
\]

Let us show a concrete example. The table of 
$\mathcal{U}(\Z/8)=\{1,3,5,7\}$
is
\begin{center}
  \begin{tabular}{l|cccc}
     &1&3&5&7 \\
    \hline
    1 & 1 & 3 & 5 & 7\\
    3 & 3 & 1 & 7 & 5\\
    5 & 5 & 7 & 1 & 3\\
    7 & 7 & 5 & 3 & 1
  \end{tabular}
\end{center}
\end{example}

\begin{exercise}
\label{xca:GxH}
        \index{Direct product!of groups}
        Let $G$ and $H$ be groups. Prove that 
        the set 
        $G\times H$
        of pairs $(g,h)$, where $g\in G$ and 
        $h\in H$ is a group with
        the operation
        \[
                (g,h)(g_1,h_1)=(gg_1,hh_1).
        \]
        This group is called the 
        \emph{direct product} of $G$ and $H$.
\end{exercise}

The construction of Example \ref{xca:GxH}
can be easily generalized to the product of 
three or more groups. 


% add two bonus exercises
% cos(a+b) and another group structure on... 