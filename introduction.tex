\thispagestyle{plain}
\section*{Introduction}

The notes correspond to the bachelor 
course \textbf{Group Theory} of the 
Vrije Universiteit Brussel, 
Faculty of Sciences, 
Department of Mathematics and Data Sciences. The course
is divided into twelve two-hour lectures. 
The topics
covered here are the following: 

\bigskip 
\contentsline {section}{\tocsubsection {\S }{1.1}{Groups}}{4}{subsection.1.1}%
\contentsline {section}{\tocsubsection {\S }{2.1}{Subgroups}}{9}{subsection.2.1}%
\contentsline {section}{\tocsubsection {\S }{2.2}{Subgroups of $\mathbb {Z}$}}{11}{subsection.2.2}%
\contentsline {section}{\tocsubsection {\S }{2.3}{Commutators}}{13}{subsection.2.3}%
\contentsline {section}{\tocsubsection {\S }{2.4}{Cyclic groups}}{14}{subsection.2.4}%
\contentsline {section}{\tocsubsection {\S }{3.1}{Lagrange's theorem}}{17}{subsection.3.1}%
\contentsline {section}{\tocsubsection {\S }{4.1}{The symmetric group}}{21}{subsection.4.1}%
\contentsline {section}{\tocsubsection {\S }{5.1}{Quotients}}{26}{subsection.5.1}%
\contentsline {section}{\tocsubsection {\S }{6.1}{Permutable subgroups}}{30}{subsection.6.1}%
\contentsline {section}{\tocsubsection {\S }{6.2}{Homomorphisms}}{31}{subsection.6.2}%
\contentsline {section}{\tocsection {\S }{7.1}{Isomorphism theorems}}{36}{subsection.7.1}%
\contentsline {section}{\tocsubsection {\S }{8.1}{Semi-direct products}}{43}{subsection.8.1}%
\contentsline {section}{\tocsubsection {\S }{9.1}{Actions of groups on sets}}{49}{subsection.9.1}%
\contentsline {section}{\tocsubsection {\S }{10.1}{$p$-groups}}{56}{subsection.10.1}%
\contentsline {section}{\tocsubsection {\S }{10.2}{Cauchy's theorem}}{57}{subsection.10.2}%
\contentsline {section}{\tocsubsection {\S }{11.1}{Sylow's theorems}}{59}{subsection.11.1}%
\contentsline {section}{\tocsubsection {\S }{12.1}{More about Sylow's theorems}}{64}{subsection.12.1}%
\contentsline {section}{\tocsubsection {\S }{12.2}{Abelian groups}}{65}{subsection.12.2}%

\tableofcontents 
\bigskip 

The material is somewhat standard. Basic texts on abstract algebra
are for example \cite{MR1129886}, \cite{MR2286236} and \cite{MR600654}. 
Lang's book \cite{MR783636} is also a standard reference, but 
maybe a bit more advanced. 

We also mention a set of 
\href{https://kconrad.math.uconn.edu/blurbs/}{great expository papers} by Keith Conrad. 
The notes are extremely well-written and are useful at  
every stage of a mathematical career. 

% Bibtex information:
% {\footnotesize\begin{verbatim}
% @misc{rings,
%     author={Vendramin, L.},
%     title={Rings and modules},
%     year={2022},
%     note={Available at www.github.com/vendramin/rings},
%     pages={106}
% }
% \end{verbatim}}

 Thanks go to Heleen Broodcoorens and 
 Senne Trappeniers. 
 %Wouter Appelmans, Arne van Antwerpen, Ilaria Colazzo, Luca Descheemaeker, {\L}ukas Kubat, Lucas Simons
% and Geoffrey Jassens. 

This version 
was compiled on \today~at~\currenttime. 
Please send comments and corrections to me at \url{Leandro.Vendramin@vub.be}. 


% \bigskip
% \begin{flushright}
% Leandro Vendramin\\Brussels, Belgium\par
% \end{flushright}
