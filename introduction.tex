\thispagestyle{plain}

\section*{Introduction}

The notes correspond to the bachelor 
course \textbf{Group Theory} of the 
Vrije Universiteit Brussel, 
Faculty of Sciences, 
Department of Mathematics and Data Sciences. The course
is divided into twelve two-hour lectures. 



The material is somewhat standard. Basic texts on abstract algebra
are for example \cite{MR1129886}, \cite{MR2286236} and \cite{MR600654}. 
Lang's book \cite{MR783636} is also a standard reference, but 
maybe a bit more advanced. 

We also mention a set of 
\href{https://kconrad.math.uconn.edu/blurbs/}{great expository papers} by Keith Conrad. 
The notes are extremely well-written and are useful at  
every stage of a mathematical career. 

The notes include Magma code, which we use to verify examples and offer alternative solutions to certain exercises. Magma \cite{zbMATH01077111} is a powerful software tool designed for working with algebraic structures. There is a free \href{https://magma.maths.usyd.edu.au/calc/}{online} version of Magma available.


% Bibtex information:
% {\footnotesize\begin{verbatim}
% @misc{rings,
%     author={Vendramin, L.},
%     title={Rings and modules},
%     year={2022},
%     note={Available at www.github.com/vendramin/rings},
%     pages={106}
% }
% \end{verbatim}}

 Thanks go to Heleen Broodcoorens, Davide Ferri and 
 Senne Trappeniers. 
 %Wouter Appelmans, Arne van Antwerpen, Ilaria Colazzo, Luca Descheemaeker, {\L}ukas Kubat, Lucas Simons
% and Geoffrey Jassens. 

This version 
was compiled on \today~at~\currenttime. 
Please send comments and corrections to me at \url{Leandro.Vendramin@vub.be}. 


% \bigskip
% \begin{flushright}
% Leandro Vendramin\\Brussels, Belgium\par
% \end{flushright}
