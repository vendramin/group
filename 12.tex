\section{02/05/2024}

\subsection{Double cosets (optional)}

The idea used in Example \ref{exa:for_HK} can be generalized. 

\begin{example}
\index{Coclase!doble}
Sea $G$ un grupo y sean $H$ y $K$ subgrupos de $G$. Hacemos
que el grupo $L=H\times K$ actúe en $G$ por
\[
(h,k)\cdot g=hgk^{-1}.
\]
Las órbitas son los conjuntos de la forma
\[
HgK=\{hgk:h\in H,\,k\in K\},
\]
estos conjuntos se llaman $(H,K)$-coclases dobles.
En particular, dos $(H,K)$-coclases dobles son disjuntas o iguales. Más aún,
$G$ se descompone como unión disjunta
\[
G=\bigcup_{i\in I}Hg_iK,
\]
para algún conjunto $I$, es decir $G$
es unión disjunta de $(H,K)$-coclases dobles.
Calculamos ahora
\[
L_g=\{(h,k)\in H\times K:hgk^{-1}=g\}=\{(h,g^{-1}hg)\in H\times K\}
\]
y vemos que $|L_g|=|H\cap gKg^{-1}|$, pues los conjuntos $L_g$ y $H\cap gKg^{-1}$ están en biyección.

Luego,
gracias al principio fundamental del conteo,
\[
|HgK|=(L:L_g)=\frac{|H\times K|}{|H\cap gKg^{-1}|}=\frac{|H||K|}{|H\cap gKg^{-1}|}.
\]
\end{example}

El ejemplo anterior puede generalizarse, lo que nos da una descomposición
de un grupo como unión disjunta de \textbf{coclases dobles}. Veremos más adelante
demostraciones alternativas de los teoremas de Sylow basadas en
coclases dobles.

\begin{quote}
Veamos una demostración alternativa del primer teorema de Sylow
que utiliza coclases dobles.
Primero demostraremos un resultado auxiliar. Si $P\in\Syl_p(G)$ y $H\leq G$, entonces
existe un $g\in G$ tal que $H\cap gPg^{-1}\in\Syl_p(H)$. En efecto, supongamos que
$|H|=p^\beta t$ con $p$ coprimo con $t$.
Si descomponemos a $G$ en $(H,P)$-coclases dobles,
\[
|G|=\sum_{i=1}^k \frac{|H||P|}{|H\cap x_iPx_i^{-1}|}.
\]
Al simplificar $|P|=p^\alpha$, tenemos que $m=\sum_{i=1}^k(H:H\cap x_iPx_i^{-1})$, lo que nos dice
que existe $i\in\{1,\dots,k\}$ tal que
$(H:H\cap x_iPx_i^{-1})$ no es divisible por $p$. Esto significa que que
$p^\beta$ divide a $|H\cap x_iPx_i^{-1}|$ y en consecuencia $H\cap x_iPx_i^{-1}\in\Syl_p(H)$.
Por el teorema de Cayley podemos suponer que nuestro subgrupo $G$ es un subgrupo
de $\GL_n(p)$ para algún $n$ y algún primo $p$. Sea $P$ un subgrupo de Sylow
del grupo $\GL_n(P)$. La observación que demostramos
aplicada al grupo $G$ nos dice que existe $g\in \GL_n(p)$ tal que $G\cap gPg^{-1}$ es un subgrupo de
Sylow de $G$.
\end{quote}

\begin{quote}
Veamos una
demostración alternativa del segundo teorema de Sylow que usa coclases dobles.
Si $P,Q\in\Syl_p(G)$ y
descomponemos a $G$ en $(P,Q)$-coclases dobles, tenemos
\[
p^\alpha m=\sum_{i=1}^k\frac{|P||Q|}{|P\cap x_iQx_i^{-1}|}
\implies
m=\sum_{i=1}^k\frac{|P|}{|P\cap x_iQx_i^{-1}|}
\]
para ciertos $x_1,\dots,x_k\in G$.
Como $m$ no es divisible por $p$, existe algún $i\in\{1,\dots,k\}$ tal que $|P|=|P\cap x_iQx_i^{-1}|$
, lo que
implica que $P=x_iQx_i^{-1}$ para algún $i\in\{1,\dots,k\}$.
\end{quote}

\begin{quote}
Una demostración alternativa del tercer teorema de Sylow basada en coclases dobles.
Sean $P\in\Syl_p(G)$ y $N=N_G(P)$. Recordemos
que $n_p(G)=(G:N)$. Si descomponemos
a $G$ en $(P,N)$-coclases dobles,
\[
|G|=\sum_{i=1}^k\frac{|P||N|}{|N\cap x_iPx_i^{-1}|}
\]
para ciertos $x_1,\dots,x_k\in G$. Sin perder generalidad podemos suponer que $x_1=1$, entonces la fó
rmula anterior queda
\[
n_p(G)=1+\sum_{i=2}^k\frac{|P|}{|N\cap x_iPx_i^{-1}|},
\]
pues $(G:N)=n_p(G)$.
El teorema quedará demostrado si vemos que la suma del miembro de la derecha es divisible por $p$. Si
 esto no pasa,
es decir si existe $i\in\{2,\dots,k\}$ tal que $|N\cap x_iPx_i^{-1}|=|P|$, entonces
$x_iPx_i^{-1}=N\cap x_iPx_i^{-1}\subseteq N$. Como entonces $P$ y también $x_iPx_i^{-1}$ son ambos $p
$-subgrupos de Sylow de $N$,
el segundo teorema de Sylow afirma que estos subgrupos tienen que ser conjugados en $N$. Por definici
ón del normalizador, $P$ es normal en $N$.
En consecuencia, $x_iPx_i=P$, es decir $x_i\in N$, una contradicción pues como $i>1$ se tiene que
$Px_iN$ y $Px_1N=PN$ son coclases dobles disjuntas.
\end{quote}