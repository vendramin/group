\section{Lecture -- Week 12}

We conclude our study of the Sylow theorems with two final results, which may be omitted on a first reading.

\begin{optional}
\begin{theorem}
    Let $N$ be a normal subgroup of a finite group 
    $G$ and $P\in\Syl_p(G)$. Then 
    $P\cap N\in\Syl_p(N)$. Moreover, every Sylow 
    $p$-subgroup of $N$  
    can be obtained this way.
\end{theorem}

\begin{proof}
    Since $N$ is normal,
    by Theorem \ref{thm:Sylow_auxiliar} applied to the group $N$, 
    there exists 
    $g\in G$ such that 
        \[
                g(P\cap N)g^{-1}=gPg^{-1}\cap gNg^{-1}=gPg^{-1}\cap N\in\Syl_p(N).
        \]
    Then $P\cap N$ is a Sylow $p$-subgroup of $g^{-1}Ng=N$.

    Let $Q\in\Syl_pN$ and $P\in\Syl_p(G)$ be such that
    $Q\subseteq P$. Then
    $Q\subseteq P\cap N$. Hence 
    $Q=P\cap N$, as $P\cap N$ is a Sylow $p$-subgroup of $N$. 
\end{proof}
\end{optional}

As a corollary, if $G$ is a finite group and 
$N$ is a normal subgroup of $G$, then 
$n_p(N)\leq n_p(G)$.

\begin{optional}
\begin{theorem}
    Let $p$ be a prime number, $G$ be a finite group, $P\in\Syl_p(G)$, 
    and $N$ be a normal subgroup of $G$. Let $\pi\colon G\to G/N$ be the canonical homomorphism. Then 
    $\pi(P)\in\Syl_p(G/N)$ and every Sylow 
    $p$-subgroup of $G/N$ can be obtained this way. 
\end{theorem}

\begin{proof}
    Since $\pi(P)=(\pi|_{P})(P)\simeq P/N\cap P$,
    the second isomorphism theorem 
    implies that 
    $\pi(P)$ is a $p$-group. Since
    $|PN|=|P||N|/|P\cap N|$,
        \[
                (G/N:\pi(P))=(G:PN)
        \]
        is not divisible by $p$. Thus $\pi(P)\in\Syl_p(G/N)$.

        If $Q\in\Syl_p(G/N)$, then $Q=\pi(H)$ for some
        subgroup $H$ of $G$
        with $N\subseteq H$. In particular,
        \[
                |Q|=|\pi(H)|=\frac{|H|}{|H\cap N|}=\frac{|H|}{|N|}.
        \]
        Thus 
        \[
                (G:H)=\frac{|G|/|N|}{|H|/|N|}=(G/N:Q)
        \]
        is not divisible by $p$. 
        
        Let $X\in\Syl_p(H)$. Since $(G:H)$ is not divisible by $p$, 
        $X\in\Syl_p(G)$. Hence 
        \[
        \pi(X)\subseteq\pi(H)=Q. 
        \]
        Thus 
        $\pi(X)=Q$, as 
        $\pi(X)\in\Syl_p(G/N)$.
\end{proof}
\end{optional}

As a corollary, if $G$ is a finite group and
$N$ is a normal subgroup of $G$, then 
\[
n_p(G/N)\leq n_p(G).
\]

\begin{optional}
\begin{corollary}
Let $G$ be a finite group. Assume that 
$G$ contains only one Sylow $p$-subgroup. Then
every subgroup and every quotient of 
$G$ 
contains only one Sylow 
$p$-subgroup. 
\end{corollary}

\begin{proof}
    If $H$ is a subgroup of $G$, then
    $n_p(H)\leq n_p(G)=1$. If $N$ is a normal subgroup of 
    $G$, then $n_p(G/N)\leq n_p(G)=1$.
\end{proof}    
\end{optional}

%Veamos una aplicación.
%
%\begin{theorem}[Wilson]
%       Sea $n\in\N$. Entonces $n$ es primo si y sólo si
%       $(n-1)!\equiv -1\bmod n$.
%\end{theorem}
%
%\begin{proof}
%       Sea $p$ un número primo.
%       El grupo $\Sym_p$ tiene $(p-1)!$ elementos de orden $p$. Cada
%       $p$-subgroupo de Sylow de $\Sym_p$ está generado por un $p$-ciclo, y luego
%       $n_p\equiv (p-2)!$. Por el tercer teorema de Sylow,
%       $(p-2)!=n_p\equiv 1\bmod p$. Al multiplicar por $p-1$, tenemos
%       $(p-1)!\equiv -1\bmod p$.
%\end{proof}

\subsection{Abelian groups}

Let $A$ be an abelian group and $x_1,\dots,x_k\in A$. In this subsection we will use additive notation.  
The subgroup $\langle x_1,\dots,x_k\rangle$ 
generated by $\{x_1,\dots,x_k\}$
is the set of integer linear combinations of the elements 
$x_1,\dots,x_k$, that is
\[ 
\langle x_1,\dots,x_k\rangle=\left\{ 
\sum_{i=1}^k m_ix_i: m_1,\dots,m_k\in\Z\right\}.
\]
We say that $\{x_1,\dots,x_k\}$ \emph{generates} $A$ if 
$A=\langle x_1,\dots,x_k\rangle$. And we say that the set 
$\{x_1,\dots,x_k\}$ is 
\emph{linearly independent} if 
\[
\sum_{i=1}^k m_ix_i=0\implies m_1x_1=\cdots=m_kx_k=0.
\]

Note that our definition of linearly independence for abelian groups is slightly 
different from that of linear algebra. For example, in the group $\Z/5$, 
one has $5x=0$ for all $x$. Thus there will be no linearly independent sets with 
the standard linear algebra definition. 

\begin{definition}
\index{Direct sum of abelian groups}
    Let $A$ be an abelian group and $X$ and $Y$ be subgroups of $A$. 
    We say that $A$ is the \emph{direct sum} of $X$ and $Y$ 
    if $A=X+Y$ and $X\cap Y=\{0\}$. In this case, we write $A=X\oplus Y$. 
\end{definition}

\begin{exercise}
    Let $A$ be an abelian group and $X$ and $Y$ be subgroups of $A$
    such that $A=X\oplus Y$.     
    Prove that 
    every element $a\in A$ can be
    written uniquely as $a=x+y$ for $x\in X$ and $y\in Y$.     
\end{exercise}

A subset $\{x_1,\dots,x_k\}$ is a \emph{basis} of $A$ 
if $A=\langle x_1\rangle\oplus\cdots\oplus \langle x_k\rangle$, that is
if $\{x_1,\dots,x_k\}$ is a 
linearly independent set of generators of $A$. 


\begin{theorem}
\label{thm:fundamental_abelian}
    Every finitely generated abelian group has a basis. In particular, 
    it is a finite direct sum of cyclic groups. 
\end{theorem}

Before proving the theorem, we need a lemma.

\begin{lemma}
\label{lem:trick_abelian}
    Let $A=\langle x_1,\dots,x_n\rangle$ be a finitely generated 
    abelian group and $c_1,\dots,c_n$ be positive integers such that 
    $\gcd(c_1,\dots,c_n)=1$. Then there exist $y_1,\dots,y_n\in A$ 
    such that 
    $A=\langle y_1,\dots,y_n\rangle$ and 
    \[ 
    y_1=c_1x_1+\cdots+c_nx_n.
    \]
\end{lemma}

\begin{proof}
    We proceed by induction on $s=c_1+\cdots+c_n$. The case $s=1$ is trivial. 
    So assume that $s\geq2$. Without loss of generality, we may assume that 
    $c_1\geq c_2>0$. Then 
    \[ 
    (c_1-c_2)+c_2+c_3+\cdots+c_n=c_1+c_3+\cdots+c_n<s.
    \]
    Moreover, $\gcd(c_1-c_2,c_2,\dots,c_n)=1$. Since $A=\langle x_1,x_1+x_2,x_3,\dots,x_n\rangle$, the inductive hypothesis implies that 
    there exist $y_1,\dots,y_n\in A$ such that 
    $A=\langle y_1,\dots,y_n\rangle$ and 
    \[ 
        y_1=(c_1-c_2)x_1+c_2(x_1+x_2)+c_3x_3+\cdots+c_nx_n
        =c_1x_1+c_2x_2+\cdots+c_nx_n.\qedhere
    \]
\end{proof}

Now we are ready to prove the main theorem of this section. 

\begin{proof}[Proof of Theorem \ref{thm:fundamental_abelian}]
    We proceed by induction on the number $n$ of generators. The case $n=1$ is trivial. So assume that the result holds for $n-1$ generators. 
    Among the generating sets $\{x_1,\dots,x_n\}$ 
    with $n$ elements, there is one 
    for which the order $|x_1|$ of $x_1$ is the smallest possible. By the inductive hypothesis, 
    the theorem will be proved if we can show that 
    \begin{equation}
    \label{eq:decomposition}
    A=\langle x_1\rangle\oplus \langle x_2,\dots,x_n\rangle
    \end{equation}
    holds. 
    Assume that \eqref{eq:decomposition} does not hold. 
    Note that 
    $A=\langle x_1\rangle+\langle x_2,\dots,x_n\rangle$, as 
    $\{x_1,\dots,x_n\}$ is a generating set of $A$. Since
    the decomposition  
    \eqref{eq:decomposition} does not hold, 
    $\langle x_1\rangle\cap \langle x_2,\dots,x_n\rangle\ne\{0\}$. Let  
    $\xi\in \langle x_1\rangle\cap \langle x_2,\dots,x_n\rangle$ be a non-zero element. Then 
    $\xi=m_1x_1=m_2x_2+\cdots+m_nx_n$ for some integers  
    $m_1\ne 0$ and $m_2,\dots,m_n\in\Z$ not all zero.  Thus 
    \[ 
    (-m_1)x_1+m_2+\cdots+m_nx_n=0.
    \]
    After changing the sign of some of the generators, 
    we produce a generating set $\{z_1,\dots,z_k\}$ of $A$ 
    such that our linear combination becomes 
    \[ 
    \lambda_1z_1+\lambda_2z_2+\cdots+\lambda_nz_n=0, 
    \]
    where $\lambda_1,\dots,\lambda_n$ are non-negative integers  
    and $0<\lambda_1<|z_1|$. Let $d=\gcd(\lambda_1,\dots,\lambda_n)$ and
    for each $i\in\{1,\dots,n\}$, let 
    $c_i=\lambda_i/d$. 
    By Lemma~\ref{lem:trick_abelian}, there exist 
    $y_1,\dots,y_n\in A$ such that 
    $A=\langle y_1,\dots,y_n\rangle$ and 
    \[ 
    y_1=c_1z_1+\cdots+c_nz_n.
    \]
    But $dy_1=\lambda_1z_1+\cdots+\lambda_nz_n=0$ and 
    $d\leq \lambda_1<|x_1|$. We have found a generating set 
    $\{y_1,\dots,y_n\}$ in which the element
    $y_1$ has order smaller than $|x_1|$, a contradiction. 
\end{proof}

The previous theorem 
translates into the following result.

\begin{theorem}
\label{thm:abelian_factors}
    Let $A$ be a non-zero finitely generated 
    abelian group. Then 
    \[
    A\simeq (\Z/{n_1})\times\cdots\times (\Z/{n_k})\times\Z^r,
    \]
    for integers $n_1,\dots,n_k\geq2$ and 
    $r\geq0$. The integers
    $n_1,\dots,n_k$ 
    can be chosen so that $n_1\geq2$ and 
    $n_j$ divides $n_{j+1}$ for all $j\in\{1,\dots,k-1\}$. 
\end{theorem}

The integer $r$ 
in Theorem \ref{thm:abelian_factors}
is uniquely determined by $A$ and 
is called the \emph{rank} of $A$. The integers $n_1,\dots,n_k$ 
in Theorem \ref{thm:abelian_factors}
are called the \emph{invariant factors} of $A$ and 
are uniquely determined by $A$. 

In these notes, we will not prove that the rank and the invariant factors are uniquely determined by the group. Additionally, we will not prove the existence 
of the invariant factors. Instead, we will explain how to obtain them with some concrete examples.

\begin{example}
    Let $A=(\Z/6)\times(\Z/100)\times(\Z/45)$. We use 
    the fact that 
    \[
    (\Z/a)\times(\Z/b)\simeq\Z/ab
    \]
    whenever $\gcd(a,b)=1$
    to decompose $A$ as follows:
    \[
    A\simeq (\Z/2\times\Z/3)\times(\Z/4\times\Z/25)\times(\Z/5\times\Z/9).
    \]
    Let us order the prime powers: 2, 4, 3, 9, 5, 25. 
    Now we collect the highest prime powers appearing in our decomposition: 4 is the highest power of 2,
    9 is the highest power of 3, and 25 is the highest power of 5. Thus 
    $s_2=4\times 9\times 25=900$ is the highest invariant factor. Now 
    2 is the highest remaining power of 2, 
    3 is the highest power of 3 and 5 is the highest power of 5. Thus 
    $s_1=2\times 3\times 5=30$ is the second invariant factor. 
    Thus 
    \[ 
    A\simeq (\Z/30)\times(\Z/900).
    \]
\end{example}

\begin{example}
    Let $A=(\Z/10)\times(\Z/15)\times(\Z/20)\times(\Z/25)$. 
    As we did in the previous example, we decompose each factor: 
    \[ 
    A\simeq (\Z/2)\times(\Z/5)\times(\Z/3)\times (\Z/5)\times(\Z/4)\times (\Z/5)\times(\Z/25). 
    \]
    The numbers we see are 2, 4, 3, 5, 25. The invariant factors are
    then $s_3=4\times 3\times 25=300$, $s_2=10$, $s_3=5$ and $s_4=5$. 
    Hence 
    \[ 
    A\simeq (\Z/5)\times(\Z/5)\times(\Z/10)\times(\Z/300).
    \]
\end{example}

\begin{example}
    There are six abelian groups of order 200, namely
    \begin{align*}
        &\Z/{200}, 
        &&\Z/{2}\times\Z/{100}, 
        &&\Z/{2}\times\Z/{2}\times\Z/{50},\\ 
        &\Z/{5}\times\Z/{40}, 
        &&\Z/{10}\times\Z/{20}, 
        &&\Z/{2}\times\Z/{10}\times\Z/{10}.
    \end{align*}
\end{example}

\begin{exercise}
\label{xca:factors:24,12,4,2}
    Find the invariant factors 
    of the group $(\Z/4)\times(\Z/6)\times(\Z/8)\times(\Z/12)$. 
\end{exercise}

\begin{exercise}
\label{xca:howmany:500}
    How many abelian groups of order 500 are there? 
\end{exercise}


    

% \subsection{Double cosets (optional)}

% The idea used in Example \ref{exa:for_HK} can be generalized. 

% \begin{example}
% \index{Coclase!doble}
% Sea $G$ un grupo y sean $H$ y $K$ subgrupos de $G$. Hacemos
% que el grupo $L=H\times K$ actúe en $G$ por
% \[
% (h,k)\cdot g=hgk^{-1}.
% \]
% Las órbitas son los conjuntos de la forma
% \[
% HgK=\{hgk:h\in H,\,k\in K\},
% \]
% estos conjuntos se llaman $(H,K)$-coclases dobles.
% En particular, dos $(H,K)$-coclases dobles son disjuntas o iguales. Más aún,
% $G$ se descompone como unión disjunta
% \[
% G=\bigcup_{i\in I}Hg_iK,
% \]
% para algún conjunto $I$, es decir $G$
% es unión disjunta de $(H,K)$-coclases dobles.
% Calculamos ahora
% \[
% L_g=\{(h,k)\in H\times K:hgk^{-1}=g\}=\{(h,g^{-1}hg)\in H\times K\}
% \]
% y vemos que $|L_g|=|H\cap gKg^{-1}|$, pues los conjuntos $L_g$ y $H\cap gKg^{-1}$ están en biyección.

% Luego,
% gracias al principio fundamental del conteo,
% \[
% |HgK|=(L:L_g)=\frac{|H\times K|}{|H\cap gKg^{-1}|}=\frac{|H||K|}{|H\cap gKg^{-1}|}.
% \]
% \end{example}

% El ejemplo anterior puede generalizarse, lo que nos da una descomposición
% de un grupo como unión disjunta de \emph{coclases dobles}. Veremos más adelante
% demostraciones alternativas de los teoremas de Sylow basadas en
% coclases dobles.

% \begin{quote}
% Veamos una demostración alternativa del primer teorema de Sylow
% que utiliza coclases dobles.
% Primero demostraremos un resultado auxiliar. Si $P\in\Syl_p(G)$ y $H\leq G$, entonces
% existe un $g\in G$ tal que $H\cap gPg^{-1}\in\Syl_p(H)$. En efecto, supongamos que
% $|H|=p^\beta t$ con $p$ coprimo con $t$.
% Si descomponemos a $G$ en $(H,P)$-coclases dobles,
% \[
% |G|=\sum_{i=1}^k \frac{|H||P|}{|H\cap x_iPx_i^{-1}|}.
% \]
% Al simplificar $|P|=p^\alpha$, tenemos que $m=\sum_{i=1}^k(H:H\cap x_iPx_i^{-1})$, lo que nos dice
% que existe $i\in\{1,\dots,k\}$ tal que
% $(H:H\cap x_iPx_i^{-1})$ no es divisible por $p$. Esto significa que que
% $p^\beta$ divide a $|H\cap x_iPx_i^{-1}|$ y en consecuencia $H\cap x_iPx_i^{-1}\in\Syl_p(H)$.
% Por el teorema de Cayley podemos suponer que nuestro subgrupo $G$ es un subgrupo
% de $\GL_n(p)$ para algún $n$ y algún primo $p$. Sea $P$ un subgrupo de Sylow
% del grupo $\GL_n(P)$. La observación que demostramos
% aplicada al grupo $G$ nos dice que existe $g\in \GL_n(p)$ tal que $G\cap gPg^{-1}$ es un subgrupo de
% Sylow de $G$.
% \end{quote}

% \begin{quote}
% Veamos una
% demostración alternativa del segundo teorema de Sylow que usa coclases dobles.
% Si $P,Q\in\Syl_p(G)$ y
% descomponemos a $G$ en $(P,Q)$-coclases dobles, tenemos
% \[
% p^\alpha m=\sum_{i=1}^k\frac{|P||Q|}{|P\cap x_iQx_i^{-1}|}
% \implies
% m=\sum_{i=1}^k\frac{|P|}{|P\cap x_iQx_i^{-1}|}
% \]
% para ciertos $x_1,\dots,x_k\in G$.
% Como $m$ no es divisible por $p$, existe algún $i\in\{1,\dots,k\}$ tal que $|P|=|P\cap x_iQx_i^{-1}|$
% , lo que
% implica que $P=x_iQx_i^{-1}$ para algún $i\in\{1,\dots,k\}$.
% \end{quote}

% \begin{quote}
% Una demostración alternativa del tercer teorema de Sylow basada en coclases dobles.
% Sean $P\in\Syl_p(G)$ y $N=N_G(P)$. Recordemos
% que $n_p(G)=(G:N)$. Si descomponemos
% a $G$ en $(P,N)$-coclases dobles,
% \[
% |G|=\sum_{i=1}^k\frac{|P||N|}{|N\cap x_iPx_i^{-1}|}
% \]
% para ciertos $x_1,\dots,x_k\in G$. Sin perder generalidad podemos suponer que $x_1=1$, entonces la fó
% rmula anterior queda
% \[
% n_p(G)=1+\sum_{i=2}^k\frac{|P|}{|N\cap x_iPx_i^{-1}|},
% \]
% pues $(G:N)=n_p(G)$.
% El teorema quedará demostrado si vemos que la suma del miembro de la derecha es divisible por $p$. Si
%  esto no pasa,
% es decir si existe $i\in\{2,\dots,k\}$ tal que $|N\cap x_iPx_i^{-1}|=|P|$, entonces
% $x_iPx_i^{-1}=N\cap x_iPx_i^{-1}\subseteq N$. Como entonces $P$ y también $x_iPx_i^{-1}$ son ambos $p
% $-subgrupos de Sylow de $N$,
% el segundo teorema de Sylow afirma que estos subgrupos tienen que ser conjugados en $N$. Por definici
% ón del normalizador, $P$ es normal en $N$.
% En consecuencia, $x_iPx_i=P$, es decir $x_i\in N$, una contradicción pues como $i>1$ se tiene que
% $Px_iN$ y $Px_1N=PN$ son coclases dobles disjuntas.
% \end{quote}