\chapter{}

\topic{Double cosets (optional)}

The idea used in Example \ref{exa:for_HK} can be generalized. 

\begin{example}
\index{Coclase!doble}
Sea $G$ un grupo y sean $H$ y $K$ subgrupos de $G$. Hacemos
que el grupo $L=H\times K$ actúe en $G$ por
\[
(h,k)\cdot g=hgk^{-1}.
\]
Las órbitas son los conjuntos de la forma
\[
HgK=\{hgk:h\in H,\,k\in K\},
\]
estos conjuntos se llaman $(H,K)$-coclases dobles.
En particular, dos $(H,K)$-coclases dobles son disjuntas o iguales. Más aún,
$G$ se descompone como unión disjunta
\[
G=\bigcup_{i\in I}Hg_iK,
\]
para algún conjunto $I$, es decir $G$
es unión disjunta de $(H,K)$-coclases dobles.
Calculamos ahora
\[
L_g=\{(h,k)\in H\times K:hgk^{-1}=g\}=\{(h,g^{-1}hg)\in H\times K\}
\]
y vemos que $|L_g|=|H\cap gKg^{-1}|$, pues los conjuntos $L_g$ y $H\cap gKg^{-1}$ están en biyección.

Luego,
gracias al principio fundamental del conteo,
\[
|HgK|=(L:L_g)=\frac{|H\times K|}{|H\cap gKg^{-1}|}=\frac{|H||K|}{|H\cap gKg^{-1}|}.
\]
\end{example}

El ejemplo anterior puede generalizarse, lo que nos da una descomposición
de un grupo como unión disjunta de \textbf{coclases dobles}. Veremos más adelante
demostraciones alternativas de los teoremas de Sylow basadas en
coclases dobles.